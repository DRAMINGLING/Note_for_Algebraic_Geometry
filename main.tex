\documentclass{ctexart}
\usepackage[a4paper,left=3cm,right=3cm,top=3cm,bottom=3cm]{geometry}
\usepackage{amssymb,amsfonts,amsmath,amsthm}
\usepackage{anyfontsize}
\usepackage{graphicx}
\usepackage{tikz-cd}
\usepackage{bm}
\usepackage{bbm}
\usepackage{mathrsfs}
\usepackage[colorlinks = true,      % 启用彩色链接而非带框链接
            linkcolor = blue,       % 内部链接(如目录、交叉引用)颜色
            urlcolor = blue,        % 外部链接颜色
            citecolor = blue]{hyperref} % 参考文献链接颜色

\newtheorem{definition}{Definition}[section]
\newtheorem{remark}{Remark}[section]
\newtheorem{theorem}{Theorem}[section]
\newtheorem{proposition}{Proposition}[section]
\newtheorem{setting}{Setting}[section]
\newtheorem{lemma}{Lemma}[section]
\newtheorem{reason}{Reason}[section]
\newtheorem{example}{Example}[section]
\newtheorem{corollary}{Corollary}[section]
\newtheorem{exercise}{Exercise}[section]

\renewcommand{\proofname}{\textit{Proof}}
\renewcommand{\contentsname}{Contents}
\renewcommand{\refname}{References}
\renewcommand{\hom}{\operatorname{Hom}}

\newcommand{\leftquotes}{``}
\newcommand{\rightquotes}{''}

\newcommand{\mapsfrom}{\mathrel{\reflectbox{\ensuremath{\mapsto}}}}
\newcommand{\longmapsfrom}{\mathrel{\reflectbox{\ensuremath{\longmapsto}}}}
\newcommand{\groupaction}{\mathrel{\text{\scalebox{1.5}{\rotatebox[origin=c]{90}{$\circlearrowright$}}}}}

\newcommand{\rank}{\operatorname{rank}}
\newcommand{\im}{\operatorname{im}}
\newcommand{\identity}{\operatorname{id}}
\newcommand{\trdeg}{\operatorname{trdeg}}
\newcommand{\codim}{\operatorname{codim}}
\newcommand{\depth}{\operatorname{depth}}
\newcommand{\height}{\operatorname{height}}
\newcommand{\projdim}{\dim_{\operatorname{proj}}}
\newcommand{\characteristic}{\operatorname{Char}}

\newcommand{\derivation}{\operatorname{Der}}
\newcommand{\ext}{\operatorname{Ext}}
\newcommand{\mor}{\operatorname{Mor}}
\newcommand{\aut}{\operatorname{Aut}}
\newcommand{\homend}{\operatorname{End}}

\newcommand{\sheafhom}{\mathcal{H}\kern -.5pt om}
\newcommand{\sheaftor}{\mathcal{T}\kern -.5pt or}
\newcommand{\sheafext}{\mathcal{E}\kern -.5pt xt}
\newcommand{\sheafend}{\mathcal{E}\kern -.5pt nd}

\newcommand{\coker}{\operatorname{coker}}
\newcommand{\coim}{\operatorname{coim}}
\newcommand{\cone}{\operatorname{cone}}
\newcommand{\exceptionallocus}{\operatorname{Ex}}
\newcommand{\totalcomplex}{\operatorname{Tot}}

\newcommand{\spec}{\operatorname{Spec}}
\newcommand{\proj}{\operatorname{Proj}}
\newcommand{\spf}{\operatorname{Spf}}
\newcommand{\fraction}{\operatorname{Frac}}
\newcommand{\singular}{\operatorname{Sing}}
\newcommand{\support}{\operatorname{Supp}}

\newcommand{\reduction}{\operatorname{red}}
\newcommand{\normalization}{\operatorname{nor}}
\newcommand{\regular}{\operatorname{reg}}
\newcommand{\smooth}{\operatorname{sm}}
\newcommand{\analytification}{\operatorname{an}}

\newcommand{\inverselimit}{\underset{\longleftarrow}{\lim}} 
\newcommand{\colimit}{\underset{\longrightarrow}{\lim}}
\newcommand{\derivedotimes}{\overset{\llcorner}{\otimes}}

\newcommand{\grassmannianscheme}[2]{\operatorname{Gr}(#1, #2)}
\newcommand{\grassmanmianfunctor}[2]{\mathcal{G}\kern -.5pt r(#1, #2)}
\newcommand{\hilbertscheme}{\operatorname{Hilb}}
\newcommand{\hilbertfunctor}{\mathcal{H}\kern -.5pt ilb}
\newcommand{\picardscheme}{\operatorname{Pic}}
\newcommand{\picardfunctor}{\mathcal{P}\kern -.5pt ic}
\newcommand{\isomorphismscheme}{\operatorname{Isom}}
\newcommand{\isomorphismfunctor}{\mathcal{I}\kern -.5pt som}

\newcommand{\abeliangroups}{\mathbf{Ab}}
\newcommand{\rings}{\mathbf{Ring}}
\newcommand{\commutativerings}{\mathbf{CRing}}

\newcommand{\module}{\operatorname{Mod}}
\newcommand{\algebra}{\operatorname{Alg}}
\newcommand{\scheme}{\operatorname{Sch}}

\newcommand{\matrices}[2]{\operatorname{Mat}_{#1}(#2)}
\newcommand{\generallineargroup}[2]{\operatorname{GL}_{#1}(#2)}
\newcommand{\speciallineargroup}[2]{\operatorname{SL}_{#1}(#2)}
\newcommand{\projectivespeciallineargroup}[2]{\operatorname{PSL}_{#1}(#2)}

\newcommand{\order}{\operatorname{ord}}
\newcommand{\divisor}{\operatorname{div}}
\newcommand{\annihilator}{\operatorname{ann}}

\title{Algebraic Geometry}
\author{}
\date{}

\begin{document}
\maketitle
\newpage

\tableofcontents
\newpage

\section{Schemes}
\label{section:Schemes}

In the whole note, all rings mentioned are commutative rings with identity.
Also, some trivial proofs are omitted.
\subsection{Spectrum}
\begin{definition}
    Let $A$ be a ring.
    The spectrum of $A$ is the set of all prime ideals of $A$, denoted by $\spec A$.
\end{definition}
\begin{definition}
    Let $A$ be a ring, $I \subseteq A$ ideal.
    Define set $V(I) = \{\mathfrak{p}\in \spec A \big{|} I \subseteq \mathfrak{p}\}$ and $D(I) = \spec A \setminus V(I)$.
    In particular, for each $f\in A$, define $V(f) = V((f))$ and $D(f) = D((f))$.
\end{definition}
\begin{remark}
    Obviously, if $i \subseteq I$, then $V(I) \supseteq V(J)$.
\end{remark}
\begin{proposition}[\textbf{\emph{Zariski Topology}}]
    Let $A$ be a ring.
    Then the above definitions satisfy that
    \begin{itemize}
        \item $\{V(I) \big{|} I \subseteq A \text{ ideal}\}$ form the closed subsets of a topology on $\spec A$, called the Zariski topology.
        \item $\{D(f) \big{|} f\in A\}$ form a topological basis, called standard open subsets.
    \end{itemize}
    \label{Proposition 1.1}
\end{proposition}
\begin{remark}
    Should think $V(f)$ means $f = 0$ while $D(f)$ means $f \neq 0$.
\end{remark}
\begin{definition}
    Let $A$ be a ring, $I \subseteq A$ ideal.
    Define the radical of $I$ to be the set $\{x\in A \big{|} \exists n\in \mathbb{N}_{+}, x^n\in I\}$, denoted by $rad(I)$.
\end{definition}
\begin{lemma}
    Let $A$ be a ring, $I \subseteq A$ ideal.
    Then $V(I) = V(rad(I))$.
    \label{Lemma 1.1}
\end{lemma}
\begin{reason}
    Immediately comes from result of commutative algebra that $rad(I) = \cap_{\mathfrak{p} \supseteq I} \mathfrak{p}$.
\end{reason}
\begin{proposition}
    Let $A$ be a ring, $I,J \subseteq A$ ideals.
    Then $V(I) \subseteq V(J)$ if and only if $rad(I) \supseteq rad(J)$.
    \label{Proposition 1.2}
\end{proposition}
\begin{proposition}
    Let $A, B$ be rings, $\varphi: A \longrightarrow B$ ring homomorphism.
    Then induced map $\varphi^{\ast}: \spec(B) \longrightarrow \spec A\quad \mathfrak{P} \longmapsto \varphi^{-1}(\mathfrak{P})$ is continuous.
    In particular, the preimage of standard open subset $D(f)$ is just $D(\varphi(f))$ and so is $V(f)$.
    \label{Proposition 1.3}
\end{proposition}
\begin{example}
    (1)Let $A$ be a ring, $I \subseteq A$ ideal.
    Assume $\varphi: A \longrightarrow A/I$ is the natural homomorphism.
    Then image of $\varphi^{\ast}$ is $V(I) \subseteq \spec A$.
    Get $\spec(A/I) \cong V(I)$ \\
    (2)Let $A$ be a ring, $f \neq 0\in A$.
    Assume $A_{f}$ is the localization and $\varphi: A \longrightarrow A_{f}\quad a \longmapsto \frac{a}{1}$.
    Then image of $\varphi^{\ast}$ is $D(f) \subseteq \spec A$.
    Get $\spec(A_{f}) \cong D(f)$.
\end{example}
\begin{definition}
    Let $X$ be a topological space.
    We say that $X$ is quasi-compact if any open covering admits a finite subcovering.
\end{definition}
\begin{remark}
    In the topology, we would call this property to be compact not quasi-compact.
    While, in the history, French mathematicians ask compact topological space also to be Hausdorff.
    To differ this definition, we say quasi-compact instead of compact.
\end{remark}
\begin{proposition}
    Let $A$ be a ring.
    Then $\spec A$ is quasi-compact as a topological space.
    \label{Proposition 1.4}
\end{proposition}
\begin{reason}
    The finite subcovering comes from the fact that any element in an ideal can be expressed as a finite sum.
\end{reason}
Since $D(f) \cong \spec(A_{f})$ as topological spaces, we immediately get the following corollary.
\begin{corollary}
    Let A be a ring, $f\in A$.
    Then $D(f)$ is quasi-compact as a topological space.
    \label{Corollary 1.1}
\end{corollary}
\subsection{Sheaves}
\begin{definition}
    Let X be a topological space and $\mathfrak{C}$ be a category.
    Assume $op(X)$ is the category of open subsets of X with morphisms are inclusions.
    A presheaf on X valued on $\mathcal{C}$ (or say of $\mathfrak{C}$) is a contravariant functor $\mathcal{F}: op(X) \longrightarrow \mathfrak{C}$.
\end{definition}
\begin{remark}
    Though we can consider this general definition of presheaf, we would only consider presheaves at least valued on $Ab$,
    where $Ab$ is the category of abelian groups.
    In addition, we need these presheaves map $\varnothing$ to $\{0\}$.
\end{remark}
\begin{definition}
    Let X be a topological space, $\mathcal{F}$ presheaf of abelian groups on X.
    For $V \subseteq V$, we call the correspondence homomorphism $\mathcal{F}(U) \longrightarrow \mathcal{F}(V)$ a restriction map, denoted by $\cdot \rho_{UV}$.
    Elements in $\mathcal{F}(U)$ is called a section of $\mathcal{F}$ on U.
    In particular, if $U = X$, we call elements of $\mathcal{F}(X)$ global sections, otherwise local sections.
\end{definition}
\begin{remark}
    For convenience, for $s\in \mathcal{F}(U)$, we would write $s\big{|}_V$ to denote image of s instead of $\rho_{UV}(s)$.
    Also, sometimes $\mathcal{F}(U)$ can be rewritten as $\Gamma(U, \mathcal{F})$.
\end{remark}
\begin{definition}
    Let X be a topological space, $\mathcal{F}$ presheaf of abelian groups on X.
    $\mathcal{F}$ is a sheaf on X if it satisfies that \\
    (1)For each open subset U with open covering $\{U_{i}\}_{i\in I}$ and $s\in \mathcal{F}(U)$, 
    if $s\big{|}_{U_{i}} = 0$ for all i, then $s = 0$. \\
    (2)For each open subset U with open covering $\{U_{i}\}_{i\in I}$,
    if there exists $s_{i}\in \mathcal{F}(U_{i})$ for all i, satisfying $s_{i}\big{|}_{U_{i} \cap U_{j}} = s_{j}\big{|}_{U_{i} \cap U_{j}}$ for all i,j,
    then there exists $s\in \mathcal{F}(U)$ such that $s\big{|}_{U_{i}} = s_{i}$ for all i.
\end{definition}
\begin{remark}
    The two conditions in fact give a exact sequence for each open subset U with open covering $\{U_{i}\}_{i\in I}$.
    \begin{equation*}
        0 \longrightarrow \mathcal{F}(U) \longrightarrow \prod_{i\in I} \mathcal{F}(U_{i}) \longrightarrow \prod_{i,j\in I} \mathcal{F}(U_{i} \cap U_{j})
    \end{equation*}
\end{remark}
\begin{example}
    (1)Let X be a topological space.
    Then $U \longmapsto \{\text{continuous functions $U \longrightarrow \mathbb{R}$}\}$ give a sheaf on X. \\
    (2)Let $X = \mathbb{R}^{n}$.
    Then $U \longmapsto \{\text{$C^{\infty}$-functions $U \longrightarrow \mathbb{R}$}\}$ gives a sheaf on X. \\
    (3)Let X be a topological space and A be an abelian group.
    Suppose A is topological space with discrete topology.
    Then $U \longmapsto \{\text{locally constant functions $U \longrightarrow A$}\}$ gives a sheaf on X, which is called the constant sheaf of A on X.
\end{example}
\begin{definition}
    Let X be a topological space, $\mathcal{F}$ (pre)sheaf of abelian groups on X.
    For $x\in X$, define the stalk of $\mathcal{F}$ at x to be $\mathcal{F}_{x} = \underset{\underset{U\ni x}{\longrightarrow}}{\lim} \mathcal{F}(U)$.
    More concretely, an element of $\mathcal{F}_{x}$ is represented by a pair $(U, s) = s_{x}$ where U is an open neighbourhood of x and $s\in \mathcal{F}(U)$, called a germ.
    Two germs $(U, s)$ and $(V, t)$ represent the same element if there exists open neighbourhood W of x such that $W \subseteq U \cap V$ and $s\big{|}_{W} = t\big{|}_{W}$.
\end{definition}
\begin{proposition}
    Let X be a topological space, $\mathcal{F}$ sheaf of abelian groups on X.
    Suppose U is an open subset of X and $s,t\in \mathcal{F}(U)$.
    Then $s = t$ if and only if $s_{x} = t_{x}$ for all $x\in U$.
    \label{Proposition 1.5}
\end{proposition}
\begin{proof}
    "$\Rightarrow$": Obviously.
    \par
    "$\Leftarrow$": Since for all $x\in U$, we have $s_{x} = t_{x}$,
    there exists open neighbourhood $W_{x}$ of x such that $W_{x} \subseteq U$ and $s\big{|}_{W_{x}} = t\big{|}_{W_{x}}$.
    Note that $\{W_{x}\}_{x\in U}$ form an open covering of U.
    By property of sheaf, we get $s = t$.
\end{proof}
\begin{definition}
    Let X be a topological space, $\mathcal{F}, \mathcal{G}$ (pre)sheaves of abelian groups on X.
    A morphism $\varphi$ from $\mathcal{F}$ to $\mathcal{G}$ is a family of group homomorphisms $\varphi(U): \mathcal{F}(U) \longrightarrow \mathcal{G}(U)$ such that for $V \subseteq U$,
    there is a commutative diagram
    \begin{equation*}
        \begin{tikzcd}
            \mathcal{F}(U) \arrow[r, "\varphi(U)"] \arrow[d, "\rho_{UV}"] &
            \mathcal{G}(U) \arrow[d, "\rho_{UV}"] \\
            \mathcal{F}(V) \arrow[r, "\varphi(V)"] &
            \mathcal{G}(V)
        \end{tikzcd}
    \end{equation*}
\end{definition}
\begin{remark}
    By property of direct limit, morphism $\varphi$ gives homomorphism $\varphi_{x}: \mathcal{F}_{x} \longrightarrow \mathcal{G}_{x}$ for all $x\in X$.
\end{remark}
\begin{definition}
    Let X be a topological space, $\varphi: \mathcal{F} \longrightarrow \mathcal{G}$ morphism of (pre)sheaves of abelian groups on X.
    We say $\varphi$ is isomorphic if there exists morphism $\psi: \mathcal{G} \longrightarrow \mathcal{F}$ such that for each open subsets U, 
    $\psi(U)\varphi(U) = \identity_{\mathcal{F}(U)}$ and $\varphi(U)\psi(U) = \identity_{\mathcal{G}(U)}$.
\end{definition}
\begin{proposition}
    Let X be a topological space, $\varphi: \mathcal{F} \longrightarrow \mathcal{G}$ morphism of sheaves of abelian groups on X.
    Then $\varphi$ is isomorphic if and only if $\varphi_{x}$ is isomorphic for all $x\in X$.
    \label{Proposition 1.6}
\end{proposition}
\begin{proof}
    "$\Rightarrow$": For all $(U, t)\in \mathcal{F}_{x}$, since $\varphi(V)$ is isomorphic,
    there exists $s\in \mathcal{F}(U)$ such that $\varphi(V)(s) = t$.
    Thus $\varphi_{x}((U, s)) = (U, t)$, get $\varphi_{x}$ is surjective.
    For all $(U, s)\in \ker(\varphi_{x})$, $(U, \varphi(U)(s)) = 0$ in $\mathcal{G}_{x}$.
    Then there exists open neighbourhood V of x such that $V \subseteq U$ and $\varphi(U)(s)\big{|}_{V} = 0$.
    By commutative diagram, $\varphi(U)(s)\big{|}_{V} = \varphi(V)(s\big{|}_{V})$.
    Since $\varphi(V)$ is isomorphic, get $s\big{|}_{V} = 0$ and so $(U, s) = 0$ in $\mathcal{F}_{x}$.
    Get $\varphi_{x}$ is isomorphic.
    \par
    "$\Leftarrow$": For each open subset U, want to show that $\varphi(U)$ is isomorphic.
    For $s\in \ker{\varphi(U)}$, $\varphi(U)(s) = 0$.
    Then for all $x\in U$, $(U, \varphi(U)(s)) = 0$ in $\mathcal{G}_{x}$.
    As $(U, \varphi(U)(s)) = \varphi_{x}((U, s))$ and $\varphi_{x}$ is isomorphic, get $(U, s) = 0$ in $\mathcal{F}_{x}$.
    Thus there exists open neighbourhood $W_{x}$ of x such that $W_{x} \subseteq U$ and $s\big{|}_{W_{x}} = 0$.
    By property of sheaf, get $s = 0$ and so $\varphi(U)$ is injective.
    \par
    For all $t\in \mathcal{G}(U)$, $(U, t)$ is an element of $\mathcal{G}_{x}$ for all $x\in U$.
    Then there exists $(V_{x}, s(x))\in \mathcal{F}_{x}$ such that $\varphi_{x}((V_{x}, s(x))) = (U, t)$ and $\varphi(V_{x})(s(x)) = t\big{|}_{V_{x}}$.
    Note that $\varphi(V_{x} \cap V_{y})(s(x)\big{|}_{V_{x} \cap V_{y}}) = (t\big{|}_{V_{x}})\big{|}_{V_{x} \cap V_{y}} = (t\big{|}_{V_{y}})\big{|}_{V_{x} \cap V_{y}} = \varphi(V_{x} \cap V_{y})(s(y)\big{|}_{V_{x} \cap V_{y}})$.
    Since $\varphi(V_{x} \cap V_{y})$ is injective, get $s(x)\big{|}_{V_{x} \cap V_{y}} = s(y)\big{|}_{V_{x} \cap V_{y}}$.
    By property of sheaf, there exists $s\in \mathcal{F}(U)$ such that $s\big{|}_{V_{x}} = s(x)$.
    So $\varphi(U)(s)\big{|}_{V_{x}} = \varphi(V_{x})(s(x)) = t\big{|}_{V_{x}}$.
    By property of sheaf, get $\varphi(U)(s) = t$ and so $\varphi(U)$ is isomorphic.
\end{proof}
\begin{definition}
    Let X be a topological space, $\varphi: \mathcal{F} \longrightarrow \mathcal{G}$ morphism of presheaves of abelian groups on X. \\
    (1)Define the kernel of $\varphi$ to be $\ker(\varphi): op(X) \longrightarrow Ab\quad U \longmapsto \ker(\varphi(U))$. \\
    (2)Define the image of $\varphi$ to be $im(\varphi): op(X) \longrightarrow Ab\quad U \longmapsto im(\varphi(U))$. \\
    (3)Define the cokernel of $\varphi$ to be $coker(\varphi): op(X) \longrightarrow Ab\quad U \longmapsto coker(\varphi(U))$.
\end{definition}
\begin{remark}
    The definition of cokernel in fact induce the definition of quotient presheaf.
\end{remark}
\begin{definition}[\textbf{\emph{Sheafification}}]
    Let X be a topological space, $\mathcal{F}$ presheaf of abelian groups on X.
    Define a sheaf $\mathcal{F}^{+}$ of abelian groups on X by $U \longmapsto \mathcal{F}^{+}(U)$, 
    where $\mathcal{F}^{+}(U) = \{(s_{x})_{x\in U} \big{|} \text{ for all $x\in U$, $\exists$ \text{open} $V_{x}\ni x$ and $t(x)\in \mathcal{F}(V_{x})$ such that $t(x)_{y} = s_{y}$, $\forall y\in V_{x}$}\}$ is a subset of $\prod_{x\in U} \mathcal{F}_{x}$.
\end{definition}
\begin{proposition}
    Let X be a topological space, $\mathcal{F}$ presheaf of abelian groups on X.
    Show that there is a natural morphism of presheaves $\varphi: \mathcal{F} \longrightarrow \mathcal{F}^{+}$ satisfies that \\
    (1)$\varphi_{x}$ is identity homomorphism. \\
    (2)If $\mathcal{F}$ is sheaf, then $\varphi$ is isomorphism. \\
    (3)For any morphism of presheaves $\psi: \mathcal{F} \longrightarrow \mathcal{G}$ where $\mathcal{G}$ is a sheaf,
    there exists unique morphism $\phi: \mathcal{F}^{+} \longrightarrow \mathcal{G}$ such that the following diagram commutes.
    \begin{equation*}
        \begin{tikzcd}
            \mathcal{F} \arrow[rr, "\psi"] \arrow[dr, "\varphi"] &&
            \mathcal{G} \\
            & \mathcal{F}^{+} \arrow[ur, "\phi"] &
        \end{tikzcd}
    \end{equation*}
    \label{Proposition 1.7}
\end{proposition}
\begin{reason}
    The construction of $\varphi$ is just $\varphi(U): s \longmapsto (s_{x})_{x\in U}$.
\end{reason}
\begin{definition}
    Let X be a topological space, $\varphi: \mathcal{F} \longrightarrow \mathcal{G}$ morphism of sheaves of abelian groups on X. \\
    (1)Define the kernel of $\varphi$ to be $\ker(\varphi): op(X) \longrightarrow Ab\quad U \longmapsto \ker(\varphi(U))$. \\
    (2)Define the image of $\varphi$ to be sheafification of $op(X) \longrightarrow Ab\quad U \longmapsto im(\varphi(U))$, denoted by $im(\varphi)$. \\
    (3)Define the cokernel of $\varphi$ to be sheafification of $op(X) \longrightarrow Ab\quad U \longmapsto coker(\varphi(U))$, denoted by $coker(\varphi)$.
\end{definition}
\begin{remark}
    The definition of cokernel in fact induce the definition of quotient sheaf.
\end{remark}
\begin{definition}
    Let X be a topological space, $\varphi: \mathcal{F} \longrightarrow \mathcal{G}$ morphism of sheaves of abelian groups on X. \\
    (1)We say that $\varphi$ is injective if $\ker(\varphi) = 0$. \\
    (2)We say that $\varphi$ is surjective if $im(\varphi) = \mathcal{G}$
\end{definition}
\begin{proposition}
    Let X be a topological space, $\varphi: \mathcal{F} \longrightarrow \mathcal{G}$ morphism of sheaves of abelian groups on X.
    Then \\
    (1)$\varphi$ is injective if and only if $\varphi(U)$ is injective for all open subset U. \\
    (2)$\varphi$ is injective if and only if $\varphi_{x}$ is injective for all $x\in X$. \\
    (3)$\varphi$ is surjective if and only if $\varphi_{x}$ is surjective for all $x\in X$. \\
    (4)$\varphi$ is isomorphic if and only if $\varphi$ is injective and surjective.
    \label{Proposition 1.8}
\end{proposition}
\begin{remark}
    The proof is similar to proof of Proposition \ref{Proposition 1.6}.
    In addition, Hartshorne exercise II 1.2 gives a more explicit sketch.
\end{remark}
\begin{definition}[\textbf{\emph{Direct Image and Inverse Image}}]
    Let $f:X \longrightarrow Y$ be continuous map of topological spaces, $\mathcal{F}$ sheaf of abelian groups on X, $\mathcal{G}$ sheaf of abelian groups on Y. \\
    (1)Define dierct image to be the sheaf $f_{\ast}\mathcal{F}: V \longmapsto f_{\ast}\mathcal{F}(V) = \mathcal{F}(f^{-1}(V))$. \\
    (2)Define inverse image to be the sheafification of $U \longmapsto \underset{\underset{V \supseteq f(U)}{\longrightarrow}}{\lim} \mathcal{G}(V)$, denoted by $f^{-1}\mathcal{G}$.
\end{definition}
\begin{remark}
    In general, we don't have $(f_{\ast}\mathcal{F})_{f(x)} \neq \mathcal{F}_{x}$,
    while there is a canonical homomorphism $(f_{\ast}\mathcal{F})_{f(x)} \longrightarrow \mathcal{F}_{x}$.
    On the other hand, we always have $(f^{-1}\mathcal{G})_{x} = \mathcal{G}_{f(X)}$.
\end{remark}
\begin{example}
    Let $i: Z \hookrightarrow X$ be an inclusion of topological spaces, $\mathcal{F}$ sheaf of abelian groups on X.
    We often write $i^{-1}\mathcal{F}$ as $\mathcal{F}\big{|}_{Z}$.
\end{example}
\subsection{Varieties}
\begin{example}[\textbf{\emph{Affine Varieties}}]
    Let k be an algebraically closed field, $V \subseteq k^{n}$ irreducible closed subset.
    Suppose $\mathfrak{p}$ is the corresponding prime ideal of $\mathfrak{p}$ in $\spec(k[x_{1}, \cdots, x_{n}])$.
    Define affine coordinate ring of V to be $k[V] = k[x_{1}, \cdots, x_{n}]/\mathfrak{p}$.
    Then $k[V]$ is a finitely generated k-algebra and an integral domain.
    Define sheaf $\mathcal{O}_{V}$ of abelian groups on V to be $U \longmapsto \{\text{regular functions $U \longrightarrow k$}\}$,
    where the definition of regular function can be seen in section 3 of Chapter I.
    We call such a pair $(V, \mathcal{O}_{V})$ affine variety.
\end{example}
\begin{proposition}
    Let k be an algebraically closed field, $V \subseteq k^{n}$ irreducible closed subset.
    Then \\
    (1)$\mathcal{O}_{V}(V) = k[V]$. \\
    (2)For all $f\in k[V]$, $\mathcal{O}_{V}(D(f)) = k[V]_{f}$. \\
    (3)For all $x\in V$, $\mathcal{O}_{V, x} = k[V][f^{-1} \big{|} f(x) \neq 0]$ is a local ring with maximal ideal $\mathfrak{m}_{x}$. \\
    (4)For all $x\in V$, $\mathcal{O}_{V, x}/\mathfrak{m}_{x} = k$.
    \label{Proposition 1.9}
\end{proposition}
\begin{example}[\textbf{\emph{Gluing Affine Varieties}}]
    Let k be an algebraically closed field.
    A abstract prevariety over k is a pair $(X, \mathcal{O}_{X})$, where X is an irreducible topological space and $\mathcal{O}_{X}$ is a sheaf of abelian groups on X,
    satisfying that there exists finite open covering $\{U_{i}\}$ of X such that $(U_{i}, \mathcal{O}\big{|}_{U_{i}}) \cong (V_{i}, \mathcal{O}_{V_{i}})$ where $(V_{i}, \mathcal{O}_{V_{i}})$ is an affine variety.
    The isomorphism is a pair $(f_{i}, f_{i}^{\sharp})$, where $f_{i}: U_{I} \longrightarrow V_{i}$ is a homeomorphism of topological spaces and $f_{i}^{\sharp}: \mathcal{O}_{V_{i}} \longrightarrow f_{\ast}\mathcal{O}_{X}\big{|}_{U_{i}}$ is an isomorphism of sheaves. 
\end{example}
\begin{example}[\textbf{\emph{Affine Line with Two Origins}}]
    Let k be an algebraically closed field.
    Assume $U_{1}, U_{2}$ are two affine line $\mathbb{A}_{k}^{1}$.
    Glue them on $U_{1} \setminus \{O_{1}\}$ and $U_{2} \setminus \{O_{2}\}$ and get an abstract variety X.
    We call X affine line with two origins.
    Obviously, $O_{1}$ and $O_{2}$ have no disjoint open neighbourhoods.
    Thus we say X is some kind of not separated, which is the case we want to avoid.
\end{example}
\begin{example}[\textbf{\emph{Product of Prevarieties}}]
    Let k be an algebraically closed field, X abstract prevariety over k.
    Suppose that $\{U_{i}\}_{i\in I}$ are affine open subsets of X.
    Consider $X \times X$ as a prevariety, glued by $\{U_{i} \times U_{j}\}_{i,j\in I}$,
    where $U_{i} \times U_{j}$ is still isomorphic to affine variety refer to Hartshorne exercise I 3.15.
\end{example}
\begin{definition}
    Let k be an algebraically closed field, X abstract prevariety over k.
    X is a abstract variety if $\delta = \{(x, x) \big{|} x\in X\}$ is closed in $X \times X$. 
\end{definition}
\subsection{Structure Sheaf}
From now on, we would consider sheaves of rings.
The above propositions about sheaves of abelian groups are still right for sheaves of rings.
Here we don't need to repeat the proofs and just acknowledge them.
Similar to the theory of abstract, we will define something basic like affine variety first.
\begin{definition}
    Let A be a ring, $X = \spec A$ spectrum.
    Define the structure sheaf $\mathcal{O}_{X}$ of rings on X to be $U \longrightarrow \mathcal{O}_{X}(U)$, 
    where $\mathcal{O}_{X}(U) = \{(a_{\mathfrak{p}})_{\mathfrak{p}\in U} \big{|} \forall \mathfrak{p}\in U, \exists f(\mathfrak{p})\in (A \setminus \mathfrak{p}) \text{ and } b(\mathfrak{p})\in A_{f(\mathfrak{p})} \text{ such that $D(f(\mathfrak{p})) \subseteq U$ and $b(\mathfrak{p})_{\mathfrak{q}} = a_{\mathfrak{q}}$, $\forall \mathfrak{q}\in D(f(\mathfrak{p}))$}\}$ is a subset of $\prod_{\mathfrak{p}\in U} A_{\mathfrak{p}}$.
\end{definition}
\begin{theorem}
    Let A be a ring, $\mathcal{O}_{Spec(A)}$ structure sheaf.
    Then for all $f\in A$ and $g\in rad(f)$, we have an isomorphism $A_{f} \overset{\sim}{\longrightarrow} \mathcal{O}_{X}(D(f))$ and the following diagram commutes
    \begin{equation*}
        \begin{tikzcd}
            A_{f} \arrow[r] \arrow[d] &
            \mathcal{O}_{X}(D(f)) \arrow[d] \\
            A_{g} \arrow[r] &
            \mathcal{O}_{X}(D(g))
        \end{tikzcd}
    \end{equation*}
    \label{Theorem 1.1}
\end{theorem}
\begin{reason}
    Construct the isomorphism as $\frac{a}{f^{n}} \longmapsto (\frac{a}{f^{n}})_{\mathfrak{p}\in D(f)}$.
    The hardest part is to show this map is surjective, which need some commutative algebra to prove.
    For all $(a_{\mathfrak{p}})_{\mathfrak{p}\in D(f)}$, $\{D(f(\mathfrak{p}))\}$ form an open covering of $D(f)$.
    By Corollary \ref{Corollary 1.1}, $D(f)$ is quasi-compact and so there exists finitely many $\mathfrak{p}_{i}\in D(f)$ such that $D(f(\mathfrak{p}_{i}))$ cover $D(f)$.
    \par
    For convenience, rewrite $f(\mathfrak{p}_{i})$ as $f_{i}$ and assume that $b(\mathfrak{p}_{i}) = \frac{a_{i}}{f_{i}^{n_{i}}}$.
    Then $\frac{a_{i}}{f_{i}^{n_{i}}}$ and $\frac{a_{j}}{f_{j}^{n_{j}}}$ coincide on $A_{\mathfrak{q}}$ for all $i,j$ and $\mathfrak{q}\in D(f_{i}f_{j})$.
    Consider $\Delta_{ij} = a_{i}f_{j}^{n_{j}} - a_{j}f_{i}^{n_{i}}$ and annihilator $ann_{A}(\Delta_{ij})$.
    If $f_{i}f_{j}\notin rad(ann_{A}(\Delta_{ij}))$, take maximal ideal $\mathfrak{m}_{ij} \supseteq ann_{A}(\Delta_{ij})$.
    Then $\mathfrak{m}_{ij}\in D(f_{i}f_{j})$ and so $\frac{a_{i}}{f_{i}^{n_{i}}} = \frac{a_{j}}{f_{j}^{n_{j}}}$ in $A_{\mathfrak{m}_{ij}}$.
    Thus there exists $s_{ij}\in (A \setminus \mathfrak{m}_{ij})$ such that $s\Delta_(ij) = 0$ and so $s\in ann_{A}(\Delta_{ij})$.
    But $ann_{A}(\Delta_{ij}) \subseteq \mathfrak{m}_{ij}$, get $s\in \mathfrak{m}_{ij}$, contradiction!
    \par
    Thus $f_{i}f_{j}\in rad(ann_{A}(\Delta_{ij}))$ and there exists $k_{ij}\in \mathbb{M}$ such that $(f_{i}f_{j})^{k_{ij}}\Delta_{ij} = 0$.
    Note that there are finitely many i, we can take a supremum k and so $(f_{i}f_{j})^{k}\Delta_{ij} = 0$ for all $i,j$.
    Let $a_{i}' = a_{i}f_{i}^{k}$ and $f_{i}' = f_{i}^{n_{i} + k}$.
    Can replace $\frac{a_{i}}{f_{i}^{n_{i}}}$ by $\frac{a_{i}'}{f_{i}'}$.
    Then $a_{i}'f_{j}' - a_{j}'f_{i}' = 0$.
    On the other hand, $D(f) \subseteq \cup_{i} D(f_{i}) = \cup_{i}D(f_{i}')$ gives that there exists $n\in \mathbb{N}$ such that $f^{n} = \sum_{i} b_{i}f_{i}'$, where $b_{i}\in A$.
    Consider $\frac{\sum_{i} b_{i}a_{i}'}{f^{n}}$.
    Then $f^{n}a_{j}' = \sum_{i} b_{i}f_{i}'a_{j}' = \sum_{i} b_{i}a_{i}'f_{j}' = (\sum_{i} b_{i}a_{i}')f_{j}'$ and so $\frac{\sum_{i} b_{i}a_{i}'}{f^{n}}$ and $\frac{a_{i}'}{f_{i}'}$ coincide on $A_{\mathfrak{q}}$ for all $\mathfrak{q}\in D(f_{i}') = D(f_{i})$.
    Get $\frac{\sum_{i} b_{i}a_{i}'}{f^{n}} \longmapsto (a_{\mathfrak{p}})_{\mathfrak{p}\in D(f)}$ and so homomorphism is surjective.
\end{reason}
\begin{corollary}
    Let A be a ring, $\mathcal{O}_{Spec(A)}$ structure sheaf.
    Then $\mathcal{O}_{Spec(A)}(Spec(A)) = A$.
    \label{Corollary 1.2}
\end{corollary}
\begin{remark}
    With structure sheaf, now we can distinguish $\spec k$ and $\spec(k')$ for different fields k and $k'$.
\end{remark}
\begin{proposition}
    Let A be a ring, $\mathcal{O}_{Spec(A)}$ structure sheaf.
    Then for all $\mathfrak{p}\in \spec A$, we have an isomorphism $A_{\mathfrak{p}} \overset{\sim}{\longrightarrow} \mathcal{O}_{Spec(A), \mathfrak{p}}$.
    \label{Proposition 1.10}
\end{proposition}
\begin{reason}
    By definition of stalk and property of direct limit, we immediately get the isomorphism.
\end{reason}
\subsection{Schemes}
\begin{definition}[\textbf{\emph{Ringed Spaces}}]
    Let X be a topological space, $\mathcal{O}_{X}$ sheaf of rings on X.
    Then pair $(X, \mathcal{O}_{X})$ is called a ringed space.
\end{definition}
\begin{definition}[\textbf{\emph{Locally Ringed Spaces}}]
    Let $(X, \mathcal{O}_{X})$ be a ringed space.
    We say $(X, \mathcal{O}_{X})$ is a locally ringed space if for all $x\in X$, 
    $\mathcal{O}_{X, x}$ is a local ring with maximal ideal $\mathfrak{m}_{x}$ and residue field $k(x) = \mathcal{O}_{X, x}/\mathfrak{m}_{x}$.
\end{definition}
\begin{remark}
    With these definitions we can evaluate sections at point of underlying topological space.
    For open subset $U \subseteq X$, $x\in U$ and $s\in \mathcal{O}_{X}(U)$, define $f(x)$ to be the image of $f_{x}$ in residue field $k(x)$.
\end{remark}
\begin{example}
    Let A be a ring.
    Then $(Spec(A), \mathcal{O}_{Spec(A)})$ is a locally ringed space.
    For all $\mathfrak{p}\in \spec A$, $k(\mathfrak{p}) = \fraction(A/\mathfrak{p})$.
    If moreover A is an integral domain and $\eta = (0)$ is the generic point, then $k(\eta) = \fraction(A)$.
\end{example}
\begin{definition}
    Let $(X, \mathcal{O}_{X})$ and $(Y, \mathcal{O}_{Y})$ be two ringed spaces.
    A morphism of ringed spaces from $(X, \mathcal{O}_{X})$ to $(Y, \mathcal{O}_{Y})$ is a pair $(f, f^{\sharp})$,
    where $f: X \longrightarrow Y$ is a continuous map and $f^{\sharp}: \mathcal{O}_{Y} \longrightarrow f_{\ast}\mathcal{O}_{X}$ is a morphism of sheaves of rings on Y.
\end{definition}
\begin{definition}
    Let $(X, \mathcal{O}_{X})$ and $(Y, \mathcal{O}_{Y})$ be two locally ringed spaces.
    A morphism of locally ringed spaces from $(X, \mathcal{O}_{X})$ to $(Y, \mathcal{O}_{Y})$ is a morphism of ringed spaces $(f, f^{\sharp})$ from $(X, \mathcal{O}_{X})$ to $(Y, \mathcal{O}_{Y})$ satisfying that for all $x\in X$ and $y = f(x)$,
    the first row of the following commutative diagram is a local homomorphism i.e. $(f_{x}^{\sharp})^{-1}(\mathfrak{m}_{x}) = \mathfrak{m}_{y}$
    \begin{equation*}
        \begin{tikzcd}
            \mathcal{O}_{Y, y} \arrow[rr, "f_{x}^{\sharp}"] \arrow[dr] &&
            \mathcal{O}_{X. x} \\
            & (f_{\ast}\mathcal{O}_{X})_{y} \arrow[ur] &
        \end{tikzcd}
    \end{equation*}
\end{definition}
\begin{remark}
    Since $f_{x}^{\sharp}$ is a local homomorphism, it induces a field extension $k(f(x)) \hookrightarrow k(x)$.
\end{remark}
\begin{definition}
    Let $(f, f^{\sharp}): (X, \mathcal{O}_{X}) \longrightarrow (Y, \mathcal{O}_{Y})$ be a morphism of (locally) ringed spaces.
    We say that $(f, f^{\sharp})$ is isomorphic if there exists morphism $(g, g^{\sharp}): (Y, \mathcal{O}_{Y}) \longrightarrow (X, \mathcal{O}_{X})$ such that \\
    (1)$gf = \identity_{X}$ and $fg = \identity_{Y}$ \\
    (2)$(gf)^{\sharp}$ is isomorphism of sheaves of rings on X and $(fg)^{\sharp}$ is isomorphism of sheaves of rings on Y.
\end{definition}
\begin{proposition}
    Let $(f, f^{\sharp}): (X, \mathcal{O}_{X}) \longrightarrow (Y, \mathcal{O}_{Y})$ be a morphism of (locally) ringed spaces.
    Then $(f, f^{\sharp})$ is isomorphic if and only if f is homeomorphism and $f_{x}^{\sharp}$ is isomorphic for all $x\in X$.
    \label{Proposition 1.11}
\end{proposition}
\begin{proof}
    "$\Rightarrow$": Obviously.
    \par
    "$\Leftarrow$": Since f is homeomorphism, assume the inverse of f is g.
    Want construct $(g, g^{\sharp})$ with $g^{\sharp}(U): \mathcal{O}_{X}(U) \longrightarrow \mathcal{O}_{Y}(f(U))$.
    We will prove that $f^{\sharp}(V)$ is isomorphic for all open subset V of Y and take $g^{\sharp}(U) = (f^{\sharp}(f(U)))^{-1}$.
    For $t\in \ker(f^{\sharp}(V))$, $f^{\sharp}(V)(t) = 0$.
    Then $f_{x}^{\sharp}((V, t)) = (f^{-1}(V), 0) = 0$ in $\mathcal{O}_{X, x}$ for all $x\in f^{-1}(V)$.
    As $f_{x}^{\sharp}$ is isomorphic, we get $(V, t) = 0$ in $\mathcal{O}_{Y, f(x)}$.
    Thus there exists open neighbourhood $V_{x}$ of $f(x)$ such that $t\big{|}_{V_{x}} = 0$.
    Note that f is homomorphism inducing that $\{V_{x}\}_{x\in f^{-1}(x)}$ is an open covering of V.
    By property of sheaf get $t = 0$ and so $f^{\sharp}(v)$ is injective. 
    \par
    For each open subset U of X and $s\in \mathcal{O}_{X}(U)$, consider $(U, s)\in \mathcal{O}_{X, x}$ for some $x\in U$.
    Since $f_{x}^{\sharp}$ is isomorphic for all $x\in X$, there exists open neighbourhood $V_{x}$ of $f(x)$ and $t(x)\in \mathcal{O}_{Y}(V_{x})$ such that $f_{x}^{\sharp}((V_{x}, t(x))) = (U, s)$.
    In fact, by choose appropriate $V_{x}$, we can have $f^{\sharp}(V_{x})(t(x)) = s\big{|}_{V_{x}}$.
    By injectivity of $f^{\sharp}(V_{x} \cap V_{x'})$, get $t(x)\big{|}_{V_{x} \cap V_{x'}} = t(x')\big{|}_{V_{x \cap V_{x'}}}$.
    Then by property of sheaf, there exists $t\in \mathcal{O}_{Y}(V)$ such that $t\big{|}_{V_{x}} = t(x)$.
    So $f^{\sharp}(t)\big{|}_{V_{x}} = f^{\sharp}(V_{x})(t(x)) = s\big{|}_{V_{x}}$.
    By property of sheaf, get $f^{\sharp}(t) = s$ and so $f^{\sharp}(f(U))$ is surjective.
    Thus $f^{\sharp}(V)$ is isomorphic for all open subset V of Y.
\end{proof}
\begin{definition}[\textbf{\emph{Affine Schemes}}]
    Let $(X, \mathcal{O}_{X})$ be a locally ringed space.
    We say that $(X, \mathcal{O}_{X})$ is an affine scheme if there exists a ring A such that $(X, \mathcal{O}_{X}) \cong (A, \mathcal{O}_{Spec(A)})$ as locally ringed spaces.
\end{definition}
\begin{remark}
    An affine variety over algebraically closed field k is an affine scheme isomorphic to $(Spec(k[V]), \mathcal{O}_{Spec(k[V])})$,
    where $k[V]$ is affine coordinate ring of some irreducible closed subset V of $k^{n}$.
\end{remark}
\begin{theorem}
    Let $\varphi: A \longrightarrow B$ be a ring homomorphism.
    Then we can associate it with a natural morphism $(f, f^{\sharp}): (Spec(B), \mathcal{O}_{Spec(B)}) \longrightarrow (Spec(A), \mathcal{O}_{Spec(A)})$ of locally ringed spaces.
    Moreover, there is a one-to-one correspondence 
    \begin{equation*}
        \hom(A, B) \longleftrightarrow Mor((Spec(B), \mathcal{O}_{Spec(B)}), (Spec(A), \mathcal{O}_{Spec(A)}))
    \end{equation*}
    \label{Theorem 1.2}
\end{theorem}
\begin{proof}
    For convenience, let $X = \spec(B)$ and $Y = \spec A$.
    Take f to be the induced map of $\varphi$.
    By Proposition \ref{Proposition 1.3}, we know that f is continuous and in particular, $f^{-1}(D(a)) = D(\varphi(a))$ for all $a\in A$.
    Note that $\varphi$ induce a homomorphism $A_{a} \longrightarrow B_{\varphi(a)}$, while $\mathcal{O}_{X}(D(\varphi(a))) = B_{\varphi(a)}$ and $\mathcal{O}_{Y}(D(a)) = A_{a}$.
    Thus we can define $f^{\sharp}(D(a)): \mathcal{O}_{Y}(D(a)) \longrightarrow \mathcal{O}_{Y}(D(\varphi(a))) = f_{\ast}\mathcal{O}_{X}(D(a))$.
    By gluing, get $f^{\sharp}: \mathcal{O}_{Y} \longrightarrow \mathcal{O}_{X}$.
    For all $\mathfrak{p}\in X$ and $\varphi^{-1}(\mathfrak{p})$, it is obvious that $f_{\mathfrak{p}^{\sharp}}$ is local.
    Thus $(f, f^{\sharp})$ is a morphism of locally ringed spaces.
    \par
    If $\varphi$ and $\varphi'$ give the same morphism $(f, f^{\sharp})$, consider $f^{\sharp}(Y)$.
    By above process, $f^{\sharp}(Y)$ is just $\varphi$ and so $\varphi = \varphi'$.
    For any morphism of locally ringed space $(f, f^{\sharp}): (X, \mathcal{O}_{X}) \longrightarrow (Y, \mathcal{O}_{Y})$, likewise consider the homomorphism of global sections $\varphi$.
    Firstly, want to show that f is the induced map of f.
    Note that for all $\mathfrak{p}\in X$, there is a commutative diagram 
    \begin{equation*}
        \begin{tikzcd}
            A \arrow[r, "\varphi"] \arrow[d] &
            B \arrow[d] \\
            A_{f(\mathfrak{p})} \arrow[r, "f_{\mathfrak{p}}^{\sharp}"] & 
            B_{\mathfrak{p}}
        \end{tikzcd}
    \end{equation*}
    Since $f_{\mathfrak{p}^{\sharp}}$ is local homomorphism and $f_{\mathfrak{p}}^{\sharp}(\frac{\varphi^{-1}(\mathfrak{p})}{1}) \subseteq \frac{\mathfrak{p}}{1} \subseteq \mathfrak{m}_{\mathfrak{p}}$,
    get $\varphi^{-1}(\mathfrak{p}) \subseteq f(\mathfrak{p})$.
    On the other hand, for all $a\in f(\mathfrak{p})$, we have $\frac{\varphi(a)}{1}\in \mathfrak{p}$ and so $f(\mathfrak{p}) = \varphi^{-1}(\mathfrak{p})$.
    \par 
    It suffices to show that $\varphi$ just induce $(f, f^{\sharp})$ and we only need to check about standard open sets.
    Note that for all $a\in A$, $\mathfrak{p}\in f^{-1}(D(a))$ if and only if $f(\mathfrak{p})\in D(a)$ if and only if $\varphi^{-1}(\mathfrak{p})\in D(a)$ if and only if $\mathfrak{p}\in D(\varphi(a))$,
    get $f^{-1}(D(a)) = D(\varphi(a))$.
    In addition, there is a commutative diagram
    \begin{equation*}
        \begin{tikzcd}
            A \arrow[r, "\varphi"] \arrow[d] &
            B \arrow[d] \\
            A_{a} \arrow[r, "f^{\sharp}(D(a))"] & 
            B_{\varphi(a)}
        \end{tikzcd}
    \end{equation*}
    For $\frac{a'}{a^{n}}\in A_{a}$, we have
    \begin{equation*}
        \begin{split}
            f^{\sharp}(D(a))(\frac{a'}{a^{n}}) & = f^{\sharp}(D(a))(\frac{a'}{1})f^{\sharp}(D(a))(\frac{1}{a^{n}}) \\
            & = \frac{\varphi(a')}{1}f^{\sharp}(D(a))(\frac{a}{1})^{-n} \\
            & = \frac{\varphi(a')}{1}(\frac{\varphi(a)}{1})^{-n} \\
            & = \frac{\varphi(a')}{\varphi(a)^{n}}
        \end{split}
    \end{equation*}
    Thus $(f, f^{\sharp})$ is induced by $\varphi$.
\end{proof}
\begin{definition}[\textbf{\emph{Schemes}}]
    Let $(X, \mathcal{O}_{X})$ be a locally ringed space.
    We say that $(X, \mathcal{O}_{X})$ is a scheme if there exists an open covering $\{U_{i}\}_{i\in I}$ of X with $(U_{i}, \mathcal{O}_{X}\big{|}_{U_{i}})$ affine scheme.
    A morphism of schemes is just a morphism of locally ringed spaces.
\end{definition}
\begin{remark}
    By property of standard open subsets, we can easily see that affine open subsets of a scheme $(X, \mathcal{O}_{X})$ form a topological basis.
\end{remark}
\begin{proposition}
    Let $(X, \mathcal{O}_{X})$ be an affine scheme.
    Assume that $X = \spec A$ and $U = \spec(B) \subseteq X$ is an affine open subset.
    Then for all $a\in A$, $U \cap D(a) = D(b)$ where b is the image of a under the restriction map.
    \label{Proposition 1.12}
\end{proposition}
\begin{proof}
    Note that for all $x\in U$, x corresponds to a prime ideal $\mathfrak{p}_{x}\in \spec A$ and a prime ideal $\mathfrak{P}_{x}\in \spec(B)$.
    Thus they have the same stalk $\mathcal{O}_{X, x}$, inducing the following commutative diagram
    \begin{equation*}
        \begin{tikzcd}
            A \arrow[r] \arrow[d] &
            B \arrow[d] \\
            A_{\mathfrak{p}_{x}} \arrow[r, "\sim"] &
            B_{\mathfrak{P}_{x}}
        \end{tikzcd}
    \end{equation*}
    Then $x\in D(b)$ if and only if $b\in (B_{\mathfrak{P}_{x}})^{\times}$ if and only if $a\in (A_{\mathfrak{p}_{x}})^{\times}$ if and only if $x\in D(a)$.
    Get $U \cap D(a) = D(b)$.
\end{proof}
\subsection{Projective Schemes}
\begin{definition}
    Let $S = \oplus_{d \ge 0} S_{d}$ be a graded ring.
    An ideal I of S is said to be homogeneous if it is generated by homogeneous elements,
    or equivalently $I = \oplus_{d \ge 0} (I \cap S_{d})$.
\end{definition}
\begin{definition}
    Let $S = \oplus_{d \ge 0} S_{d}$ be a graded ring.
    Define $\proj(S)$ to be the set of all homogeneous prime ideals of S not containing $S_{+}$,
    where $S_{+} = \oplus_{d > 0} S_{d}$.
\end{definition}
Our goal is to define scheme structure on $\proj(S)$.
So, similar to definition of projective variety in classic algebraic geometry, we will define Zariski topology on $\proj(S)$.
\begin{definition}
    Let $S = \oplus_{d \ge 0} S_{d}$ be a graded ring, $I \subseteq S$ homogeneous ideal.
    Define $V_{+}(I) = \{\mathfrak{p}\in \proj(S) \big{|} I \subseteq \mathfrak{p}\}$ and $D_{+}(I) = \proj(S) \setminus V_{+}(I)$.
    In particular, $V_{+}(S_{+}) = \varnothing$ and $V_{+}(0) = \proj(S)$.
\end{definition}
\begin{proposition}[\textbf{\emph{Zariski Topology}}]
    Let $S = \oplus_{d \ge 0} S_{d}$ be a graded ring.
    Then the above definitions satisfy: \\
    (1)$\{V_{+}(I) \big{|} I \subseteq S \text{ homogeneous ideal}\}$ form the closed subsets of a topology on $\proj(S)$ called the Zariski topology. \\
    (2)$\{D_{+}(f) \big{|} f\in S\}$ form a basis of the topology, called standard open subsets.
    \label{Proposition 1.13}
\end{proposition}
\begin{remark}
    The proof has nothing more than Proposition \ref{Proposition 1.1}, but it is necessary to check ideals are homogeneous.
    In addition, one can check that the Zariski topology on $\proj(S)$ is in fact the induced topology of Zariski topology on $\spec A$.
\end{remark}
\begin{lemma}
    Let S be a graded ring, $I,J \subseteq S$ homogeneous ideals.
    Then $V_{+}(I) = V_{+}(J)$ if and only if $J \cap S_{+} \subseteq rad(I)$.
    \label{Lemma 1.2}
\end{lemma}
To give a appropriate sheaf of rings on $\proj(S)$, we need the following lemma
\begin{lemma}
    Let $S = \oplus_{d \ge 0} S_{d}$ be a graded ring, $f\in S$ homogeneous element of degree $r > 0$.
    Then \\
    (1)$u_{f}: D_{+}(f) \longrightarrow \spec(S_{f})\quad \mathfrak{p} \longmapsto \mathfrak{p}S_{f} \cap S_{(f)}$ is bijective,
    where $S_{f} \subseteq S_{f}$ is the subset of all homogeneous elements of degree 0. \\
    (2)Suppose there exists homogeneous element $s\in S$ of degree $> 0$ with $D_{+}(g) \subseteq D_{+}(f)$.
    Then there exists ring homomorphism $S_{(f)} \longrightarrow S_{(g)}$.
    \label{Lemma 1.3}
\end{lemma}
\begin{proof}
    (1): For $\mathfrak{q}\in \spec(S_{(f)})$, we have that $\mathfrak{q}S_{f}$ is a homogeneous ideal of $S_{f}$ and so is $rad(\mathfrak{q}S_{f})$.
    Let $\mathfrak{p}\in D_{+}(f)$ be the preimage of $rad(\mathfrak{q}S_{f})$ in S (In fact, $\mathfrak{p} = rad(\mathfrak{q}S_{f}) \cap S$).
    Want to show that $rad(\mathfrak{q}S_{f})$ is prime.
    If $a,b\in S_{f}$ homogeneous with $ab\in rad(\mathfrak{q}S_{f})$, then $a^{m}b^{m}\in \mathfrak{q}S_{f}$ for some $m\in \mathbb{N}_{+}$.
    Get $(\frac{a^{r}}{f^{\deg(a)}})^{m}(\frac{b^{r}}{f^{\deg(b)}})^{m}\in \mathfrak{q}S_{f} \cap S_{(f)} = \mathfrak{q}$.
    As $\mathfrak{q}$ is prime, $a^{r}\in \mathfrak{q}S_{f}$ or $b^{r}\in \mathfrak{q}S_{f}$.
    Thus $a\in rad(\mathfrak{q}S_{f})$ or $b\in rad(\mathfrak{q}S_{f})$ and so $rad(\mathfrak{q}S_{f})$ is prime.
    Then $\mathfrak{p}$ is prime and $u_{f}(\mathfrak{p}) = \mathfrak{q}$.
    \par
    For $\mathfrak{p},\mathfrak{p}'\in D_{+}(f)$, we claim that $\mathfrak{p}' \subseteq \mathfrak{p}$ if and only if $u_{f}(\mathfrak{p}') \subseteq \mathfrak{p}$.
    From the above process, we can easily see that $\mathfrak{p} = rad(u_{f}(\mathfrak{p})S_{f}) \cap S$.
    By the same process, we can prove the claim.
    And the claim immediately gives the injectivity of $u_{f}$.
    \par
    (2): If $D_{+}(g) \subseteq D_{+}(f)$, then $g^{m} = bf$ for some $b\in S$ and $m\in \mathbb{N}_{+}$.
    Note that since f,g are both homogeneous, get b is also homogeneous.
    Then $\frac{a}{f^{n}} \longmapsto \frac{ab}{g^{mn}}$ gives the homomorphism.
\end{proof}
\begin{theorem}
    Let $S = \oplus_{d \ge 0} S_{d}$ be a graded ring, $f\in S$ homogeneous element of degree $r > 0$.
    Then \\
    (1)$u_{f}: D_{+}(f) \longrightarrow \spec(S_{f})$ as above is a homeomorphism. \\
    (2)If $f,g\in S$ are homogeneous elements of degree $> 0$ with $D_{+}(g) \subseteq D_{+}(f)$,
    then the following diagram commutes 
    \begin{equation*}
        \begin{tikzcd}
            D_{+}(g) \arrow[r, "u_{g}"] \arrow[d, hookrightarrow] &
            \spec(S_{(g)}) \arrow[d] \\
            D_{+}(f) \arrow[r, "u_{f}"] &
            \spec(S_{(f)})
        \end{tikzcd}
    \end{equation*}
    (3)Via the homeomorphism $u_{f}$, we equip $D_{+}(f)$ with a structure sheaf $\mathcal{O}_{D_{+}(f)}$.
    And $\{\mathcal{O}_{D_{+}(f)}\}$ glue to a structure sheaf $\mathcal{O}_{Proj(S)}$ on $\proj(S)$.
    In particular, $(Proj(S), \mathcal{O}_{Proj(S)})$ is a scheme. \\
    (4)For all $\mathfrak{p}\in \proj(S)$, $\mathcal{O}_{Proj(S), \mathfrak{p}} = S_{(\mathfrak{p})}$.
    \label{Theorem 1.3}
\end{theorem}
\begin{proof}
    (1): By Lemma \ref{Lemma 1.3}, we have already known that $u_{f}$ is bijective. 
    As the Zariski topology of $\proj(S)$ is the induced topology of $\spec(S)$.
    We can view $u_{f}$ as the restriction to $D_{+}(f)$ of $\spec(S_{f}) \longrightarrow \spec(S_{(f)})$.
    Thus $u_{f}$ is continuous.
    On the other hand, want to show that $u_{f}$ is closed.
    This comes from a similar claim:
    for $\mathfrak{p}\in D_{+}(f)$ and homogeneous ideal $I \subseteq \mathfrak{p}$, we have $I \subseteq \mathfrak{p}$ if and only if $(IS_{f} \cap S_{(f)}) \subseteq (\mathfrak{p}S_{f} \cap S_{(f)})$.
    The proof is similar to the proof of Lemma \ref{Lemma 1.3}.
    Thus with the claim, get $u_{f}$ is homogeneous. 
    \par
    (2): Obvious 
    \par
    (3): Since ${D_{+}(f)}_{f\in S}$ is a topological basis stable under finite intersection, 
    we only need to show their structure sheaves coincide on the intersection set, which comes from (2). 
    \par
    (4): Same argument as the affine case.
\end{proof}
Now with well defined affine and projective scheme, we can restate definition of affine and projective abstract variety.
\begin{definition}[\textbf{\emph{Affine Abstract Varieties}}]
    Let k be an algebraically closed field.
    An affine abstract variety over k is a scheme of the form $\spec(k[x_{0}, \cdots, x_{n}]/\mathfrak{p})$ for some $\mathfrak{p}\in \spec(k[x_{0}, \cdots, x_{n}])$.
\end{definition}
\begin{definition}[\textbf{\emph{Projective Abstract Varieties}}]
    Let k be an algebraically closed field.
    A projective abstract variety over k is a scheme of the form $\proj(k[x_{0}, \cdots, x_{n}]/\mathfrak{p})$ for some $\mathfrak{p}\in \proj(k[x_{0}, \cdots, x_{n}])$.
\end{definition}

\section{Morphisms of Schemes}
\label{section:Morphisms of Schemes}

Recall a morphism of schemes $(f, f^{\sharp}): (X, \mathcal{O}_{X}) \longrightarrow (Y, \mathcal{O}_{Y})$ is a morphism of locally ringed spaces.
From now on, for convenience, we would just write $f: X \longrightarrow Y$ for a morphism of schemes from $(X, \mathcal{O}_{X})$ to $(Y, \mathcal{O}_{Y})$.
\begin{lemma}
    Let $f: X \longrightarrow Y$ be a morphism of schemes.
    Suppose that $U \subseteq X$ is an open subset.
    Then f induce a morphism $f\big{|}_{U} : U \longrightarrow Y$, where U is scheme $(U, \mathcal{O}_{X}\big{|}_{U})$.
    \label{Lemma 2.1}
\end{lemma}
\begin{proof}
    Obviously, the continuous map of underlying topological spaces can be taken as $f: U \longrightarrow Y$ sending $x$ to $f(x)$.
    For each $V \subseteq Y$ open subset, define $f\big{|}_{U}^{\sharp}(V): \mathcal{O}_{Y}(V) \longrightarrow \mathcal{O}_{X}\big{|}_{U}(f\big{|}_{U}^{-1}(V)) = \mathcal{O}_{X}(f^{-1}(V) \cap U)$ by sending $t$ to $f^{\sharp}(V)(t)\big{|}_{f^{-1}(V) \cap U}$.
    It suffices to check naturality.
    Consider the following diagram for open subsets $V' \subseteq V \subseteq Y$,
    \begin{equation*}
        \begin{tikzcd}[row sep = large]
            \mathcal{O}_{Y}(V) \arrow[r, "(f\big{|}_{U})^{\sharp}(V)"] \arrow[d, "\rho_{VV'}"] &
            \mathcal{O}_{X}\big{|}_{U}(f\big{|}_{U}^{-1}(V)) \arrow[d, "\rho_{VV'}"] \\
            \mathcal{O}_{Y}(V') \arrow[r, "(f\big{|}_{U})^{\sharp}(V')"] &
            \mathcal{O}_{X}\big{|}_{U}(f\big{|}_{U}^{-1}(V'))
        \end{tikzcd}
    \end{equation*}
    Then for all $t\in \mathcal{O}_{Y}(V)$
    \begin{equation*}
        \begin{split}
            \rho_{VV'} \circ (f\big{|}_{U})^{\sharp}(V)(t) & = (f^{\sharp}(V)(t)\big{|}_{f^{-1}(V) \cap U})\big{|}_{f\big{|}_{U}^{-1}(V')} \\
            & = f^{\sharp}(V)(t)\big{|}_{f^{-1}(V') \cap U} \\
            & = (f^{\sharp}(V)(t)\big{|}_{f^{-1}(V')})\big{|}_{f^{-1}(V') \cap U} \\
            & = (f^{\sharp}(V')(t\big{|}_{V'}))\big{|}_{f^{-1}(V') \cap U} \\
            & = (f\big{|}_{U})^{\sharp}(V')(t\big{|}_{V'}) \\
            & = (f\big{|}_{U})^{\sharp}(V') \circ \rho_{VV'}(t)
        \end{split}
    \end{equation*}
    Get the diagram commutes and done!
\end{proof}
\begin{lemma}
    Let $f: X \longrightarrow Y$ be a morphism of schemes.
    Suppose that $V \subseteq Y$ is an open subset and $f(X) \subseteq V$.
    Then there exists unique morphism $g: X \longrightarrow V$ such that $f = i \circ g$, where $i: V \hookrightarrow Y$ is the inclusion.
    \label{Lemma 2.2}
\end{lemma}
\begin{reason}
    The part of construction is similar to proof of Lemma \ref{Lemma 2.1}.
    For uniqueness, if there are another $\widetilde{g}$ satisfies the conditions, firstly we can easily see that $g = \widetilde{g}$ in the underlying topological space.
    While each ring homomorphism in $g^{\sharp}$ and $\widetilde{g}^{\sharp}$ is a ring homomorphism in $f^{\sharp}$, get $g = \widetilde{g}$ as morphisms of schemes.
\end{reason}
\begin{lemma}[\textbf{\emph{Gluing Lemma}}]
    Let X,Y be two schemes.
    Assume that $\{U_{i}\}_{i\in I}$ is an open covering of X.
    Then giving a morphism $f: X \longrightarrow Y$ is equivalent to giving a family of morphisms ${f_{i}: U_{i} \longrightarrow Y}_{i\in I}$ such that $f_{i}\big{|}_{U_{i} \cap U_{j}} = f_{j}\big{|}_{U_{i} \cap U_{j}}$.
    \label{Lemma 2.3}
\end{lemma}
\begin{proof}
    "$\Rightarrow$": Obvious.
    \par
    "$\Leftarrow$": We need to construct a morphism $f: X \longrightarrow Y$ at first.
    It is obvious that we can get a continuous map f of underlying topological spaces be gluing $f_{i}$, satisfying that $f\big{|}_{U_{i}} = f_{i}$.
    For each $V \subseteq Y$ open subset and $t\in \mathcal{O}_{Y}(V)$, we have images ${f_{i}^{\sharp}(V)(t)}_{i\in I}$.
    Note that $f_{i}^{\sharp}(t)\in \mathcal{O}_{X}(f^{-1}(V) \cap U_{i})$, while $f^{-1}(V) \cap U_{i}$ form an open covering of $f^{-1}(V)$ and $f_{i}^{\sharp}(t)$ coincide on $f^{-1}(U) \cap U_{i} \cap U_{j}$ for all $i,j\in I$.
    By property of sheaf, there exists $s\in \mathcal{O}_{X}(f^{-1}(V))$ such that $s\big{|}_{f^{-1} \cap U_{i}} = f_{i}^{\sharp}(t)$.
    Let $f^{\sharp}(V)$ map t to s.
    By property of sheaf, it is easty to check naturality and that $f^{\sharp}(V)$ is a ring homomorphism.
    Get a morphism f of schemes from X to Y and $f\big{|}_{U_{i}} = f_{i}$.
\end{proof}
\begin{lemma}
    Let X be a scheme, $x\in X$.
    Then there exists a morphism of schemes $\spec(\mathcal{O}_{X, x}) \longrightarrow X$.
    \label{Lemma 2.4}
\end{lemma}
\begin{reason}
    For affine case, assume $X = \spec A$.
    For all $x\in X$, it corresponds to a prime ideal $\mathfrak{p}$ of A and there is a natural homomorphism $A \longrightarrow A_{\mathfrak{p}}$,
    where $A_{\mathfrak{p}}$ is just $\mathcal{O}_{X, x}$, inducing a morphism $\spec(\mathcal{O}_{X, x}) \longrightarrow X$.
    \par
    For general case, we can take an affine open neighbourhood U of x, which gives a morphism $\spec(\mathcal{\mathcal{O}_{X, x}}) = (\mathcal{O}_{X}\big{|}_{U})_{x} \longrightarrow U$.
    Composing with $i: U \hookrightarrow X$, get $\spec(\mathcal{O}_{X, x}) \longrightarrow X$.
    To show that the morphism is independent of choose of U.
    Consider two affine open neighbourhoods $V \subseteq U$ of x and there is a commutative diagram
    \begin{equation*}
        \begin{tikzcd}
            \mathcal{O}_{X}(U) \arrow[rr, "\rho_{UV}"] \arrow[dr] &&
            \mathcal{O}_{X}(V) \arrow[dl] \\
            & \mathcal{O}_{X,x} &
        \end{tikzcd}
    \end{equation*}
    which shows that $\spec(\mathcal{O}_{X, x}) \longrightarrow U$ is composition of $\spec(\mathcal{O}_{X, x}) \longrightarrow V$ and $V \hookrightarrow U$.
\end{reason}
\begin{remark}
    Since there is a natural morphism $\spec(k(x)) \longrightarrow \spec(\mathcal{O}_{X, x})$, we also get $\spec(k(x)) \longrightarrow X$.
\end{remark}
\subsection{S-schemes}
\begin{definition}
    Let S be a schemes.
    An S-scheme (or scheme over S) is a scheme X equipped with a morphism of schemes $X \longrightarrow S$.
    A morphism of S-schemes is a morphism of schemes $X \longrightarrow Y$ such that the following diagram commutes
    \begin{equation*}
        \begin{tikzcd}
            X \arrow[rr] \arrow[dr] &&
            Y \arrow[dl] \\
            & S &
        \end{tikzcd}
    \end{equation*}
    Define $Mor_{S}(X, Y)$ to be the set of all morphism of S-schemes from X to Y.
    In particular, if $S = \spec A$ is an affine scheme, we write A-scheme for short.
\end{definition}
\begin{definition}
    Let A be a ring, B an A-algebra, X an A-scheme.
    Define the set of B-points of X to be $X(B) = Mor_{Spec(A)}(Spec(B), X)$. 
\end{definition}
\begin{example}
    (1)Let k be a field, $l/K$ field extension, X a k-scheme.
    Then an l-point of X
    \begin{equation*}
        \begin{tikzcd}[sep = tiny]
            \spec(l) \arrow[rr] \arrow[dr] &&
            X \arrow[dl] \\
            & \spec k &
        \end{tikzcd}
        \longleftrightarrow
        \begin{tikzcd}[sep = tiny]
            l \arrow[rr, leftarrow] \arrow[dr, leftarrow] &&
            \mathcal{O}_{X, x} \arrow[dl, leftarrow] \\
            & k &
        \end{tikzcd}
        \text{ for some $x\in X$}
    \end{equation*}
    where the first row of the second diagram is a local homomorphism.
    In particular, $X(k) \longleftrightarrow \{x\in X \big{|} k(x) = k\}$. \\
    (2)Let $X = \mathbb{A}_{\mathbb{R}}^{1} = \spec(\mathbb{R}[t])$.
    There are 2 $\mathbb{C}$-points of X where image is $(t^{2} + 1)\in X$ with residue field isomorphic to $\mathbb{C}$.
    The fact comes from another fact that the following commutative diagram has two choices
    \begin{equation*}
        \begin{tikzcd}
            \mathbb{C} \arrow[rr, leftarrow] \arrow[dr, leftarrow] &&
            \mathbb{R}[t]/(t^{2} + 1) \arrow[dl, leftarrow] \\
            & \mathbb{R} &
        \end{tikzcd}
    \end{equation*}
    (3)Let k be a field, X a k-scheme.
    Assume that $k[\varepsilon] = k[t]/(t^{2})$ and $\spec(k[\varepsilon]) = \{(\overline{t})\}$.
    Then a $k[\varepsilon]$-point of X
    \begin{equation*}
        \begin{tikzcd}[sep = tiny]
            \spec k[\varepsilon] \arrow[rr] \arrow[dr] &&
            X \arrow[dl] \\
            & \spec k &
        \end{tikzcd}
        \longleftrightarrow
        \begin{tikzcd}[sep = tiny]
            k[\varepsilon] \arrow[rr, leftarrow] \arrow[dr, leftarrow] &&
            \mathcal{O}_{X, x} \arrow[dl, leftarrow] \\
            & k &
        \end{tikzcd}
        \text{ for some $x\in X$}
    \end{equation*}
    where the first row of the second diagram is a local homomorphism.
    Moreover, here $k(x) = k$.
\end{example}
\begin{definition}[\textbf{\emph{Tangent Spaces}}]
    Let X be a scheme.
    Assume that $x\in X$ with $k(x) = k$ for some field k.
    Define the tangent space of X at x to be $T_{x}(X) = \hom_{k}(\mathfrak{m}_{x}/\mathfrak{m}_{x}^{2}, k)$, 
    which is the dual space of $\mathfrak{m}_{x}/\mathfrak{m}_{x}^{2}$ as a k-vector space.
\end{definition}
\begin{remark}
    With this language of tangent space, we can conclude example 2.1 (3) that $X(k[\varepsilon]) = \{(x, v) \big{|} x\in X \text{ such that } k(x) = k, v\in T_{x}(X)\}$.
\end{remark}
\subsection{Open Immersions and Closed Immersions}
\begin{definition}
    Let X be a scheme.
    An open subscheme of X is an open subset $U \subseteq X$ equipped with sheaf $\mathcal{O}_{X}\big{|}_{U}$.
\end{definition}
\begin{definition}[\textbf{\emph{Open Immersions}}]
    Let X,Y be two schemes.
    An open immersion from X to Y is a morphism of schemes $f: X \longrightarrow Y$ inducing an isomorphism between X and some open subset of Y.
\end{definition}
\begin{definition}[\textbf{\emph{Closed Immersions}}]
    Let X,Y be two schemes.
    A closed immersion from X to Y is a morphism of schemes $f: X \longrightarrow Y$ such that \\
    (1)f on underlying topological spaces is a homeomorphism between X and some closed subset of Y. \\
    (2)$f^{\sharp}$ is surjective.
\end{definition}
\begin{remark}
    A closed subscheme of Y is a scheme X with a closed immersion of $f: X \longrightarrow Y$.
    And we would identify two closed subschemes $f: X \longrightarrow Y$ and $f': X' \longrightarrow Y$ if there is an isomorphism $g: X \longrightarrow X'$ such that $f = f' \circ g$.
\end{remark}
\begin{lemma}
    Let $f: X \longrightarrow Y$ be a morphism of schemes.
    Suppose that f on underlying spaces is a homeomorphism between X and some closed subset $V \subseteq Y$.
    Then f is closed immersion if and only if $f_{x}^{\sharp}$ is surjective for all $x\in X$.
    \label{Lemma 2.5}
\end{lemma}
\begin{proof}
    "$\Rightarrow$": For all $x\in X$ and $(U, s)\in \mathcal{O}_{X, x}$, we have $f(U)$ is open in V.
    Thus there exists $W\subseteq Y$ open such that $W \cap V = f(U)$ and so $f^{-1}(W) = U$.
    Consider $(W, s)\in (f_{\ast}\mathcal{O}_{X})_{f(x)}$.
    Since f is closed immersion, get there exists open neighbourhood $W' \subseteq W$ of $f(x)$ and $t\in \mathcal{O}_{Y}(W')$ such that $f^{\sharp}(W')(t) = s\big{|}_{f^{-1}(W')}$.
    Note that $x\in f^{-1}(W')$ and $(U, s) = (f^{-1}(W'), s\big{|}_{f^{-1}(W')}) = (f^{-1}(W'), f^{\sharp}(W')(t)) = f_{x}^{\sharp}((W', t))$ in $\mathcal{O}_{X, x}$.
    Get $f_{x}^{\sharp}$ is surjective.
    \par
    "$\Leftarrow$": For all $y\in Y$, want to show that $f_{y}^{\sharp}: \mathcal{O}_{Y, y} \longrightarrow (f_{\ast}\mathcal{O}_{X})_{y}$ is surjective.
    Note that if $y\notin V$, then $(f_{\ast}\mathcal{O}_{X})_{y} = 0$, we only need to prove for $y\in V$.
    For all $(W, s)\in (f_{\ast}\mathcal{O}_{X})_{y}$ where $s\in \mathcal{O}_{X}(f^{-1}(W))$, consider $(f^{W}, s)\in \mathcal{O}_{X, x}$ for all $x\in f^{-1}(W)$.
    As $f_{x}^{\sharp}$ is surjective, there exists open neighbourhood $W_{x}$ of $f(x)$ and $t(x)\in \mathcal{O}_{Y}(W_{x})$ such that $f^{\sharp}(W_{x})(t(x)) = s\big{|}_{f^{-1}(W_{x})}$.  
    Since $y\in V$, there exists $x\in f^{-1}(W)$ such that $y = f(x)$.
    Now take $(W_{x}, t(x))\in \mathcal{O}_{Y, y}$ and get $f_{y}^{\sharp}((W_{x}, t(x))) = (W_{x}, s\big{|}_{f^{-1}(W_{x})}) = (W, s)$ in $(f_{\ast}\mathcal{O}_{X})_{y}$.
    Thus $f^{\sharp}$ is surjective.                                                                                                                                                                                                                                                                                                                                                                                                                                                                                                                                                                                                                                                                                                                                                                                                                                                                                                                                                                                                                                                                                                                                                               
\end{proof}
\begin{definition}
    Let X be a scheme.
    For $f\in \mathcal{O}_(X)(X)$, define $X_{f}$ to be the set of all $x\in X$ such that $f(x)\neq 0$.
\end{definition}
\begin{lemma}
    Let X be a scheme.
    Then for all $f\in \mathcal{O}_{X}(X)$ and $x\in X$, we have \\
    (1)$X_{f}$ is open in X. \\
    (2)Suppose that X can be covered by finitely many affine open subsets $\{U_{i}\}_{i\in I}$ satisfying that each $U_{i} \cap U_{j}$ is also covered by finitely many affine open subsets.
    Then $\rho_{XX_{f}}: \mathcal{O}_{X}(X) \longrightarrow \mathcal{O}_{X}(X_{f})$ induces $\mathcal{O}_{X}(X)_{f} \overset{\sim}{\longrightarrow} \mathcal{O}_{X}(X_{f})$. \\
    (3)Let $f: X \longrightarrow Y$ be a morphism of schemes. 
    Then for all $g\in \mathcal{O}_{Y}(Y)$, $f^{-1}(Y_{g}) = X_{h}$, where h is the image of g in $\mathcal{O}_{X}(X)$.
    \label{Lemma 2.6}
\end{lemma}
\begin{proof}
    (1): For all $x\in X_{f}$, since $f(x)\neq 0$, we have $f_{x}\in \mathcal{O}_{X, x}^{\times}$.
    Assume that $(U, s)$ is the inverse of $f_{x}$.
    Then $(U, \rho_{XU}(f)s) = 1$ in $\mathcal{O}_{X, x}$.
    Thus there exists an open neighbourhood $U' \subseteq U$ of x such that $(\rho_{XU}(f)s)\big{|}_{U'} = 1$ and so $\rho_{XU'}(f)\in \mathcal{O}_{X}(U')^{\times}$.
    Get $U' \subseteq X_{f}$ and so $X_{f}$ is open. 
    \par
    (2): For all $x\in X$, as process in (1), there exists open neighbourhood $U_{x} \subseteq X_{f}$ of x and $s(x)\in \mathcal{O}_{X}(U_{x})$ such that $f\big{|}_{U_{x}}s(x) = 1$.
    Note that ${U_{x}}_{x\in X_{f}}$ form an open covering of $X_{f}$ and $s(x)\big{|}_{U_{x} \cap U_{y}} = (f\big{|}_{U_{x} \cap U_{y}})^{-1} = s(y)\big{|}_{U_{x} \cap U_{y}}$.
    By property of sheaf, there exists $g\in \mathcal{O}_{X}(X_{f})$ such that $g\big{|}_{U_{x}} = s(x)$.
    Define $\varphi: \frac{s}{f^{n}} \longmapsto s\big{|}_{X_{f}}g^{n}$.
    It is easy to check that $\varphi$ is a well defined ring homeomorphism.
    \par
    Want to show that $\varphi$ is isomorphic.
    As X is covered by ${U_{i}}_{i\in I}$, $X_{f}$ is covered by ${V_{i}}_{i\in I}$, where $V_{i} = X_{f} \cap U_{i} = D(f\big{|}_{U_{i}})$.
    By property of sheaf and flatness of localization, there is a commutative diagram with exact rows
    \begin{equation*}
        \begin{tikzcd}
            0 \arrow[r] &
            \mathcal{O}_{X}(X)_{f} \arrow[r] \arrow[d, "\varphi"] &
            \prod_{i\in I} \mathcal{O}_{X}(U_{i})_{f} \arrow[r] \arrow[d, "\sim"] &
            \prod_{i,j\in I} \mathcal{O}_{X}(U_{i} \cap U_{j})_{f} \arrow[d, "\psi"] \\
            0 \arrow[r] &
            \mathcal{O}_{X}(X_{f}) \arrow[r] &
            \prod_{i\in I} \mathcal{O}_{X}(V_{i}) \arrow[r] &
            \prod_{i,j\in I} \mathcal{O}_{X}(V_{i} \cap V_{j})
        \end{tikzcd}    
    \end{equation*}
    By commutativity of diagram, it is easy to show that $\varphi$ is injective.
    While for $\psi$, we can view $U_{1} \cap U_{j}$ as another scheme with finite affine open covering, inducing that $\psi$ is injective.
    By 5 Lemma in category theory, we get $\varphi$ is surjective and thus isomorphic.
    \par
    (3)Consider the following commutative diagram for all $x\in X$,
    \begin{equation*}
        \begin{tikzcd}
            \mathcal{O}_{Y}(Y) \arrow[r, "f^{\sharp}(Y)"] \arrow[d] &
            \mathcal{O}_{X}(X) \arrow[d] \\
            \mathcal{O}_{Y, f(x)} \arrow[r, "f_{x}^{\sharp}"] &
            \mathcal{O}_{X, x}
        \end{tikzcd}
    \end{equation*}
    As the last row is a local homomorphism, we get the result.
\end{proof}
\begin{proposition}[\textbf{\emph{Criterion of Affineness}}]
    Let X be a scheme.
    Suppose that there are finitely many elements $f_{1}, \cdots, f_{r}\in \mathcal{O}_{X}$ such that $(f_{1}, \cdots, f_{r}) = (1) = \mathcal{O}_{X}(X)$ and each open set $X_{f_{i}}$ is affine.
    Then X is affine.
    \label{Proposition 2.1} 
\end{proposition}
\begin{proof}
    Set $A = \mathcal{O}_{X}(X)$ and $A_(i) = \mathcal{O}_{X}(X_{f_{i}})$.
    Since $(f_{1}, \cdots, f_{n}) = \mathcal{O}_{X}(X)$, $\{X_{f_{i}}\}_{i\in I}$ cover X.
    By Lemma \ref{Lemma 2.6}, there is an isomorphism $A_{f_{i}} \overset{\sim}{\longrightarrow} A_{i}$.
    Thus there are morphisms $u_{i}: X_{f_{i}} = \spec(A_{f_{i}}) \longrightarrow \spec A$.
    Check if $u_{i}\big{|}_{X_{f_{i}} \cap X_{f_{j}}} = u_{j}\big{|}_{X_{f_{i}} \cap X_{f_{j}}}$.
    Note that $X_{f_{i}} \cap X_{f_{j}} = D(f_{i}\big{|}_{X_{f_{j}}})$ in $X_{f_{j}}$ and $X_{f_{i}} \cap X_{f_{j}} = D(f_{j}\big{|}_{X_{f_{i}}})$ in $X_{f_{i}}$.
    Get $u_{i}\big{|}_{X_{f_{i}} \cap X_{f_{j}}} = (Spec(A[\frac{1}{f_{i}}, \frac{1}{f_{j}}]) \longrightarrow \spec A) = u_{j}\big{|}_{X_{f_{i}} \cap X_{f_{j}}}$.
    By Gluing Lemma, there exists morphism $u: X \longrightarrow \spec A$.
    It is easy to see that for all $x\in X$, $u_{x}$ is isomorphic and so is u. 
\end{proof}
\begin{lemma}
    Let $\varphi: A \longrightarrow B$ be a ring homomorphism.
    If $\varphi$ is injective, then the associated morphism $f: \spec(B) \longrightarrow \spec A$ has dense image.
    Moreover, $f^{\sharp}: \mathcal{O}_{Spec(A)} \longrightarrow f_{\ast}\mathcal{O}_{Spec(B)}$ is also injective.
    \label{Lemma 2.7}
\end{lemma}
\begin{proof}
    Note for all $a\in A$, $f(D(a)) = D(\varphi(a))$.
    To show f has dense image, it suffices to show $D(\varphi(a)) \neq \varnothing$ if $D(a) \neq \varnothing$, which immediately comes from $\varphi$ is injective.
    As $f^[\sharp]$ is injective if and only if $f^{\sharp}(D(a))$ is injective for all standard open subset $D(a) \subseteq \spec A$.
    While $f^{\sharp}(D(a)): A_{a} \longrightarrow B_{\varphi(a)}$ is obviously injective, get $f^{\sharp}$ is injective.  
\end{proof}
\begin{theorem}                                                                                                                                                                                                                                                                             
    Let $X = \spec A$ be an affine scheme, $i: Z \hookrightarrow X$ closed immersion.
    Then $Z \cong \spec(A/I)$ for some ideal $I \subseteq A$.
    \label{Theorem 2.1}
\end{theorem}
\begin{proof}
    First show that Z is affine.
    Cover Z by affine open subsets ${V_{j}}_{j\in J}$.
    Since i is a closed immersion, there exists ${U_{j}}_{j\in J}$ such that $i^{-1}(U_{j}) = V_{j}$.
    Take standard open covering $\{U_{jk} = D(f_{jk})\}$ of $U_{j}$, then Z is covered by $i^{-1}(U_{jk})$.
    Note that there is a commutative diagram for all $x\in Z$,
    \begin{equation*}
        \begin{tikzcd}
            A \arrow[r, "i^{\sharp}(X)"] \arrow[d] &
            \mathcal{O}_{Z}(Z) \arrow[d] \\
            \mathcal{O}_{X, i(x)} \arrow[r, "i_{x}^{\sharp}"] &
            \mathcal{O}_{Z, x}
        \end{tikzcd}
    \end{equation*}
    Assume $h_{jk} = i^{\sharp}(X)(f_{jk})$.
    As $i_{x}^{\sharp}$ is a local homomorphism and surjective, $x\in Z_{h_{jk}}$ if and only if $h_{jk}(x)\in \mathcal{O}_{Z, x}^{\times}$ if and only if $f_{jk}(i(x))\in \mathcal{O}_{X, i(x)}^{\times}$ if and only if $i(x)\in D(f_{jk})$.
    Get $i^{-1}(U_{jk}) = Z_{h_{jk}}$.
    While $i^{-1}(U_{jk}) \subseteq i^{-1}(U_{j})$ and $i^{-1}(U_{j}) = V_{j}$ is affine, get $i^{-1}(U_{jk}) = Z_{h_{jk}} \cap V_{j}$ is affine.
    Since $X = i(Z) \cup (X \setminus i(Z)) = (\cup_{j, k} U_{jk}) \cup (\cup_{r} D(f_{r}))$ and X is quasi-compact, 
    there exists finitely many j,k such that $U_{l}$ cover X and $i^{-1}(D(f_{l}))$ is affine open subset of Z or empty set.
    Thus we get a finite affine open covering $\{Z_{h_{l}}\}$ of Z.
    As $(h_{l} \big{|} l) \supseteq i^{\sharp}(X)((f_{l} \big{|} l))$ and $1\in (f_{l} \big{|} l)$, get $(h_{l} \big{|} l) = \mathcal{O}_{Z}(Z)$.
    By Proposition \ref{Proposition 2.1}, Z is affine.
    \par
    Now we can assume $Z = \spec(B)$.
    Then $i: Z \longrightarrow X$ associates with a ring homomorphisms $\varphi: A \longrightarrow B$.
    It suffices to prove that $\varphi$ is surjective.
    Consider injective homomorphism $\widetilde{\varphi}: A/\ker(\varphi) \hookrightarrow B$ and take $X' = \spec(A/\ker(\varphi))$.
    Get a commutative diagram
    \begin{equation*}
        \begin{tikzcd}
            Z \arrow[rr, "i"] \arrow[dr, "g"] &&
            X \\
            & X' \arrow[ur, "i'"] &
        \end{tikzcd}
    \end{equation*}
    Want to show g is isomorphic.
    As other two morphisms are both injective, i is closed immersion and $i'$ is closed, get g is injective and closed.
    By Lemma \ref{Lemma 2.7}, get g has dense image and $g^{\sharp}$ is injective.
    Thus g on underlying topological spaces is homeomorphism.
    While for all $x\in Z$, $i_{x}^{\sharp}$ and ${i'}_{g(x)}^{\sharp}$ are both surjective, get $g_{x}^{\sharp}$ is surjective.
    By Proposition \ref{Proposition 1.11}, g is isomorphic.
\end{proof}
\subsection{Finite Type and Finite Morphisms}
\begin{definition}[\textbf{\emph{Local on the Base}}]
    Let P be a property of morphisms of schemes.
    We say that P is local on the base if for any morphism $f: X \longrightarrow Y$ and affine open covering $\{V_{i}\}_{i\in I}$ of Y,
    the following conditions are equivalent \\
    (1)f satisfies P \\
    (2)each induced morphism $f_{i}: f^{-1}(V_{i}) \longrightarrow V_{i}$ satisfies P.
\end{definition}
\begin{lemma}
    Let P be a property of morphisms of schemes local on the base.
    Then for any morphism $f: X \longrightarrow Y$, the following conditions are equivalent \\
    (1)f satisfies P \\
    (2)For any affine open covering $\{V_{i}\}_{i\in I}$, induced morphisms $f_{i}$ satisfies P \\
    (3)There exists an affine open covering $\{V_{i}\}_{i\in I}$ such that induced morphisms $f_{i}$ satisfies P
    \label{Lemma 2.9}
\end{lemma}
\begin{remark}
    Open, closed immersion are local on the base.
\end{remark}
\begin{definition}[\textbf{\emph{Quasi-compact}}]
    Let $f: X \longrightarrow Y$ be a morphism of schemes.
    We say that f is quasi-compact if for all affine open subset $V\in Y$, the preimage of V in X is quasi-compact.
\end{definition}
\begin{proposition}
    Let $f: X \longrightarrow Y$ be a morphism of schemes.
    Suppose Y admits an affine open covering $\{V_{i}\}_{i\in I}$ such that each $f^{-1}(V_{i})$ is quasi-compact.
    Then f is quasi-compact.
    \label{proposition 2.2}
\end{proposition}
\begin{proof}
    Let $V \subseteq Y$ be an affine open subset.
    Take standard open covering $\{V_{ik}\}$ of $V_{i} \cap V$, where $V_{ik} = D(f_{ik})$ in $V_{i}$ for some $f_{ik}\in \mathcal{O}_{Y}(V_{i})$.
    Since all such ${V_{ik}}$ cover V and V is quasi-compact, there is finite open subcovering $\{V_{l}\}$.
    As $f^{-1}(V) = \cup_{l} f^{-1}(V_{l})$, it suffices to show that $f^{-1}(V_{l})$ is quasi-compact.
    While $f^{-1}(V_{i})$ is quasi-compact, there is finite affine open covering $\{U_{ij}\}$.
    Since $V_{l} \subseteq V_{i}$ for some $i\in I$, $f^{-1}(V_{l})$ is covered by $U_{ijl} = U_{ij} \cap f^{-1}(V_{l})$.
    Note that $f^{-1}(V_{l}) = X_{h_{l}}$ by Lemma \ref{Lemma 2.6}, $U_{ijl}$ is affine.
    Thus $f^{-1}(V_{l})$ is quasi-compact and so is $f^{-1}(V)$.
    Get f is quasi-compact.
\end{proof}
\begin{definition}[\textbf{\emph{Locally of Finite Type}}]
    Let $f: X \longrightarrow Y$ be a morphism of schemes.
    We say that f is locally of finite type if for all affine open subset $V = \spec A \subseteq Y$, 
    there exists affine open covering $\{U_{i} = \spec(B_{i})\}$ of $f^{-1}(V)$ such that $B_{i}$ are all A-algebra of finite type.
    Or equivalently, for all affine open subset $U = \spec(B) \subseteq f^{-1}(V)$, B is an A-algebra of finite type.
\end{definition}
\begin{lemma}
    Let A be a ring, B an A-algebra.
    Then \\
    (1)Suppose that $\spec(B)$ covered by $\{D(b_{i})\}_{i\in I}$, where $b_{i}\in B$.,
    satisfying that each $B_{b_{i}}$ is an A-algebra of finite type.
    Then B is also an A-algebra of finite type. \\
    (2)If $\spec(B) \longrightarrow \spec A$ is an open immersion, then B is an A-algebra of finite type.
    \label{Lemma 2.10}
\end{lemma}
\begin{proof}
    (1): Since $\spec(B)$ is quasi-compact, we can assume that $\{D(b_{i})\}_{i\in I}$ is finite.
    As $B_{b_{i}}$ is A-algebra of finite type, assume that $B_{b_{i}} = A[\frac{a_{ij}}{b_{i}^{n_{ij}}}]$, where $a_{ij}\in B$.
    Consider the A-subalgebra C of B generated by $\{b_{i}\}$ and $\{a_{ij}\}$.
    Then $B_{b_{i}} \subseteq C_{b_{i}} \subseteq B_{b_{i}}$, get $C_{b_{i}} = B_{b_{i}}$.
    Since $D(b_{i})$ cover B, we have $\sum_{i\in I} b_{i}'b_{i} = 1_{B}$ where $b_{i}'\in B$.
    Consider the A-subalgebra D of B generated by C and $\{b_{i}'\}$.
    Obviously, D is of finite type over A.
    While for all $b\in B$, since $\frac{b}{1}\in B_{b_{i}}$, there exists $k_{i}\in \mathbb{N}$ such that $b_{i}^{k_{i}}b = b_{i}^{k_{i}}c_{i}$ for some $c_{i}\in C$.
    By finiteness of i, we can take a supremum k of $k_{i}$.
    Thus $b = b(\sum_{i} b_{i}'b_{i})^{\sharp(I)k} = b\sum_{i} b_{i}^{k}d_{ik} = \sum_{i} b_{i}^{k}c_{i}d_{ik}\in D$.
    Get $B = D$ and so B is of finite type over A.
    \par
    (2): View $\spec(B)$ is an open subset of $\spec A$.
    Cover $\spec(B)$ by finite standard open subsets $\{D(a_{i}) = D(b_{i})\}$.
    Since $B_{b_{i}} = \mathcal{O}_{Spec(B)}(D(b_{i})) = \mathcal{O}_{Spec(A)}(D(a_{i})) = A_{a_{i}}$, 
    get $B_{b_{i}}$ is of finite type over $A_{a_{i}}$ and so is of finite type over A.
    BY (1), get B is of finite type over A.
\end{proof}
\begin{proposition}
    Let $f: X \longrightarrow Y$ be a morphism of schemes.
    Suppose Y admits an affine open covering $\{V_{i}\}_{i\in I}$ with each $f^{-1}(V_{i})$ covered by affine open subsets $\{U_{ij}\}$ such that $\mathcal{O}_{X}(U_{ij})$ is an $\mathcal{O}_{Y}(V_{i})$-algebra of finite type via $\rho_{f^{-1}(V_{i}U_{ij})} \circ f^{\sharp}(V_{i})$.
    Then f is locally of finite type.
    \label{Proposition 2.3}
\end{proposition}
\begin{proof}
    Let $V = \spec A \subseteq Y$ be an affine open subset.
    Similar to proof of Proposition \ref{proposition 2.2}, we can take standard open covering ${V_{ik}}$ of $V \cap V_{i}$, where $V_{ik} = D(g_{ik})$ in $V_{i}$.
    Each $f^{-1}(V_{i})$ can be covered by affine open subsets $\{U_{ij}\}$.
    Consider $U_{ijk} = U_{ij} \cap f^{-1}(V_{ik})$.
    As $f^{-1}(V_{ik}) = X_{h_{ik}}$, get $U_{ijk}$ is affine.
    Also since $V_{ik}$ is localization of $V_{i}$, get $\mathcal{O}_{X}(U_{ijk})$ is of finite type over $\mathcal{O}_{Y}(V_{ik})$.
    Note $V_{ik} \hookrightarrow V$ is an open immersion, by Lemma \ref{Lemma 2.10}, get $\mathcal{O}_{Y}(V_{ik})$ is of finite type over $\mathcal{O}_{Y}(V)$.
    Thus we can cover $f^{-1}(V)$ with affine open subsets $\{U_{ijk}\}$ such that $\mathcal{O}_{X}(U_{ijk})$ is of finite type over $\mathcal{O}_{Y}(V)$.
    \par
    Let $U \subseteq f^{-1}(V)$ be affine open subset.
    For all $x\in U$, there exists standard open subset $D(b(x)) \subseteq U$ of x such that $D(b_{i}) \subseteq U_{ijk}$ for some $U_{ijk}$.
    Thus $D(b(x))$ form an open covering of U.
    Note for each $D(b(x))$, $\mathcal{O}_{X}(D(b(x)))$ is localization of $\mathcal{O}_{X}(U_{ijk})$ and so is of finite type over $\mathcal{O}_{Y}(V)$.
    By Lemma \ref{Lemma 2.10}, get $\mathcal{O}_{X}(U)$ is of finite type over $\mathcal{O}_{Y}(V)$.
    Thus f is locally of finite type.
\end{proof}
\begin{definition}[\textbf{\emph{of Finite Type}}]
    Let $f: X \longrightarrow Y$ be a morphism of schemes.
    We say that f is of finite type if f is quasi-compact and locally of finite type.
\end{definition}
\begin{remark}
    With Proposition \ref{proposition 2.2} and Proposition \ref{Proposition 2.3}, of finite type is local on the base.
\end{remark}
\begin{example}
    (1)Let $f: \spec(B) \longrightarrow \spec A$ be a morphism of affine schemes.
    Then f is of finite type if and only if the associated ring homomorphisms $\varphi: A \longrightarrow B$ is of finite type. \\
    (2)Let k be an algebraically closed field k.
    Affine or projective abstract varieties over k is of finite type over $\spec k$. \\
    (3)Let k be an algebraically closed field k.
    k-morphisms between affine or projective abstract varieties over k is locally of finite type. \\
    (4)An open immersion is always locally of finite type but not quasi-compact in general.
    For example, we can take $Y = \spec(k[x_{1}, x_{2}, \cdots])$, $X = Y \setminus \{(0)\}$ and $i: X \hookrightarrow Y$,
    then $i^{-1}(Y) = X$ is not quasi-compact.
\end{example}
\begin{proposition}
    (1)The composition of two morphisms of finite type (resp. locally of finite type, quasi-compact) is still of finite type (resp. locally of finite type, quasi-compact). \\
    (2)Let $f: X \longmapsfrom Y$ and $g: Y \longrightarrow Z$ be two morphisms of schemes.
    If $g \circ f$ is locally of finite type, then f is locally of finite type.
    \label{Proposition 2.4}
\end{proposition}
\begin{proof}
    (1): Trivial. 
    \par
    (2): Take affine open covering $\{W_{i}\}$ of Z.
    Then $g^{-1}(W_{i})$ form an open covering of Y and for each $W_{i}$, 
    $f^{-1}g^{-1}(W_{i})$ has an affine open covering $\{U_{ij}\}$ such that $\mathcal{O}_{X}(U_{ij})$ is of finite type over $\mathcal{O}_{Z}(W_{j})$.
    For all affine open subset $V \subseteq Y$, can cover $V \cap g^{-1}(W_{i})$ with standard open subsets $\{V_{ik}\}$, where $V_{ik} = D(g_{ik})$ in $g^{-1}(W_{i})$.
    Consider $U_{ijk} = U_{ij} \cap f^{-1}(V_{ik})$ which is affine.
    Thus $f^{-1}(V)$ is covered by $U_{ijk}$.
    \par
    For all affine open subset $U \subseteq f^{-1}(V)$ and $x\in U$, can take standard open neighbourhood $D(b(x)) \subseteq U$ of x such that $D(b(x)) \subseteq U_{ijk}$ for some $U_{ijk}$.
    While $\mathcal{O}_{X}(D(b(x)))$ is localization of $\mathcal{O}_{X}(U_{ijk})$ and so is of finite type over $\mathcal{O}_{X}(U_{ijk})$, consider if $\mathcal{O}_{X}(U_{ijk})$ is of finite type over $\mathcal{O}_{Y}(V)$.
    Note that there is a commutative diagram
    \begin{equation*}
        \begin{tikzcd}
            \mathcal{O}_{X}(U_{ijk}) &&
            \mathcal{O}_{Z}(V_{i}) \arrow[dl] \arrow[ll] \\
            & \mathcal{O}_{Y}(V_{ik}) \arrow[ul] &
        \end{tikzcd}
    \end{equation*}
    Get $\mathcal{O}_{X}(U_{ijk})$ is of finite type over $\mathcal{O}_{Y}(V_{ik})$.
    In addition, $\mathcal{O}_{Y}(V_{ik})$ is of finite type over $\mathcal{O}_{Y}(V)$.
    Get $\mathcal{O}_{X}(D(b(x)))$ is of finite type over $\mathcal{O}_{Y}(V)$.
    By Lemma \ref{Lemma 2.10}, get $\mathcal{O}_{X}(U)$ is of finite type over $\mathcal{O}_{Y}(V)$.
\end{proof}
\begin{definition}[\textbf{\emph{Affine}}]
    Let $f: \spec(B) \longrightarrow \spec A$ be a morphism of schemes.
    We say that f is affine if for all affine open subset $V \subseteq Y$,
    $f^{-1}(V)$ is affine.
\end{definition}
\begin{definition}[\textbf{\emph{Finite}}]
    Let $f: \spec(B) \longrightarrow \spec A$ be a morphism of schemes.
    We say that f is finite if for all affine open subset $V \subseteq Y$,
    $f^{-1}(V)$ is affine and $\mathcal{O}_{X}(f^{-1}(V))$ is finite over $\mathcal{O}_{Y}(V)$.
\end{definition}
\begin{proposition}
    Composition of two finite (resp. affine) morphisms is still finite (resp. affine).
    \label{Proposition 2.5} 
\end{proposition}
\begin{lemma}
    Let A be a ring, B an A-algebra via ring homomorphism $\varphi: A \longrightarrow B$.
    Suppose that $\spec A$ is covered by $\{D(a_{i})\}_{i\in I}$ such that each $B_{\varphi(a_{i})}$ is finite over $A_{a_{i}}$.
    Then B is finite over A.
    \label{Lemma 2.11}
\end{lemma}
\begin{proof}
    Since $\spec A$ is quasi-compact, we can assume that $\{D(a_{i})\}_{i\in I}$ is finite.
    Let $b_{i} = \varphi(a_{i})$.
    As $B_{b_{i}}$ is finite $A_{a_{i}}$-algebra, assume that $B_{b_{i}} = \sum_{j} A_{a{i}}\frac{b_{ij}}{b_{i}^{n_{ij}}}$, where $b_{ij}\in B$.
    Consider the A-submodule C of B generated by $\{b_{ij}\}$.
    Then $B_{b_{i}} \subseteq C_{b_{i}} \subseteq B_{b_{i}}$, get $C_{b_{i}} = B_{b_{i}}$.
    Since $D(a_{i})$ cover B, we have $\sum_{i\in I} a_{i}'a_{i} = 1_{A}$ where $a_{i}'\in A$.
    While for all $b\in B$, since $\frac{b}{1}\in B_{b_{i}}$, there exists $k_{i}\in \mathbb{N}$ such that $b_{i}^{k_{i}}b = b_{i}^{k_{i}}c_{i}$ for some $c_{i}\in C$.
    By finiteness of i, we can take a supremum k of $k_{i}$.
    Thus 
    \begin{equation*}
       \begin{split}
            b & = (\sum_{i} a_{i}'a_{i})^{\sharp(I)k}b \\
            & = \sum_{i} a_{ik}a_{i}^{k}b \\ 
            & = \sum_{i} a_{ik}b_{i}^{k}b \\
            & = \sum_{i} a_{ik}b_{i}^{k}c_{i} \\
            & = \sum_{i} a_{ik}a^{k}c_{i}\in C 
       \end{split}
    \end{equation*}
    Get $B = C$ and so B is finite over A.
\end{proof}
\begin{proposition}
    Let $f: \spec(B) \longrightarrow \spec A$ be a morphism of schemes.
    Suppose that Y admits an affine open covering $\{V_{i}\}$ such that each $U_{i} = f^{-1}(V_{i})$ is affine.
    Then \\
    (1)f is affine \\
    (2)If moreover each $\mathcal{O}_{X}(U_{i})$ is finite over $\mathcal{O}_{Y}(V_{i})$, then f is finite.
    \label{Proposition 2.6}
\end{proposition}
\begin{proof}
    (1): Let $V \subseteq Y$ be an affine open subset.
    For all $y\in V \cap V_{i}$, take standard open subset $D(a(y)) \subseteq V$ such that $y\in D(a(y)) \subseteq V \cap V_{i}$.
    Now $D(a(y))$ is an affine subset of $V_{i}$.
    Take standard open subset $D(b(y)) \subseteq V_{i}$ such that $y\in D(b(y))$.
    By Proposition \ref{Proposition 1.12}, we have that $D(a(y)) \cap D(b(y))$ is a standard open subset of $D(a(y))$.
    Thus we can take $V_{ik}$ cover V, satisfying that $V_{ik}$ is standard open subset both in V and $V_{i}$.
    Since V is quasi-compact, can assume that $\{V_{ik}\}$ is finite.
    Thus U can be covered by finitely many $f^{-1}(V_{ik})$.
    While $f^{-1}(V_{ik}) = f^{-1}(V_{ik}) \cap f^{-1}(V_{i})$ is affine, get $f^{-1}(V_{ik})$ is affine subset of U of the form $U_{h}$ for some $h\in \mathcal{O}_{X}(U)$.
    Assume $V_{ik} = D(g_{ik})$ in V, get $f^{-1}(V_{ik}) = U_{f^{\sharp}(V)(g_{ik})}$.
    In addition, since $(f^{\sharp}(V)(g_{ik}) \big{|} i,k) \supseteq f^{\sharp}(V)((g_{ik} \big{|} i,k))$ and $1\in (g_{ik} \big{|} i,k)$, we have $(f^{\sharp}(V)(g_{ik}) \big{|} i,k) = \mathcal{O}_{X}(U)$.
    Thus by criterion of affineness, get U is affine.
    \par
    (2): By assumption, $\mathcal{O}_{X}(f^{-1}(V_{i}))$ is finite over $\mathcal{O}_{Y}(V_{i})$.
    Then their localizations still satisfy that $\mathcal{O}_{X}(f^{-1}(V_{ik}))$ is finite over $\mathcal{O}_{Y}(V_{ik})$.
    By Lemma \ref{Lemma 2.11}, $\mathcal{O}_{X}(U)$ is finite over $\mathcal{O}_{Y}(V)$.
    Thus f is finite.
\end{proof}
\begin{example}
    (1)Closed immersion is finite. \\
    (2)Open immersion is not finite in general, as $A_{f}$ is not finite over A in general. \\
    (3)Open immersion is not affine in general, as $\mathbb{A}_{k}^{1} \setminus \{(0)\}$ is not affine. \\
    (4)Let $f: \spec(B) \longrightarrow \spec A$ be a morphism of affine schemes.
    Then f is finite of and only if the associated ring homomorphism $\varphi: A \longrightarrow B$ makes B finite over A.
\end{example}
\subsection{Gluing and Fibre Product}
\begin{lemma}[\textbf{\emph{Gluing Lemma}}]
    Let S be a scheme, $\{X_{i}\}_{i\in I}$ a collection of S-schemes.
    For each i, give open subschemes $\{X_{ij}\}_{j\in I}$ of $X_{i}$ and for each i,j, give S-isomorphism $f_{ij}: X_{ij} \overset{\sim}{\longrightarrow} X_{ji}$,
    satisfying that \\
    (1)$f_{ij} = f_{ji}^{-1}$. 
    In particular, $f_{ii}$ is identity. \\
    (2)$f_{ij}(X_{ij} \cap X_{ik}) = X_{ji} \cap X_{jk}$. \\
    (3)$(f_{jk} \circ f_{ij})\big{|}_{X_{ij} \cap X_{ik}} = f_{ik}\big{|}_{X_{ij} \cap X_{ik}}$.
    \par
    Then there exists an S-scheme X unique up to isomorphism equipped with S-open immersions $g_{i}: X_{i} \longrightarrow X$ such that $X = \cup_{i\in I} g_{i}(X_{i})$ and $(g_{j} \circ f_{ij})\big{|}_{X_{ij}} = g_{i}\big{|}_{X_{ij}}$.
    X is called the gluing S-scheme of $\{X_{i}\}_{i\in I}$ along $\{X_{ij}\}$.
    \label{Lemma 2.12}
\end{lemma}
\begin{remark}
    (1)Let $S = \spec(\mathbb{Z})$.
    Then this lemma gives a way to glue scheme. \\
    (2)Gluing $\{X_{i}\}_{i\in I}$ along $\{X_{ij} = \varnothing\}$ gives $X = \sqcup_{i\in I} X_{i}$.
\end{remark}
\begin{proof}
    Consider topological space $X = (\sqcup_{i\in I} X_{i})/\sim$, 
    where $\sim$ is an equivalence relation and $x \sim y$ if $x\in X_{ij}$, $y\in X_{ji}$ and $y = f_{ij}(x)$ for some i,j.
    Get open inclusions $g_{i}: X_{i} \hookrightarrow X$ and $g_{j} \circ f_{ij} = g_{i}\big{|}_{X_{ij}}$.
    Need to construct sheaf of rings on X.
    Assume that $U_{i} = g_{i}(U_{i})$.
    Take sheaf $\mathcal{O}_{U_{i}}$ to be $(g_{i})_{\ast}\mathcal{O}_{X_{i}}$.
    Note that $\mathcal{O}_{U_{i}}\big{|}_{U_{i} \cap U_{j}} = \mathcal{O}_{U_{j}}\big{|}_{U_{i} \cap U_{j}}$ and $U_{i}$ cover X.
    By gluing lemma for sheaves, we get a sheaf $\mathcal{O}_{X}$ on X such that $\mathcal{O}_{X}\big{|}_{U_{i}} = \mathcal{O}_{U_{i}}$.
    As $g_{i}$ is isomorphic, take $h_{i}: U_{i} \overset{g_{i}^{-1}}{\longrightarrow} X_{i} \longrightarrow S$ and we have that $h_{i}\big{|}_{U_{i} \cap U_{j}} = h_{j}\big{|}_{U_{i} \cap U_{j}}$.
    By gluing lemma of morphisms, get $X \longrightarrow S$ and so X is an S-scheme.
    Uniqueness up to isomorphism is obvious.
\end{proof}
\begin{definition}[\textbf{\emph{Fibre Product}}]
    Let S be a scheme, X,Y two S-schemes.
    Define the fiber product of X and Y over S to be a S-scheme $W = X \times_{S} Y$ equipped with two S-morphisms $p: W \longrightarrow X$ and $q: W \longrightarrow Y$ satisfying the following universal property.
    \begin{equation*}
        \begin{tikzcd}
            \forall Z \arrow[drr, "g", bend left] \arrow[dr, "\exists!" ,dashrightarrow] \arrow[ddr, "f", bend right] && \\
            & W \arrow[r, "q"] \arrow[d, "p"] &
            Y \arrow[d] \\
            & X \arrow[r] &
            S
        \end{tikzcd}
    \end{equation*}
\end{definition}
\begin{proposition}[\textbf{\emph{Existence and Uniqueness of Fibre Product}}]
    Let S be a scheme, X,Y two S-schemes.
    Then the fiber product of X and Y over S exists and is unique up to isomorphism.
    \label{Proposition 2.7}
\end{proposition}
\begin{proof}
    Uniqueness up to isomorphism immediately comes from universal property.
    Here we only talk about the existence.
    For affine case, if $X = \spec A$, $Y = \spec(B)$ and $S =Spec(C)$, then we can take $X \times_{S} Y = \spec(A \otimes_{C} B)$.
    With universal property of tensor product, we can easily check $\spec(A \otimes_{C} B)$ is fiber product.
    \par
    If $X \times_{S} Y$ exists, then obviously $Y \times_{S} X$ exists and $U \times_{S} Y = p^{-1}(U)$ exists for all open subset $U \subseteq X$.
    Consider the case that Y,S are still affine while X abtrary.
    Cover X by affine open subsets $U_{i}$.
    Then $U_{i} \times_{S} Y$ exists and so does $(U_{i} \cap U_{j}) \times_{S} Y$.
    Taking $X_{i} = U_{i} \times_{S} Y$ and $X_{ij} = (U_{i} \cap U_{j}) \times_{S} Y$, we get isomorphisms $f_{ij}: X_{ij} \longrightarrow X_{ji}$.
    Bu gluing lemma of schemes, there exists gluing scheme W.
    It is easy to check that W satisfies the conditions for fiber product.
    \par
    Consider the case that S is affine while X,Y abtrary.
    Cover Y by affine open subsets $V_{i}$.
    Then $X \times_{S} V_{i}$ exists.
    Similar to the above process, get $X \times_{S} Y$ exists.
    \par
    For general case, cover S by affine open subsets $W_{i}$.
    Assume $f: X \longrightarrow S$ and $g: Y \longrightarrow S$.
    Then $f^{-1}(W_{i}) \times_{W_{i}} g^{-1}(W_{i})$ exists for all i.
    Thus $f^{-1}(W_{i} \cap W_{j}) \times_{W_{i}} g^{-1}(W_{i} \cap W_{j})$ exists for all i,j.
    Note that $f^{-1}(W_{i} \cap W_{j}) \times_{W_{i}} g^{-1}(W_{i} \cap W_{j})$ and $f^{-1}(W_{i} \cap W_{j}) \times_{W_{i} \cap W_{j}} g^{-1}(W_{i} \cap W_{j})$ satisfy the same universal property.
    We can view $f^{-1}(W_{i} \cap W_{j}) \times_{W_{i} \cap W_{j}} g^{-1}(W_{i} \cap W_{j})$ as open subset of $f^{-1}(W_{i}) \times_{W_{i}} g^{-1}(W_{i})$.
    Take $X_{i} = f^{-1}(W_{i}) \times_{W_{i}} g^{-1}(W_{i})$ and $X_{ij} = f^{-1}(W_{i} \cap W_{j}) \times_{W_{i} \cap W_{j}} g^{-1}(W_{i} \cap W_{j})$.
    There are isomorphisms $X_{ij} \overset{\sim}{\longrightarrow} X_{ji}$.
    By gluing lemma of schemes, $X \times_{S} Y$ exists.
\end{proof}
\begin{proposition}
    Let S be a scheme, X,Y S-schemes, .
    Then \\
    (1)$X \times_{S} S \cong X$ \\
    (2)$X \times_{S} Y \cong Y \times_{S} X$ \\
    (3)$(X \times_{S} Y) \times_{S} Z \cong X \times_{S} (Y \times_{S} Z)$. \\
    (4)If moreover Z is a Y-scheme, then $(X \times_{S} Y) \times_{Y} Z \cong X \times_{S} Z$. 
    \label{Proposition 2.8}
\end{proposition}
\begin{example}
    Let X be an affine or projective abstract variety over algebraically closed field k, $l/k$ field extension.
    Then $X \times_{k} l$ is affine or projective abstract "variety" over l and $(X \times_{k} l)(l) = X(l)$.
\end{example}
\begin{definition}
    Let $f: X \longrightarrow Y$ be a morphism of schemes.
    For $y\in Y$ with residue field $k(y)$, define the fiber of f over y to be the $k(y)$-scheme $X_{y} = X \times_{Y} k(y)$.
\end{definition}
\begin{proposition}
    Let $f: X \longrightarrow Y$ be a morphism of schemes, $y\in Y$.
    Then projection $p: X_{y} \longrightarrow X$ induces a homeomorphism between $X_{y}$ and $f^{-1}(y)$.
    \label{Proposition 2.9}
\end{proposition}
\begin{proof}
    Let $V \subseteq Y$ be an open subset.
    Then $X_{y} = X \times_{Y} k(y) \cong (X \times_{Y} V) \times_{V} k(y)$.
    Note that $X \times_{Y} V = f^{-1}(V)$, get $X_{y} \cong f^{-1}(V)_{y}$.
    Thus by taking affine open neighbourhood of y, we can assume that $Y = \spec A$.
    \par
    Let $U \subseteq X$ be an open subset.
    Then $p^{-1}(U) \cong U \times_{Y} k(y) = U_{y}$ so that we can also assume $X = \spec(B)$.
    Since f now is a morphism of affine schemes, we can take its associated ring homomorphisms $\varphi: A \longrightarrow B$ and y corresponds to a prime ideal $\mathfrak{p}$ of A.
    Then p corresponds to the following commutative diagram
    \begin{equation*}
        \begin{tikzcd}
            B \arrow[rr] \arrow[dr] &&
            B \otimes_{A} \fraction(A/\mathfrak{p}) \\
            & B \otimes_{A} A_{\mathfrak{p}} \arrow[ur] &
        \end{tikzcd}
    \end{equation*}
    For all $\mathfrak{q}\in im(p)$, we have that $\mathfrak{q}\supseteq \varphi(\mathfrak{p})B$ and $\mathfrak{q} \cap \varphi(A \setminus \mathfrak{p}) = \varnothing$.
    While for all $\mathfrak{q}\in \{\mathfrak{q}\in \spec(B) \big{|} \mathfrak{q}\supseteq \varphi(\mathfrak{p})B, \mathfrak{q} \cap \varphi(A \setminus \mathfrak{p}) = \varnothing\} = f^{-1}(\mathfrak{p})$,
    consider subset $\mathfrak{P} = \{b \otimes \frac{\overline{a'}}{\overline{a}} \big{|} b\in \mathfrak{q}, \overline{a},\overline{a'}\in A/\mathfrak{p}\} \subseteq B \otimes_{A} A/\mathfrak{p}$.
    With homomorphisms $B \otimes_{a} \fraction(A/\mathfrak{p}) \longrightarrow \fraction(B/\mathfrak{q})$, it is easy to see that $\mathfrak{P}$ is a prime ideal of $B \otimes_{a} \fraction(A/\mathfrak{p})$ and $p(\mathfrak{P}) = \mathfrak{q}$.
    Thus $im(p) = f^{-1}(\mathfrak{p})$.
    \par
    Want to show that $\widetilde{p}: X_{y} \longrightarrow im(p)$ is an open map.
    It suffices to prove that $p(D(b \otimes 1))$ is open in $im(p)$ for all $b\in B$.
    Obviously, we have $p(D(b \otimes 1)) \subseteq D(b) \cap im(p)$.
    While for all $\mathfrak{q}\in D(b) \cap im(p)$, there exists $\mathfrak{P}\in X_{y}$ such that $p(\mathfrak{P}) = \mathfrak{q}$.
    If $b \times 1\in \mathfrak{P}$, then $b\in p(\mathfrak{P}) = \mathfrak{q}$, contradiction.
    Thus $\mathfrak{P}\in D(b \otimes 1)$. 
    Thus p induces a homeomorphism between $X_{f}$ and $f^{-1}(\mathfrak{p})$.
\end{proof}
\begin{definition}[\textbf{\emph{Base Change}}]
    Let S be a scheme, X an S-scheme.
    If $S'$ is an S-scheme, we call $X \times_{S} S' \longrightarrow S \times_{S} S' \cong S'$ the base change of X in the morphism $S' \longrightarrow S$.
    \begin{equation*}
        \begin{tikzcd}
            X \times_{S} S' \arrow[r] \arrow[d] &
            S' \arrow[d] \\
            X \arrow[r] &
            S
        \end{tikzcd}
    \end{equation*}
    A property of morphisms P is called stable under base change if for all $X \longrightarrow Y$ satisfying P and $Y'$ Y-scheme, 
    $X \times_{Y} Y' \longrightarrow Y'$ still satisfies P.  
\end{definition}
\begin{proposition}
    The following propositions of morphisms are stable under base change. \\
    (1)open immersion \\
    (2)closed immersion \\
    (3)locally of finite type \\
    (4)quasi-compact \\
    (5)of finite type \\
    (6)affine \\
    (7)finite
    \label{Proposition 2.10}
\end{proposition}
\begin{corollary}
    Let S be a scheme, X,Y two S-schemes of finite type (resp. locally of finite type, quasi-compact, affine, finite).
    Then so is $X \times_{S} Y$ over S.
    \label{Corollary 2.1}
\end{corollary}
\begin{reason}
    These properties are all stable under composition and base change.
\end{reason}
\begin{corollary}
    Let $f: X \longrightarrow Y$ be a morphism of schemes of finite type.
    Then for all $y\in Y$, $X_{y}$ is finite type over $k(y)$.
    \label{Corollary 2.2}
\end{corollary}

\section{Global Properties of Schemes}
\label{section:Global Properties of Schemes}

\begin{definition}
    Let $X$ be a topological space.
    We say that $X$ is noetherian if all decreasing sequence of closed subsets stabilizes.
    Or equivalently, all increasing sequence of open subsets stabilizes.
\end{definition}
\begin{remark}
    If $A$ is a noetherian ring, then $\spec A$ is noetherian.
    But if $\spec A$ is noetherian, $A$ need not be noetherian in general.
    For example, we can take $A = k[x_{1}, x_{2}, \cdots]/(x_{1}, x_{2}^{2}, \cdots)$.
    \label{remark:noetherian_affine_scheme_not_noetherian_ring_in_general}
\end{remark}
\begin{proposition}
    Let X be a topological space.
    Then \\
    (1)If X is noetherian, then all open or closed subsets of X is also noetherian. \\
    (2)X is noetherian if and only if all open subsets of X are quasi-compact. \\
    (3)If X is covered by finitely many $\{X_{i}\}$ which are noetherian, then X is noetherian.
    \label{Proposition 3.1}
\end{proposition}
\begin{definition}
    Let X be a topological space.
    We say that X is irreducible if X is nonempty and X cannot be covered by two proper closed subsets.
    Or equivalently, X is nonempty and any 2 nonempty open subsets intersect.
\end{definition}
\begin{proposition}
    Let X be a topological space.
    Then \\
    (1)If X is irreducible, then every nonempty open subset of X is also irreducible. \\
    (2)Let $Y \subseteq X$ be a topological subspace.
    If Y is irreducible, then closure $\overline{Y}$ of Y in X is also irreducible.
    \label{Proposition 3.2}
\end{proposition}
\begin{theorem}
    Let X be a noetherian topological space.
    Then every nonempty closed subset Y of X can be uniquely written as a finite union $Y = \cup_{i} Y_{i}$, where $Y_{i}$ is irreducible closed subset and $Y_{i} \not\supseteq Y_{j}$ for all $i \neq j$.
    The decomposition of Y is called the irreducible components of Y.
    \label{Theorem 3.1}
\end{theorem}
\begin{proof}
    Existence: suppose there exists nonempty closed subset Y of X with no such finite decomposition.
    As X is noetherian, we can take a minimal Y.
    Obviously, Y is not irreducible.
    Thus $Y = Y_{1} \cup Y_{2}$, where $Y_{1}$ and $Y_{2}$ are proper closed subsets.
    As $Y_{1}$ and $Y_{2}$ are nonempty, get $Y_{1}$ and $Y_{2}$ can be written as such finite decompositions, inducing that Y can be written as such finite decomposition, contradiction.
    Thus all nonempty closed subsets have such finite decomposition.
    \par
    Uniqueness: Assume $Y = \cup_{i} Y_{i}$ and $Y = \cup_{j} Z_{j}$ are two decompositions.
    Then $Z_{j} \subseteq Y = \cup_{i} Y_{i}$ and so $Z_{j} = \cup_{i} (Z_{j} \cap Y_{i})$.
    Since $Z_{j}$ is irreducible, there exists i such that $Z_{j} \subseteq Y_{i}$.
    While for $Y_{i}$ there also exists $j'$ such that $Y_{i} \subseteq Z_{j'}$, inducing $Z_{j} \subseteq Z_{j'}$.
    By conditions for decomposition, get $j = j'$ and $Y_{i} = Z_{j}$.
    Thus we give a bijection between $\{Y_{i}\}$ and $\{Z_{j}\}$ so that $Y = \cup_{i} Y_{i}$ and $Y = \cup_{j} Z_{j}$ are the same decomposition.
\end{proof}
\begin{remark}
    Without noetherian condition, we can use the Axiom of Choice to prove weak edition of this theorem i.e. decomposition without finiteness.
\end{remark}
\subsection{Noetherian Schemes}
\begin{definition}[\textbf{\emph{Locally Noetherian Schemes}}]
    Let X be a scheme.
    We say that X is locally noetherian if for all affine open subset $U = \spec A$ of X, A is noetherian.
\end{definition}
\begin{lemma}
    Let X be a locally noetherian scheme.
    Then the minimal irreducible closed subsets of X are closed points.
    \label{Lemma 3.1}
\end{lemma}
\begin{proof}
    Assume V is one minimal irreducible closed subset of X.
    Then for all $x\in V$, $\overline{\{x\}} = V$.
    If there are two points in V, suppose they are x and y. 
    Take a noetherian affine open neighbourhood $U = \spec A$ of x.
    Then since $\overline{\{y\}} = V$, we have that $y\in X$.
    While minimal irreducible closed subsets in an affine scheme are the maximal ideals.
    Get $x = y$, thus V is one-point set.
\end{proof}
\begin{definition}[\textbf{\emph{Noetherian Schemes}}]
    Let X be a scheme.
    We say that X is noetherian if X can be covered by finitely many affine open subset $U = \spec A$ of X with A is noetherian.
\end{definition}
\begin{remark}
    If X is noetherian scheme, then the underlying topological space of X is noetherian.
    While if the underlying topological space of X is noetherian, X is not noetherian in general.
    Example can be taken the same as Remark \ref{remark:noetherian_affine_scheme_not_noetherian_ring_in_general}.
\end{remark}
\begin{proposition}
    Let X be a noetherian scheme, $U \subseteq X$ open subset.
    Then U is noetherian as scheme.
    \label{Proposition 3.3}
\end{proposition}
\begin{proposition}
    Let X be a noetherian scheme.
    If $i: U \hookrightarrow X$ is an open immersion, then i is of finite type.
    \label{Proposition 3.4}
\end{proposition}
\begin{reason}
    By noetherian, we know i is quasi-compact.
    By open immersion, we know is locally of finite type.
\end{reason}
\begin{proposition}
    Let S be a scheme, X,Y two S-schemes.
    If X is noetherian and of finite type over S,
    then all S-morphism $f: X \longrightarrow Y$ are of finite type.
    \label{Proposition 3.5}
\end{proposition}
\begin{reason}
    By Proposition \ref{Proposition 2.4}, we know that f is locally of finite type.
    Since X is noetherian, get f is also quasi-compact.
\end{reason}
\begin{remark}
    All affine or projective abstract varieties over algebraically closed field k is noetherian.
    Thus k-morphism between affine or projective abstract varieties is of finite type.
\end{remark}
\begin{theorem}
    Let X be a scheme.
    Suppose that X admits an affine open covering $\{U_{i} = Sepc(A_{i})\}$ with each $A_{i}$ noetherian.
    Then X is locally noetherian.
    \label{Theorem 3.2}
\end{theorem}
\begin{proof}
    Let U be an affine open subset of X.
    Similarly, we can cover U by $U_{ik}$ such that $U_{ik} \subseteq U$ is of the form $D(f_{ik})$ and $U_{ik} \subseteq U_{i}$ is of the form $D(g_{ik})$.
    Note that localization of noetherian ring is noetherian, get $\mathcal{O}_{X}(U_{ik})$ is noetherian.
    Remains to show that if $\spec A$ is covered by $\{D(f_{i}) = \spec(A_{f_{i}})\}$ with each $A_{f_{i}}$ noetherian,
    then A is noetherian.
    \par
    Since $\spec A$ is quasi-compact, can assume $\{D(f_{i}) = \spec(A_{f_{i}})\}$ finite.
    Consider $\varphi_{i}: A \longrightarrow A_{f_{i}}$ localization.
    For any increasing sequence of ideals of A, $I_{1} \subseteq I_{2} \subseteq \cdots$, by localization, we get increasing sequence of ideals of $A_{f_{i}}$.
    Since $A_{f_{i}}$ is noetherian and there are only finitely many i, there exists n such that $\forall k \ge n$, $\varphi_{i}(I_{k})A_{f_{i}} = \varphi_{i}(I_{k + 1})A_{f_{i}}$.
    Thus it suffices to show that $I_{k} = \cap_{i} (\varphi_{i}(I_{k})A_{f_{i}} \cap A)$.
    For $s\in \cap_{i} (\varphi_{i}(I_{k})A_{f_{i}} \cap A)$, assume that $s = \frac{a_{i}}{f_{i}^{n_{i}}}$ in $A_{f_{i}}$ where $a_{i}\in I_{k}$
    Similar to proof of Theorem \ref{Theorem 1.1}, we can take $a_{i}'$ and $f_{i}'$ such that $\frac{a_{i}'}{f_{i}'} = s$, $a_{i}'f_{j}' = a_{j}'f_{i}'$ and $\sum_{i} b_{i}f_{i}' = 1$ where $b_{i}\in A$.
    Consider element $\frac{\sum_{i} b_{i}a_{i}'}{1}$.
    Then as $a_{j}' = \sum_{i} b_{i}f_{i}'a_{j}' = \sum_{i}b_{i}a_{i}'f_{j}'$, $\frac{\sum_{i} b_{i}a_{i}'}{1} = s$ in $A_{f_{j}}$ for all j.
    Thus $s = \frac{\sum_{i} b_{i}a_{i}'}{1}$ and so $s\in I_{k}$.
    Get $I_{k} = \cap_{i} (\varphi_{i}(I_{k})A_{f_{i}} \cap A)$.
\end{proof}
\subsection{Reduced and Integral Schemes}
\begin{definition}[\textbf{\emph{Reduced Schemes}}]
    Let X be a scheme.
    A scheme X is called reduced if for all $x\in X$, the local ring $\mathcal{O}_{X, x}$ are reduced.
\end{definition}
\begin{remark}
    In an affine scheme, viewing each ring element as a function, then function vanishing at all prime ideals is equivalent to be nilpotent.
    Thus, geometrically, reducedness means that function is determined by its values at each points.
\end{remark}
\begin{proposition}
    Let X be a scheme.
    Then X is reduced if and only if for all open subset $U \subseteq X$, $\mathcal{O}_{X}(U)$ is reduced.
    \label{Proposition 3.6}
\end{proposition}
\begin{proof}
    "$\Rightarrow$": If $s\in \mathcal{O}_{X}(U)$ satisfies that there exists $n\in \mathbb{N}_{+}$ such that $s^{n} = 0$,
    then for all $x\in U$, $(s_{x})^{n} = 0$ in $\mathcal{O}_{X, x}$.
    Since $\mathcal{O}_{X, x}$ is reduced, get $s_{x} = 0$.
    Thus there exists open neighbourhood $U_{x}$ of x such that $s\big{|}_{U_{x}} = 0$.
    As $U_{x}$ cover U, by property of sheaf, $s = 0$.
    Thus $\mathcal{O}_{X}(U)$ is reduced.
    \par
    "$\Leftarrow$": For all $x\in X$, if $(U, s)\in \mathcal{O}_{X, x}$ satisfies that there exists $n\in \mathbb{N}_{+}$ such that $(U, s)^{n} = 0$,
    then there exists open neighbourhood V of x such that $(s\big{|}_{V})^{n} = 0$.
    Since $\mathcal{O}_{X}(V)$ is reduced, get $s\big{|}_{V} = 0$.
    As $(U, s) = (V, s\big{|}_{V})$, $\mathcal{O}_{X, x}$ is reduced.
\end{proof}
\begin{example}
    (1)Let A be a ring.
    Then $\spec A$ is reduced if and only if A is reduced. \\
    (2)Let A be a ring.
    Consider $A_{red} = A/rad(0)$.
    There is a closed immersion $\spec(A_{red}) \longrightarrow \spec A$.
    Reduced scheme can also be defined for abtrary scheme, see in Hartshorne exercise II 2.3. \\
    (3)Let A be a ring.
    Then reduced closed subscheme of $\spec A$ is given by some radical ideal $rad(I)$ of A.
    For general case, if V is a closed subset of scheme X, then there exists unique structure of reduced closed subscheme on V gluing by reduced closed subscheme $V \cap V_{i}$ of affine open subset $V_{i}$.
    Details can be seen in Hartshorne Chapter II example 3.2.6, p86.
\end{example}
\begin{definition}[\textbf{\emph{Integral Schemes}}]
    Let X be a scheme.
    We say that X is integral if X is irreducible and reduced.
\end{definition}
\begin{proposition}
    Let X be a scheme.
    If X is noetherian and connected, then X is integral if and only if for all $x\in X$, $\mathcal{O}_{X, x}$ is integral domain.
    \label{Proposition 3.7}
\end{proposition}
\begin{proof}
    "$\Rightarrow$": For all $x\in X$, if $(U, s),(V, t)\in \mathcal{O}_{X, x}$ satisfy that $(U, s)(V, t) = 0$,
    then there exists affine open neighbourhood W of x such that $s\big{|}_{W}t\big{|}_{W} = 0$.
    Since X is irreducible, by Proposition \ref{Proposition 3.2}, W is irreducible.
    While $D(s\big{|}_{W}) \cap D(t\big{|}_{W}) = D(s\big{|}_{W}t\big{|}_{W}) = \varnothing$, get $D(s\big{|}_{W}) = \varnothing$ or $D(t\big{|}_{W}) = \varnothing$.
    Thus $(U, s)$ is nilpotent or $(V, t)$ is nilpotent.
    Since $\mathcal{O}_{X, x}$ is reduced, get $\mathcal{O}_{X, x}$ is integral domain.
    \par
    "$\Leftarrow$": Since $\mathcal{O}_{X, x}$ is integral domain, $\mathcal{O}_{X, x}$ is reduced and so X is reduced.
    Want to show that X is irreducible.
    Suppose that X is not irreducible.
    Since X is noetherian, by Theorem \ref{Theorem 3.1}, X has finite decomposition of irreducible components.
    Assume $X = \cup_{i} X_{i}$ and $X \neq X_{1}$.
    As X is connected, $X_{1} \cap (\cup_{i \neq 1} X_{i}) \neq \varnothing$.
    Without loss of generality, can assume that $X_{1} \cap X_{2} \neq \varnothing$ and take $x\in X_{1} \cap X_{2}$.
    Consider an affine open neighbourhood $U = \spec A$ of X.
    Take generic points $\eta_{1}\in X_{1}$ and $\eta_{2}\in X_{2}$ (By Proposition \ref{Proposition 3.10}).
    Then $\eta_{1},\eta_{2}\in U$ and correspond to different prime ideals $\mathfrak{p}_{1}$ and $\mathfrak{p}_{2}$ respectively.
    While x also corresponds to a prime ideal $\mathfrak{p}$, we have $\mathfrak{p}_{1} \subseteq \mathfrak{p}$ and $\mathfrak{p}_{2} \subseteq \mathfrak{p}$,
    inducing that there are two minimal prime ideals in $A_{\mathfrak{p}}$ contradicting to that $A_{\mathfrak{p}}$ is integral domain.
\end{proof}
\begin{proposition}
    Let $X = \spec A$ be an affine scheme.
    Then \\
    (1)Closed subset $V(I) \subseteq X$ is irreducible if and only if $rad(I)$ is prime.
    In particular, X is irreducible if and only if X has a unique minimal prime ideal. \\
    (2)X is integral if and only if A is integral domain. \\
    (3)If A is noetherian, then we have that irreducible components of X are $\{V(\mathfrak{p}_{i})\}$, where $\mathfrak{p}_{i}$ are minimal prime ideals of A.
    \label{Proposition 3.8}
\end{proposition}
\begin{proposition}
    Let X be a nonempty scheme.
    Then X is integral if and only if for all nonempty open subset U of X, $\mathcal{O}_{X}(U)$ is integral domain.
    \label{Proposition 3.9}
\end{proposition}
\begin{reason}
    Proof of "$\Rightarrow$" is similar to proof of Proposition \ref{Proposition 3.7}.
    For "$\Leftarrow$", it suffices to show that all affine open subsets are irreducible, which comes from Proposition \ref{Proposition 3.8}.
\end{reason}
\begin{example}
    Let A be an integral domain.
    Then $\mathbb{A}_{A}^{n}$ and $\mathbb{P}_{A}^{n}$ are integral.
\end{example}
\begin{proposition}
    Let X be an irreducible scheme.
    Then X contains a unique point $\eta$ such that $\overline{\{\eta\}} = X$ called the generic point of X.
    If moreover X is integral, then $\mathcal{O}_{X, \eta}$ is a field called the function field of X.
    \label{Proposition 3.10}
\end{proposition}
\begin{remark}
    If X is integral, then for all affine open neighbourhood $U = \spec A$ of $\eta$, we have that $\mathcal{O}_{X, \eta} = \fraction(A)$. 
\end{remark}
\begin{proof}
    For affine case, obviously, $\eta = (0)$.
    For abtrary X, if $\eta$ exists, then for all affine open subset $U = \spec A$ of X, we have $\eta\in U$.
    Thus $\eta$ is also generic point of U, inducing the uniqueness of generic point.
    \par
    On the other hand, take $\eta = \eta_{U}$ for some fixed affine open subset U of X.
    For all open subset V of X, since X is irreducible, $U \cap V$ is nonempty.
    Thus $U \cap V$ is a nonempty open subset of U and so $\eta\in U \cap V \subseteq V$.
    Get $\overline{\{\eta\}} = X$ and $\eta$ is the wanted generic point.
\end{proof}
\subsection{Geometrically Global Properties}
Now let $X$ be a $k$-scheme for some field $k$, we are interested in $X_{\overline{k}} = X \times_{k} \overline{k}$.
\begin{definition}
    Let X be a k-scheme for some field k.
    We say that X is geometrically integral (resp. geometrically irreducible, geometrically reduced) if $X_{\overline{k}}$ is integral (resp. irreducible, reduced).
\end{definition}
\begin{remark}
    (1)These properties are dependent on the base field k.
    For instance, $\spec(\mathcal{C})$ is geometrically irreducible over $\mathbb{C}$ but not over $\mathbb{R}$. \\
    (2)Geometrically irreducible induces irreducible but the oppsite is false.
    For $\spec(\mathbb{C})$ over $\mathbb{R}$ is irreducible but not geometrically irreducible. \\
    (3)Geometrically reduced induces reduced but the oppsite is false.
    For instance, take $X = \spec(\mathbb{R}[x]/(x^{2} + 1))$ over $\mathbb{R}$, then X is reduced but not geometrically reduced. \\
    (4)Geometrically integral induces but the oppsite is false.
    Can take the same example as (2) or (3).
    In addition, if take $k = \mathbb{F}_{p}(t)$ and $X = \spec(k[x]/(x^{p} - t)) = k(t^{\frac{1}{p}})$, 
    then this is an example that is integral but neither geometrically irreducible nor geometrically reduced.
\end{remark}
\begin{proposition}
    Let k be a perfect field, X integral k-scheme, K function field of X.
    Then the following conditions are equivalent \\
    (1)X is geometrically integral. \\
    (2)The ring $K \otimes_{k} \overline{k}$ is an integral scheme. \\
    (3)k is algebraically closed in K. \\
    (4)For all finite field extension $l/k$, $X \times_{k} l$ is integral. \\
    (5)For all algebraic field extension $l/k$, $X \times_{k} l$ is integral. \\
    (6)For all field extension $l/k$, $X \times_{k} l$ is integral.
    \label{Proposition 3.11}
\end{proposition}
\subsection{Dimension Theory}
\begin{theorem}[\textbf{\emph{Cohen-Seidenberg, Going-up Theorem}}]
    Let $A \hookrightarrow B$ be an injective ring homomorphism.
    Suppose that B is integral over A.
    Then \\
    (1)The associated map of spectrums $\spec(B) \longrightarrow \spec A$ is surjective. \\
    (2)$\dim(A) = \dim(B)$.
    \label{Theorem 3.3}
\end{theorem}
\begin{remark}
    Precisely, Going-up Theorem is saying that if $\mathfrak{p}_{1} \subsetneq \mathfrak{p}_{2}$ are two prime ideals of A and there exists $\mathfrak{P}_{1}\in \spec(B)$ over $\mathfrak{p}_{1}$,
    then there also exists $\mathfrak{P}_{2}\in \spec(B)$ over $\mathfrak{p}_{2}$ such that $\mathfrak{P}_{1} \subsetneq \mathfrak{P}_{2}$.
\end{remark}
\begin{theorem}[\textbf{\emph{Noetherian Normalization Theorem}}]
    Let A be an algebra of finite type over a field k.
    Then \\
    (1)There exists $y_{1}, \cdots, y_{r}\in A$ algebraically independent over k such that A is integral over $k[y_{1}, \cdots, y_{r}]$. \\
    (2)If A is integral domain, then $\dim(A)$ equals to the transcandence degree of $\fraction(A)$ over k, which by (1) is r.
    In addition, for all prime ideal $\mathfrak{p}\in \spec A$, we have that $\dim(A) = \dim(A/\mathfrak{p}) + ht(\mathfrak{p})$.
    \label{Theorem 3.4}
\end{theorem}
\begin{definition}
    Let X be a nonempty topological space.
    Define the dimension of X is the supremum of n for which there exists $V_{0} \supsetneq V_{1} \supsetneq \cdots \supsetneq V_{n}$ sequence of irreducible closed subsets of X.
\end{definition}
\begin{proposition}
    Let $X = \spec A$ be an affine scheme.
    Then we have $\dim(X) = \dim(A)$.
    \label{Proposition 3.12}
\end{proposition}
\begin{reason}
    Comes from the fact irreducible closed subset one-to-one corresponds to prime ideal of A.
\end{reason}
\begin{proposition}
    Let X be a topological space.
    Then \\
    (1) For all subset $Y \subseteq X$ with induced topology, then $\dim(Y) \le \dim(X)$. \\
    (2) Suppose that X is irreducible and finite-dimensional.
    If $Y \subseteq X$ is a closed subset and $\dim(X) = \dim(Y)$, then $X = Y$. \\
    (3)If X is noetherian as topological space, then $\dim(X) = \max_{\text{irreducible components}}\{\dim(V)\}$. \\
    (4)If $\{U_{i}\}$ is an open covering of X, then $\dim(X) = \sup_{i}\{\dim(U_{i})\}$.
    \label{Proposition 3.13}
\end{proposition}
\begin{example}
    (1)Let X be an irreducible scheme.
    If $\dim(X) = 0$, then underlying topological space of X is a one-point set. \\
    (2) Let X be a noetherian scheme.
    If $\dim(X) = 0$, then irreducible components of X are all one-point sets and disjoint. \\
    (3)Let A be a noetherian ring.
    Then $\dim(\mathbb{A}_{A}^{n}) = \dim(\mathbb{P}_{A}^{n}) = \dim(A) + n$. \\
    (4)$\dim(X) = \dim(X_{red})$. \\
    (5)Let X be a scheme, $U \subseteq X$ dense open subset.
    It can happen that $\dim(U) = \dim(X)$ even when X is noetherian integral scheme.
    For instance, take $X = \spec A$ for some discrete valuation ring A. 
\end{example}
\begin{definition}
    Let X be a topological space, $Y \subseteq X$ irreducible closed.
    Define the codimension $codim_{X}(Y)$ to be the supremum of n for which there exists $V_{0} \supsetneq V_{1} \supsetneq \cdots \supsetneq V_{n} = Y$ sequence of irreducible closed subsets of X.
\end{definition}
\begin{remark}
    Obviously, $\dim(X) \ge \dim(Y) + codim_{X}(Y)$.
\end{remark}
\begin{theorem}
    Let X be an integral scheme of finite type over field k, $K$ function field of X.
    Then \\
    (1)$\dim(X) = tr\dim(K/k) < \infty$ \\
    (2)For all nonempty open subset U of X, $\dim(X) = \dim(U)$. \\
    (3)For all closed point $x\in X$, $\dim(X) = \dim(\mathcal{O}_{X, x})$. 
    \label{Theorem 3.5}
\end{theorem}
\begin{proof}
    (1): By Proposition \ref{Proposition 3.13}, $\dim(X)$ equals to the supremum of dimension of affine open subsets.
    While for all affine open subset $U = \spec A$ of X, we have that A is of finite type over k, $\dim(U) = \dim(A)$ and $\fraction(A) = K$.
    By Noetherian Normalization Theorem, get $\dim(U) = tr\dim(K/k)$ and so $\dim(X) = tr\dim(K/k)$.
    \par
    (2): Since U is also integral scheme with function field K of finite type over k, by (1), $\dim(U) = \dim(X)$. 
    \par
    (3): For all closed point $x\in X$, take affine open neighbourhood $U = \spec A$ of x.
    Then x corresponds to a prime ideal $\mathfrak{p}_{x}$ of A and $\mathcal{O}_{X, x} = A_{\mathfrak{p}}$.
    While x is also closed point in U, get $\mathfrak{p}_{x}$ is maximal ideal.
    Thus by Noetherian Normalization Theorem, $\dim(\mathcal{O}_{X, x}) = \dim(A) = \dim(U) = \dim(X)$. 
\end{proof}
\begin{corollary}
    Let X be a scheme of finite type over field k. 
    Then $\dim(X) < \infty$.
    If moreover X is irreducible, then $\dim(X) = \dim(U)$ for all nonempty open subset U of X.
    \label{Corollary 3.1}
\end{corollary}
\begin{proof}
    Since X is of finite type over k, get X is noetherian and so X has finite decomposition of irreducible components $\{X_{i}\}$.
    Then $\dim(X) = \max_{i}\{\dim(X_{i})\}$.
    While $\dim(X_{i}) = \dim((X_{i})_{red})$ where $(X_{i})_{red}$ is integral scheme of finite type over k.
    By Proposition \ref{Theorem 3.5}, get $\dim(X)$ finite.
\end{proof}
\begin{definition}[\textbf{\emph{Pure Schemes}}]
    Let X be a noetherian scheme.
    We say that X is pure (or say equidimensional) if all irreducible components of X have the same finite dimension.
\end{definition}
\begin{proposition}
    Let X be a scheme of finite type over field k.
    Suppose X is pure.
    Then \\
    (1)For all nonempty open subset U of X, $\dim(U) = \dim(X)$. \\
    (2)For all irreducible closed subset $Y \subseteq X$, $\dim(X) = \dim(Y) + codim_{X}(Y)$. 
    \label{Proposition 3.14} 
\end{proposition}
\begin{reason}
    The proof is similar to proof of Theorem \ref{Theorem 3.5} and Corollary \ref{Corollary 3.1}.
\end{reason}
\begin{theorem}
    Let X be a scheme of finite type over field k.
    Then the set of all closed points of X is dense in X.
    \label{Theorem 3.6}
\end{theorem}
\begin{proof}
    By taking reduced scheme of X, can assume that X is reduced.
    If suffices to show that there exists some closed point in U for all affine open subset $U = \spec A$of X.
    Take $x\in U$ such that x corresponds to maximal ideal $\mathfrak{m}_{x}\subseteq A$.
    Want to show that x is a closed point in X.
    Since X is of finite type over k, get X is noetherian and so that X has finite decomposition of irreducible components.
    Suppose that $x\in X_{i}$ where $X_{i}$ is some irreducible component.
    Only need to show that x is closed point in $X_{i}$. 
    \par
    Note that $X_{i}$ with reduced closed subscheme structure is integral scheme of finite type over k.
    Replace X by $X_{i}$.
    If there is another point $y\in \overline{\{x\}}$, then for all open neighbourhood V of y, $x\in V$.
    Take $V = \spec(B)$ to be affine and assume that x,y correspond to prime ideals $\mathfrak{q}_{x}$ and $\mathfrak{q}_{y}$ respectively.
    Then it is easy to see that $\mathfrak{q}_{x} \subseteq \mathfrak{q}_{y}$.
    Thus $\dim(X) = \dim(U) = \dim(A_{\mathfrak{m}_{x}}) = \dim(B_{\mathfrak{q}_{x}}) < \dim(B_{\mathfrak{q}_{y}}) \le \dim(B) = \dim(V) = \dim(X)$, contradiction!
    Get x is a closed point in X.
    So it is obviously that the set of all closed points of X is dense in X.
\end{proof}
\begin{corollary}
    Let X be a scheme of finite type over field k.
    Assume Z is an irreducible component of X.
    Then there exists closed point $x\in Z$ such that x is not in any other irreducible component of X.
    \label{Corollary 3.2}
\end{corollary}
\begin{proof}
    Since X is of finite type over k, X is noetherian.
    Thus X has finitely many irreducible components.
    Assume other irreducible components are $Z_{1}, \cdots, Z_{n}$ and $\Sigma_{1} \subseteq Z_{1}, \cdots, \Sigma_{n} \subseteq Z_{n}$ are subsets of closed points respectively.
    If all closed points of Z are in some other irreducible component, then the subset of all closed points in X is $\cup_{i = 1}^{n} \Sigma_{i}$.
    By Theorem \ref{Theorem 3.6}, we have that $X = \overline{\cup_{i = 1}^{n} \Sigma_{i}} \subseteq \overline{\cup_{i = 1}^{n} Z_{i}} = \cup_{i = 1}^{n} Z_{i}$, contradiction!
\end{proof}
\begin{theorem}
    Let $f: X \longrightarrow Y$ be a finite and surjective morphism of schemes.
    Then $\dim(X) = \dim(Y)$.
    \label{Theorem 3.7}
\end{theorem}
\begin{proof}
    Since f is affine, can cover both X and Y by affine open subsets.
    Thus the question can be reduced to the affine case.
    Assume that $X = \spec(B)$ and $Y = \spec A$ with A,B reduced.
    Note that B is finite over A and so B is integral over A.
    By Cohen-Seidenberg, $\dim(X) = \dim(A) = \dim(B) = \dim(Y)$.
\end{proof}
\begin{definition}
    Let $f: X \longrightarrow Y$ be morphism of integral schemes.
    We say that f is dominant if $f(X)$ is dense in Y.
    Or equivalently, $f(X)$ contains the generic point of Y.
    Or equivalently, $f(\eta_{X}) = \eta_{y}$.
\end{definition}
\begin{theorem}
    Let X,Y be two integral schemes of finite type over field k, $f: X \longrightarrow Y$ dominant k-morphism, K function field of X, F function field of Y.
    Then \\
    (1)The generic fiber $X_{\eta_{Y}}$ is an integral F-scheme whose function field is K. \\
    (2)$\dim(X_{\eta_{Y}}) = \dim(X) - \dim(Y)$. 
    \label{Theorem 3.8}
\end{theorem}
\begin{remark}
    (1): Take affine open covering $\{V_{i} = \spec(A_{i})\}$ of Y with $\fraction(A_{i}) = F$.
    Then we can coevr X by affine open subsets $U_{ij} = \spec(B_{ij}) \subseteq f^{-1}(V_{i})$.
    Note that $X_{\eta_{Y}} = \cup_{i,j} (U_{ij} \times_{V_{i}} F) = \cup_{i,j} \spec(B_{ij} \otimes_{A_{i}} F)$.
    It suffices to show that $B_{ij} \otimes_{A_{i}} F$ is integral domain with fraction field K.
    As f is dominant, $f(\eta_{X}) = \eta_{Y}$.
    Consider the associated ring homomorphism $\varphi_{ij}: A_{i} \longrightarrow B_{ij}$.
    Then $\ker(\varphi_{ij}) = \{0\}$.
    Since F is a localization of $A_{i}$, $B_{ij} \otimes_{A_{i}} F$ is a localization of $B_{ij}$ so that $B_{ij} \otimes_{A_{i}} F$ is integral domain with fraction field $K = \fraction(B_{ij})$.
    \par
    (2): By Theorem \ref{Theorem 3.5}, $\dim(X_{\eta_{Y}}) = tr\dim(K/F) = tr\dim(K/k) - tr\dim(F/k) = \dim(X) = \dim(Y)$.
\end{remark}
\begin{remark}
    Fibre over generic point is integral doesn't induce other fibers are integral.
\end{remark}

\section{Global Properties of Morphisms}
\label{section:Global Properties of Morphisms}

\subsection{Separated Morphisms}
\begin{definition}
    Let $f: X \longrightarrow Y$ be a morphism of schemes.
    Define the diagonal morphism associated to f to be $\Delta_{X/Y}: X \longrightarrow X \times_{Y} X$ induced by $(\identity_{X}, \identity_{X})$ with the following commutative diagram.
    \begin{equation*}
        \begin{tikzcd}
            X \arrow[drr, "\identity_{X}", bend left] \arrow[dr, "\Delta_{X/Y}"] \arrow[ddr, "\identity_{X}", bend right] && \\
            & X \times_{Y} X \arrow[r] \arrow[d] & 
            X \arrow[d, "f"] \\
            & X \arrow[r, "f"] &
            Y
        \end{tikzcd}
    \end{equation*}
\end{definition}
\begin{remark}
    For $x\in X \times_{Y} X$, if images of x under two projections are same, we don't have that $x\in \Delta_{X/Y}(X)$.
    For example, take $X = \spec(\mathbb{C})$ and $Y = \spec(\mathbb{R})$.
    Then X has only one point but $X \times_{Y} Y$ has two different points.
    \par
    In addition, if take $W = \cup U \times_{V} U$, where $V \subseteq Y$ and $U \subseteq f^{-1}(V)$ are both affine open and $U$ consist of an open covering of $X$,
    then $\Delta(X) \subseteq W$ is closed in each $U \times_{V} U$ so that it is closed in $W$.
\end{remark}
\begin{definition}[\textbf{\emph{Separated Morphisms}}]
    Let $f: X \longrightarrow Y$ be a morphism of schemes.
    We say that f is separated if the diagonal morphism associated to f is a closed immersion.
    In particular, we say that a S-scheme X is separated over S if $X \longrightarrow S$ is separated.
\end{definition}
\begin{proposition}
    Let $f: X \longrightarrow Y$ be a morphism of affine schemes.
    Then f is separated.
    \label{Proposition 4.1}
\end{proposition}
\begin{proof}
    Assume that $X = \spec(B)$ and $\spec A$.
    Then $X \times_{Y} X = \spec(B \otimes_{A} B)$ and diagonal morphism associates to a ring homomorphism $\varphi: B \otimes_{A} B \longrightarrow B$.
    Since $\varphi$ is obviously surjective, get $B \cong (B \otimes_{A} B)/I$ for some ideal I and so $\spec(B) = \spec((B \otimes_{A} B)/I)$ is a closed subscheme of $X \times_{Y} X$.
    Thus $\Delta_{X/Y}$ is a closed immersion.
\end{proof}
With this definition, we can state definition of abstract variety over general field k.
\begin{definition}[\textbf{\emph{Abstract Varieties}}]
    Let k be a field.
    An abstract variety over k is a separated and integral scheme of finite type over k.
\end{definition}
\begin{proposition}
    Let $f: X \longrightarrow Y$ be a morphism of schemes.
    Then f is separated if and only if $\Delta_{X/Y}(X)$ is closed in $X \times_{Y} X$.
    \label{Proposition 4.2}
\end{proposition}
\begin{proof}
    "$\Rightarrow$": Obvious.
    \par
    "$\Leftarrow$": There is a commutative diagram
    \begin{equation*}
        \begin{tikzcd}
            X \arrow[rr, "\identity_{X}"] \arrow[dr, "\Delta_{X/Y}"] &&
            X \\
            & X \times_{Y} X \arrow[ur, "p"] &
        \end{tikzcd}
    \end{equation*}
    It is easy to see that $\Delta_{X/Y}$ induces a homeomorphism between X and $\Delta_{X/Y}(X)$ in this case.
    Just need to show that $\Delta_{X/Y}^{\sharp}$ is surjective.
    \par
    By Lemma \ref{Lemma 2.5}, if suffices to prove that $\Delta_{X/Y, x}^{\sharp}$ is surjective for all $x\in X$.
    Take affine open neighbourhood V of $f(x)$ and affine open neighbourhood $U \subseteq f^{-1}(V)$ of x.
    Then $\Delta_{X/Y, x}^{\sharp} = \Delta_{U/V, x}^{\sharp}$.
    Note that $\Delta_{U/V}$ is a closed immersion, done!
\end{proof}
\begin{proposition}
    Let $f: X \longrightarrow Y$ be a morphism of schemes.
    Suppose that Y admits an open covering $\{Y_{i}\}$ such that $f^{-1}(Y_{i}) \longrightarrow Y_{i}$ is separated.
    Then f is separated.
    \label{Proposition 4.3}
\end{proposition}
\begin{proof}
    Let $X_{i} = f^{-1}(Y_{i})$.
    Then $X \times_{Y} X$ is covered by $X_{i} \times_{Y_{i}} X_{i}$ with each $X_{i} \longrightarrow Y_{i}$ separated.
    Thus $\Delta_{X/Y}(X) \cap (X_{i} \times_{Y_{i}} X_{i}) = \Delta_{X_{i}/Y_{i}}(X_{i})$ is closed in $X_{i} \times_{Y_{i}} X_{i}$.
    Get $\Delta_{X/Y}(X)$ is closed in $X \times_{Y} X$.
    By Proposition \ref{Proposition 4.2}, f is separated.
\end{proof}
\begin{proposition}[\textbf{\emph{Criterion of Separated}}]
    Let $S = \spec A$ be an affine scheme, X an S-scheme.
    The following conditions are equivalent \\
    (1)X is separated over S. \\
    (2)For all U,V affine open subsets of X, $U \cap V$ is still affine and $\varphi_{UV}: \mathcal{O}_{X}(U) \otimes_{A} \mathcal{O}_{X}(V) \longrightarrow \mathcal{O}_{X}(U \cap V)$ is surjective. \\
    (3)There exists affine open covering $\{U_{i}\}$ of X such that $U_{i} \cap U_{j}$ is still affine and $\varphi_{U_{i}U_{j}}: \mathcal{O}_{X}(U_{i}) \otimes_{A} \mathcal{O}_{X}(U_{j}) \longrightarrow \mathcal{O}_{X}(U_{i} \cap U_{j})$ is surjective
    \label{Proposition 4.4}
\end{proposition}
\begin{proof}
    Prove as $(1) \Rightarrow (2) \Rightarrow (3) \Rightarrow (1)$.
    \par
    $(1) \Rightarrow (2)$: For all U,V affine subsets of X, $U \cap V = \Delta_{X/S}^{-1}(U \times_{S} V)$.
    Since $\Delta_{X/S}$ is closed immersion, get $\Delta_{X/S}\big{|}_{U \cap V}: U \cap V \longrightarrow U \times_{S} V$ still closed immersion.
    As $U \times_{X} V$ is affine, $U \cap V$ is affine and $\varphi_{UV}: \mathcal{O}_{X}(U) \otimes_{A} \mathcal{O}_{X}(V) \longrightarrow \mathcal{O}_{X}(U \cap V)$ is surjective.
    \par
    $(2) \Rightarrow (3)$: Obvious. 
    \par
    $(3) \Rightarrow (1)$: Similarly, it suffices to show that $\Delta_{X/S}(X)$ is closed in $X \times_{S} X$.
\end{proof}
\begin{proposition}
    (1)Open or closed immersions are separated. \\
    (2)The composition of two separated morphisms is separated. \\
    (3)Separated is stable under base change. \\
    (4)Let $f: X \longrightarrow Y$ and $g: Y \longrightarrow Z$ be two morphisms of schemes.
    If $g \circ f$ is separated, then f is separated.
    In particular, k-morphisms between abstract k-varieties are separated.
    \label{Proposition 4.5}
\end{proposition}
\begin{proof}
    Here only prove (4).
    Consider $h: X \times_{Y} X \longrightarrow X \times_{Z} X$ with the commutative diagram.
    \begin{equation*}
        \begin{tikzcd}
            X \times_{Y} X \arrow[drr, "p_{2}", bend left] \arrow[dr, "h"] \arrow[ddr, "p_{1}", bend right] && \\
            & X \times_{Z} X \arrow[r, "q_{2}"] \arrow[d, "q_{1}"] & 
            X \arrow[d, "g \circ f"] \\
            & X \arrow[r, "g \circ f"] &
            Z
        \end{tikzcd}
    \end{equation*}
    Want to show $\Delta_{X/Y}(X) = h^{-1}(\Delta_{X/Z})$ which is closed in $X \times_{Y} X$.
    Obviously, $\Delta_{X/Y}(X) \subseteq h^{-1}(\Delta_{X/Z})$.
    For the other side, if for $s,t\in X \times_{Y} X$ satisfy that $h(s) = h(t)\in \Delta_{X/Z}(X)$, we need $s = t$.
    Assume that $h(s) = h(t) = \Delta_{X/Z}(x)$.
    Take affine open neighbourhoods $U, V, W$ of $x, f(x), g(f(x))$ respectively such that $U \subseteq f^{-1}(V)$ and $V \subseteq g^{-1}(W)$.
    Then $h\big{|}_{U \times_{V} U}$ is injective.
    It is easy to see that s and t are both in $U \times_{V} U$.
    Thus $s = t$.  
\end{proof}
\begin{proposition}
    Let S be a scheme, X reduced S-scheme, Y separated S-scheme.
    Suppose that $f,g: X \longrightarrow Y$ are two S-morphisms coincider on a dense open subset $U \subseteq X$.
    Then $f = g$.
    \label{Proposition 4.6}
\end{proposition}
\begin{proof}
    Consider $h: X \longrightarrow Y \times_{S} Y$ induced by $(f, g)$.
    We have that $\Delta_{Y/S} \circ f$ coincide with h on U.
    Thus $h(U) = \Delta_{Y/S} \circ f(U) \subseteq \Delta_{Y/S}(Y)$ so that $U \subseteq h^{-1}(\Delta_{Y/S}(Y))$.
    Note that $h^{-1}(\Delta_{Y/S}(Y))$ is closed in X.
    While U is dense in X, get $h^{-1}(\Delta_{Y/S}(Y)) = X$ and so $h(X) \subseteq \Delta_{Y/S}(Y)$.
    \par
    For all $x\in X$, we have that $p_{1} \circ h(x) = f(x)$ and $p_{2} \circ h(x) = g(x)$.
    While $p_{1} \circ \Delta_{Y/S} = \identity_{Y} = p_{2} \circ \Delta_{Y/S}$, get $f(x) = g(x)$ for all $x\in X$.
    Now it suffices to show that $f^{\sharp}(V) = g^{\sharp}(V)$ for all affine open subset V of Y.
    Thus can assume that $Y = \spec A$.
    Also, by taking affine open covering of X, can assume $X = \spec(B)$.
    Then $f,g: X \longrightarrow Y$ associate to $\varphi, \psi: A \longrightarrow B$.
    For all $a\in A$, take $b = \varphi(a) - \psi(a)\in B = \mathcal{O}_{X}(X)$.
    Then $b\big{|}_{U} = 0$ so that $U \subseteq V(b)$.
    Note that $V(b)$ is closed, get $V(b) = \spec(B)$ so that $b\in rad(B) = 0$.
    Thus $\varphi = \psi$ and so $f = g$.
\end{proof}
\subsection{Proper Morphisms}
\begin{definition}[\textbf{\emph{Universally Closed Morphisms}}]
    Let $f: X \longrightarrow Y$ be a morphism of schemes.
    We say that f is universally closed if for any base change $Y' \longrightarrow Y$, $X \times_{Y} Y' \longrightarrow Y'$ is closed.
\end{definition}
\begin{definition}[\textbf{\emph{Proper Morphisms}}]
    Let $f: X \longrightarrow Y$ be a morphism of schemes.
    We say that f is proper if f is separated, of finite type and universally closed.
    In particular, an S-scheme X is proper over S if $X \longrightarrow S$ is proper.
\end{definition}
\begin{example}
    (1)Closed immersion is always proper. \\
    (2)Open immersion is not proper in general, since it is not quasi-compact in general. \\
    (3)Let k be an algebraically closed field.
    Then $\mathbb{A}_{k}^{1} \longrightarrow k$ is not proper, since $\mathbb{A}_{k}^{1} \times_{k} \mathbb{A}_{k}^{1} \longrightarrow \mathbb{A}_{k}^{1}$ is not closed.
    The image of $\spec(k[x, y]/(xy - 1))$ is $\mathbb{A}_{k}^{1} \setminus \{(x)\}$.
    Will see that projective abstract k-varieties are proper over k.
\end{example}
\begin{proposition}
    Let $f: X \longrightarrow Y$ be a morphism of schemes.
    Suppose that Y admits an open covering $\{Y_{i}\}$ such that $f^{-1}(Y_{i}) \longrightarrow Y_{i}$ is proper.
    Then f is proper.
    \label{Proposition 4.7}
\end{proposition}
\begin{proof}
    We already know of finite type and separated are local on the base.
    It suffices to show that universally closed is local on the base.
    Let $X_{i} = f^{-1}(Y_{i})$.
    For all base change $g: Y' \longrightarrow Y$, consider the following commutative diagram. 
    \begin{equation*}
        \begin{tikzcd}
            X \times_{Y} Y' \arrow[r, "\widetilde{f}"] \arrow[d, "\widetilde{g}"] &
            Y' \arrow[d, "g"] \\
            X \arrow[r, "f"] &
            Y
        \end{tikzcd}
    \end{equation*}
    Then $X \times_{Y} Y'$ can be covered by $X_{i} \times_{Y_{i}} g^{-1}(Y_{i})$ which is a base change of $X_{i} \longrightarrow Y_{i}$.
    Thus $X_{i} \times_{Y_{i}} g^{-1}(Y_{i}) \longrightarrow g^{-1}(Y_{i})$ is closed.
    For all closed subset V of $X \times_{Y} Y'$, $V \cap X_{i} \times_{Y_{i}} g^{-1}(Y_{i})$ is closed in $X_{i} \times_{Y_{i}} g^{-1}(Y_{i})$ so that $\widetilde{f}(V) \cap g^{-1}(Y_{i})$ is closed in $g^{-1}(Y_{i})$.
    As $g^{-1}(Y_{i})$ cover $Y'$, get $\widetilde{f}(V)$ is closed in $Y'$.
    Thus $X \longrightarrow Y$ is universally closed. 
\end{proof}
\begin{theorem}
    Let $f: X \longrightarrow Y$ be a finite morphism of schemes.
    Then f is proper.
    \label{Theorem 4.1}
\end{theorem}
\begin{proof}
    Obviously, f is of finite type.
    Since proper is local on the base, can assume that $Y = \spec A$.
    While f is affine, get $X = f^{-1}(Y)$ affine.
    Assume that $X = \spec(B)$.
    Now f is a morphism of affine schemes, automatically being separated.
    \par
    Consider the associated ring homomorphism $\varphi: A \longrightarrow B$.
    $\varphi$ factors $A \longrightarrow A/\ker(\varphi) \longrightarrow B$.
    Can assume that $\varphi$ is injective.
    As f is finite, B is finite over A.
    By Cohen-Seidenberg, f is closed.
    Similar discussion show that f is universally closed.
\end{proof}
\begin{proposition}
    (1)The composition of two proper morphisms is proper. \\
    (2)Proper is stable under base change. \\
    (3)Let $f: X \longrightarrow Y$ and $g: Y \longrightarrow Z$ be two morphisms of schemes.
    If $g \circ f$ is proper and g is separated, then f is proper.
    In particular, k-morphisms between abstract k-varieties are proper.
    \label{Proposition 4.8}
\end{proposition}
\begin{theorem}[\textbf{\emph{Cancellation Theorem}}]
    Let P be a property of morphisms that is stable under composition and base change.
    Assume that $f: X \longrightarrow Y$ and $g: Y \longrightarrow Z$ are two morphisms of schemes.
    If $g \circ f$ satisfies P and g is separated, then f satisfies P.
    \label{Theorem 4.2}
\end{theorem}
\begin{reason}
    Comes from the following commutative diagram. 
    \begin{equation*}
        \begin{tikzcd}[column sep = tiny]
            & X \cong X \times_{Y} Y \arrow[rr] \arrow[dl] &&
            X \times_{Z} Y \arrow[rr] \arrow[dr] \arrow[dl] &&
            Z \times_{Z} Y \cong Y \arrow[dr] & \\
            Y \arrow[rr, "\Delta_{Y/Z}"] && 
            Y \times_{Z} Y &&
            X \arrow[rr, "g \circ f"] &&
            Z
        \end{tikzcd}
    \end{equation*}
\end{reason}
\begin{lemma}
    Let $\varphi: A \longrightarrow B$ be injective ring homomorphism.
    If the associated map of spectrum $f: \spec A \longrightarrow \spec(B)$ is closed, then $\varphi^{-1}(B^{\times}) = A^{\times}$.
    \label{Lemma 4.1}
\end{lemma}
\begin{proof}
    By Lemma \ref{Lemma 2.7}, the image of f is dense in $\spec A$.
    While f is closed, get f is surjective.
    Thus for all $b\in B^{\times}$ and $a\in \varphi^{-1}(b)$, if $a\notin A^{\times}$, 
    then there exists maximal ideal $\mathfrak{m}\in \spec A$ such that $a\in \mathfrak{m}$.
    Since f is surjective, there exists prime ideal $\mathfrak{P}\in \spec(B)$ such that $\varphi^{-1}(\mathfrak{P}) = \mathfrak{m}$.
    Thus $b = \varphi(a) \in \mathfrak{P}$, contradiction.
    Get $\varphi^{-1}(B^{\times}) = A^{\times}$.
\end{proof}
\begin{lemma}
    Let $\varphi: A \longrightarrow B$ be injective ring homomorphism.
    If the associated map of spectrum $f: \spec(A[t]) \longrightarrow \spec(B[t])$ is closed, then B is integral over A.
    \label{Lemma 4.2}
\end{lemma}
\begin{proof}
    Consider ideal $I = (bt - 1) \subseteq B[t]$ and $J = I \cap A[t] \subseteq A[t]$.
    Since $\varphi$ is injective, the induced homomorphism $A[t] \longrightarrow B[t]$ is also injective and so is $\varphi': A[t]/J \longrightarrow B[t]/I$.
    And $\spec(B[t]/I)  \longrightarrow \spec(A[t]/J)$ is closed.
    By Lemma \ref{Lemma 4.1}, we have that ${\varphi'}^{-1}((B[t]/I)^{\times}) = (A[t]/J)^{\times}$.
    Note that $\overline{b}\overline{t} = \overline{1}$, get $\overline{t}\in (B[t]/I)^{\times}$.
    As $\widetilde{\overline{t}} = \overline{t}$, $\overline{t}\in (A[t]/J)^{\times}$.
    Thus there exists polynomial $a_{n}t^{n} + \cdots + a_{1}t + a_{0}\in A[t]$ such that $\overline{t}\overline{a_{n}t^{n} + \cdots + a_{1}t + a_{0}} = \overline{1}$ in $A[t]/J$ so that in $B[t]/I$.
    Multiplied by $\overline{b^{n + 1}}$, get $\overline{a_{n} + \cdots + a_{1}b^{n - 1} + a_{0}b^{n}} = \overline{b^{n + 1}}$.
    Thus $b^{n + 1} - a_{0}b^{n} - \cdots - a_{n - 1}b - a_{n}\in I = (bt - 1)$.
    Since $I \cap B = \{0\}$, get $b^{n + 1} - a_{0}b^{n} - \cdots - a_{n - 1}b - a_{n} = 0$ and so b is integral over A.
    In conclusion, B is integral over A.
\end{proof}
\begin{proposition}
    Let $f: \spec(B) \longrightarrow \spec A$ be a proper morphism of affine schemes.
    Then B is finite over $A$-module.
\end{proposition}
\begin{proof}
    Consider the associated ring homomorphism $\varphi: A \longrightarrow B$ and $\ker(\varphi)$.
    There is a commutative diagram.
    \begin{equation*}
        \begin{tikzcd}
            \spec(B) \arrow[rr, "f"] \arrow[dr, "g"] &&
            \spec A \\
            & \spec(A/\ker(\varphi)) \arrow[ur, "h"] &
        \end{tikzcd}
    \end{equation*}
    As h is separated, by Proposition \ref{Proposition 4.8}, g is proper.
    Note that $A/\ker(\varphi)$ is finite over A, can assume that $\varphi$ is injective.
    \par
    Obviously, B is of finite type over A.
    It suffices to show that B is integral over A.
    Note that $B[t] \cong A[t] \otimes_{A} B$, get $\spec(B[t]) \longrightarrow \spec(A[t])$ is given by base change $\spec(A[t]) \longrightarrow \spec A$ on f.
    As f is universally closed, $\spec(B[t]) \longrightarrow \spec(A[t])$ is closed.
    Thus by Lemma \ref{Lemma 4.2}, B is integral over A.
\end{proof}
\subsection{Projective Morphisms}
\begin{definition}
    Let Y be a scheme.
    Define projective n-space over Y to be $\mathbb{P}_{Y}^{n} = \mathbb{P}_{\mathbb{Z}}^{n} \times_{\mathbb{Z}} Y$,
    where $\mathbb{P}_{\mathbb{Z}}^{n} = \proj(\mathbb{Z}[x_{0}, \cdots, x_{n}])$.
\end{definition}
\begin{definition}[\textbf{\emph{Projective Morphisms}}]
    Let $f: X \longrightarrow Y$ be a morphism of schemes.
    We say that f is projective if it factors through for some projective n-space over Y,
    \begin{equation*}
        \begin{tikzcd}
            X \arrow[r, "i"] \arrow[dr, "f"] &
            \mathbb{P}_{Y}^{n} \arrow[d, "p"] \\
            & Y
        \end{tikzcd}
    \end{equation*}
    where i is a closed immersion and p is the projection.
\end{definition}
\begin{remark}
    This definition is same to the definition in Hartshorne, which is more general than the definition in EGA.
    For instance, finite induce EGA's projective but not Hartshorne's projective.
\end{remark}
\begin{lemma}
    Let A be a ring, B an A-algebra, S a graded A-algebra.
    Then $S \otimes_{A} B$ is a graded B-algebra graded by $(S \otimes_{A} B)_{d} = S_{d} \otimes B$.
    In addition, we have that $\proj(S \otimes_{A} B) = \proj(S) \times_{A} B$.
    \label{Lemma 4.3}
\end{lemma}
\begin{proof}
    It is easy to check $S \otimes_{A} B$ is a graded B-algebra.
    There is a graded homomorphism $\varphi: S \longrightarrow S \otimes_{A} B\quad s \longmapsto s \otimes 1_{B}$ and $\varphi(S_{+})(S \otimes_{A} B) = (S \otimes_{A} B)_{+}$.
    Get A-morphism $g: \proj(S \otimes_{A} B) \longrightarrow \proj(S)$.
    On the other hand, $B \longrightarrow S \otimes_{A} B$ gives A-morphism $\pi: \proj(S \otimes_{A} B) \longrightarrow \spec(B)$.
    By universal property of $\proj(S) \times_{A} B$, there is an A-morphism $h: \proj(S \otimes_{A} B) \longrightarrow \proj(S) \times_{A} B$ induced by $(g, \pi)$.
    And we have that $h^{-1}(D_{+}(f) \times_{A} B) = g^{-1}(D_{+}(f)) = D_{+}(\varphi(f))$.
    $D_{+}(\varphi(f)) \longrightarrow D_{+}(f) \times_{A} B$ associates to ring homomorphism $\psi: S_{(f)} \otimes_{A} B \longrightarrow (S \otimes_{A} B)_{(\varphi(f))}$.
    It suffices to show $\psi$ is isomorphic.
    While $S_{(f)} \otimes_{A} B$ is the subset of elements of degree 0 in $S_{f} \otimes_{A} B$, $(S \otimes_{A} B)_{(\varphi(f))}$ is the subset of elements of degree 0 in $(S \otimes_{A} B)_{\varphi(f)}$ and $S_{f} \otimes_{A} B \cong (S \otimes_{A} B)_{\varphi(f)}$, get $\psi$ isomorphic.
\end{proof}
\begin{theorem}
    Let $f: X \longrightarrow Y$ be a projective morphism of schemes.
    Then f is proper.
    \label{Theorem 4.3}
\end{theorem}
\begin{proof}
    Since proper is stable under composition and closed immersion is proper, we only need to show that $\mathbb{P}_{Y}^{n} \longrightarrow Y$ is proper.
    In addition, as proper is stable under base change, it suffices to prove $p: \mathbb{P}_{\mathbb{Z}}^{n} \longrightarrow \mathbb{Z}$ is proper.
    Note that $\mathbb{P}_{\mathbb{Z}}^{n} = \cup_{i = 0}^{n} D_{+}(x_{i})$.
    It is easy to use the affine open covering to show p is of finite type and separated.
    \par
    To show p is universally closed, need to prove $\mathbb{P}_{S}^{n} \longrightarrow S$ is closed.
    While closed is local on the base, can assume $S = \spec A$ so that by Lemma \ref{Lemma 4.3}, $\mathbb{P}_{S}^{n} = \proj(A[x_{0}, \cdots, x_{n}])$.
    It is easy to show that $\spec(A[x_{0}, \cdots, x_{n}]_{(\frac{1}{x_{i}})}) \longrightarrow \spec A$ is proper for all $0 \le i \le n$.
    Thus $f$ is proper.
\end{proof}
\begin{theorem}[\textbf{\emph{Valuation Criterion for Separated and Proper}}]
    Let $f: X \longrightarrow Y$ be a morphism of schemes with X noetherian.
    Then \\
    (1)f is separated if and only if for all valuation ring R with $Rrad(R) = K$ and commutative diagram
    \begin{equation*}
        \begin{tikzcd}
            \spec K \arrow[r] \arrow[d] &
            X \arrow[d, "f"] \\
            \spec(R) \arrow[ur, dashrightarrow] \arrow[r] &
            Y
        \end{tikzcd}
    \end{equation*}
    there exists at more one morphism $\spec(R) \longrightarrow X$ making the diagram commutative. \\
    (2)If moreover f is of finite type,
    then f is proper if and only if for all valuation ring R with $Rrad(R) = K$ and commutative diagram
    \begin{equation*}
        \begin{tikzcd}
            \spec K \arrow[r] \arrow[d] &
            X \arrow[d, "f"] \\
            \spec(R) \arrow[ur, "\exists!", dashrightarrow] \arrow[r] &
            Y
        \end{tikzcd}
    \end{equation*}
    there exists unique morphism $\spec(R) \longrightarrow X$ making the diagram commutative.
    \label{Theorem 4.4}
\end{theorem}
\begin{proof}
    See in the Hartshorne Chapter II Theorem 4.3, p97 and Theorem 4.7, p101.
\end{proof}

\section{Local Properties of Schemes and Morphisms}
\label{section:Local Properties of Schemes and Morphisms}

\subsection{Normal Schemes}
\begin{definition}
    Let X be a scheme.
    We say that X is normal at $x\in X$ if the local ring $\mathcal{O}_{X, x}$ is integrally closed domain.
    In particular, we say that X is normal if X is normal at x for all $x\in X$.
\end{definition}
\begin{example}
    Let A be an integral domain.
    Then $\spec A$ is normal if and only if A is integrally closed.
\end{example}
\begin{theorem}[\textbf{\emph{Algebraic Hartogs}}]
    Let A be a noetherian integrally closed domain, $K = \fraction(A)$.
    Then $A = \cap_{\mathfrak{p}\in \spec A, ht(\mathfrak{p}) \le 1} A_{\mathfrak{p}} \subseteq K$.
    \label{Theorem 5.1}
\end{theorem}
\begin{remark}
    In algebraic geometry, this fact can be translated into that functions defined in codim 1 are defined everywhere.
\end{remark}
\begin{corollary}
    Let X be a noetherian normal scheme.
    Assume that $Z \subseteq X$ is a closed subset and all irreducible components of Z are of codimension $\ge 2$.
    Then the restriction $\mathcal{O}_{X}(X) \longrightarrow \mathcal{O}_{X}(X \setminus Z)$ is isomorphic.
    \label{Corollary 5.1}
\end{corollary}
\begin{proof}
    Since X is integral, both $\mathcal{O}_{X}(X)$ and $\mathcal{O}_{X}(X \setminus Z)$ are subrings of the function field K of X.
    Suffice to prove the surjectivity for the case that Z is irreducible.
    For all affine open subset $U \subseteq X$ meeting Z, we have $codim_{X}(Z) = codim_{U}(Z \cap U)$.
    By gluing lemma, can assume that $X = \spec A$ so that $Z = V(\mathfrak{q})$ for some $\mathfrak{q}\in X$.
    \par
    For all $\mathfrak{p}\in X$ with $ht(\mathfrak{p}) \le 1$, we have that $\mathfrak{p}\in X \setminus Z$.
    Now we can view $\mathcal{O}_{X}(X \setminus Z)$ as a subring of $A_{\mathfrak{p}}$.
    Thus by Theorem \ref{Theorem 5.1} $A \subseteq \mathcal{O}_{X}(X \setminus Z) \subseteq \cap_{\mathfrak{p}\in \spec A, ht(\mathfrak{p}) \le 1} A_{\mathfrak{p}} = A$, done!
\end{proof}
\begin{definition}
    Let X be a scheme, $x\in X$.
    We say that x is of codimension r if codimension of the irreducible closed subset $\overline{\{x\}}$ is r.
\end{definition}
\begin{proposition}
    Let X be a scheme, $x\in X$.
    Then x is of codimension r if and only if $\dim(\mathcal{O}_{X, x}) = r$.
    \label{Proposition 5.1}
\end{proposition}
\begin{reason}
    Take affine open neighbourhood U of x.
\end{reason}
\begin{lemma}
    Let S be a scheme, X,Y two S-schemes with Y locally of finite type.
    Assume that either X integral or S locally noetherian.
    For all $x\in X$ and S-morphism $f_{x}: \spec(\mathcal{O}_{X, x}) \longrightarrow Y$,
    $f_{x}$ extends an open subset of X containing x.
    \label{Lemma 5.1}
\end{lemma}
\begin{proof}
    Can assume X, Y and S are affine.
    Set $X =Spec(B), Y = \spec A$ and $S = \spec(C)$.
    Want all C-homomorphism $\varphi: A \longrightarrow B_{\mathfrak{p}}$ factors as 
    \begin{equation*}
        \begin{tikzcd}
            A \arrow[rr, "\varphi"] \arrow[dr] &&
            B_{\mathfrak{p}} \\
            & B_{b} \arrow[ur] &
        \end{tikzcd}
    \end{equation*}
    for some $b\notin \mathfrak{p}$.
    \par
    If B is integral domain, choose generators of A as C-algebra $\{a_{1}, \cdots, a_{r}\}$.
    Then we can take $b\notin \mathfrak{p}$ such that $\varphi(a_{i})\in B_{b}$ for all i, done!
    \par
    If C is noetherian, write $A = C[t_{1}, \cdots, t_{n}]/I$, where $I = (P_{1}, \cdots, P_{m})$ is finitely generated.
    Now since $P_{j}(\overline{t_{1}}, \cdots, \overline{t_{n}}) = 0$, $P_{j}(\varphi(\overline{t_{1}}), \cdots, \varphi(\overline{t_{n}})) = 0$ in $B_{\mathfrak{p}}$.
    By finiteness of j, can take $b\notin \mathfrak{p}$ such that $bP_{j}(\varphi(\overline{t_{1}}), \cdots, \varphi(\overline{t_{n}})) = 0$ in B, done!
\end{proof}
\begin{theorem}
    Let S be a noetherian scheme, X,Y two S-schemes of finite type.
    Assume that Y is proper over S and X is normal with function field K.
    For all nonempty open subset U of X and S-morphism $f: U \longrightarrow Y$ (resp. $f_{\eta_{X}}: \spec K \longrightarrow Y$),
    there exists open subset $V \subseteq U$ (resp. nonempty open subset V) of X, containing all points of codimension 1 of X, such that f extends uniquely to V.  
    In particular, if $\dim(X) = 1$, then f extends to whole X.
    \label{Theorem 5.2}
\end{theorem}
\begin{proof}
    Uniqueness: For open subset case, if there are two extension to V, denote by $f_{1}$ and $f_{2}$.
    Then $f_{1}$ and $f_{2}$ coincide on U which is a dense open subset of V.
    Note that X is reduced and Y is separated over S, by Proposition \ref{Proposition 4.6}, $f_{1} = f_{2}$.
    \par
    Existence: For each case, there exists S-morphism $f_{\eta_{X}}: \spec K \longrightarrow Y$.
    For all point $x\in X$ of codimension 1, $\mathcal{O}_{X, x}$ is 1-dimensional, noetherian and integrally closed local domain so that is a discrete valuation ring.
    By valuation criterion of properness, $f_{\eta_{X}}$ extends uniquely to S-morphism $f_{x}: \spec(\mathcal{O}_{X, x}) \longrightarrow Y$.
    Thus by Lemma \ref{Lemma 5.1}, $f_{x}$ extends to $g_{x}: U_{x} \longrightarrow Y$ for some open neighbourhood $U_{x}$ of x.
    Note that $f(\eta_{X}) = g_{x}(\eta_{X})$.
    Take affine affine open covering $f(\eta_{X})\in \spec A \subseteq Y$ and affine open covering $\spec(B) \subseteq f^{-1}(Spec(A)) \cap g^{-1}(Spec(A))$.
    Then f and $g_{X}$ coincide on $\spec(B)$ induced by the following commutative diagram.
    \begin{equation*}
        \begin{tikzcd}
            A \arrow[r, bend left] \arrow[r, bend right] &
            B \arrow[r, hookrightarrow] &
            K
        \end{tikzcd}
    \end{equation*}
    Get f and $g_{x}$ coincide on $U \cap U_{x}$.
    For $x'\in X$ another point of codimension 1, similar argument gives f, $g_{x}$ and $g_{x'}$ coincide on the intersections of U, $U_{x}$ and $U_{x'}$.
    By gluing, f can extends to $U \cup (\cup_{\text{x of codimension 1}} U_{x})$.
\end{proof}
\begin{definition}
    Let X,Y be two abstract k-varieties.
    We say that X and Y are k-birational if their function fields are k-isomorphic.
\end{definition}
\begin{corollary}
    Let X,Y be two normal and complete curves over k.
    If X,Y are k-birational, then X,Y are k-isomorphic.
    \label{Corollary 5.2}
\end{corollary}
\begin{definition}[\textbf{\emph{Normalization}}]
    Let $X = \spec A$ be an affine scheme with A integral domain.
    Consider $\widetilde{X} = \spec(\widetilde{A})$, where $\widetilde{A}$ is the integral closure of A in $\fraction(A)$.
    We say that $\widetilde{X}$ is the normalization of X with the $\pi: \widetilde{X} \longrightarrow X$ and universal property:
    for all dominant morphism $f: Y \longrightarrow X$ with Y normal, there exists morphism $\widetilde{f}: Y \longrightarrow \widetilde{X}$ such that $f = \pi \circ \widetilde{f}$.
    By gluing, we can also define the normalization of abtrary integral scheme X over its function field with the same universal property.
\end{definition}
\begin{example}
    (1)Let $X = \spec(\mathbb{Z}[\sqrt{-3}])$ not normal.
    Then $\widetilde{X} = \spec(\mathbb{Z}[\frac{-1 + \sqrt{-3}}{2}])$. \\
    (2)Let $X = \spec(k[x, y]/(y^{2} - x^{3}))$ not normal.
    Then $\widetilde{X} = \spec(k[t])$.
    Morphism $\pi: \widetilde{X} \longrightarrow X$ is given by $x \longmapsto t^{2}$ and $y \longmapsto t^{3}$. \\
    (3)Let $X = \spec(k[x, y]/(y^{2} - x^{2}(x + 1)))$ not normal.
    Then $\widetilde{X} = \spec(k[t])$.
    Morphism $\pi: \widetilde{X} \longrightarrow X$ is given by $x \longmapsto t^{2} - 1$ and $y \longmapsto t(t^{2} - 1)$. 
\end{example}
\begin{remark}
    (1)Let A be an integral domain, $K = \fraction(A)$, $L/K$ field extension.
    Since we can define the integral closure $\widetilde{A}_{L}$ of A in L, we can also define the normalization of $Sepc(A)$ in L to be $\spec(\widetilde{A}_{L})$.
\end{remark}
\begin{proposition}
    Let X be an integral scheme, K function field of X.
    Then the natural morphism $\pi: \widetilde{X} \longrightarrow X$ is affine and surjective.
    If moreover X is of finite type over k for some field k, $\pi$ is finite.
    \label{Proposition 5.2}
\end{proposition}
\begin{proof}
    The question is local.
    Can assume that $X = \spec A$ is affine.
    Then $\widetilde{X} = \spec(\widetilde{A})$ and $K = \fraction(A)$.
    Obviously $\pi: \spec(\widetilde{A}) \longrightarrow \spec A$ is affine.
    And since $\widetilde{A}$ is integral over A, by Cohen-Seidenberg Theorem, get $\pi$ is surjective.
    \par
    If moreover A is of finite type over k, by noetherian normalization Theorem, get there exists $x_{1}, \cdots, x_{r}\in A$ algebraically independent over k such that A is integral over $k[x_{1}, \cdots, x_{r}]$.
    Thus K is a finite extension over $k(x_{1}, \cdots, x_{r})$ and $\widetilde{A}$ is the integral closure of $k[x_{1}, \cdots, x_{r}]$ in K.
    The proof of the fact that $\widetilde{A}$ is of finite type over $k[x_{1}, \cdots, x_{r}]$ is similar to the proof in \emph{Algebraic Nunber Theory}, Milne, p35.
    Thus $\widetilde{A}$ is finite over $k[x_{1}, \cdots, x_{n}]$ and so that $\widetilde{A}$ is finite over A.
\end{proof}
\subsection{Regular Schemes}
\begin{definition}
    Let X be a locally noetherian scheme.
    We say that X is regular at $x\in X$ if the local ring $\mathcal{O}_{X, x}$ is regular.
    If not, then x is called a singular point of X.
    In particular, we say that X is regular if X is regular at all points.
\end{definition}
\begin{example}
    (1)Let X be a normal and locally noetherian scheme.
    Since discrete valuation ring is regular local ring, all points of codimension 1 in X are nonsingular points. \\
    (2)Let $X = \spec(k[x, y, z]/(z^{2} - xy))$.
    Then X is normal and noetherian but not regular at $(\overline{x}, \overline{y}, \overline{z})$.
\end{example}
\begin{proposition}
    Let X be a locally noetherian scheme.
    If X is regular at closed points, then X is regular.
    \label{Proposition 5.3}
\end{proposition}
\begin{proof}
    For all $x\in X$, $\overline{\{x\}}$ is irreducible.
    Consider the minimal irreducible closed subset $V \subseteq \{x\}$.
    By Lemma \ref{Lemma 3.1}, V must be a one-point set.
    Assume that $V = \{y\}$.
    Take an affine open neighbourhood U of y, then $x\in U$.
    As $y\in \overline{\{x\}}$, get $\mathcal{O}_{X, x}$ is a localization of $\mathcal{O}_{X, y}$.
    Since X is regular at y, X is also regular at x.
\end{proof}
\begin{remark}
    Let X be a locally noetherian scheme, $x\in X$ closed point.
    Recall that $T_{X, x} = \hom_{k(x)}(\mathfrak{m}_{x}/\mathfrak{m}_{x}^{2}, k(x))$.
    Thus X is regular at x if and only if $\dim_{k(x)}(T_{X, x}) = \dim(\mathcal{O}_{X, x})$.
\end{remark}
\begin{lemma}
    Let $V = k^{n}$ be a vector space over field k, $y\in V$, $V^{\ast}$ the dual space.
    Take $J \subseteq k[x_{1}, \cdots, x_{n}]$ to be the maximal ideal corresponding to y.
    Define map D to be $D: k[x_{1}, \cdots, x_{n}] \longrightarrow V^{\ast}\quad P \longmapsto ((t_{1}, \cdots, t_{n} \mapsto \sum_{j} \frac{\partial P}{\partial x_{j}}(y)t_{j}))$.
    Then $D\big{|}_{J}$ induces an isomorphism $J/J^{2} \overset{\sim}{\longrightarrow} V^{\ast}$. 
    \label{Lemma 5.2}
\end{lemma}
\begin{proof}
    Assume $y = (y_{1}, \cdots, y_{n})$.
    Foa all $P\in J$, $P(y) = 0$.
    Consider Taylor Expansion of P at y is $P = P(y_{1}, \cdots, y_{n}) + \sum_{j} \frac{\partial P}{\partial x_{j}}(y)(x_{j} - y_{j}) + Q$.
    Note that $P\in J$ and $Q\in J^{2}$, get $\overline{P} = \overline{\sum_{j} \frac{\partial P}{\partial x_{j}}(y)(x_{j} - y_{j})}$ in $J/J^{2}$.
    Thus for all $P\in J^{2}$, get $\sum_{j} \frac{\partial P}{\partial x_{j}}(y)(x_{j} - y_{j})\in J^{2}$.
    It is easy to see that $\frac{\partial P}{\partial x_{j}}(y) = 0$ for all j.
    Thus $D(P)$ is zero map so that D induces an isomorphism $J/J^{2} \overset{\sim}{\longrightarrow} V^{\ast}$.
\end{proof}
\begin{theorem}[\textbf{Jacobson Criterion}]
    Let k be a field and $X = \spec(k[x_{1}, \cdots, x_{n}]/I)$ be an affine k-variety where $I = (P_{1}, \cdots, P_{r})$, $x\in X$ closed point with $k(x) = k$.
    Then X is regular at x if and only if the Jacobson matrix $J_{x} = (\frac{\partial P_{i}}{\partial x_{j}}(x))_{i,j}$ is of rank $n - \dim(X)$.
    \label{Theorem 5.3}
\end{theorem}
\begin{remark}
    More generally, for closed point $x\in X$ with residue field $k(x)$.
    If $J_{x}$ is of rank $n - \dim(X)$, then X is regular at x.
    On the other hand, only when $k(x)$ is separable over k can we get $J_{x}$ is of rank $n - \dim(X)$ if X is regular at x.
\end{remark}
\begin{proof}
    Let $f: X \longrightarrow \mathbb{A}_{k}^{n}$ be the closed immersion, $y = f(x)$.
    Set $J \subseteq k[x_{1}, \cdots, x_{n}]$ be the corresponding maximal ideal to $y\in Y$.
    Note that $k \subseteq k(y) \subseteq k(x) = k$, get $k(y) = k$.
    Thus $k[x_{1}, \cdots, x_{n}]/J \cong k$.
    Assume the images of $x_{j}$ under the natural homomorphism $k[x_{1}, \cdots, x_{n}] \longrightarrow k[x_{1}, \cdots, x_{n}]/J$ are $y_{j}$ respectively.
    Then by the Lemma \ref{Lemma 5.1}, get $J/J^{2} \cong V^{n}$, where $V = k^{n}$.
    As the maximal corresponding to x is $J/I$, $\mathfrak{m}_{x}/\mathfrak{m}_{x}^{2} \cong (J/I)/(J/I)^{2} \cong J/(J^{2} \cap I)$.
    Consider the following exact sequence
    \begin{equation*}
        0 \longrightarrow I/(J^{2} \cap I) \longrightarrow I/J^{2} \longrightarrow J/(J^{2} \cap I) \longrightarrow 0
    \end{equation*}
    Thus 
    \begin{equation*}
        \begin{split}
            dim_{k}{T_{X, x}} & = \dim_{k_{x}}(\mathfrak{m}/\mathfrak{m}^{2}) \\
             & = \dim_{k{x}}(J/J^{2}) - \dim_{k{x}}(I/(J^{2} \cap I)) \\
             & = n - rank(J_{x})
        \end{split}
    \end{equation*}
    Thus X is regular at x if and only if the Jacobson matrix $J_{x} = (\frac{\partial P_{i}}{\partial x_{j}}(x))_{i,j}$ is of rank $n - \dim(X)$.
\end{proof}
\begin{example}
    Let k be an algebraically field and $X = \spec(k[x_{1}, \cdots, x_{m}])/(P)$ be an affine k-variety where P is irreducible.
    Now $\dim(X) = n - 1$.
    And X is regular at closed point $x\in X$ if and only if at least one $\frac{\partial P}{\partial x_{j}}(x) \neq 0$.
\end{example}
\subsection{Flat Morphisms}
\begin{definition}
    Let $f: X \longrightarrow Y$ be a morphism of schemes.
    We say that f is flat at $x\in X$ if the induced homomorphism $f_{x}^{\sharp}: \mathcal{O}_{Y, f(x)} \longrightarrow \mathcal{O}_{X, x}$ is flat.
    In particular, we say that f is flat if it is flat at all points.
\end{definition}
\begin{remark}
    (1)Open immersions are flat. \\
    (2)Flatness is stable under composition and base change.
\end{remark}
\begin{proposition}
    Let $f: X \longrightarrow Y$ be a finite morphism of schemes with Y integral and locally noetherian.
    Then f is flat if and only if for all $y\in Y$, the fiber $X_{y}$ is of constant dimension.
    \label{Proposition 5.4}
\end{proposition}
\begin{proposition}
    Let $f: X \longrightarrow Y$ be a finite and surjective morphism between regular schemes.
    Then f is flat.
    \label{Proposition 5.5}
\end{proposition}
\begin{remark}
    Surjectivity of f and regularity of Y are both necessary, while regularity of X can be reduced to Cohen-Macaulay condition.
\end{remark}
\begin{theorem}[\textbf{\emph{Going-down Theorem for Flat Condition}}]
    Let $\varphi: A \longrightarrow B$ be a flat ring homomorphism.
    If $\mathfrak{p}_{1},\mathfrak{p}_{2} \subseteq A$ are prime ideals such that $\mathfrak{p}_{1} \supseteq \mathfrak{p}_{2}$ and $\mathfrak{P}_{1} \subseteq B$ is prime ideal over $\mathfrak{p}_{1}$,
    then there exists $\mathfrak{P}_{2} \subseteq B$ prime ideal over $\mathfrak{p}_{2}$.
    \label{Theorem 5.4}
\end{theorem}
\begin{theorem}
    Let $f: X \longrightarrow Y$ be flat morphism of finite type between noetherian schemes.
    Then f is open.
    \label{Theorem 5.5}
\end{theorem}
\begin{reason}
    Comes from Chavalley's Theorem and Going-down Theorem.
    Proof of Chavalley's Theorem can be seen in the Midterm24.pdf.
\end{reason}
\begin{corollary}
    Let $f: X \longrightarrow Y$ be flat morphism of finite type between noetherian schemes.
    If Y is irreducible, then all nonempty open subset $U \subseteq X$ dominates Y i.e. $f(U)$ is dense in Y.
    Moreover, all irreducible components of X dominate Y.
    \label{Corollary 5.3}
\end{corollary}
\begin{lemma}
    Let A be a reduced noetherian ring, $a\in A$ zero-divisor.
    Then a is in a minimal prime ideal of A.
    \label{Lemma 5.3}
\end{lemma}
\begin{proof}
    Since $a$ is a zero-divisor, the natural homomorphism $A \longrightarrow A_{a}$ is not injective with nontrivial kernel $I$.
    While $A$ is reduced, get $V(I) \neq \spec A$.
    Note that $D(a) \subseteq V(I)$, $\spec A$ contains an irreducible component whose generic point $\eta$ is not in $D(a)$.
    While $\eta$ corresponds to a minimal prime ideal $\mathfrak{p}$ of $A$, get $a\in \mathfrak{p}$.
\end{proof}
\begin{theorem}
    Let $f: X \longrightarrow Y$ be a morphism of schemes.
    Assume that $X$ is noetherian and reduced and $Y$ is noetherian, normal and one-dimensional.
    Suppose that all irreducible components of X dominate $Y$.
    Then $f$ is flat.
    \label{Theorem 5.6}
\end{theorem}
\begin{proof}
    For generic point $\eta_{Y}$, $\mathcal{O}_{Y, \eta_{Y}}$ is a field, so there is nothing to prove.
    For other $x\in X$ which not maps to $\eta_{Y}$, since Y is one-dimensional, $y = f(x)$ is a closed point.
    Then $\mathcal{O}_{Y, y}$ is a discrete valuation ring.
    Take uniformizer $\omega$ of $\mathcal{O}_{Y, y}$ and assume $f_{x}^{\sharp}(\omega) = t$.
    If t is in a minimal prime ideal $\mathfrak{p}$ of $\mathcal{O}_{X, x}$, 
    consider the generic point $\eta$ of some irreducible component containing x corresponds to $\mathfrak{p}$.
    As irreducible components of X dominate Y, $f(\eta) = \eta_{Y}$.
    Then there is commutative diagram
    \begin{equation*}
        \begin{tikzcd}
            \mathcal{O}_{Y, y} \arrow[r, "f_{x}^{\sharp}"] \arrow[d] &
            \mathcal{O}_{X, x} \arrow[d] \\
            \fraction(\mathcal{O}_{Y, y}) \arrow[r, "f_{\eta}^{\sharp}"] &
            \mathcal{O}_{X, \eta}
        \end{tikzcd}
    \end{equation*}
    while $\omega$ is not zero in $\fraction(\mathcal{O}_{Y, y})$, its image in $\mathcal{O}_{X, \eta}$ should be invertible, contradicting to $t\in \mathfrak{p}$.
    Thus t is not in some minimal prime ideal of A.
    By Lemma \ref{Lemma 5.3}, get t is not zero-divisor.
    Thus $\mathcal{O}_{X, x}$ is torsion-free as $\mathcal{O}_{Y, y}$-module.
    While $\mathcal{O}_{Y, y}$ is principal ideal domain, get $\mathcal{O}_{X, x}$ free so that is flat.
\end{proof}
\begin{theorem}
    Let A be a noetherian ring, $a\in A$ not invertible.
    Then each minimal prime ideal of A containing a has height $\le 1$ and "$=$" holds if a is not a zero-divisor.
    \label{Theorem 5.7}
\end{theorem}
\begin{remark}
    Geometrically, for $X = \spec A$ affine scheme and $a\in A$ not invertible, each irreducible component of $V(a)$ has codimension $\le 1$ and "$=$" holds if a is not a zeo-divisor.
\end{remark}
\begin{corollary}
    Let A be a noetherian local ring, $\mathfrak{m} \subseteq A$ maximal ideal, $a\in \mathfrak{m}$.
    Then $\dim(A/(a)) \ge \dim(A) - 1$ and "$=$" holds if a is not a zero-divisor.
    \label{Corollary 5.4}
\end{corollary}
\begin{theorem}
    Let $f: X \longrightarrow Y$ be a flat morphism between locally noetherian schemes, $x\in X$, $y = f(x)$.
    Then $\dim(\mathcal{O}_{X_{y}, x}) = \dim(\mathcal{O}_{X, x}) - \dim(\mathcal{O}_{Y, y})$.
    \label{Theorem 5.8}
\end{theorem}
\begin{proof}
    Can assume $Y = \spec A$ affine.
    Consider base change $\spec(\mathcal{O}_{Y, y}) \longrightarrow Y$.
    Get $X \times_{Y} \mathcal{O}_{Y, y} \longrightarrow \mathcal{O}_{Y, y}$ is flat.
    Thus by replacing Y by $\mathcal{O}_{Y, y}$, can assume that A is noetherian local ring and y corresponds to the maximal ideal $\mathfrak{m}$ of A.
    \par
    Induct on dimension of A.
    If $\dim(A) = 0$, then A is field and Y is one-point set.
    Now $X_{y} \cong X$ so that $\dim(\mathcal{O}_{X_{y}, x}) = \dim(\mathcal{O}_{X, x}) - \dim(\mathcal{O}_{Y, y})$.
    \par
    Suppose that statement is ok for $\dim(A) = d$.
    When $\dim(A) = d + 1$, consider base change $Y_{red} \longrightarrow Y$.
    Can assume that A is reduced.
    Let $t\in A$ not invertible and not a zero-divisor, which means $t\in \mathfrak{m}$ and t is not in some minimal prime ideal. 
    t exists because $\cup_{\mathfrak{p} \text{ minimal prime ideal}} \mathfrak{p} \subsetneq \mathfrak{m}$.
    Now $A \longrightarrow \mathcal{O}_{X, x}$ is flat.
    Denote $\mathcal{O}_{X, x}$ by B.
    Since $A \overset{\cdot t}{\longrightarrow} A$ is injective, $B \overset{\cdot t}{\longrightarrow} B$ is also injective.
    Thus the image of t in B is not zeo-divisor.
    By Corollary \ref{Corollary 5.4}, $\dim(A/tA) = \dim(A) - 1$ and $\dim(B/tB) = \dim(B) - 1$.
    \par
    Set $Y' = \spec(A/tA) \subseteq Y$ and $X' = X \times_{Y} Y'$, then $X' \longrightarrow Y'$ is flat.
    While $t\in \mathfrak{m}$, get $y\in Y'$ and $X'_{y} = X_{y}$
    By induction, $\dim(\mathcal{O}_{X_{y}', x}) = \mathcal{O}_{X', x} - \dim(\mathcal{O}_{Y', y})$ so that $\dim(\mathcal{O}_{X_{y}, x}) = \dim(\mathcal{O}_{X, x}) - \dim(\mathcal{O}_{Y, y})$.
\end{proof}
\begin{remark}[\textbf{\emph{Partial converse of Theorem "Miracle Flatness"}}]
    Let $f: X \longrightarrow Y$ be a morphism between regular schemes.
    If for all $x\in X$ and $y = f(x)$, we have that $\dim(\mathcal{O}_{X_{y}, x}) = \dim(\mathcal{O}_{X, x}) - \dim(\mathcal{O}_{Y, y})$, then f is flat.
\end{remark}
\begin{lemma}
    Let $f: X \longrightarrow Y$ be a morphism of schemes, $x\in X$, $y = f(x)$.
    Then the local ring of $X_{y}$ at x is $\mathcal{O}_{X_{y}, x} \cong \mathcal{O}_{X, x}/\mathfrak{m}_{y}\mathcal{O}_{X, x}$.
    \label{Lemma 5.4}
\end{lemma}
\begin{proof}
    Since we are discussing about local rings, we can assume that $X = \spec(B)$ and $Y = \spec A$.
    Then f corresponds a ring homomorphism $\varphi: A \longrightarrow B$.
    Assume that x corresponds to $\mathfrak{q}\in \spec(B)$ and y corresponds to $\mathfrak{p}\in \spec A$.
    Then 
    \begin{equation*}
        \begin{split}
            \mathcal{O}_{X_{y}, x} & = (B \otimes_{A} k(\mathfrak{p}))_{\mathfrak{q} \otimes_{A} k(\mathfrak{p})} \\
            & \cong B_{\mathfrak{q}} \otimes_{A} k(\mathfrak{p}) \\
            & \cong B_{\mathfrak{q}} \otimes_{A} A_{\mathfrak{p}}/\mathfrak{p}A_{\mathfrak{p}} \\
            & \cong B_{\mathfrak{q}} \otimes_{A} A_{\mathfrak{p}} \otimes_{A} A/\mathfrak{p} \\
            & \cong B_{\mathfrak{q}} \otimes_{A} A/\mathfrak{p} \\
            & \cong B_{\mathfrak{q}}/\mathfrak{p}B_{\mathfrak{q}} \\
            & \cong \mathcal{O}_{X, x}/\mathfrak{m}_{y}\mathcal{O}_{X, x}
        \end{split}
    \end{equation*}
\end{proof}
\begin{theorem}
    Let $f: X \longrightarrow Y$ be a flat k-morphism between schemes of finite type over field k.
    Suppose Y is irreducible and X is pure.
    Then for all $y\in Y$, the fiber $X_{y}$ is either empty or pure with $\dim(X_{y}) = \dim(X) - \dim(Y)$.
    \label{Theorem 5.9}
\end{theorem}
\begin{proof}
    For $y\in f(X)$, consider $Z \subseteq X_{y}$ irreducible component.
    As $X_{y}$ is of finite type over $k(y)$, by Corollary \ref{Corollary 3.2}, we can choose $x\in Z$ closed point not in any other irreducible component of $X_{y}$.
    Assume $x\in X_{0} \subseteq X$ with $X_{0}$ irreducible component of X.
    Then since X is pure,
    \begin{equation*}
        \begin{split}
            \dim(\mathcal{O}_{X, x}) & = \dim(\mathcal{O}_{X_{0}, x}) \\
            & = \dim(X_{0}) - \dim(\overline{\{x\}}) \\
            & = \dim(X) - \dim(\overline{\{x\}})
        \end{split}
    \end{equation*}
    We also have that $\dim(\mathcal{O}_{Z, x}) = \dim(Z)$ and $\dim(\mathcal{O}_{Y, y}) = \dim(Y) - \dim(\overline{\{y\}})$.
    By Theorem \ref{Theorem 5.8}, $\dim(\mathcal{O}_{Z, x}) = \dim(\mathcal{O}_{X_{y}, x}) = \dim(\mathcal{O}_{X, x}) - \dim(\mathcal{O_{Y, y}})$.
    It remains to show that $\dim(\overline{\{x\}}) = \dim(\overline{\{y\}})$.
    \par
    Note that if we take open neighbourhood $V = \spec A$ of y and y corresponds to $\mathfrak{p}$, then $\dim(\overline{\{y\}}) = \dim(A/\mathfrak{p}) = tr\dim(k(y)/k)$.
    Similarly, we also have $\dim(\overline{\{x\}}) = tr\dim(k(x)/k)$.
    While by Lemma \ref{Lemma 5.4}, $\mathcal{O}_{X_{y}, x} \cong \mathcal{O}_{X, x}/\mathfrak{m}_{y}\mathcal{O}_{X, x}$, get the residue field of x in $X_{y}$ is also $k(x)$.
    Take affine open neighbourhood $U = \spec(B)$ of x in $X_{y}$.
    Then x corresponds to a maximal ideal $\mathfrak{m}$ of B and $k(x) = B/\mathfrak{m}$.
    Thus $B/\mathfrak{m}$ is a field of finite type over $k(y)$.
    By Zariski's Lemma, $B/\mathfrak{m}$ is finite over $k(y)$ so that it is algebraic over $k(y)$, inducing $tr\dim(k(x)/k) = tr\dim(k(y)/k)$.
\end{proof}
\subsection{Etale Morphisms and Smooth Schemes and Morphisms}
\begin{definition}
    Let $f: X \longrightarrow Y$ be a morphism of finite type between locally noetherian schemes, $x\in X$, $y = f(x)$.
    We say that f is unramified at x if \\
    (1)$\mathfrak{m}_{y}\mathcal{O}_{X, x} = \mathfrak{m}_{x}$. \\
    (2)The field extension $k(x)/k(y)$ is finite and separable.
    \par
    In particular, we say that f is unramified if it is unramified at all points.
\end{definition}
\begin{remark}
    Recall that $\mathcal{O}_{X_{y}, x} \cong \mathcal{O}_{X, x}/\mathfrak{m}_{y}\mathcal{O}_{X, x}$.
    Thus if f is unramified at x, then $\mathcal{O}_{X_{y}, x}$ is a field.
\end{remark}
\begin{proposition}
    Let $f: X \longrightarrow Y$ be a morphism of finite type between locally noetherian schemes.
    Then f is unramified if and only if for all $y\in Y$, $X_{y}$ is finite over $k(y)$ and reduced and for all $x\in X_{y}$, the extension $k(x)/k(y)$ is finite and separable.
    \label{Proposition 5.6}
\end{proposition}
\begin{definition}
    Let $f: X \longrightarrow Y$ be a morphism of finite type between locally noetherian schemes.
    We say that f is etale at x if f is both flat and unramified at x.
    In particular, we say that f is etale if it is etale at all points.
\end{definition}
\begin{example}
    (1)$\spec K \longrightarrow \spec k$ is unramified/etale if and only if the field extension $K/k$ is finite and separable. \\
    (2)Close immersions are unramified. \\
    (3)Open immersions are etale.
\end{example}
\begin{proposition}
    unramified and etale are both stable under composition and base change.
    \label{Proposition 5.7}
\end{proposition}
\begin{proposition}
    Let $f: X \longrightarrow Y$ be an etale morphism between locally noetherian schemes, $x\in X$, $y = f(x)$.
    Then \\
    (1)$\dim(\mathcal{O}_{X, x}) = \dim(\mathcal{O}_{Y, y})$. \\
    (2)The tangent map $T_{X, x} \longrightarrow T_{Y, y} \otimes_{k(y)} k(x)$ is an isomorphism.
    \label{Proposition 5.8}
\end{proposition}
\begin{proof}
    (1): By Theorem\ref{Theorem 5.8}, $\dim(\mathcal{O}_{X_{y}, x}) = \dim(\mathcal{O}_{X, x}) - \dim(\mathcal{O}_{Y, y})$.
    While f is unramified, get $\mathcal{O}_{X_{y}, x}$ is a field so that $\dim(\mathcal{O}_{X_{y}, x}) = 0$.
    Thus $\dim(\mathcal{O}_{X, x}) = \dim(\mathcal{O}_{Y, y})$.
    \par
    (2): Note that $T_{Y, y} \otimes_{k(y)} k(x) = \hom_{k(x)}(\mathfrak{m}_{y}/\mathfrak{m}_{y}^{2} \otimes_{k(y)} k(x), k(x))$.
    It suffices to prove $\mathfrak{m}_{x}/\mathfrak{m}_{x}^{2} \cong \mathfrak{m}_{y}/\mathfrak{m}_{y}^{2} \otimes_{k(y)} k(x)$.
    Firstly, we have that
    \begin{equation*}
        \begin{split}
            \mathfrak{m}_{y}/\mathfrak{m}_{y}^{2} \otimes_{k(y)} k(x) & \cong (\mathfrak{m}_{y} \otimes_{\mathcal{O_{Y}, y}} k(y)) \otimes_{k(y)} k(x) \\
            & \cong \mathfrak{m}_{y} \otimes_{\mathcal{O}_{Y, y}} k(x)
        \end{split}
    \end{equation*}
    and 
    \begin{equation*}
        \mathfrak{m}_{x}/\mathfrak{m}_{x}^{2} \cong \mathfrak{m}_{x} \otimes_{\mathcal{O}_{X, x}} k(x) 
    \end{equation*}
    Since f is unramified, $\mathfrak{m}_{x} = \mathfrak{m}_{y}\mathcal{O}_{X, x}$ so that $\mathfrak{m}_{y} \otimes_{\mathcal{O}_{Y, y}} \mathcal{O}_{X, x} \longrightarrow \mathfrak{m}_{x}$ is surjective.
    Since f is flat, $\mathfrak{O}_{X, x}$ is flat over $\mathfrak{O}_{Y, y}$ so that $\mathfrak{m}_{y} \otimes_{\mathcal{O}_{Y, y}} \mathcal{O}_{X, x} \longrightarrow \mathfrak{m}_{x}$ is injective.
    Thus $\mathbf{m}_{y} \otimes_{\mathcal{O}_{Y, y}} \mathcal{O}_{X, x} = \mathfrak{m}_{x}$ so that $\mathfrak{m}_{x}/\mathfrak{m}_{x}^{2} \cong \mathfrak{m}_{y}/\mathfrak{m}_{y}^{2} \otimes_{k(y)} k(x)$.
\end{proof}
\begin{proposition}
    Let X be a k-scheme of finite type. \\
    (1)Let $K/k$ be a field extension, $x\in X$ closed point with residue field k.
    Assume $x_{K}\in X_{K} = X \times_{k} K$ is a closed point over x with residue field K.
    Then X is regular at x if and only if $X_{K}$ is regular at $x_{K}$. \\
    (2)Suppose k is perfect field with algebraic closure $\overline{k}$.
    Then X is regular if and only if $X_{\overline{k}}$ is regular.
    \label{Proposition 5.9}
\end{proposition}
\begin{proof}
    (1): Can assume that $X = \spec A$ affine.
    As X is of finite type over k, A is of the form $k[x_{1}, \cdots, x_{r}]/I$.
    Thus $X_{k} = \spec(A \otimes_{k} K) \cong \spec(K[x_{1}, \cdots, x_{r}])/I$ so that the Jacobson matrixes at $x\in X$ and $x_{K}\in X_{K}$ are same.
    Note that $X_{K} \longrightarrow X$ is given by $\spec K \longrightarrow \spec k$ under base change $X \longrightarrow \spec k$, $X_{K} \longrightarrow X$ is flat.
    Assume x corresponds to maximal ideal $\mathfrak{m}$ of A.
    Then $\mathfrak{m}_{x}$ is $\mathfrak{m}A_{\mathfrak{m}}$ and $\mathfrak{m}_{x_{K}} = \mathfrak{m}A_{\mathfrak{m}} \otimes_{k} K$.
    Thus $\mathfrak{m}_{x_{K}} = \mathfrak{m}_{x}\mathcal{O}_{X_{K}, x_{K}}$, inducing that the local ring of fiber ${X_{K}}_{x}$ is a field.
    By Theorem \ref{Theorem 5.8}, $\dim(\mathcal{O}_{X, x}) = \dim(\mathcal{O}_{X_{k}, x_{K}})$.
    Thus by Jacobson Criterion, get X is regular at x if and only if $X_{K}$ is regular at $x_{K}$.
    \par
    (2): For $x\in X$ closed point, assume the residue field of x is $k(x)$.
    Consider $X' = X \times_{k} k(x)$.
    Take affine open neighbourhood $U = \spec A$ of x in X.
    Then $\spec(A \otimes_{k} k(x))$ is an affine open subset of $X'$.
    While A is of finite type over k, by Noetherian Normalization Theorem, there exists $X_{1}, \cdots, x_{r}\in A$ algebraically independent over k such that A is integral over $k[x_{1}, \cdots, x_{r}]$.
    It is easy to show that $(x_{1}, \cdots, x_{r})$ is a maximal ideal of A so that $(x_{1}, \cdots, x_{r}) \otimes_{k} k(x)$ is a maximal ideal of $A \otimes_{k} k(x)$.
    Thus we can take the closed point $x'$ corresponding to $(x_{1}, \cdots, x_{r}) \otimes_{k} k(x)$ with residue field equal to $k(x)$.
    By (1), it suffices to prove that X is regular at x if and only if $X'$ is regular at $x'$.
    \par
    Since $k(x)/k$ is finite and separable, $\spec(k(x)) \longrightarrow \spec k$ is etale.
    As etale is stable under base change, get $X' \longrightarrow X$ is etale.
    By Proposition \ref{Proposition 5.8}, we have that $\dim_{k(x)}(T_{X', x'}) = \dim_{k(x)}(T_{X, x} \otimes_{k(x)} k(x)) = \dim_{k(x)}(T_{X, x})$ and $\dim(\mathcal{O}_{X', x'}) = \dim(\mathcal{O}_{X, x})$.
    Thus X is regular at x if and only if X' is regular at x' as Remark 5.3.
\end{proof}
\begin{definition}
    Let X be a scheme of finite type over field k, $\overline{k}$ algebraic closure of k.
    We say that X is smooth (or say nonsingular) over k if $X_{\overline{k}} = X \times_{k} \overline{k}$ is regular.
\end{definition}
\begin{example}
    Let k be a field.
    Then $\mathbb{A}_{k}^{n}$ and $\mathbb{P}_{k}^{n}$ are smooth over k.
\end{example}
\begin{remark}
    By Proposition \ref{Proposition 5.9} (2), it is easy to see that if k is perfect, then smoothness over k is equivalent to regularity over k.
    In addition, X is smooth over k if and only if for all $K/k$ field extension, $X_{k} = X \times_{k} K$ is regular.
\end{remark}
\begin{definition}
    Let $f: X \longrightarrow Y$ be a morphism of finite type between locally noetherian schemes.
    We say that f is smooth if it is flat and for all $y\in Y$, the fiber $X_{y}$ is smooth over $k(y)$.
    In addition, we say that f is smooth of relative dimension n if moreover all nonempty fibers are pure of dimension n.
\end{definition}
\begin{remark}
    Somewhat etale is equivalent to smooth of relative dimension 0.
\end{remark}
\begin{theorem}
    Let $f: X \longrightarrow Y$ be a smooth morphism between locally noetherian schemes.
    If Y is regular, then X is regular.
    \label{Theorem 5.10}
\end{theorem}
\begin{remark}
    If X is regular, then we also have Y is regular even when f is just flat.
\end{remark}
\begin{proof}
    For $x\in X$ and $y = f(x)$, set $m = \dim(\mathcal{O}_{X, x})$ and $n = \dim(\mathcal{O}_{Y, y})$.
    Then by Theorem \ref{Theorem 5.8}, $\mathcal{O}_{X_{y}, x} = m - n$.
    Since f is smooth, $X_{y}$ is smooth over $k(y)$ so that $X_{y}$ is regular at $x\in X_{y}$.
    Thus the maximal ideal of $\mathcal{O}_{X_{y}, x}$ can be generated by $m - n$ elements $b_{n + 1}, \cdots, b_{m}$.
    Since $\mathcal{O}_{X_{y}, x} = \mathcal{O}_{X, x}/\mathfrak{m}_{y}\mathcal{O}_{X, x}$, can lift $b_{i}$ to $a_{i}\in \mathfrak{m}_{x}$ and complete them with $a_{1}, \cdots, a_{n}\in \mathfrak{m}_{y}$ generators of $\mathfrak{m}_{y}$.
    Get m generators $a_{1}, \cdots, a_{m}\in \mathfrak{m}_{x}$.
    Thus $\mathcal{O}_{X, x}$ is regular. 
\end{proof}
\begin{corollary}
    Smooth morphisms are stable under composition and base change.
    \label{Corollary 5.5}
\end{corollary}
\begin{reason}
    Theorem \ref{Theorem 5.10} and remark.
\end{reason}

\section{Sheaves of Modules}
\label{section:Sheaves of Modules}

\subsection{Modules over Ringed Space}
\label{subsection:Modules over Ringed Space}
\begin{definition}
    Let X be a scheme or $(X, \mathcal{O}_{X})$ be a ringed space.
    A sheaf of $\mathcal{O}_{X}$-modules is a sheaf of abelian groups $\mathcal{F}$ on X such that for all $U \subseteq X$ open, 
    $\mathcal{F}(U)$ is a module over $\mathcal{O}_{X}(U)$ which is compatible with restriction maps.
\end{definition}
\begin{definition}
    Let X be a scheme or $(X, \mathcal{O}_{X})$ be a ringed space, $\mathcal{F},\mathcal{G}$ two $\mathcal{O}_{X}$-modules.
    A morphism $\varphi$ from $\mathcal{F}$ to $\mathcal{G}$ is a morphism of sheaves compatible with the module structure i.e. for all $U \subseteq X$ open, 
    $\varphi(U): \mathcal{F}(U) \longrightarrow \mathcal{G}(U)$ is a homomorphism of $\mathcal{O}_{X}(U)$-modules.
\end{definition}
\begin{remark}
    Kernel, image and cokernel of a morphism of $\mathcal{O}_{X}$-modules are still $\mathcal{O}_{X}$-modules.
    Can also define submodule, quotient module and direct sum.
\end{remark}
\begin{definition}
    Let X be a scheme or $(X, \mathcal{O}_{X})$ be a ringed space.
    An sheaf ideal on X is an $\mathcal{O}_{X}$-submodule of $\mathcal{O}_{X}$.
\end{definition}
\begin{definition}[\textbf{\emph{Tensor Products}}]
    Let X be a scheme or $(X, \mathcal{O}_{X})$ be a ringed space, $\mathcal{F},\mathcal{G}$ two $\mathcal{O}_{X}$-modules.
    Define tensor product $\mathcal{F} \otimes_{\mathcal{O}_{X}} \mathcal{G}$ to be the sheafification of $U \longmapsto \mathcal{F}(U) \otimes_{\mathcal{O}_{X}(U)} \mathcal{G}(U)$.
\end{definition}
\begin{remark}
    The stalk of $\mathcal{F} \otimes_{\mathcal{O}_{X}} \mathcal{G}$ at $x\in X$ is just $\mathcal{F}_{x} \otimes_{\mathcal{O}_{X, x}} \mathcal{G}_{x}$.
\end{remark}
\begin{definition}[\textbf{\emph{Free $\mathcal{O}_{X}$-modules}}]
    Let X be a scheme or $(X, \mathcal{O}_{X})$ be a ringed space, $\mathcal{F}$ an $\mathcal{O}_{X}$-module.
    We say that $\mathcal{F}$ is free if $\mathcal{F}$ is isomorphic to a direct sum of $\mathcal{O}_{X}$.
    In addition, we say that $\mathcal{F}$ is free of rank r if $\mathcal{F} \cong \mathcal{O}_{X}^{r}$.
\end{definition}
\begin{definition}[\textbf{\emph{Locally Free $\mathcal{O}_{X}$-modules}}]
    Let X be a scheme or $(X, \mathcal{O}_{X})$ be a ringed space, $\mathcal{F}$ an $\mathcal{O}_{X}$-module.
    We say that $\mathcal{F}$ is locally free if X can be covered by open subsets $\{U_{i}\}_{i\in I}$ such that $\mathcal{F}\big{|}_{U_{i}}$ is free as $\mathcal{O}_{U_{i}}$-module.
    In addition, we say that $\mathcal{F}$ is locally free of rank r if X can be covered by open subsets $\{U_{i}\}_{i\in I}$ such that $\mathcal{F}\big{|}_{U_{i}} \cong \mathcal{O}_{U_{i}}^{r}$ as $\mathcal{O}_{U_{i}}$-module.
\end{definition}
\begin{definition}[\textbf{\emph{Invertible $\mathcal{O}_{X}$-modules}}]
    Let X be a scheme or $(X, \mathcal{O}_{X})$ be a ringed space, $\mathcal{F}$ an $\mathcal{O}_{X}$-module.
    We say that $\mathcal{F}$ is invertible if it is locally free of rank 1.
\end{definition}
\begin{remark}
    If take $\mathcal{G} = \sheafhom_{\mathcal{O}_{X}}(\mathcal{F}, \mathcal{O}_{X})$, then $\mathcal{F} \otimes_{\mathcal{O}_{X}} \mathcal{G} \cong \mathcal{O}_{X}$.
\end{remark}
\begin{lemma}
    Let $f: X \longrightarrow Y$ be a morphism of schemes or ringed spaces, $\mathcal{F}$ an $\mathcal{O}_{X}$-module, $\mathcal{G}$ an $\mathcal{O}_{Y}$-module.
    Then the direct image $f_{\ast}\mathcal{F}$ of $\mathcal{F}$ is an $\mathcal{O}_{Y}$-module induced by $f^{\sharp}$.
    On the other hand, the inverse image $f^{\ast}\mathcal{G} = f^{-1}\mathcal{G} \otimes_{f^{-1}\mathcal{O}_{Y}} \mathcal{O}_{X}$ of $\mathcal{G}$ is an $\mathcal{O}_{X}$-module where $f^{-1}\mathcal{O}_{Y} \longrightarrow \mathcal{O}_{X}$ is induced by $f^{\sharp}$.
    \label{Lemma 6.1}
\end{lemma}
\subsection{Quasi-coherent and Coherent Sheaves}
\begin{definition}
    Let A be a ring, $X = \spec A$, $M\in Mod_{A}$.
    Define an $\mathcal{O}_{X}$-module $\widetilde{M}$ on X by the same process as $\mathcal{O}_{X}$ replacing A by M.
\end{definition}
\begin{remark}
    The definition gives a fully faithful functor from $Mod_{A}$ to category of $\mathcal{O}_{Spec(A)}$-modules $M \longmapsto \widetilde{M}$.
\end{remark}
\begin{proposition}
    Let A be a ring, $X = \spec A$, $M\in Mod_{A}$.
    Then \\
    (1)$\widetilde{M}(X) = M$, \\
    (2)$\widetilde{M}(D(f)) = M_{f} = M \otimes_{A} A_{f}$ \\
    (3)$\widetilde{M}_{\mathfrak{p}} = M_{\mathfrak{p}} = M \otimes_{A} A_{\mathfrak{p}}$.
    \label{Proposition 6.1}
\end{proposition}
\begin{proposition}
    Let A be a ring, $X = \spec A$.
    Then \\
    (1)Let $M_{i}$ be $A$-modules. 
    Then $\widetilde{\oplus_{i} M_{i}} \cong \oplus_{i} \widetilde{M_{i}}$. \\
    (2)Let M,N be $A$-modules.
    Then $\widetilde{M \otimes_{A} N} \cong \widetilde{M} \otimes_{\mathcal{O}_{X}} \widetilde{N}$. \\
    (3)Let $L,M,N$ be $A$-modules.
    Then $L \longrightarrow M \longrightarrow N$ is exact if and only if $\widetilde{L} \longrightarrow \widetilde{M} \longrightarrow \widetilde{N}$ is exact. \\
    (4)Let $f: \spec(B) \longrightarrow \spec A$ be a morphism of affine schemes corresponding to ring homomorphism $\varphi: A \longrightarrow B$, $M\in Mod_{A}$, $N\in Mod_{B}$.
    Then $f_{\ast}\widetilde{N} \cong \widetilde{N}$, where $N$ is viewed as $A$-module, and $f^{\ast}\widetilde{M} \cong \widetilde{M \otimes_{A} B}$.
    \label{Proposition 6.2}
\end{proposition}
\begin{definition}[\textbf{\emph{Quasi-coherent Sheaves}}]
    Let X be a scheme, $\mathcal{F}$ a sheaf of abelian groups on X.
    We say that $\mathcal{F}$ is quasi-coherent if for all affine open subset $U = \spec A \subseteq X$,
    $\mathcal{F}\big{|}_{U} \cong \widetilde{M}$ for some $A$-module M.
\end{definition}
\begin{definition}[\textbf{\emph{Coherent Sheaves}}]
    Let X be a noetherian scheme, $\mathcal{F}$ a sheaf of abelian groups on X.
    We say that $\mathcal{F}$ is coherent if for all affine open subset $U = \spec A \subseteq X$,
    $\mathcal{F}\big{|}_{U} \cong \widetilde{M}$ for some finite $A$-module M.
\end{definition}
\begin{theorem}
    Let X be a scheme, $\mathcal{F}$ an $\mathcal{O}_{X}$-module.
    Suppose X admits an affine open subset covering $\{U_{i} = \spec(A_{i})\}_{i}$ such that for all i, $\mathcal{F}\big{|}_{U_{i}} \cong \widetilde{M_{i}}$ for some $A_{i}$-module $M_{i}$.
    Then $\mathcal{F}$ is quasi-coherent.
    If moreover X is noetherian and $M_{i}$ are finite over $A_{i}$, then $\mathcal{F}$ is coherent.
    \label{Theorem 6.1} 
\end{theorem}
\begin{proof}
    Let $U = \spec A \subseteq X$ be an affine open subset.
    Cover each $U \cap U_{i}$ with $\{U_{ik}\}$ such that $U_{ik} \subseteq U_{i}$ of the form $D(g_{ik})$.
    Then $\mathcal{F}\big{|}_{U_{ik}} \cong \widetilde{{M_{i}}_{g_{ik}}}$.
    Can assume $X = U = \spec A$ and finitely many affine open subset $U_{i}$ such that $\mathcal{F}\big{|}_{U_{i}} \cong \widetilde{M_{i}} = \widetilde{\mathcal{F}(U_{i})}$.
    \par
    To prove $\mathcal{F} \cong \widetilde{\mathcal{F}(X)}$, it suffices to show that for all $f\in A$, $\mathcal{F}(X)_{f} = \mathcal{F}(X) \otimes_{A} A_{f} \longrightarrow \mathcal{F}(D(f))$ is isomorphic.
    Set $V_{i} = U_{i} \cap D(f) = D(f\big{|}_{U_{i}})$.
    As tensor product commutes with finite direct sum, there is commutative diagram with exact rows,
    \begin{equation*}
        \begin{tikzcd}
            0 \arrow[r] &
            \mathcal{F}(X)_{f} \arrow[r] \arrow[d] &
            \prod_{i} \mathcal{F}(U_{i})_{f} \arrow[r] \arrow[d] &
            \prod_{i,j} \mathcal{F}(U_{i} \cap U_{j})_{f} \arrow[d] \\
            0 \arrow[r] &
            \mathcal{F}(D(f)) \arrow[r] &
            \prod_{i} \mathcal{F}(V_{i}) \arrow[r] &
            \prod_{i,j} \mathcal{F}(V_{i} \cap V_{j})
        \end{tikzcd}
    \end{equation*}
    Similar to proof of Lemma \ref{Lemma 2.6}, get $\mathcal{F}$ is quasi-coherent.
    \par
    Further if A is noetherian, $\mathcal{F} = \widetilde{M}$.
    Can cover X by finitely many $\{D(f_{i})\}$ such that $\mathcal{F}(D(f_{i})) = M_{f_{i}}$ is finite over $A_{f_{i}}$
    Assume $M_{f_{i}} = \sum_{j} \frac{m_{ij}}{1}A_{f_{i}}$.
    Take $N = \sum_{i,j} m_{i,j}A$.
    Then N is a finitely generated A-submodule of M and for all i, $N_{f_{i}} = M_{f_{i}}$.
    Thus there is a natural morphism $\widetilde{N} \longrightarrow \widetilde{M}$ and $\widetilde{N}\big{|}_{D(f_{i})} \overset{\sim}{\longrightarrow} \widetilde{M}\big{|}_{D(f_{i})}$ is isomorphic for all i.
    As $d(f)_{i}$ cover X, get $\widetilde{N} \longrightarrow \widetilde{M}$ is isomorphic so that $N = M$ and M is finitely generated.
    Thus $\mathcal{F}$ is coherent.
\end{proof}
\begin{proposition}
    Let X be a scheme.
    Then \\
    (1)Kernel, image and cokernel of morphism of quasi-coherent sheaves are still quasi-coherent. \\
    (2)Direct sum of quasi-coherent sheaves is quasi-coherent. \\
    (3)Tensor product of two quasi-coherent sheaves is quasi-coherent.
    \label{Proposition 6.3}
\end{proposition}
\begin{remark}
    If X is moreover noetherian, then (1) and (3) still hold for coherent and (2) is also ok if the direct sum is finite.
\end{remark}
\begin{proposition}
    Let X be an affine scheme.
    Then for all exact sequence of quasi-coherent sheaves 
    \begin{equation*}
        0 \longrightarrow \mathcal{F} \longrightarrow \mathcal{G} \longrightarrow \mathcal{H} \longrightarrow 0
    \end{equation*}
    the sequence of global sections 
    \begin{equation*}
        0 \longrightarrow \mathcal{F}(X) \longrightarrow \mathcal{G}(X) \longrightarrow \mathcal{H}(X) \longrightarrow 0
    \end{equation*}
    is exact.
    \label{Proposition 6.4}
\end{proposition}
\begin{proposition}
    Let X be a scheme, $\mathcal{F}$ an $\mathcal{O}_{X}$-module.
    Then $\mathcal{F}$ is quasi-coherent if and only if for all $x\in X$, 
    there exists open neighbourhood $U\ni x$ such that there is an exact sequence
    \begin{equation*}
        \mathcal{O}_{U}^{\oplus J} \longrightarrow \mathcal{O}_{U}^{\oplus I} \longrightarrow \mathcal{F}\big{|}_{U} \longrightarrow 0
    \end{equation*}
    If moreover X is noetherian,
    then $\mathcal{F}$ is coherent if and only if for all $x\in X$, 
    there exists open neighbourhood $U\ni x$ such that there is an exact sequence with I,J finite
    \begin{equation*}
        \mathcal{O}_{U}^{\oplus J} \longrightarrow \mathcal{O}_{U}^{\oplus I} \longrightarrow \mathcal{F}\big{|}_{U} \longrightarrow 0
    \end{equation*}
    \label{Proposition 6.5}
\end{proposition}
\begin{proof}
    "$\Rightarrow$": For $\mathcal{F}$ quasi-coherent and all $x\in X$, 
    there is affine open neighbourhood $U = \spec A\ni x$ such that $\mathcal{F}\big{|}_{U} \cong \widetilde{M}$ for some $A$-module M.
    Note that there is an exact sequence for some index sets I,J
    \begin{equation*}
        A^{\oplus J} \longrightarrow A^{\oplus I} \longrightarrow M \longrightarrow 0
    \end{equation*}
    By Proposition \ref{Proposition 6.2} (3), get exact sequence
    \begin{equation*}
        \mathcal{O}_{U}^{\oplus J} \longrightarrow \mathcal{O}_{U}^{\oplus I} \longrightarrow \mathcal{F}\big{|}_{U} \longrightarrow 0
    \end{equation*}
    \par
    "$\Leftarrow$": For all $x\in X$, there exists open neighbourhood $U\ni x$ such that there is an exact sequence
    \begin{equation*}
        \mathcal{O}_{U}^{\oplus J} \longrightarrow \mathcal{O}_{U}^{\oplus I} \longrightarrow \mathcal{F}\big{|}_{U} \longrightarrow 0
    \end{equation*}
    Can assume that U is affine.
    By Proposition \ref{Proposition 6.4}, get exact sequence
    \begin{equation*}
        \mathcal{O}_{U}(U)^{\oplus J} \longrightarrow \mathcal{O}_{U}(U)^{\oplus I} \longrightarrow \mathcal{F}\big{|}_{U}(U) \longrightarrow 0
    \end{equation*}
    By Proposition \ref{Proposition 6.2}, get exact sequence
    \begin{equation*}
        \mathcal{O}_{U}^{\oplus J} \longrightarrow \mathcal{O}_{U}^{\oplus I} \longrightarrow \widetilde{\mathcal{F}\big{|}_{U}(U)} \longrightarrow 0
    \end{equation*}
    Thus $\mathcal{F}\big{|}_{U} \cong \widetilde{\mathcal{F}\big{|}_{U}(U)}$.
    By Theorem \ref{Theorem 6.1}, $\mathcal{F}$ is quasi-coherent.
    For coherent, the same argument would make sense.
\end{proof}
\begin{example}
    (1)Let X be a scheme.
    Then locally free $\mathcal{O}_{X}$-modules are quasi-coherent. \\
    (2)Let X be a noetherian scheme.
    Then locally free $\mathcal{O}_{X}$-modules of rank $< \infty$ are coherent. \\
    (3)Let $i: Y \longrightarrow X$ be a closed immersion with X,Y noetherian.
    Then $i_{\ast}\mathcal{O}_{Y}$ is coherent on X. \\
    (4)Let X be an integral noetherian scheme with function field K.
    Then constant sheaf $K_{X}$ is quasi-coherent but not coherent in general. \\
    (5)Let X be a scheme, $U \subseteq X$ open subset with inclusion $j: U \longrightarrow X$, $\mathcal{F}$ sheaf on U.
    Define $j_{!}\mathcal{F}$ to be the sheafification of $V \longmapsto
    \left\{
        \begin{aligned}
            & \mathcal{F}(V) & \text{if } V \subseteq U \\
            & 0 & otherwise
        \end{aligned}
    \right.
    $.
    Then stalk $(j_{!}\mathcal{F})_{x} =
    \left\{
        \begin{aligned}
            & \mathcal{F}_{x} & \text{if } x\in U \\
            & 0 & otherwise
        \end{aligned}
    \right.
    $.
    Now let $X = \spec A$ be an affine scheme with A integral domain, $U \subsetneq X$ nonempty open subset with inclusion $j: U \longrightarrow X$.
    Then $j_{!}\mathcal{O}_{U}$ is an $\mathcal{O}_{X}$-module but not quasi-coherent.
\end{example}
\subsection{Direct and Inverse Images}
\begin{lemma}
    Let $f: X \longrightarrow Y$ be a morphism of schemes, $\mathcal{M}, \mathcal{N}$ two $\mathcal{O}_{Y}$-modules.
    Then $f^{\ast}\mathcal{M} \otimes_{\mathcal{O}_{X}} f^{\ast}\mathcal{N} \cong f^{\ast}(\mathcal{M} \otimes_{\mathcal{O}_{Y}} \mathcal{N})$.
    \label{Lemma 6.2}
\end{lemma}
\begin{proof}
    By definition, it suffices to prove that $f^{-1}\mathcal{M} \otimes_{f^{-1}\mathcal{O}_{Y}} f^{-1}\mathcal{N} \cong f^{-1}(\mathcal{M} \otimes_{\mathcal{O}_{Y}} \mathcal{N})$.
    Firstly, want to show that
    \begin{equation*}
        \underset{\underset{V \supseteq f(U)}{\longrightarrow}}{\lim} \mathcal{M}(V) \otimes_{\underset{\underset{V \supseteq f(U)}{\longrightarrow}}{\lim} \mathcal{O}_{Y}(V)} \underset{\underset{V \supseteq f(U)}{\longrightarrow}}{\lim} \mathcal{N}(V)
    \end{equation*}
    is isomorphic to $\underset{\underset{V \supseteq f(U)}{\longrightarrow}}{\lim} (\mathcal{M}(V) \otimes_{\mathcal{O}_{Y}(V)} \mathcal{N}(V))$ for all opne subset $U \subseteq X$.
    Note that there are two ring homomorphisms $\varphi: \sum_{i} (m_{i}, V_{i}) \otimes (n_{i}, V_{i}') \longmapsto (\sum_{i} m_{i}\big{|}_{V} \otimes n_{i}\big{|}_{V}, V)$, where $V = \cap_{i} (V_{i} \cap V_{i}')$,
    and $\psi: (\sum_{i} m_{i} \otimes n_{i}, V) \longmapsto \sum_{i} (m_{i}, V) \otimes (n_{i}, V)$.
    Obviously, $\varphi \circ \psi(\sum_{i} m_{i} \otimes n_{i}, V) = \sum_{i} m_{i} \otimes n_{i}, V$ and $\psi \circ \varphi(\sum_{i} (m_{i}, V_{i}) \otimes (n_{i}, V_{i}')) = \sum_{i} (m_{i}, V_{i}) \otimes (n_{i}, V_{i}')$ so that get the isomorphism.
    \par
    Now we have ring homomorphisms $\mathcal{M}(V) \otimes_{\mathcal{O}_{Y}(V)} \mathcal{N}(V) \longrightarrow (\mathcal{M} \otimes_{\mathcal{O}_{Y}} \mathcal{N})(V)$ compatible with restriction maps for each open subset $V \supseteq f(U)$.
    By property of directed limit, there is an induced ring homomorphism $\underset{\underset{V \supseteq f(U)}{\longrightarrow}}{\lim} (\mathcal{M}(V) \otimes_{\mathcal{O}_{Y}(V)} \mathcal{N}(V)) \longrightarrow \underset{\underset{V \supseteq f(U)}{\longrightarrow}}{\lim} (\mathcal{M} \otimes_{\mathcal{O}_{Y}} \mathcal{N})(V)$.
    For $U' \subseteq U$ inclusion of open subsets of X, it is easy to show the following diagram commutes
    \begin{equation*}
        \begin{tikzcd}
            \underset{\underset{V \supseteq f(U)}{\longrightarrow}}{\lim} (\mathcal{M}(V) \otimes_{\mathcal{O}_{Y}(V)} \mathcal{N}(V)) \arrow[r] \arrow[d] & 
            \underset{\underset{V \supseteq f(U)}{\longrightarrow}}{\lim} (\mathcal{M} \otimes_{\mathcal{O}_{Y}} \mathcal{N})(V) \arrow[d] \\
            \underset{\underset{V \supseteq f(U')}{\longrightarrow}}{\lim} (\mathcal{M}(V) \otimes_{\mathcal{O}_{Y}(V)} \mathcal{N}(V)) \arrow[r] &
            \underset{\underset{V \supseteq f(U')}{\longrightarrow}}{\lim} (\mathcal{M} \otimes_{\mathcal{O}_{Y}} \mathcal{N})(V)
        \end{tikzcd}
    \end{equation*}
    By sheafification, get a morphism of $\mathcal{O}_{X}$-modules $f^{-1}\mathcal{M} \otimes_{f^{-1}\mathcal{O}_{Y}} f^{-1}\mathcal{N} \longrightarrow f^{-1}(\mathcal{M} \otimes_{\mathcal{O}_{Y}} \mathcal{N})$.
    As their stalks are same everywhere, it is an isomorphism.
\end{proof}
\begin{proposition}
    Let $f: X \longrightarrow Y$ be a morphism of schemes, $\mathcal{G}$ a quasi-coherent $\mathcal{O}_{Y}$-module.
    Then $f^{\ast}\mathcal{G}$ is a quasi-coherent $\mathcal{O}_{X}$-module.
    \label{Proposition 6.6}
\end{proposition}
\begin{proof}
    Can assume $X = \spec(B)$ and $Y = \spec A$.
    Then $\mathcal{G} = \widetilde{M}$ for some $A$-module M so that $f^{\ast} = \widetilde{M \otimes_{A} B}$.
    Get $f^{\ast}\mathcal{G}$ is a quasi-coherent $\mathcal{O}_{X}$-module.
\end{proof}
\begin{proposition}
    Let $f: X \longrightarrow Y$ be a morphism between noetherian schemes, $\mathcal{G}$ a coherent $\mathcal{O}_{Y}$-module.
    Then $f^{\ast}\mathcal{G}$ is a coherent $\mathcal{O}_{X}$-module.
    \label{Proposition 6.7}
\end{proposition}
\begin{proof}
    Can assume $X = \spec(B)$ and $Y = \spec A$.
    Then $\mathcal{G} = \widetilde{M}$ for some finitely generated $A$-module M so that $f^{\ast} = \widetilde{M \otimes_{A} B}$.
    Since M is finitely generated $A$-module, get $M \otimes_{A} B$ finitely generated as $B$-module.
    Get $f^{\ast}\mathcal{G}$ is a coherent $\mathcal{O}_{X}$-module.
\end{proof}
\begin{remark}
    For direct image, if N is finitely generated $B$-module, we don't have that N is finitely generated as $A$-module in general.
    Thus the similar property for direct image is much more complicated.
\end{remark}
\begin{theorem}
    Let $f: X \longrightarrow Y$ be a morphism of schemes.
    Assume that either X is noetherian or f is quasi-compact and separated.
    Then \\
    (1)If $\mathcal{F}$ is quasi-coherent sheaf on X, then $f_{\ast}\mathcal{F}$ is quasi-coherent. \\
    (2)If $\mathcal{F}$ is coherent on X with X,Y noetherian and f is finite, then $f_{\ast}\mathcal{F}$ is coherent.
    \label{Theorem 6.2} 
\end{theorem}
\begin{proof}
    For general case, can assume that $Y = \spec A$.
    Let $g\in A$ with image $h\in B = \mathcal{O}_{X}(X)$.
    Thus $f^{-1}(D(g)) = X_{h}$ so that $(f_{\ast}\mathcal{F})(D(g)) = \mathcal{F}(X_{h})$.
    It suffices to show that $\mathcal{F}(X_{h}) = f_{\ast}\mathcal{F}(Y)_{g} = \mathcal{F}(X) \otimes_{A} A_{g}$.
    Note that $B \otimes_{A} A_{g} \cong B_{h}$, we have $\mathcal{F}(X) \otimes_{A} A_{g} \cong \mathcal{F}(X)_{h}$.
    Cover X be affine open subsets $\{U_{i}\}$.
    As either X is noetherian or f is quasi-compact and separated, the covering is finite.
    There is a commutative diagram with exact rows,
    \begin{equation*}
        \begin{tikzcd}
            0 \arrow[r] &
            \mathcal{F}(X)_{h} \arrow[r] \arrow[d] &
            \prod_{i} \mathcal{F}(U_{i})_{h} \arrow[r] \arrow[d] &
            \prod_{i,j} \mathcal{F}(U_{i} \cap U_{j})_{h} \arrow[d] \\
            0 \arrow[r] &
            \mathcal{F}(X_{h}) \arrow[r] &
            \prod_{i} \mathcal{F}(X_{h} \cap U_{i}) \arrow[r] &
            \prod_{i,j} \mathcal{F}(X_{h} \cap U_{i} \cap U_{j})
        \end{tikzcd}
    \end{equation*}
    On the one hand, when X is noetherian, $U_{i} \cap U_{j}$ can still be covered by finitely many affine open subsets.
    On the other hand, when f is quasi-compact and separated, $U_{i} \cap U_{j}$ is affine.
    Thus similar to proof of Lemma \ref{Lemma 2.6}, get $\mathcal{F}(X_{h}) \cong \mathcal{F}(X)_{h}$ so that $\mathcal{F}$ is quasi-coherent.
    \par
    For finite case, can assume $X = \spec(B)$ and $Y = \spec A$ with B finite over A.
    Since $\mathcal{F}$ is coherent, get $\mathcal{F} = \widetilde{N}$ for some finitely generated $B$-module.
    By Proposition \ref{Proposition 6.2}, get $f_{\ast}\mathcal{F} = f_{\ast}\widetilde{N} = \widetilde{N \text{ as $A$-module}}$.
    While N is finitely generated and B is finite over A, get N is also finitely generated as $A$-module.
    Thus $f_{\ast}\mathcal{F}$ is coherent.
\end{proof}
\begin{example}[\textbf{\emph{Ideal Sheaves}}]
    Let $i: Y \longrightarrow X$ be a closed immersion.
    Define the ideal sheaf of Y to be $I_{Y} = \ker(\mathcal{O}_{X} \longrightarrow i_{\ast}\mathcal{O}_{Y})$.
    Then $I_{Y}$ is quasi-coherent and coherent if X,Y are noetherian.
    Conversely, all quasi-coherent ideal sheaves $\mathcal{F}$ on X are of the form $I_{Y}$ for a uniquely determined closed subscheme Y.
    Precisely, we have that $Y = \{x\in X \big{|} (\mathcal{O}_{X}/\mathcal{F})_{x} \neq 0\}$.
    \par
    In particular, if $X = \spec A$, there are some one-to-one correspondences
    \begin{equation*}
        \begin{tikzcd}[column sep= tiny]
            \{\text{ideals of A}\} \arrow[rr, leftrightarrow] \arrow[dr, leftrightarrow] &&
            \{\text{closed subschemes of X}\} \\
            & \{\text{quasi-coherent ideal sheaves on X}\} \arrow[ur, leftrightarrow] &
        \end{tikzcd}
    \end{equation*}
\end{example}
\subsection{Quasi-coherent Sheaves on Projective Schemes}
Assume that S is a graded ring.
Let $M = \oplus_{d\in \mathbb{Z}} M_{d}$ be a graded $C$-module.
Define $\widetilde{M}$ on $X = \proj(S)$ by the same process on $\mathcal{O}$ replacing S by M.
In particular, $\widetilde{M}\big{|}_{D_{+}(f)} = \widetilde{M_{(f)}}$ and $\widetilde{M}_{\mathfrak{p}} = M_{(\mathfrak{p})}$ for $\mathfrak{p}\in X$.
Then $\widetilde{M}$ is quasi-coherent on X and coherent if X is noetherian and M is finitely generated.
In addition, we should note that $\widetilde{M}$ doesn't determine M.
For instance, if $M = \oplus_{d \ge 0} M_{d}$ and $N = \oplus_{d \ge d_{0}} M_{d}$ for some $d_{0} > 0$, then $\widetilde{M} = \widetilde{N}$.
\begin{definition}[\textbf{\emph{Twisting}}]
    Let S be a graded ring and $X = \proj(S)$, $\mathcal{F}$ an $\mathcal{O}_{X}$-module.
    Set $\mathcal{O}_{X}(n) = \widetilde{S(n)}$ where $S(n)$ is the graded $C$-module given by $S(n)_{d} = S_(n + d)$.
    Also set $\mathcal{F}(n) = \mathcal{F} \otimes_{\mathcal{O}_{X}} \mathcal{O}_{X}(n)$.
\end{definition}
\begin{proposition}
    Let S be a graded ring and $X = \proj(S)$.
    Assume that S is generated by $S_{1}$ as $S_{0}$-algebra.
    Then $\mathcal{O}_{X}$ is an invertible sheaf.
    In addition, For all M graded $S$-module, we have $\widetilde{M(n)} = \widetilde{M}(n)$.
    In particular, $\mathcal{O}_{X}(n) \otimes_{\mathcal{O}_{X}} \mathcal{O}_{X}(m) \cong \mathcal{O}_{X}(m + n)$.
    \label{Proposition 6.8}
\end{proposition}
\begin{proof}
    Since S is generated by $S_{1}$ as $S_{0}$-algebra, $\{D_{+}(f)\}_{f\in S_{1}}$ cover X.
    Suffice to show for all $f\in S_{1}$, $S(n)_{(f)}$ is free of rank 1 over $S_{(f)}$.
    Note that there is a canonical isomorphism $S(n)_{(f)} \longrightarrow S_{(f)}\quad s \longmapsto f^{-n}s$, done! 
\end{proof}
\begin{lemma}[\textbf{\emph{qcqs Lemma}}]
    Let X be a scheme, $\mathcal{F}$ quasi-coherent sheaf on X, $\mathcal{L}$ invertible sheaf on X, $s\in \mathcal{L}(X)$.
    Assume that X is either noetherian or quasi-compact and separated over $\spec(\mathbb{Z})$.
    Then \\
    (1)Let $f\in \mathcal{F}(X)$.
    If $f\big{|}_{X_{s}} = 0$, then there exists $n > 0$ such that $f \otimes s^{n} = 0$ in $(\mathcal{F} \otimes_{\mathcal{O}_{X}} \mathcal{L}^{n})(X)$,
    where $X_{s} = \{x\in X \big{|} s_{x}\mathcal{O}_{X, x} = \mathcal{L}_{x}\}$ is open and $\mathcal{L}^{n}$ denotes $\mathcal{L}^{\otimes n}$. \\
    (2)Let $g\in \mathcal{F}(X_{s})$.
    Then there exists $n_{0} > 0$ such that for all $n \ge n_{0}$,
    $g \otimes (s^{n}\big{|}_{X_{s}})$ extends to a section in $(\mathcal{F} \otimes_{\mathcal{O}_{X}} \mathcal{L}^{n})(X)$.
    \label{Lemma 6.3}
\end{lemma}
\begin{proof}
    Cover X by finitely many affine open subsets $\{U_{i}\}$ such that $\mathcal{L}\big{|}_{U_{i}}$ is free and generated by $e_{i}\in \mathcal{L}(U_{i})$.
    Then $s_{i} = s\big{|}_{U_{i}} = h_{i}e_{i}$ for some $h_{i}\in \mathcal{O}_{X}(U_{i})$.
    Get $X_{s} \cap U_{i} = D(h_{i}) \subseteq U_{i}$.  
    \par
    (1): Note that $\mathcal{F}\big{|}_{U_{i}} \cong \widetilde{\mathcal{F}(U_{i})}$.
    Since $f\big{|}_{D(h_{i})}$, there exists $n > 0$ such that for all i, $f_{i} = f\big{|}_{U_{i}}$ satisfies that $h_{i}^{n}f_{i} = 0$ in $\mathcal{F}(U_{i})$.
    Thus $(f \otimes s)\big{|}_{U_{i}} = f_{i} \otimes s_{i}^{n} = h_{i}^{n}f_{i} \otimes e_{i}^{n} = 0$.
    As $U_{i}$ cover X, get $f \otimes s^{n} = 0$.
    \par
    (2): Set $g_{i} = g\big{|}_{D(h_{i})}$,
    Then there exists $m > 0$ such that for all i, there exists $g_{i}'\in \mathcal{F}(U_{i})$ such that $h_{i}^{m}g_{i} = g_{i}'$.
    Set $t_{i} = g_{i}' \otimes e_{i}^{m}\in (\mathcal{F} \otimes_{\mathcal{O}_{X}} \mathcal{L}^{m})(U_{i})$.
    Note that $t_{i}$ and $t_{j}$ coincide on $X_{s} \cap U_{i} \cap U_{j}$.
    Thus $(t_{i}\big{|}_{U_{i} \cap U_{j}} - t_{j}\big{|}_{U_{i} \cap U_{j}})\big{|}_{X_{s} \cap U_{i} \cap U_{j}}$.
    As $U_{i} \cap U_{j}$ is either noetherian or affine, by (1), there exists $p > 0$ such that $(t_{i}\big{|}_{U_{i} \cap U_{j}} - t_{j}\big{|}_{U_{i} \cap U_{j}}) \otimes s\big{|}_{U_{i} \cap U_{j}}^{p} = 0$ in $((\mathcal{F} \otimes_{\mathcal{O}_{X}} \mathcal{L}^{m})\big{|}_{U_{i} \cap U_{j}} \otimes_{\mathcal{O}_{U_{i} \cap U_{j}}} (\mathcal{L}\big{|}_{U_{i} \cap U_{j}})^{p})(U_{i} \cap U_{j})$.
    While $(\mathcal{F} \otimes_{\mathcal{O}_{X}} \mathcal{L}^{m})\big{|}_{U_{i} \cap U_{j}} \otimes_{\mathcal{O}_{U_{i} \cap U_{j}}} (\mathcal{L}\big{|}_{U_{i} \cap U_{j}})^{p} \cong (\mathcal{F} \otimes_{\mathcal{O}_{X}} \mathcal{L}^{m + p})\big{|}_{U_{i} \cap U_{j}}$, 
    get $t_{i} \otimes s_{i}^{p}$ and $t_{j} \otimes s_{j}^{p}$ coincide on $U_{i} \cap U_{j}$.
    Thus we can glue them to obtain $t\in (\mathcal{F} \otimes_{\mathcal{O}_{X}} \mathcal{L}^{m + p})(X)$ such that $t\big{|}_{X_{s}} = g \otimes (s\big{|}_{X_{s}}^{m + s})$.
    Take $n_{0} = m + p$, done! 
\end{proof}
\begin{definition}
    Let X be a scheme, $\mathcal{F}$ an $\mathcal{O}_{X}$-module.
    We say that $\mathcal{F}$ is generated by global sections (or say globally generated) if there exists a family of global sections $\{s_{i}\}$ such that for all $x\in X$,
    $\{(s_{i})_{x}\}$ generate $\mathcal{F}_{x}$.
\end{definition}
\begin{remark}
    Note that $\mathcal{F}$ is generated by global sections if and only if $\mathcal{F}$ is a quotient of a free $\mathcal{O}_{X}$-module.
\end{remark}
\begin{example}
    (1)Let $X = \mathbb{P}_{k}^{n}$ be projective k-scheme.
    Then $\mathcal{O}_{X}(-1)$ is not generated by global sections since there is no global sections at all. \\
    (2)Let $X = \spec A$ be an affine scheme.
    Then all quasi-coherent sheaves on X are generated by global sections.
\end{example}
\begin{definition}
    Let A be a ring and X be a projective scheme over A with commutative diagram
    \begin{equation*}
        \begin{tikzcd}
            X \arrow[r, "i"] \arrow[dr] &
            \mathbb{P}_{A}^{d} \arrow[d] \\
            & \spec A
        \end{tikzcd}
    \end{equation*}
    Set $\mathcal{O}_{X}(n) = i^{\ast}\mathcal{O}_{\mathbb{P}_{A}^{d}}(n)$.
    For $\mathcal{F}$ an $\mathcal{O}_{X}$-module, also set $\mathcal{F}(m) = \mathcal{F} \otimes_{\mathcal{O}_{X}} \mathcal{O}_{X}(n)$.
\end{definition}
\begin{lemma}[\textbf{\emph{Projection Formula for Affine Case}}]
    Let $f: X \longrightarrow Y$ be an affine morphism, $\mathcal{F}$ a quasi-coherent sheaf on X, $\mathcal{G}$ a quasi-coherent sheaf on Y.
    Then $f_{\ast}(\mathcal{F} \otimes_{\mathcal{O}_{X}} f^{\ast}\mathcal{G}) \cong f_{\ast}\mathcal{F} \otimes_{\mathcal{O}_{Y}} \mathcal{G}$.
    \label{Lemma 6.4}
\end{lemma}
\begin{proof}
    Can assume that $X = \spec(B)$, $Y = \spec A$, $\mathcal{F} = \widetilde{N}$ and $\widetilde{M}$.
    Then $f_{\ast}(\mathcal{F} \otimes_{\mathcal{O}_{X}} f^{\ast}\mathcal{G}) \cong \widetilde{P}$,
    where $P = N \otimes_{B} (M \otimes_{A} B)$ as $A$-module.
    On the other hand, $f_{\ast}\mathcal{F} \otimes \mathcal{G} \cong \widetilde{Q}$,
    where $Q = N \otimes_{A} M$ as $A$-module.
    As $P \cong Q$, done!
\end{proof}
\begin{theorem}[\textbf{\emph{Serre's Theorem}}]
    Let A be a noetherian ring and X be a projective scheme over A with commutative diagram, 
    \begin{equation*}
        \begin{tikzcd}
            X \arrow[r, "i"] \arrow[dr] &
            \mathbb{P}_{A}^{d} \arrow[d] \\
            & \spec A
        \end{tikzcd}
    \end{equation*}
    Assume that $\mathcal{F}$ is a coherent sheaf on X.
    Then there exists $n_{0} > 0$ such that for all $n \ge n_{0}$, $\mathcal{F}(n)$ is generated by finitely many global sections.
    \label{Theorem 6.3}
\end{theorem}
\begin{proof}
    Since i is a closed immersion, i is finite so that $i_{\ast}\mathcal{F}$ is coherent.
    By Lemma \ref{Lemma 6.4}, $i_{\ast}(\mathcal{F}(n)) \cong (i_{\ast}\mathcal{F})(n)$.
    Thus global sections of $i_{\ast}(\mathcal{F}(n))$ are equal to global sections of $(i_{\ast}\mathcal{F})(n)$ and stalks of $i_{\ast}(\mathcal{F}(n))$ are equal to stalks of $(i_{\ast}\mathcal{F})(n)$.
    It suffices to prove Theorem for $X = \mathbb{P}_{A}^{d} = \proj(A[x_{0}, \cdots, x_{d}])$.
    \par
    Cover X by $\{U_{i} = D_{+}(x_{i})\}$.
    Each $\mathcal{F}\big{|}_{U_{i}}$ is generated by finitely many sections $g_{ij}\in \mathcal{F}(U_{i})$.
    By Lemma \ref{Lemma 6.3}, there exists $n_{0} > 0$ such that $g_{ij} \otimes x_{i}^{n}$ extends to a global section for all i,j.
    Thus $\mathcal{F}(n)$ is generated by these extensions.
\end{proof}
\begin{corollary}
    Let A be a noetherian ring and X be a projective scheme over A with commutative diagram, 
    \begin{equation*}
        \begin{tikzcd}
            X \arrow[r, "i"] \arrow[dr] &
            \mathbb{P}_{A}^{d} \arrow[d] \\
            & \spec A
        \end{tikzcd}
    \end{equation*}
    Assume that $\mathcal{F}$ is a coherent sheaf on X.
    Then there exists $m\in \mathbb{Z}$ and $r < \infty$ such that $\mathcal{F}$ is a quotient of $\mathcal{O}_{X}(m)^{\oplus r}$.
    \label{Corollary 6.1}
\end{corollary}
\begin{reason}
    Since $\mathcal{F}(n)$ is generated bv global sections for large enough $n$,
    $\mathcal{F}(n)$ is quotient of $\mathcal{O}_{X}^{\oplus I}$ with finite index set.
    Tensor by $\mathcal{O}_{X}(-n)$, done!
\end{reason}
\begin{definition}
    Let S be a graded ring generated by finitely many elements of $S_{1}$ as $S_{0}$-algebra and $X = \proj(S)$, $\mathcal{F}$ an $\mathcal{O}_{X}$-module.
    Define a graded $C$-module $\Gamma_{\ast}(\mathcal{F}) = \oplus_{n\in \mathbb{Z}} \Gamma(X, \mathcal{F}(n))$.
\end{definition}
\begin{proposition}
    Let A be a ring, $S = A[x_{0}, \cdots, x_{d}]$ with $d > 0$ and $X = \proj(S) = \mathbb{P}_{A}^{d}$.
    Then $\Gamma_{\ast}(\mathcal{O}_{X}) = S$.
    \label{Proposition 6.9}
\end{proposition}
\begin{proof}
    Want to show $\Gamma(X, \mathcal{O}_{X}(n)) = 
    \left\{
        \begin{aligned}
            & S_{n} & n \ge 0 \\
            & 0 & n < 0
        \end{aligned}
    \right.
    $.
    Let $B = A[x_{0}, \cdots, x_{d}, x_{0}^{-1}, \cdots, x_{d}^{-1}]$.
    Then any global section  of $\mathcal{O}_{X}(n)$ gives an element $f\in B$ which is in $x_{0}^{n}\mathcal{O}_{X}(D_{+}(x_{0}))$ by restriction.
    Thus f is of the form $\frac{P}{x_{0}^{m}}$ where P is a homogeneous polynomial.
    But there is also a restriction in $x_{1}^{n}\mathcal{O}_{X}(D_{+}(x_{1}))$, get f is homogeneous polynomial of degree n if $n \ge 0$ and $f = 0$ if $n < 0$.
    Conversely, such an f is obviously in $\Gamma(X, \mathcal{O}_{X}(n))$.
\end{proof}
\begin{theorem}
    Let S be a graded ring generated by finitely many elements of $S_{1}$ as $S_{0}$-algebra and $X = \proj(S)$.
    Then for all quasi-coherent sheaf $\mathcal{F}$ on X, we have $\mathcal{F} \cong \widetilde{\Gamma_{\ast}(\mathcal{F})}$.
    \label{Theorem 6.4}
\end{theorem}
\begin{proof}
    Let $s\in S_{1}$.
    Set $\Gamma_{\ast}(\mathcal{F}) = M$ and $U = D_{+}(s)$.
    Want to show that there is a canonical isomorphism $\varphi_{s}: M_{(s)} \overset{\sim}{\longrightarrow} \mathcal{F}(U)$.
    Assume $s^{-n}t\in M_{(s)}$ with $t\in \Gamma(X, \mathcal{F}(n))$.
    Note that $\mathcal{F}(n) \otimes_{\mathcal{O}_{X}} \mathcal{O}_{X}(-n) \cong \mathcal{F}$, get $t\big{|}_{U} \otimes s^{-n}\in \Gamma(U, \mathcal{F})$.
    Then define $\varphi: s^{-n}t \longmapsto t\big{|}_{U} \otimes s^{-n}$.
    By Lemma \ref{Lemma 6.3}, for each element $g\in \mathcal{F}(U)$, there exists $m > 0$ such that $g \otimes s^{m}$ extends to $t\in (\mathcal{F} \otimes_{\mathcal{O}_{X}} \mathcal{O}_{X}(m))(X) \subseteq M$.
    Thus $\varphi_{s}(\frac{t}{s^{m}}) = t\big{|}_{U} \otimes s^{-m} = g \otimes s^{m} \otimes s^{-m} = g$ so that $\varphi_{s}$ is surjective.
    On the other hand, if $s^{-n}t\in \ker(\varphi_{s})$, then $t\big{|}_{U} \otimes s^{-n} = 0$ in $\mathcal{F}(U)$.
    Note that $t\big{|}_{U} \otimes s^{-n} \otimes s^{n} = t\big{|}_{U}$, by Lemma \ref{Lemma 6.3}, get there exists $m > 0$ such that $t \otimes s^{m} = 0$ in $\mathcal{F}(m)(X)$ so that $s^{-n}t = 0$ in $M_{(s)}$.
    Thus $\varphi_{s}$ is isomorphic.
    By gluing lemma of morphism, get $\widetilde{M} \overset{\sim}{\longrightarrow} \mathcal{F}$ isomorphic.
\end{proof}
\begin{theorem}
    Let A be a ring, $S = A[x_{0}, \cdots, x_{d}]$ with $d > 0$ and $X = \proj(S) = \mathbb{P}_{A}^{d}$.
    Then any closed subscheme Z of X is of the form $\proj(S/I)$ for uniquely determined homogeneous ideal I.
    \label{Theorem 6.5}
\end{theorem}
\begin{proof}
    Can assume that $d > 0$ since when $d = 0$, $\proj(X) = D_{+}(x_{0})$ is just the affine case.
    Take $\mathcal{I}$ to be the ideal sheaf of Z.
    Then $\mathcal{I}$ is quasi-coherent
    By Theorem \ref{Theorem 6.4}, $\mathcal{I} = \widetilde{I}$ where $I = \Gamma_{\ast}(\mathcal{I})$.
    Since I is a homogeneous submodule of $\Gamma_{\ast}(\mathcal{O}_{X})$, by Theorem \ref{Theorem 6.3}, I is a homogeneous ideal of S.
    Restrict to standard open subsets and by gluing lemma, we get $(Z, \mathcal{O}_{Z}) \cong (Proj(S/I), \mathcal{O}_{Proj(S/I)})$.
\end{proof}
\subsection{Ample and Very Ample Invertible Sheaves}
\begin{proposition}
    Let A be a ring, X an A-scheme, $Y = \proj(A[x_{0}, \cdots, x_{d}]) = \mathbb{P}_{A}^{d}$.
    Then \\
    (1)Let $f: X \longrightarrow Y$ be an A-morphism.
    Then $f^{\ast}\mathcal{O}_{Y}(1)$ is an invertible sheaf on X, generated by $d + 1$ global sections. \\
    (2)Conversely, if $\mathcal{L}$ is an invertible sheaf on X, generated by global sections $s_{0}, \cdots, s_{d}$,
    then there exists unique A-morphism $f: X \longrightarrow Y$ such that $\mathcal{L} \cong f^{\ast}\mathcal{O}_{Y}(1)$ and $f^{\ast}x_{i}$ is identified with $s_{i}$.
    \label{Proposition 6.10}
\end{proposition}
\begin{proof}
    (1): $\mathcal{O}_{Y}(1)$ is generated by $d + 1$ global sections, inducing global sections $s_{1}, \cdots, s_{d}$ of $f^{\ast}\mathcal{O}_{Y}(1)$.
    While for all $x\in X$ and $y = f(x)$, then $(f^{\ast}\mathcal{O}_{Y}(1))_{x} = \mathcal{O}_{Y}(1)_{y} \otimes_{\mathcal{O}_{Y, y}} \mathcal{O}_{X, x}$.
    Thus ${s_{0}}_{x}, \cdots, {s_{d}}_{x}$ generate $(f^{\ast}\mathcal{O}_{Y})_{x}$.
    \par
    (2): As assumption, X is covered by $\{X_{s_{i}}\}$.
    For all i, want to define $f_{i}: X_{s_{i}} \longrightarrow D_{+}(x_{i})$.\
    By Hartshorne exercise II 2.4, it is equivalent to give a ring homomorphism between global sections $A[\frac{x_{0}}{x_{i}}, \cdots, \widehat{\frac{x_{i}}{x_{i}}}, \cdots, \frac{x_{d}}{x_{i}}] \longrightarrow \mathcal{O}_{X_{s_{i}}}$,
    which can be defined by mapping $\frac{x_{j}}{x_{i}}$ to $\frac{s_{j}}{s_{i}}$,
    where $\frac{s_{j}}{s_{i}}$ is the unique element $a\in \mathcal{O}_{X}(X_{s_{i}})$ such that $s_{j}\big{|}_{X_{s_{i}}} = as_{i}\big{|}_{X_{s_{i}}}$.
    It is easy to check that $f_{i}$ satisfy conditions for gluing lemma.
    Get $f: X \longrightarrow Y$.
\end{proof}
\begin{remark}
    Intuitively, f should be $x \longmapsto (s_{0}(x) : \cdots : s_{d}(x))$.
    In addition, f is a closed immersion if and only if for all i, $X_{s_{i}}$ is affine and the A-algebra is generated by $\{s_{j}/s_{i}\}$.
\end{remark}
\begin{definition}[\textbf{\emph{Immersions}}]
    Let $f: X \longrightarrow Y$ be a morphism of schemes.
    We say that f is an immersion if f factors as $X \overset{\text{open immersion}}{\longrightarrow} Z \overset{\text{closed immersion}}{\longrightarrow} Y$.
\end{definition}
\begin{proposition}
    There are lots of properties of immersions. \\
    (1)Closed immersions are immersions. \\
    (2)Composition of immersions are immersions. \\
    (3)Immersions are stable under base change. 
    \label{Proposition 6.11}
\end{proposition}
\begin{remark}
    Since (1),(2),(3) are in fact the properties (a),(b),(c) in Hartshorne exercise ii 4.8,
    by the exercise, we immediately get that (d),(e),(f) also holds for immersion.
\end{remark}
\begin{definition}[\textbf{\emph{Quasi-projective}}]
    Let A be a ring, X an A-scheme.
    We say that X is quasi-projective over A if $X \longrightarrow \spec A$ factors as 
    \begin{equation*}
        \begin{tikzcd}
            X \arrow[r, "immersion"] \arrow[dr] &
            \mathbb{P}_{A}^{d} \arrow[d] \\
            & \spec A
        \end{tikzcd}
    \end{equation*}
\end{definition}
In Midterm24.pdf, there is also a definition of immersion which is a little different from here and we would rename it.
\begin{definition}[\textbf{\emph{Locally Closed Immersions}}]
    Let $f: X \longrightarrow Y$ be a morphism of schemes.
    We say that f is a locally closed immersion if f factors as $X \overset{\text{closed immersion}}{\longrightarrow} Z \overset{\text{open immersion}}{\longrightarrow} Y$.
\end{definition}
\begin{proposition}
    Let $f: X \longrightarrow Y$ be a morphism of schemes.
    If f is a immersion, then f is a locally closed immersion.
    On the other hand, if f is a locally closed immersion and suppose that either X is noetherian or f is quasi-compact, then f is an immersion.
    \label{Proposition 6.12}
\end{proposition}
\begin{definition}[\textbf{\emph{Very Ample Invertible Sheaves}}]
    Let A be a ring, X an A-scheme, $\mathcal{L}$ an invertible sheaf on X.
    We say that $\mathcal{L}$ is very ample relative to A if there exists A-immersion $f: X \longrightarrow \mathbb{P}_{A}^{d}$ such that $\mathcal{L} \cong \mathcal{O}_{X}(1) = f^{\ast}\mathcal{O}_{\mathbb{P}_{A}^{d}}(1)$.
\end{definition}
\begin{remark}
    $\mathcal{L}$ is very ample if and only if $\mathcal{L}$ can be generated by finitely many global sections such that the associated A-morphism $f: X \longrightarrow \mathbb{P}_{A}^{d}$ is an immersion.
    If moreover $X \longrightarrow \spec A$ is proper, then f is a closed immersion.
    Thus X is projective over A if and only if X is proper over A and admits a very ample invertible sheaf.
\end{remark}
\begin{definition}[\textbf{\emph{Ample Invertible Sheaves}}]
    Let X be a noetherian scheme, $\mathcal{L}$ an invertible sheaf on X.
    We say that $\mathcal{L}$ is ample if for all coherent sheaf $\mathcal{F}$ on X,
    there exists $n_{0} > 0$ such that for all $n \ge n_{0}$, $\mathcal{F} \otimes_{\mathcal{O}_{X}} \mathcal{L}^{n}$ is generated by global sections.
\end{definition}
\begin{remark}
    Serre's Theorem are saying that for $X \longrightarrow \spec A$ proper with A noetherian,
    a very ample invertible sheaf $\mathcal{L}$ on X is ample.
    In fact, can remove proper by reducing to proper case with the following nontrivial fact which is Hartshorne exercise ii 5.15.
    \par
    Fact: Let X be a noetherian scheme, $U \subseteq X$ open subset, $\mathcal{F}$ coherent sheaf on U.
    Then there exists coherent sheaf on X whose restriction to U is just $\mathcal{F}$.
\end{remark}
\begin{example}
    (1)Any invertible sheaf on an affine noetherian scheme is ample. \\
    (2)If $\mathcal{L}$ is ample, then $\mathcal{L}^{r}$ is ample for all $r > 0$. 
    Conversely, if $\mathcal{L}^{r}$ is ample for some $r > 0$, then $\mathcal{L}$ is ample. \\
    (3)Let $X = \mathbb{P}_{A}^{d}$.
    Then $\mathcal{O}_{X}(n)$ is very ample/ample if and only if $n > 0$.
\end{example}
\begin{lemma}
    Let X be a noetherian scheme, $\mathcal{L}$ ample invertible sheaf on X.
    For all $x\in X$, there exists $n > 0$ and $s\in \mathcal{L}^{n}(X)$ such that $x\in X_{s} \subseteq X$ and $X_{s}$ is an affine open subset.
    \label{Lemma 6.5}
\end{lemma}
\begin{proof}
    Let $U \subseteq X$ be an affine open neighbourhood of x such that $\mathcal{L}\big{|}_{U}$ is free.
    Our goal is to find an affine open subset inside U of the form $X_{s}$ containing x.
    Consider $Z = X \setminus U$ with the reduced scheme structure and the ideal sheaf $\mathcal{I}$ on Z.
    For all $n > 0$, since $\otimes_{\mathcal{O}_{X}} \mathcal{L}^{n}$ is exact functor, $I\mathcal{L}^{n} \subseteq \mathcal{L}^{n}$ is a subsheaf.
    As $\mathcal{L}$ is ample, for n big enough, $\mathcal{I}\mathcal{L}^{n}$ is generated by global sections.
    In particular, since $x\notin Z$, there exists global section $s\in \mathcal{I}\mathcal{L}^{n}(X) \subseteq \mathcal{L}^{n}(X)$ such that $s_{x}$ generates $(\mathcal{I}\mathcal{L}^{n})_{x} \cong \mathcal{O}_{X, x} \otimes_{\mathcal{O}_{X, x}} \mathcal{L}_{x}^{n} \cong \mathcal{L}_{x}^{n}$.
    Thus $x\in X_{s}$.
    On the other hand, $X_{s} \subseteq U$ since if $y\in X_{s}$, then $(\mathcal{I}\mathcal{L}^{n})_{u}$ contains $s_{y}$ which generates $\mathcal{L}_{y}^{n}$.
    Thus $X_{s} = X_{s} \cap U$ is affine.
\end{proof}
\begin{theorem}
    Let A be a noetherian ring and X be a scheme of finite type over A, $\mathcal{L}$ an ample invertible sheaf on X.
    Then there exists $r > 0$ such that $\mathcal{L}^{r}$ is very ample.
    \label{Theorem 6.6}
\end{theorem}
\begin{proof}
    Since X is noetherian, we can cover X by finitely many $\{X_{s_{i}}\}$ as in Lemma \ref{Lemma 6.5}.
    Can also assume $s_{i}\in \mathcal{L}^{n}(X)$ for same n and can even assume that $n = 1$ by replacing $\mathcal{L}$ by $\mathcal{L}^{n}$.
    Note that X is of finite type over A, $\mathcal{O}_{X}(X_{s_{i}})$ is generated as A-algebra by finitely many $\{g_{ij}\}$.
    By Lemma \ref{Lemma 6.3}, there exists $r > 0$ such that $g_{ij} \otimes s_{i}\big{|}_{X_{s_{i}}}^{r}$ extends to a global section $s_{ij}\in (\mathcal{O}_{X} \otimes_{\mathcal{O}_{X}} \mathcal{L}^{r})(X) = \mathcal{L}^{r}(X)$ for all i,j.
    Now $\{s_{i}^{r}\}$ generate $\mathcal{L}^{r}$ since $\{X_{s_{i}} = X_{s_{i}^{r}}\}$ cover X.
    Thus $\{s_{i}^{r}, s_{ij}\}$ also generate $\mathcal{L}^{r}$.
    \par
    By Proposition \ref{Proposition 6.10}, there is a morphism $f: X \longrightarrow \proj(A[x_{i}, x_{ij}]) = \mathbb{P}$ such that $f^{\ast}\mathcal{O}_{\mathbb{P}}(1) \cong \mathcal{L}^{n}$.
    Now for all i, $\mathcal{O}_{\mathbb{P}}(D_{+}(x_{i})) \longrightarrow \mathcal{O}_{X}(X_{s_{i}})\quad x_{ij}/x_{i} \longmapsto g_{ij}$ is surjective, inducing by $g_{ij} \otimes s\big{|}_{X_{s_{i}}}^{r}$ extends to $s_{ij}$.
    Then f is a closed immersion from X to $\cup_{i} D_{+}(x_{i}) \subseteq \mathbb{P}$, since $X_{s_{i}}$ and $D_{+}(x_{i})$ are both affine.
    Get f is locally closed immersion.
    As X is Noetherian, by Proposition \ref{Proposition 6.12}, get f is immersion.
    Thus $\mathcal{L}^{r}$ is very ample by remark 6.9.
\end{proof}
\begin{corollary}
    Let A be a noetherian ring and X be a scheme of finite type over A, $\mathcal{L}$ an ample invertible sheaf on X.
    Then X is quasi-projective (resp. projective) over A if and only if X admits an ample invertible sheaf (resp. X admits an ample invertible sheaf and X is proper).
    \label{Corollary 6.2}
\end{corollary}
\begin{remark}
    Let A be a noetherian ring and X be a scheme of finite type over A, $\mathcal{L}$ an ample invertible sheaf on X.
    If X admits an ample invertible sheaf, then X is separated.
\end{remark}

\section{Sheaf Cohomology}
\label{section:Sheaf Cohomology}

\subsection{Cohomology Group}
There are lots of abelian categories in Algebraic Geometry.
\begin{example}
    (1)$Ab$ is the category of abelian groups. \\
    (2)$Mod_{A}$ is the category of $A$-modules. \\
    (3)$Ab(X)$ is the category of sheaves of abelian groups on a topological space X. \\
    (4)$Mod(\mathcal{O}_{X})$ is the category of $\mathcal{O}_{X}$-modules on a ringed space $(X, \mathcal{O}_{X})$. \\
    (5)$Qcoh(X)$ is the category of quasi-coherent sheaves on a scheme X. \\
    (6)$Coh(X)$ is the category of coherent sheaves on a noetherian scheme X.
\end{example}
There are some special functors in Algebraic Geometry.
\begin{example}
    (1)$\hom_{\mathcal{C}}(A, \cdot): \mathcal{C} \longrightarrow Ab$ is left exact covariant functor. \\
    (2)$\hom_{\mathcal{C}}(\cdot, A): \mathcal{C} \longrightarrow Ab$ is left exact contravariant functor. \\
    (3)$\Gamma(X, \cdot): Ab(X) \longrightarrow Ab$ is left exact covariant functor. \\
    (4)$M \otimes_{A} \cdot: Mod_{A} \longrightarrow Mod_{A}$ is right exact covariant functor.
    In particular, $M \otimes_{A} \cdot$ is exact if and only if M is flat $A$-module. 
\end{example}
\begin{definition}
    Let $\mathcal{C}$ be a category with enough injective objects, $\mathcal{F}$ a left exact covariant functor, $J\in \mathcal{C}$.
    We say that J is acyclic for the functor $\mathcal{F}$ (or say $\mathcal{F}$-acyclic) if $R^{i}\mathcal{F}(J) = 0$ for all $i > 0$.
\end{definition}
\begin{proposition}
    Let $\mathcal{C}$ be a category with enough injective objects, $\mathcal{F}$ a left exact covariant functor, $A\in \mathcal{C}$.
    Assume that there is an $\mathcal{F}$-acyclic resolution $J^{\ast}$ of A i.e. $0 \longrightarrow A \longrightarrow J_{0} \longrightarrow J_{1} \longrightarrow \cdots$ exact with each $J_{i}$ $\mathcal{F}$-acyclic.
    Then $R^{i}\mathcal{F}(A) \cong H^{i}(\mathcal{F}(J^{\ast}))$.
    \label{Proposition 7.1} 
\end{proposition}
\begin{proposition}
    Let $(X, \mathcal{O}_{X})$ be a ringed space.
    Then $Mod(\mathcal{O}_{X})$ has enough injective objects.
    In particular, $Ab(X)$ has enough injective objects if viewing X as $(X, \mathbb{Z}_{X})$ ringed space, where $\mathbb{Z}_{X}$ is the constant sheaf.
    \label{Proposition 7.2}
\end{proposition}
\begin{proof}
    Let $\mathcal{F}$ be an $\mathcal{O}_{X}$-module.
    For all $x\in X$, can embed $\mathcal{F}_{x}$ in an injective $\mathcal{O}_{X, x}$-module $I_{x}$ since $Mod_{A}$ has enough injective objects for all ring A.
    View $I_{x}$ as a sheaf on $\{x\}$.
    Set $\mathcal{I} = \prod_{x\in X} j_{\ast}I_{X}$ where $j: \{x\} \longrightarrow X$ is inclusion.
    For all $\mathcal{G}$ $\mathcal{O}_{X}$-module, we have $\hom_{\mathcal{O}_{X}}(\mathcal{G}, \mathcal{I}) = \prod_{x\in X} \hom_{\mathcal{O}_{X}}(\mathcal{G}, j_{\ast}I_{x}) = \prod_{x\in X} \hom_{\mathcal{O}_{X, x}}(\mathcal{G}_{x}, I_{x})$.
    With $\mathcal{F}_{x} \hookrightarrow I_{x}$, get a monomorphism $\mathcal{F} \longrightarrow \mathcal{I}$ with $\mathcal{I}$ injective.
\end{proof}
\begin{definition}[\textbf{\emph{Sheaf Cohomology}}]
    Let X be a topological space, $\Gamma(X, \cdot): Ab(X) \longrightarrow Ab$.
    Define the cohomology functor $H^{i}(X, \cdot) = R^{i}\Gamma(X, \cdot)$.
    For $\mathcal{F}\in Ab(X)$, we call $H^{i}(X, \mathcal{F})$ the ith cohomology group of $\mathcal{F}$.
\end{definition}
\begin{definition}
    Let X be a topological space, $\mathcal{F}\in Ab(X)$.
    We say that $\mathcal{F}$ is flasque if for all $V \subseteq U$ inclusion of open subsets,
    the restriction map $\mathcal{F}(U) \longrightarrow \mathcal{F}(V)$ is surjective.
\end{definition}
\begin{proposition}
    (1)Direct products of flasque sheaves are flasque. \\
    (2)Direct image of flasque sheaves are flasque. \\
    (3)quotient of flasque sheaves by flasque sheaves are flasque. \\
    (4)If $0 \longrightarrow \mathcal{F} \longrightarrow \mathcal{G} \longrightarrow \mathcal{H} \longrightarrow 0$ is exact with $\mathcal{F}$ flasque,
    then $0 \longrightarrow \mathcal{F}(U) \longrightarrow \mathcal{G}(U) \longrightarrow \mathcal{H}(U) \longrightarrow 0$ is exact for all open subset $U \subseteq X$.
    \label{Proposition 7.3}
\end{proposition}
\begin{proposition}
    Let $(X, \mathcal{O}_{X})$ be a ringed space.
    Then every injective $\mathcal{O}_{X}$-module is flasque.
    \label{Proposition 7.4}
\end{proposition}
\begin{proof}
    Consider $U \subseteq X$ is an inclusion of open subsets.
    Set $\mathcal{G}_{U} = j_{!}\mathcal{O}_{U}$.
    Let $\mathcal{F}$ be an injective $\mathcal{O}_{X}$-module, $V \subseteq U$ open subset.
    Then the natural morphism $\mathcal{G}_{V} \longrightarrow \mathcal{G}_{U}$ is monic so that $0 \longrightarrow \mathcal{G}_{V} \longrightarrow \mathcal{G}_{U}$ is exact.
    Since $\mathcal{F}$ is injective, get $\hom_{\mathcal{O}_{X}}(\mathcal{G}_{U}, \mathcal{F}) \longrightarrow \hom_{\mathcal{O}_{X}}(\mathcal{G}_{V}, \mathcal{F}) \longrightarrow 0$ exact.
    Note that $\hom_{\mathcal{O}_{X}}(\mathcal{G}_{U}, \mathcal{F})\big{|}_{U} = \hom_{\mathcal{O}_{U}}(\mathcal{O}_{U}, \mathcal{F}\big{|}_{U}) = \mathcal{F}(U)$.
    Thus $\mathcal{F}(U) \longrightarrow \mathcal{F}(V) \longrightarrow 0$ is exact so that $\mathcal{F}$ is flasque.
\end{proof}
\begin{corollary}
    Let X be a topological space.
    Then every injective object in $Ab(X)$ is flasque by viewing X as ringed space $(X, \mathbb{Z}_{X})$ where $\mathbb{Z}_{X}$ is the constant sheaf.
    \label{Corollary 7.1}
\end{corollary}
\begin{proposition}
    Let X be a topological space.
    Then every flasque sheaf on X is acyclic for the functor $\Gamma(X, \cdot)$.
    \label{Proposition 7.5}
\end{proposition}
\begin{proof}
    Let $\mathcal{F}$ be a flasque sheaf on X.
    Consider exact sequence $0 \longrightarrow \mathcal{F} \longrightarrow \mathcal{I} \longrightarrow \mathcal{G} \longrightarrow 0$ where $\mathcal{I}$ is an injective object in $Ab(X)$.
    By Corollary \ref{Corollary 7.1}, $\mathcal{I}$ is flasque.
    Thus $\mathcal{G}$ is flasque since it is quotient of flasque sheaf by flasque sheaf.
    Take the long exact sequence $0 \longrightarrow \mathcal{F}(X) \longrightarrow \mathcal{I}(X) \longrightarrow \mathcal{G}(X) \longrightarrow H^{1}(X, \mathcal{F}) \longrightarrow H^{1}(X, \mathcal{I}) \longrightarrow \cdots$.
    While $\mathcal{F}$ is flasque, we have $0 \longrightarrow \mathcal{F}(X) \longrightarrow \mathcal{I}(X) \longrightarrow \mathcal{G}(X) \longrightarrow 0$ is exact.
    Also, as $\mathcal{I}$ is injective, $H^{i}(X, \mathcal{I}) = 0$ for all $i > 0$.
    Thus $H^{i}(X, \mathcal{F}) = 0$ for all $i > 0$ so that $\mathcal{F}$ is acyclic for the functor $\Gamma(X, \cdot)$. 
\end{proof}
\begin{corollary}
    Let $(X, \mathcal{O}_{X})$ be a ringed space.
    Then the right derived functor of $\Gamma(X, \cdot): Mod(\mathcal{O}_{X}) \longrightarrow Ab$ coincide with $H^{i}(X, \cdot)$.
    \label{Corollary 7.2}
\end{corollary}
\begin{reason}
    Take injective resolution in $Mod(\mathcal{O}_{X})$.
    By Proposition \ref{Proposition 7.4} and Proposition \ref{Proposition 7.5}, it is a $\Gamma(X, \cdot)$-acyclic resolution.
    Thus by Proposition \ref{Proposition 7.1}, the two right derived functors coincide.
\end{reason}
\begin{corollary}
    Let X be a topological space, $Y \subseteq X$ closed subset, $i: Y \longrightarrow X$ inclusion.
    Then for all $\mathcal{F}\in Ab(Y)$, we have $H^{i}(Y, \mathcal{F}) \cong H^{i}(X, j_{\ast}\mathcal{F})$.
    \label{Corollary 7.3}
\end{corollary}
\begin{reason}
    Take injective resolution $J^{\ast}$ of $\mathcal{F}$ in $Ab(Y)$.
    Then $f_{\ast}J^{\ast}$ is a flasque resolution of $f_{\ast}\mathcal{F}$.
\end{reason}
\begin{remark}[\textbf{\emph{Gabber Theorem}}]
    Let X be a scheme.
    Then $Qcoh(X)$ has enough injective objects.
\end{remark}
\begin{remark}
    Let X be a scheme.
    Suppose that X is either noetherian or quasi-compact and separated.
    Then the right derived functor of $\Gamma(X, \cdot): Qcoh(X) \longrightarrow Ab$ coincide with $H^{i}(X), \cdot$.
    In fact, this property is equivalent to each injective object is $\Gamma(X, \cdot)$-acyclic.
\end{remark}
\begin{theorem}[\textbf{\emph{Grothendieck}}]
    Let X be a noetherian topological space of dimension $n < \infty$.
    Then for all $\mathcal{F}\in Ab(X)$, we have that $H^{i}(X, \mathcal{F}) = 0$ for all $i > n$.
    \label{Theorem 7.1}
\end{theorem}
\subsection{Cohomology of Noetherian Affine Schemes}
\begin{theorem}[\textbf{\emph{Artin-Rees}}]
    Let A be a noetherian ring, $M \subseteq N$ finitely generated $A$-modules, $\mathfrak{a} \subseteq A$ ideal.
    Then for all $n > 0$, there exists $m > n$ such that $\mathfrak{a}M \supseteq M \cap a^{m}N$.
    \label{Theorem 7.2}
\end{theorem}
\begin{remark}
    This theorem says that the $\mathfrak{a}$-adci topology of M is compatible with the induced topology of $\mathfrak{a}$-adic topology of N on M. 
\end{remark}
\begin{lemma}
    Let A be a noetherian ring, I an injective $A$-module, $\mathfrak{a} \subseteq A$ ideal.
    Assume $J \subseteq I$ is the A-submodule consist of $x\in I$ such that $\mathfrak{a}^{n}x= 0$ for some $n > 0$.
    Then J is injective.
    \label{Lemma 7.1}
\end{lemma}
\begin{proof}
    It suffices to show that for all $\mathfrak{b} \subseteq A$ ideal and $\varphi: \mathfrak{b} \longrightarrow J$ homomorphism of $A$-modules,
    there exists $\psi: A \longrightarrow J$ extending $\varphi$ satisfies the following diagram commutes
    \begin{equation*}
        \begin{tikzcd}
            \mathbf{b} \arrow[r, hookrightarrow] \arrow[d, "\varphi"] &
            A \arrow[dl, "\psi"] \\
            J &
        \end{tikzcd}
    \end{equation*}
    \par
    Since A is noetherian, B is finitely generated.
    Thus there exists $n > 0$ such that $\mathfrak{a}^{n}\varphi(\mathfrak{b}) = 0$.
    By Artin-Rees, there exists $m > n$ such that $\mathfrak{a}^{n}\mathfrak{b} \supseteq \mathfrak{b} \cap \mathfrak{a}^{m}$.
    Thus $\varphi(\mathfrak{b} \cap \mathfrak{a}^{m}) = 0$ so that $\varphi$ factors through $\mathfrak{b}/(\mathfrak{b} \cap \mathfrak{a}^{m})$.
    Consider $\varphi_{I}: \mathfrak{b}/(\mathfrak{b} \cap \mathfrak{a}^{m}) \longrightarrow J \hookrightarrow I$.
    As I is injective, $\varphi_{I}$ extends to $\psi_{I}: A/\mathfrak{a}^{m} \longrightarrow I$.
    Note that $im(\psi_{I}) \subseteq J$ since $\mathfrak{a}^{m}im(\psi{I}) = 0$, get $\psi$. 
\end{proof}
\begin{lemma}
    Let A be a noetherian ring, I an injective $A$-module.
    Then for all $f\in A$, the localization $\theta: I \longrightarrow I_{f}$ is surjective.
    \label{Lemma 7.2}
\end{lemma}
\begin{proof}
    For all $i > 0$, set $\mathfrak{b}_{i} = ann_{A}(f^{i})$.
    Get an increasing sequence of ideals $\mathfrak{b}_{1} \subseteq \mathfrak{b}_{2} \subseteq \cdots$.
    As A is noetherian, $\{\mathfrak{b}_{i}\}$ stabilize from $\mathfrak{b}_{r}$ for some $r < \infty$.
    For $x\in I_{f}$, we can write $x = \frac{\theta(y)}{f^{n}}$ where $y\in I$ and $n > 0$.
    There is a canonical homomorphism $\varphi: (f^{n + r}) \longrightarrow I\quad f^{n + r} \longmapsto f^{r}y$.
    Easy to check that $\varphi$ is well defined.
    Since I is injective, $\varphi$ extends to $\psi: A \longrightarrow I$.
    Set $z = \psi(1)$, then $f^{n + r}z = f^{r}y$.
    Thus $\theta(z) = \frac{z}{1} = \frac{f^{n + r}z}{f^{n + r}} = \frac{f^{r}y}{f^{n + r}} = \frac{\theta(y)}{f^{n}} = x$ so that $\theta$ is surjective.
\end{proof}
\begin{proposition}
    Let A be a noetherian ring, I an injective $A$-module.
    Then $\widetilde{I}$ is flasque on $X = \spec A$.
    \label{Proposition 7.6}
\end{proposition}
\begin{proof}
    Set $Y = Supp(\widetilde{I}) = \{x\in X \big{|} \widetilde{I}_{x} \neq 0\} \subseteq X$ closed.
    If $Y = \varnothing$, then $\widetilde{I}$ is flasque.
    If $Y \neq \varnothing$, since A is noetherian, we can assume for all injective $A$-module J with support $Supp(\widetilde{J}) \subsetneq Y$, $\widetilde{J}$ is flasque.
    Want to show $\widetilde{I}(X) \longrightarrow \widetilde{I}(U)$ is surjective for all open subset $U \subseteq X$.
    \par
    If $Y \cap U = \varnothing$, then $\widetilde{I}(U) = 0$, done!
    If $Y \cap U \neq \varnothing$, take standard open subset $D(f) \subseteq U$ and $D(f) \cap Y \neq \varnothing$.
    Set $Z = X \setminus D(f)$.
    Let $s\in \widetilde{I}(U)$, then $s\big{|}_{D(f)} \subseteq \widetilde{I}(D(f)) = I_{f}$.
    By Lemma \ref{Lemma 7.2}, we can lift $s\big{|}_{D(f)}$ to $t\in \widetilde{I}(X) = I$.
    Thus $(s - t\big{|}_{U})\big{|}_{D(f)} = 0$ so that $(s - t\big{|}_{U})\in \Gamma_{Z \cap U}(U, \widetilde{I}\big{|}_{U})$ i.e. support of it is in Z.
    Remains to show $\Gamma_{Z}(X, \widetilde{I}) \longrightarrow \Gamma_{Z \cap U}(U, \widetilde{I}\big{|}_{U})$.
    \par
    Note that $J = \Gamma_{Z}(X, \widetilde{I})$ is exactly the submodule consist of $x\in I$ such that $f^{n}x = 0$ for some $n > 0$.
    By Lemma \ref{Lemma 7.1}, J is injective $A$-module with $Supp(\widetilde{J}) \subsetneq Y$.
    By induction hypothesis, $\widetilde{J}$ is flasque.
    Thus $\widetilde{J}(X) \longrightarrow \widetilde{J}(U)$ is surjective.
    While $\widetilde{J}(X) = \Gamma_{Z}(X, \widetilde{I})$ and $\widetilde{J}(U) = \Gamma_{Z \cap U}(U, \widetilde{I}\big{|}_{U})$, done!
\end{proof}
\begin{theorem}[\textbf{\emph{Serre}}]
    Let A be a noetherian ring and $X = \spec A$.
    Then for all $\mathcal{F}$ quasi-coherent sheaf in X, we have that $H^{i}(X, \mathcal{F}) = 0$ for all $i > 0$.
    \label{Theorem 7.3}
\end{theorem}
\begin{proof}
    For $\mathcal{F}$ quasi-coherent, set $M = \mathcal{F}(X)$.
    Take injective resolution $I^{\ast}$ of M in $Mod_{A}$.
    Thus $0 \longrightarrow \mathcal{F} \longrightarrow \widetilde{I^{0}} \longrightarrow \cdots$ is exact, which is a $\Gamma(X, \cdot)$-acyclic resolution by Proposition \ref{Proposition 7.6}.
    As $\widetilde{I^{i}}$ is $\Gamma(X, \cdot)$-acyclic, can apply $\Gamma(X, \cdot)$ to calculate $H^{i}(X, \mathcal{F})$.
    But $\Gamma(X, \cdot)$ gives back to $0 \longrightarrow M \longrightarrow I^{0} \longrightarrow \cdots$ exact, get $H^{i}(X, \mathcal{F}) = 0$ for all $i > 0$. 
\end{proof}
\begin{corollary}
    Let X be a noetherian affine scheme.
    Then for all exact sequence $0 \longrightarrow \mathcal{F} \longrightarrow \mathcal{G} \longrightarrow \mathcal{H} \longrightarrow 0$ of $\mathcal{O}_{X}$-module with $\mathcal{F}$ quasi-coherent,
    we have that $0 \longrightarrow \mathcal{F}(X) \longrightarrow \mathcal{G}(X) \longrightarrow \mathcal{H}(X) \longrightarrow 0$ is exact.
    \label{Corollary 7.4}
\end{corollary}\
\begin{corollary}
    Let X be a noetherian scheme.
    Suppose that $0 \longrightarrow \mathcal{F} \longrightarrow \mathcal{G} \longrightarrow \mathcal{H} \longrightarrow 0$ is an exact sequence of $\mathcal{O}_{X}$-module.
    If $\mathcal{F}$ and $\mathcal{H}$ are both quasi-coherent (resp. coherent), then $\mathcal{G}$ is quasi-coherent (resp. coherent).
    \label{Corollary 7.5} 
\end{corollary}
\begin{proof}
    Can assume that $X = \spec A$ is affine.
    We have a commutative diagram with exact rows
    \begin{equation*}
        \begin{tikzcd}
            0 \arrow[r] &
            \widetilde{\mathcal{F}(X)} \arrow[r] \arrow[d, "\sim"] &
            \widetilde{\mathcal{G}(X)} \arrow[r] \arrow[d] &
            \widetilde{\mathcal{H}(X)} \arrow[r] \arrow[d, "\sim"] & 
            0 \\
            0 \arrow[r] &
            \mathcal{F} \arrow[r] &
            \mathcal{G} \arrow[r] &
            \mathcal{H} \arrow[r] & 
            0
        \end{tikzcd}
    \end{equation*}
    Thus by 5 Lemma, $\mathcal{G}$ is quasi-coherent.
    \par
    If moreover $\mathcal{F}$ and $\mathcal{H}$ are coherent, then $\mathcal{F}(X)$ and $\mathcal{H}(X)$ are finitely generated.
    Then $\mathcal{G}(X)$ is also finitely generated so that $\mathcal{G}$ is coherent.
\end{proof}
\begin{lemma}
    Let X be a noetherian scheme.
    Assume that $H^{1}(X, \mathcal{I}) = 0$ for all $\mathcal{I}$ coherent ideal sheaf.
    Then we can cover X by affine open subsets of the form $X_{f}$ for $f\in A = \Gamma(X, \mathcal{O}_{X})$.
    \label{Lemma 7.3}
\end{lemma}
\begin{proof}
    Suffice to show that every closed point $x\in X$ admits an affine open neighbourhood $X_{f}$.
    For $x\in X$ closed point, choose affine open neighbourhood U of x and set $Y = X \setminus U$.
    Get $0 \longrightarrow \mathcal{I}_{Y \cup \{x\}} \longrightarrow \mathcal{I}_{Y} \longrightarrow i_{\ast}\mathcal{O}_{\{x\}} \longrightarrow 0$  is exact with reduced scheme structures on each set,
    where $i: \{x\} \longrightarrow X$ is inclusion.
    By assumption, $\Gamma(X, \mathcal{I}_{Y}) \longrightarrow \Gamma(X, i_{\ast}\mathcal{O}_{\{x\}})$ is surjective.
    Take $f\in \Gamma(X, \mathcal{I}_{Y})$ such that f is mapped to 1.
    Then $x\in X_{f}$.
    On the other hand, $X_{f} \subseteq U$, get $X_{f}$ is affine.
\end{proof}
\begin{lemma}
    Let X be a noetherian scheme.
    Assume that $H^{1}(X, \mathcal{I}) = 0$ for all $\mathcal{I}$ coherent ideal sheaf.
    Then for all $e > 0$ and $\mathcal{F} \subseteq \mathcal{O}_{X}^{\oplus r}$ coherent, we have that $H^{1}(X, \mathcal{F}) = 0$.
    \label{Lemma 7.4}
\end{lemma}
\begin{proof}
    Define filtration $\mathcal{F} \supseteq \mathcal{F} \cap \mathcal{O}_{X}^{\oplus (r - 1)} \supseteq \cdots \supseteq \mathcal{F} \cap \mathcal{O}_{X} \supseteq 0$ such that the associated quotients are coherent ideal sheaves.
    By long exact sequence and induction, get $H^{1}(X, \mathcal{F} \cap \mathcal{O}_{X}^{i}) = 0$ for all $i > 0$ so that $H^{1}(X, \mathcal{F}) = 0$.
\end{proof}
\begin{theorem}[\textbf{\emph{Serre, Cohomology Criterion of Affineness}}]
    Let X be a noetherian scheme.
    Then the following conditions are equivalent: \\
    (1)X is affine \\
    (2)$H^{i}(X, \mathcal{F}) = 0$ for all $i > 0$ and $\mathcal{F}$ quasi-coherent. \\
    (3)$H^{1}(X, \mathcal{I}) = 0$ for all $\mathcal{I}$ coherent ideal sheaf.
    \label{Theorem 7.4}
\end{theorem}
\begin{proof}
    See in the Hartshorne Chapter III p.215-216.
\end{proof}
\begin{remark}
    In fact, noetherian condition is not necessary.
\end{remark}
\subsection{Čech Cohomology}
\begin{definition}
    Let X be a topological space, $\mathcal{U} = \{U_{i}\}_{i\in I}$ open covering of X with well-ordered index set I, $\mathcal{F}$ sheaf of abelian groups on X.
    For $i_{0}, \cdots, i_{p}\in I$, set $U_{i_{0}, \cdots, i_{p}} = U_{i_{0}} \cap \cdots \cap U_{i_{p}}$.
    Define complex $C^{\ast}(\mathcal{I}, \mathcal{F})$ to be 
    \begin{equation*}
        C^{p}(\mathcal{U}, \mathcal{F}) = \prod_{i_{0}, \cdots, i_{p}} \mathcal{F}(U_{i_{0}, \cdots, i_{p}})
    \end{equation*}
    with coboundary map 
    \begin{equation*}
        \begin{split}
            d^{p}: C^{p}(\mathcal{U}, \mathcal{F}) & \longrightarrow C^{p + 1}(\mathcal{U}, \mathcal{F}) \\
            \alpha = (\alpha_{i_{0}, \cdots, i_{p}})_{i_{0}, \cdots, i_{p}} & \longmapsto (\sum_{k = 0}^{p + 1} (-1)^{k}\alpha_{i_{0}, \cdots, \widehat{i_{k}}, \cdots, i_{p + 1}}\big{|}_{U_{i_{0}, \cdots, i_{p + 1}}})_{i_{0}, \cdots, i_{p + 1}}
        \end{split}
    \end{equation*}
\end{definition}
\begin{remark}
    If we don't assume that I is well-ordered, we need to set $U_{i_{0}, \cdots, i_{p}} = 0$ if there are repeated index and $\alpha_{i_{0}, \cdots, i_{p}} = (-1)^{sgn(\sigma)}\alpha_{i_{\sigma(0)}, \cdots, i_{\sigma(p)}}$.
\end{remark}
\begin{definition}
    Let X be a topological space, $\mathcal{U} = \{U_{i}\}_{i\in I}$ open covering of X with well-ordered index set I, $\mathcal{F}$ sheaf of abelian groups on X.
    Define $\check{H}^{p}(\mathcal{U}, \mathcal{F}) = h^{p}(C^{\ast}(\mathcal{U}, \mathcal{F}))$, called the pth Čech cohomology of $\mathcal{F}$ with respect to $\mathcal{U}$.
\end{definition}
\begin{example}
    (1)For any $\mathcal{F}$ and $\mathcal{U}$, $\check{H}^{0}(\mathcal{U}, \mathcal{F}) = \Gamma(X, \mathcal{F})$. \\
    (2)For $\mathcal{U} = \{X\}$, $\check{H}^{p}(\mathcal{U}, \mathcal{F}) = 0$ for all $p > 0$.
\end{example}
\begin{lemma}
    Let X be a topological space, $\mathcal{U} = \{U_{i}\}_{i\in I}$ open covering of X with well-ordered index set I, $\mathcal{F}$ sheaf of abelian groups on X.
    Suppose that $\mathcal{U}$ contains $U_{i} = X$ for some i.
    Then $0 \longrightarrow \mathcal{F}(X) \longrightarrow C^{\ast}(\mathcal{U}, \mathcal{F})$ is exact.
    \label{Lemma 7.5}
\end{lemma}
\begin{proof}
    Obviously, exactness at $\mathcal{F}(X)$ and $C^{0}(\mathcal{U}, \mathcal{F})$ is ok.
    For $p \ge 1$, without loss of generality, may assume that $i = 0$ which is the smallest element.
    Define homotopy $k^{p}: C^{p}(\mathcal{U}, \mathcal{F}) \longrightarrow C^{p - 1}(\mathcal{U}, \mathcal{F})\quad \alpha \longmapsto (\alpha_{0, i_{0}, \cdots, i_{p - 1}})_{i_{0}, \cdots, i_{p - 1}}$.
    Check that $d^{p - 1} \circ k^{p} + k^{p + 1} \circ d^{p} = \identity_{C^{\ast}(\mathcal{U}, \mathcal{F})}$.
    Thus identity is homotopic to zero map, which inducing $\check{H}^{p}(\mathcal{U}, \mathcal{F}) = 0$ for all $p > 0$.. 
\end{proof}
\subsection{Comparison Theorem}
\begin{definition}
    Let X be a topological space, $\mathcal{U} = \{U_{i}\}_{i\in I}$ open covering of X with well-ordered index set I, $\mathcal{F}$ sheaf of abelian groups on X.
    Define complex of sheaves on X $\mathcal{C}^{\ast}(\mathcal{U}, \mathcal{F})$ to be
    \begin{equation*}
        \mathcal{C}^{p}(\mathcal{U}, \mathcal{F}) = \prod_{i_{0}, \cdots, i_{p}} f_{\ast}(\mathcal{F}\big{|}_{U_{i_{0}, \cdots, i_{p}}})
    \end{equation*}
    where f is the inclusion, with coboundary map
    \begin{equation*}
        d^{p}: \mathcal{C}^{p}(\mathcal{U}, \mathcal{F}) \longrightarrow \mathcal{C}^{p + 1}(\mathcal{U}, \mathcal{F})
    \end{equation*}
    defined accordingly.
\end{definition}
\begin{remark}
    Obviously, $\Gamma(X, \mathcal{C}^{p}(\mathcal{U}, \mathcal{F})) = C^{p}(\mathcal{U}, \mathcal{F})$.
\end{remark}
\begin{proposition}
    Let X be a topological space, $\mathcal{U} = \{U_{i}\}_{i\in I}$ open covering of X with well-ordered index set I, $\mathcal{F}$ sheaf of abelian groups on X.
    Then $0 \longrightarrow \mathcal{F} \longrightarrow \mathcal{C}^{\ast}(\mathcal{U}, \mathcal{F})$ is exact.
    \label{Proposition 7.7}
\end{proposition}
\begin{reason}
    Since exactness of complex of sheaves is equivalent to exactness at all stalks.
    Set $x\in X$, we can represent elements in stalks by sections on V, which is small enough to be contained in $U_{j}$ for some j.
    Apply Lemma \ref{Lemma 7.5}, get exactness of complex of sections on V.
\end{reason}
\begin{proposition}
    Let X be a topological space, $\mathcal{U} = \{U_{i}\}_{i\in I}$ open covering of X with well-ordered index set I, $\mathcal{F}$ flasque sheaf of abelian groups on X.
    Then $\check{H}^{p}(\mathcal{U}, \mathcal{F}) = 0$ for all $p > 0$.
    \label{Proposition 7.8}
\end{proposition}
\begin{proof}
    Since flasque is stable under restriction to open subset, direct limit and product, $\mathcal{C}^{p}(\mathcal{U}, \mathcal{F})$ is flasque for all p.
    Thus $\mathcal{C}^{\ast}(\mathcal{U}, \mathcal{F})$ is a flasque resolution of $\mathcal{F}$.
    By Proposition \ref{Proposition 7.5} and Proposition \ref{Proposition 7.1}, get
    \begin{equation*}
        \check{H}^{p}(\mathcal{U}, \mathcal{F}) = h^{p}(\Gamma(X, \mathcal{C}^{\ast}(\mathcal{U}, \mathcal{F}))) = H^{p}(X, \mathcal{F}) = 0
    \end{equation*}
\end{proof}
\begin{lemma}
    Let X be a topological space, $\mathcal{U} = \{U_{i}\}_{i\in I}$ open covering of X with well-ordered index set I, $\mathcal{F}$ sheaf of abelian groups on X.
    Then for all $p \ge 0$, there is a natural homomorphism $\check{H}^{p}(\mathcal{U}, \mathcal{F}) \longrightarrow H^{p}(X, \mathcal{F})$.
    \label{Lemma 7.6}
\end{lemma}
\begin{proof}
    Let $0 \longrightarrow \mathcal{F} \longrightarrow \mathcal{I}^{\ast}$ be an injective resolution of $\mathcal{F}$ in $Ab(X)$. 
    By Comparison Theorem in Homological Algebra, there exists a morphism of complexes which is unique up to homotopy,
    inducing the natural homomorphisms we want.
\end{proof}
\begin{theorem}[\textbf{\emph{Comparison Theorem}}]
    Let X be a topological space, $\mathcal{U} = \{U_{i}\}_{i\in I}$ open covering of X with well-ordered index set I, $\mathcal{F}$ sheaf of abelian groups on X.
    Suppose that for all p and all $i_{0}, \cdots, i_{p}$, we have that $H^{q}(U_{i_{0}, \cdots, i_{p}}, \mathcal{F}\big{|}_{U_{i_{0}, \cdots, i_{p}}}) = 0$ for all $q > 0$.
    Then the natural homomorphism $\check{H}^{p}(\mathcal{U}, \mathcal{F}) \longrightarrow H^{p}(X, \mathcal{F})$ is isomorphic for all p.
    \label{Theorem 7.5}
\end{theorem}
\begin{proof}
    For $p = 0$, the natural homomorphism is identity map of global sections of $\mathcal{F}$.
    For $p > 0$, embed $\mathcal{F}$ into an injective sheaf $\mathcal{I}$.
    There is an exact sequence
    \begin{equation*}
        0 \longrightarrow \mathcal{F} \longrightarrow \mathcal{I} \longrightarrow \mathcal{G} \longrightarrow 0
    \end{equation*}
    By assumption, for all $i_{0}, \cdots, i_{p}$,
    \begin{equation*}
        0 \longrightarrow \mathcal{F}(U_{i_{0}, \cdots, i_{p}})\longrightarrow \mathcal{I}(U_{i_{0}, \cdots, i_{p}}) \longrightarrow \mathcal{G}(U_{i_{0}, \cdots, i_{p}}) \longrightarrow 0
    \end{equation*}
    is exact.
    Taking products, get
    \begin{equation*}
        0 \longrightarrow C^{\ast}(\mathcal{U}, \mathcal{F}) \longrightarrow C^{\ast}(\mathcal{U}, \mathcal{I}) \longrightarrow C^{\ast}(\mathcal{U}, \mathcal{G}) \longrightarrow 0
    \end{equation*}
    By Homological Algebra, there is an long exact sequence of cohomological groups
    \begin{equation*}
        \cdots \longrightarrow \check{H}^{p - 1}(\mathcal{U}, \mathcal{F}) \longrightarrow \check{H}^{p - 1}(\mathcal{U}, \mathcal{I}) \longrightarrow \check{H}^{p - 1}(\mathcal{U}, \mathcal{G}) \longrightarrow \check{H}^{p}(\mathcal{U}, \mathcal{F}) \longrightarrow \cdots
    \end{equation*}
    Since $\mathcal{I}$ is injective, get $\mathcal{I}$ is flasque and $\check{H}^{p}(\mathcal{U}, \mathcal{I}) = 0$ for all $p > 0$.
    Thus $\check{H}^{p}(\mathcal{U}, \mathcal{F}) = \check{H}^{p - 1}(\mathcal{U}, \mathcal{G})$.
    By induction, done!
\end{proof}
\begin{corollary}
    Let X be a (noetherian) separated scheme, $\mathcal{U}$ affine open covering of X, $\mathcal{F}$ quasi-coherent sheaf on X.
    Then the natural homomorphism $\check{H}^{p}(\mathcal{U}, \mathcal{F}) \longrightarrow H^{p}(X, \mathcal{F})$ is isomorphic for all p.
    \label{Corollary 7.6}
\end{corollary}
\begin{remark}
    This corollary immediately comes from Serre Vanishing Theorem.
\end{remark}
\begin{corollary}
    Let X be a (noetherian) separated scheme covered by $r + 1$ affine open subsets.
    Then for all $\mathcal{F}$ quasi-coherent on X, we have that $H^{i}(X, \mathcal{F}) = 0$ for all $i > r$.
    In particular, when $r = 0$, this specializes to Serre Vanishing Theorem.
    \label{Corollary 7.7}
\end{corollary}
\subsection{Cohomology of Projective Space}
\label{subsection:Cohomology of Projective Space}
\begin{theorem}
    Let A be a noetherian ring, $S= A[x_{0}, \cdots, x_{r}]$, $X = \proj(S) = \mathbb{P}_{A}^{r}$, $r \ge 1$.
    Then we have that \\ 
    (1)$H^{i}(X, \mathcal{O}_{X}(n)) = 0$ for all $0 < i < r$ and all $n\in \mathbb{Z}$ \\
    (2)$H^{r}(X, \mathcal{O}_{X}(-r - 1)) \cong A$ is free $A$-module of rank 1 \\
    (3)the natural pairing $H^{0}(X, \mathcal{O}_{X}(n)) \times H^{r}(X, \mathcal{O}_{X}(-n - r - 1)) \longrightarrow H^{r}(X, \mathcal{O}_{X}(-r - 1))$ is a perfect pairing of finite free $A$-modules for all $n\in \mathbb{Z}$.
    \label{Theorem 7.6}
\end{theorem}
\begin{remark}
    In fact, we have already known that $H^{0}(X, \mathcal{O}_{X}(n)) =
    \left\{
        \begin{aligned}
            & S_{n} & n \ge 0 \\
            & 0 & n < 0
        \end{aligned}
    \right.$
    and $H^{i}(X, \mathcal{O}_{X}(n)) = 0$ for all $i > r$ and all $n\in \mathbb{Z}$.
    Thus with this theorem, we have figured out all the cohomological groups of $\mathbb{P}_{A}^{r}$.
    Especially, all of them are finite free $A$-modules.
\end{remark}
\begin{proof}
    Consider $\mathcal{F} = \oplus_{n\in \mathbb{Z}} \mathcal{O}_{X}(n)$ quasi-coherent.
    Since cohomology commutes with direct sum, $H^{i}(X, \mathcal{F}) = \oplus_{n\in \mathbb{Z}} H^{i}(X, \mathcal{F})$.
    In particular, $H^{0}(X, \mathcal{F}) = \Gamma_{\ast}(\mathcal{O_{X}}) \cong S$.
    Note that X is noetherian, separated and covered by $r + 1$ affine open subsets.
    By Corollary \ref{Corollary 7.7}, get $\check{H}^{p}(\mathcal{U}, \mathcal{F}) \cong H^{p}(X, \mathcal{F})$ for all p.
    \par
    For all $i_{0}, \cdots, i_{p}$, note that $U_{i_{0}, \cdots, i_{p}} = D_{+}(x_{i_{0}}, \cdots, x_{i_{p}})$.
    Get $\mathcal{F}(U_{i_{0}, \cdots, i_{p}}) \cong S_{x_{i_{0}}, \cdots, x_{i_{p}}}$. 
    Thus $C^{\ast}(\mathcal{U}, \mathcal{F})$ is given by 
    \begin{equation*}
        0 \longrightarrow \prod_{i_{0}} S_{x_{i_{0}}} \longrightarrow \cdots \longrightarrow S_{x_{0}, \cdots, x_{r}}.
    \end{equation*}
    Since $\check{H}^{r}(\mathcal{U}, \mathcal{F})$ is the cokernel of $d^{r - 1}: \prod_{k} S_{x_{0}, \cdots, \widehat{x_{k}}, \cdots, x_{r}} \longrightarrow S_{x_{0}, \cdots, x_{r}}$.
    Note that $S_{x_{0}, \cdots, x_{r}}$ is free $A$-module spanned by $\{x_{0}^{l_{0}}..x_{r}^{l_{r}}\}$, where $l_{0}, \cdots, l_{r}\in \mathbb{Z}$.
    In addition, $im(d^{r - 1})$ is free $A$-module spanned by $\{x_{0}^{l_{0}}\cdots x_{r}^{l_{r}}\}$, where $l_{0}, \cdots, l_{r}\in \mathbb{Z}$ and at least one $l_{i} \ge 0$.
    Get $H^{r}(\mathcal{U}, \mathcal{F})$ is free $A$-module spanned by $\{x_{0}^{l_{0}}\cdots x_{r}^{l_{r}}\}$, where $l_{i} < 0$.
    While the only "monomial" of degree $-r - 1$ is $x_{0}^{-1}\cdots x_{r}^{-1}$, get $H^{r}(X, \mathcal{O}_{X}(-r - 1)) \cong A$.
    \par
    For (3), we also find $H^{r}(X, \mathcal{O}_{X}(-n - r - 1)) = 0$ if $n < 0$.
    On the other hand, $H^{0}(X, \mathcal{O}_{X}(n)) = 0$ if $n < 0$.
    Thus, (3) is trivially true for $n < 0$.
    For $n \ge 0$, $H^{0}(X, \mathcal{O}_{X}(n))$ has basis $\{x_{0}^{{m_{0}}}\cdots x_{r}^{m_{r}}\}$,
    where $m_{i} \ge 0$ and $\sum_{i = 0}^{r} m_{i} = n$.
    The natural pairing is generated by $(x_{0}^{m_{0}}\cdots x_{r}^{m_{r}}, x_{0}^{l_{0}}\cdots x_{r}^{l_{r}}) \longmapsto x_{0}^{m_{0} + l_{0}}\cdots x_{r}^{m_{r} + l_{r}}$,
    here that $x_{i}^{m_{i} + l_{i}} = 0$ if $m_{i} + l_{i} \ge 0$.
    Then it is a perfect pairing since it is easy to see that dual basis of $x_{0}^{m_{0}}\cdots x_{r}^{m_{r}}$ is $x_{0}^{-m_{0} - 1}\cdots x_{r}^{-m_{r} - 1}$.
    \par
    For (1), take induction on r.
    When $r = 1$, there is nothing to prove.
    Suppose that $r \ge 2$.
    Consider localization of $C^{\ast}(\mathcal{U}, \mathcal{F})$ with respect to $x_{r}$.
    By Theorem \ref{Theorem 2.1}, the localization gives the Čech complex of $\mathcal{F}\big{|}_{U_{r}}$ with respect to $\mathcal{U}\big{|}_{U_{r}}$.
    As localization is exact and $U_{r}$ is affine, $H^{i}(X, \mathcal{F})_{x_{r}} = H^{i}(U_{r}, \mathcal{F}\big{|}_{U_{r}}) = 0$.
    Suffice to show that for all $0 < i < r$, $x_{r}$ acts injectively on $H^{i}(X, \mathcal{F})$.
    View $x_{r}$ as a homomorphism of $A$-modules like below
    \begin{equation*}
        x_{r}: H^{i}(X, \mathcal{F}(-1)) \longrightarrow H^{i}(X, \mathcal{F})
    \end{equation*}
    here $\mathcal{F}(-1)$ is same sheaf on $\mathcal{F}$ with grading shifted.
    Consider exact sequence of graded $C$-modules 
    \begin{equation*}
        0 \longrightarrow S(-1) \overset{x_{r}}{\longrightarrow} S \longrightarrow S/(x_{r}) \longrightarrow 0
    \end{equation*}
    Get exact sequence of sheaves 
    \begin{equation*}
        0 \longrightarrow \mathcal{O}_{X}(-1) \overset{x_{r}}{\longrightarrow} \mathcal{O}_{X} \longrightarrow j_{\ast}\mathcal{O}_{H} \longrightarrow 0
    \end{equation*}
    where H is the hyperplane given by $x_{r} = 0$ and $j: H \hookrightarrow X$ is closed immersion. 
    Take direct sum, get exact sequence
    \begin{equation*}
        0 \longrightarrow \mathcal{F}(-1) \overset{x_{r}}{\longrightarrow} \mathcal{F} \longrightarrow \mathcal{F}_{H} \longrightarrow 0
    \end{equation*}
    where $\mathcal{F}_{H} = \oplus_{n\in \mathbb{Z}} j_{\ast}(\mathcal{O}_{H}(n))$. 
    Get long exact sequence
    \begin{equation*}
        \cdots \longrightarrow H^{i - 1}(X, \mathcal{F}_{H}) \longrightarrow H^{i}(X, \mathcal{F}(-1)) \overset{x_{r}}{\longrightarrow} H^{i}(X, \mathcal{F}) \longrightarrow H^{i}(X, \mathcal{F})_{H} \longrightarrow \cdots
    \end{equation*}
    Now $H \cong \mathbf{P}^{r - 1}_{A}$ and since j is closed immersion, $H^{i}(X, \mathcal{F}_{H}) = H^{i}(H, \oplus_{n\in \mathbb{Z}} O_{H}(n))$.
    By induction hypothesis, $H^{i}(X, \mathcal{F}_{H}) = 0$ for all $0 < i < r - 1$.
    Thus $x_{r}$ is injective on $H^{i}(X, \mathcal{F}(-1))$ for all $1 < i < r$.
    When $i = 1$, find that $H^{0}(X, \mathcal{F}) = S \longrightarrow H^{0}(X, \mathcal{F}_{H}) = S/(x_{r})$ is surjective.
    Get $x_{r}$ is injective on $H^{i}(X, \mathcal{F}(-1))$ for all $0 < i < r$, done!
\end{proof}
\begin{theorem}[\textbf{\emph{Serre}}]
    Let $A$ be a noetherian ring, $X$ projective scheme over $A$, $\mathcal{F}$ coherent sheaf on X.
    Then \\
    (1)for all $i \ge 0$, $H^{i}(X, \mathcal{F})$ is a finite $A$-module. \\
    (2)for $j: X \hookrightarrow \mathbb{P}_{A}^{r}$ closed immersion, there exists integer $n_{0}$ relative to $\mathcal{F}$ such that $H^{i}(X, \mathcal{F}(n)) = 0$ for all $i > 0$ and $n \ge n_{0}$.
    \label{Theorem 7.7}
\end{theorem}
\begin{proof}
    Since $j$ is finite, $j_{\ast}(\mathcal{F})$ is coherent on $\mathbb{P}_{A}^{r}$ and by Projection Formula, $j_{\ast}(\mathcal{F}(n)) = (j_{\ast}(\mathcal{F}))(n)$.
    What's more, since j is closed immersion, $H^{i}(X, \mathcal{F}) = H^{i}(\mathbb{P}_{A}^{r}, j_{\ast}(\mathcal{F}))$.
    Thus we can reduce this problem to $X = \mathbb{P}_{A}^{r}$.
    For $\mathcal{F} = \mathcal{O}_{X}(m)$, this is just things told by \ref{Theorem 7.6}.
    Also, for general $\mathcal{F}$, as $\mathbb{P}_{A}^{r}$ can be covered by $r + 1$ affine open subsets, 
    $H^{i}(X, \mathcal{F}(n)) = 0$ for all $i > r$ and $n\in \mathbb{Z}$.
    \par
    For (1), by induction, suppose theorem is correct for $i + 1$.
    By Corollary \ref{Corollary 6.1}, there exists $m\in \mathbb{Z}$ and $r < \infty$ such that $\mathcal{F}$ is quotient of $\mathcal{O}_{X}(m)^{r}$,
    inducing the following exact sequence of coherent sheaves
    \begin{equation*}
        0 \longrightarrow \mathcal{G} \longrightarrow \mathcal{O}_{X}(m)^{r} \longrightarrow \mathcal{F} \longrightarrow 0
    \end{equation*} 
    Consider the long exact sequence
    \begin{equation*}
        \cdots \longrightarrow H^{i}(X, \mathcal{O}_{X}(m)^{r}) \longrightarrow H^{i}(X, \mathcal{F}) \longrightarrow H^{i + 1}(X, \mathcal{G}) \longrightarrow \cdots
    \end{equation*}
    By hypothesis, $H^{i + 1}(X, \mathcal{G})$ is finite $A$-module.
    As $A$ is noetherian, $H^{i}(X, \mathcal{F})$ is also finite.
    \par
    For (2), by above discuss, we only need to check for $0 < i \le r$.
    Induct on i, assume for $i + 1$ and all coherent sheaf $\mathcal{F}$,
    there exists $n_{i + 1}(\mathcal{F})$ such that $H^{i + 1}(X, \mathcal{F}(n)) = 0$ for all $n \ge n_{i + 1}(\mathcal{F})$.
    Take same short exact sequence.
    Tensor the short exact sequence by $\mathcal{O}_{X}(n)$.
    Get exact sequence
    \begin{equation*}
        0 \longrightarrow \mathcal{G}(n) \longrightarrow \mathcal{O}_{X}(m + n)^{r} \longrightarrow \mathcal{F}(n) \longrightarrow 0
    \end{equation*}
    Consider the long exact sequence
    \begin{equation*}
        \cdots \longrightarrow H^{i}(X, \mathcal{O}_{X}(m + n)^{r}) \longrightarrow H^{i}(X, \mathcal{F}(n)) \longrightarrow H^{i + 1}(X, \mathcal{G}(n)) \longrightarrow \cdots
    \end{equation*}
    By hypothesis, there exists $n_{i + 1}(\mathcal{G})$ such that $H^{i + 1}(X, \mathcal{G}(n)) = 0$ for all $n \ge n_{i + 1}(\mathcal{G})$.
    By Theorem \ref{Theorem 7.6}, $H^{i}(X, \mathcal{O}_{X}(m + n)^{r}) = 0$ for large enough $n$.
    Thus there exists $n_{i}(\mathcal{F})$ such that $H^{i}(X, \mathcal{F}(n)) = 0$ for all $n \ge n_{i}(\mathcal{F})$.
    As there are finitely many $i$, we can take a uniform upper $n_{0}(\mathcal{F})$ bound for $n_{i}(\mathcal{F})$.
\end{proof}
\begin{corollary}
    Let $f: X \longrightarrow Y$ be a projective morphism with $X, Y$ noetherian, $\mathcal{F}$ coherent sheaf on $X$.
    Then $f_{\ast}\mathcal{F}$ is coherent on $Y$.
    \label{Corollary 7.8}
\end{corollary}
\begin{reason}
    Can assume $Y = \spec A$.
    Then $f_{\ast}\mathcal{F} \cong \widetilde{H^{0}(X, \mathcal{F})}$ which is a finite $A$-module by Theorem \ref{Theorem 7.7}.
\end{reason}
\begin{corollary}
    Let $X$ be a geometrically integral scheme projective over field $k$.
    Then $\Gamma(X, \mathcal{O}_{X}) = k$.
    \label{Corollary 7.9} 
\end{corollary}
\begin{reason}
    By Theorem \ref{Theorem 7.7}, $L = \Gamma(X, \mathcal{O}_{X})$ is finite over $k$.
    While $X$ is integral, get $L$ is integral domain.
    Thus by Noetherian Normalization Lemma, $L/k$ is finite field extension.
    Note by Proposition \ref{Proposition 3.11}, geometrically integral if and only if $k$ is algebraically closed in function field $k(X)$.
    Get $L = k$.
\end{reason}
\begin{corollary}
    Let $A$ be a noetherian ring, $X$ proper over $\spec A$, $\mathcal{L}$ ample invertible sheaf on X.
    Then for all $\mathcal{F}$ coherent sheaf on X, there exists $n_{0}$ relative to $\mathcal{F}$ such that $H^{i}(X, \mathcal{F} \otimes_{\mathcal{O}_{X}} \mathcal{L}^{n}) = 0$ for all $i > 0$ and $n \ge n_{0}$. 
    \label{Corollary 7.10}
\end{corollary}
\begin{reason}
    By Theorem \ref{Theorem 6.6}, there exists $m > 0$ such that $\mathcal{L}^{m}$ is very ample.
    Get closed immersion $i: X \hookrightarrow \mathbb{P}_{A}^{r}$ such that $\mathcal{L}^{n} = \mathcal{O}_{X}(1)$.
    Apply Theorem \ref{Theorem 7.7} to $\mathcal{F}, \mathcal{F} \otimes_{\mathcal{O}_{X}} \mathcal{L}, \cdots, \mathcal{F} \otimes_{\mathcal{O}_{X}} \mathcal{L}^{m - 1}$.
\end{reason}
\begin{theorem}[\textbf{\emph{Serre Cohomological Criteria for Ampleness}}]
    Let $A$ be a noetherian ring, $X$ proper over $\spec A$, $\mathcal{L}$ invertible sheaf on X.
    Then the following conditions are equivalent \\
    (1)$\mathcal{L}$ is ample \\
    (2)for all $\mathcal{F}$ coherent sheaf on X, 
    there exists $n_{0}$ relative to $\mathcal{F}$ such that $H^{i}(X, \mathcal{F} \otimes_{\mathcal{O}_{X}} \mathcal{L}^{n}) = 0$ for all $i > 0$ and $n \ge n_{0}$. \\
    (3)for all $\mathcal{F}$ coherent sheaf on X, 
    there exists $n_{0}$ relative to $\mathcal{F}$ such that $H^{1}(X, \mathcal{F} \otimes_{\mathcal{O}_{X}} \mathcal{L}^{n}) = 0$.
    \label{Theorem 7.8}
\end{theorem}
\begin{proof}
    Step 1.
    Since by Corollary \ref{Corollary 7.10}, we have $(1) \Rightarrow (2)$, and $(2) \Rightarrow (3)$ is obvious.
    Only need to prove $(3) \Rightarrow (1)$.
    Moreover, suffices to prove that for all $\mathcal{F}$ coherent sheaf on $X$ and closed point $x\in X$,
    there exists open neighbourhood $U_{x}\ni x$ and $n_{0}(x) > 0$ such that for all $n \ge n_{0}(x)$ and $y\in U_{x}$,
    the stalk $(\mathcal{F} \otimes_{\mathcal{O}_{X}} \mathcal{L}^{n})_{y}$ is generated by global sections of $\mathcal{F} \otimes_{\mathcal{O}_{X}} \mathcal{L}^{n}$,
    since $X$ noetherian, it can be covered by only finitely many $U_{x}$ so that we can take a uniform upper bound of $n_{0}(x)$.
    \par
    Step 2.
    Claim that it suffices to prove that for all $\mathcal{F}$ coherent sheaf on $X$, closed point $x\in X$ and large enough $n$,
    there exists open neighbourhood $U_{x}\ni x$ depending on $n$ such that for all  $y\in U_{x}$,
    the stalk $(\mathcal{F} \otimes_{\mathcal{O}_{X}} \mathcal{L}^{n})_{y}$ is generated by global sections of $\mathcal{F} \otimes_{\mathcal{O}_{X}} \mathcal{L}^{n}$.
    Assume we have proved this.
    Apply it to $\mathcal{F} = \mathcal{O}_{X}$, there exists large enough $n_{1}$ and $V\ni x$ such that for all $y\in V$, $(\mathcal{L}^{n_{1}})_{y}$ is generated by global sections.
    Then apply this to $\mathcal{F}$, there exists large enough $n_{0}$ and open neighbourhoods $U_{0}, U_{1}, \cdots, U_{n_{1} - 1}$ of $x$ such that for all $i$ and $y\in U_{i}$,
    $(\mathcal{F} \otimes_{\mathcal{O}_{X}} \mathcal{L}^{n + i})_{y}$ is generated by global sections.
    Take $U_{x} = V \cap U_{0} \cap \cdots \cap U_{n_{1} - 1}$.
    For all $n \ge n_{0}$ and $y\in U_{x}$, assume $n = n_{0} + mn_{1} + i$, 
    then $(\mathcal{F} \otimes_{\mathcal{O}_{X}} \mathcal{L}^{n})_{y} = (\mathcal{F} \otimes_{\mathcal{O}_{X}} \mathcal{L}^{n_{0} + i})_{y} \otimes_{\mathcal{O}_{X, y}} (\mathcal{L}^{n_{1}})_{y}^{\otimes m}$ is generated by global sections.
    \par
    Step 3.
    Let $x\in X$ be a closed point, $\mathcal{I}_{x}$ the ideal sheaf of $\{x\} \overset{i}{\hookrightarrow} X$,
    where $\{x\}$ has the same scheme structure as $\spec(k(x))$ with structure sheaf denoted by $k(x)$ for convenience.
    We have short exact sequence
    \begin{equation*}
        0 \longrightarrow \mathcal{I}_{X} \longrightarrow \mathcal{O}_{x} \longrightarrow i_{\ast}k(x) \longrightarrow 0
    \end{equation*}
    Tensor by $\mathcal{F}$, get
    \begin{equation*}
        \mathcal{I}_{x}\mathcal{F} \longrightarrow \mathcal{F} \longrightarrow i_{\ast}k(x) \otimes_{\mathcal{O}_{X}} \mathcal{F} \longrightarrow 0
    \end{equation*}
    Tensor by $\mathcal{L}^{r}$, get
    \begin{equation*}
        \mathcal{I}_{x}\mathcal{F} \otimes_{\mathcal{O}_{X}} \mathcal{L}^{n} \longrightarrow \mathcal{F} \otimes_{\mathcal{O}_{X}} \mathcal{L}^{n} \longrightarrow i_{\ast}k(x) \otimes_{\mathcal{O}_{X}} \mathcal{F} \otimes_{\mathcal{O}_{x}} \mathcal{L}^{n}\longrightarrow 0
    \end{equation*}
    Note that, by hypothesis, $H^{1}(X, \mathcal{I}_{X}\mathcal{F} \otimes_{\mathcal{O}_{X}} \mathcal{L}^{n}) = 0$ for all $n \ge n_{0}$.
    So $\Gamma(X, \mathcal{F} \otimes_{\mathcal{O}_{X}} \mathcal{L}^{n}) \longrightarrow \Gamma(X, i_{\ast}k(x) \otimes_{\mathcal{O}_{X}} \mathcal{F} \otimes_{\mathcal{O}_{X}} \mathcal{L}^{n})$ is surjective for all $n \ge n_{0}$.
    Note that $i_{\ast}k(x) \otimes_{\mathcal{O}_{X}} \mathcal{F} \otimes_{\mathcal{O}_{X}} \mathcal{L}^{n}$ is only supported at $x$,
    we have that $\Gamma(X, i_{\ast}k(x) \otimes_{\mathcal{O}_{X}} \mathcal{F} \otimes_{\mathcal{O}_{X}} \mathcal{L}^{n}) = (i_{\ast}k(x) \otimes_{\mathcal{O}_{X}} \mathcal{F} \otimes_{\mathcal{O}_{X}} \mathcal{L}^{n})_{x} = k(x) \otimes_{\mathcal{O}_{X, x}} (\mathcal{F} \otimes_{\mathcal{O}_{X}} \mathcal{L}^{n})_{x}$.
    Denote $(\mathcal{F} \otimes_{\mathcal{O}_{X}} \mathcal{L}^{n})_{x}$ by $M$.
    As $M/\mathfrak{m}_{x}M$ is generated by global sections, by Nakayama's Lemma, $M$ is generated by global sections.
    \par
    As $\mathcal{F} \otimes_{\mathcal{O}_{X}} \mathcal{L}^{n}$ is coherent,
    take an affine open neighbourhood $U = \spec A$ of $x$ so that $(\mathcal{F} \otimes_{\mathcal{O}_{X}} \mathcal{L}^{n})\big{|}_{U} = \widetilde{N}$ for some finitely generated $A$-module.
    Assume $x$ corresponds to maximal ideal $\mathfrak{m}$ and $N$ is generated by $m_{1}, \cdots, m_{l}$.
    Since $N_{\mathfrak{m}}$ is generated by global sections, $\frac{m_{k}}{1} = \frac{\sum_{j} a_{kj}s_{j, x}}{b_{k}}$,
    where $a_{kj}\in A$, $b_{k}\notin \mathfrak{m}$ and $s_{j}$ are global sections.
    Then there exists $c_{k}\notin \mathfrak{m}$ such that $(b_{k}m_{k} - \sum_{j} a_{kj}s_{j, x})c_{k} = 0$.
    Consider $b = \prod_{k = 1}^{l} b_{k}c_{k}$, then $\mathfrak{m}\in D(b)$.
    Obviously, for each $\mathfrak{p}\in D(b)$, $\frac{m_{k}}{1}$ can be generated by global sections in $N_{\mathfrak{p}}$.
    Thus for all $n \ge n_{0}$, there exists open neighbourhood $U_{x}\ni x$ depending on $n$ such that for all $y\in U_{x}$,
    $(\mathcal{F} \otimes_{\mathcal{O}_{X}} \mathcal{L}^{n})_{y}$ is generated by global sections, done!
\end{proof}
\begin{remark}
    In fact, all finiteness results still hold for $X$ proper over $A$.
    Prooves can be seen in EGA or Stacks Project.
\end{remark}
\subsection{Higher Direct Images}
\begin{definition}
    Let $f: X \longrightarrow Y$ be a morphism of schemes (can be replaced by topological spaces or ringed space).
    For all $i \ge 0$, define $R^{i}f_{\ast}: Ab(X) \longrightarrow Ab(Y)$ as the right derived functor of $f_{\ast}: Ab(X) \longrightarrow Ab(Y)$,
    which is called the higher direct image of $f$.
\end{definition}
\begin{remark}
    In fact, for $\mathcal{F} \in Ab(X)$, $R^{i}f_{\ast}\mathcal{F}$ is the sheaf associated with the presheaf $V \subseteq Y \longmapsto H^{i}(f^{-1}(V), \mathcal{F}\big{|}_{f^{-1}(V)})$.
    Details can be seen in Hartshorne Chapter III section 8. 
    In particular, flasque sheaves are acyclic for the functor $f_{\ast}$.
    It follows that the derived functors of $f_{\ast}: Mod(\mathcal{O}_{X}) \longrightarrow Mod(\mathcal{O}_{Y})$ coincide with $R^{i}f_{\ast}$.
    \par
    Moreover, if $X$ is noetherian and $Y = \spec A$ is affine,
    then for all $\mathcal{F}$ quasi-coherent sheaf on X, we have that $R^{i}f_{\ast}\mathcal{F} \cong \widetilde{H^{i}(X, \mathcal{F})}$.
    Details can also be seen in Hartshorne Chapter III section 8. 
    \par
    In particular, for $f: X \longrightarrow Y$ morphism of schemes with X noetherian and $\mathcal{F}$ quasi-coherent sheaf on X,
    the derived functor $R^{i}f_{\ast}\mathcal{F}$ is quasi-coherent on Y.
    Also true if $f: X \longrightarrow Y$ is quasi-compact and separated.
\end{remark}
\begin{proposition}
    Let $f: X \longrightarrow Y$ be an affine morphism between noetherian schemes.
    Then for $\mathcal{F}$ quasi-coherent on X, we have \\
    (1)$R^{i}f_{\ast}\mathcal{F} = 0$ for all $i > 0$ \\
    (2)$H^{i}(X, \mathcal{F}) = H^{i}(Y, f_{\ast}\mathcal{F})$ for all $i$ \\
    \label{Proposition 7.9}
\end{proposition}
\begin{proof}
    For (1), note that $R^{i}f_{\ast}\mathcal{F}$ is the sheaf associated with the presheaf $V \subseteq Y \longmapsto H^{i}(f^{-1}(V), \mathcal{F}\big{|}_{f^{-1}(V)})$,
    consider affine open subset $V \subseteq Y$.
    Since $f$ is affine, $f^{-1}(V)$ is affine.
    Note $\mathcal{F}\big{|}_{f^{-1}(V)}$ is still quasi-coherent on $f^{-1}(V)$,
    by Theorem \ref{Theorem 7.4}, get $H^{i}(f^{-1}(V), \mathcal{F}\big{|}_{f^{-1}(V)}) = 0$ for all $i > 0$.
    By definition of sheafification, it is easy to see that $R^{i}f_{\ast}\mathcal{F} = 0$.
    \par
    For (2), the general proof, we need to use property of spectrum sequence.
    Here, we only gieve a proof assuming $Y$ separated.
    Take affine open covering $\{V_{i}\}_{i\in I}$ of $Y$.
    Since $f$ is affine, get $\{U_{i} = f^{-1}(V_{i})\}_{i\in I}$ is an affine open covering of $X$.
    As we can compute cohomology with Čech complexes, suffices to show they have same complexes.
    For open covering $\{V_{i}\}_{i\in I}$ of $Y$ and $f_{\ast}\mathcal{F}$, the Čech complex should be
    \begin{equation*}
        0 \longrightarrow f_{\ast}(\mathcal{F})(Y) = \mathcal{F}(X) \longrightarrow \prod_{i\in I} f_{\ast}\mathcal{F}(V_{i}) = \prod_{i\in I} \mathcal{F}(U_{i}) \longrightarrow \cdots   
    \end{equation*}
    Thus $H^{i}(X, \mathcal{F}) = H^{i}(Y, f_{\ast}\mathcal{F})$ for all $i$.
\end{proof}
\begin{proposition}
    Let $f: X\longrightarrow Y$ be a projective morphism between noetherian schemes.
    Then for each $\mathcal{F}$ coherent sheaf on X, $R^{k}f_{\ast}\mathcal{F}$ is coherent on $Y$ for all $k \ge 0$.
    \label{Proposition 7.10}
\end{proposition}
\begin{proof}
    Can assume that $Y = \spec A$ is affine.
    Want to show that $R^{k}f_{\ast}\mathcal{F} \cong \widetilde{H^{k}(X, \mathcal{F})}$,
    which is a finitely generated $A$-module by Theorem \ref{Theorem 7.7}.
    Suffice to prove that for all $f\in A$, $H^{k}(f^{-1}(D(f)), \mathcal{F}\big{|}_{f^{-1}(D(f))}) = H^{k}(X, \mathcal{F})_{f}$.
    Note that $D(f) = Y_{f}$, by Lemma \ref{Lemma 2.6}, get $f^{-1}(D(f)) = f^{-1}(Y_{f}) = X_{g}$, where g is the image of f under the map $f^{\sharp}(Y)$.
    Since $f$ is projective, by Theorem \ref{Theorem 4.3}, $f$ is proper so that $f$ is of finite type and separated.
    Thus $X$ can be covered by finitely many affine open subsets $U_{1}, \cdots, U_{r}$.
    And $U_{i} \cap U_{j}$ is still affine.
    Consider the commutative diagram
    \begin{equation*}
        \begin{tikzcd}
            0 \arrow[r] &
            \mathcal{F}(X) \arrow[r] \arrow[d] &
            \prod_{i = 1}^{r} \mathcal{F}(U_{i}) \arrow[r] \arrow[d] &
            \prod_{i = 1}^{r}\prod_{j = 1}^{r} \mathcal{F}(U_{i} \cap U_{j}) \arrow[r] \arrow[d] &
            \cdots \\
            0 \arrow[r] &
            \mathcal{F}(X_{g}) \arrow[r] &
            \prod_{i = 1}^{r} \mathcal{F}(X_{g} \cap U_{i}) \arrow[r] &
            \prod_{i = 1}^{r}\prod_{j = 1}^{r} \mathcal{F}(X_{g} \cap U_{i} \cap U_{j}) \arrow[r] &
            \cdots
        \end{tikzcd}
    \end{equation*}
    By similar argument in Lemma \ref{Lemma 2.6}, we can show that $\mathcal{F}(X_{g}) \cong \mathcal{F}(X)_{g} \cong \mathcal{F}(X)_{f}$.
    Moreover, we have $\mathcal{F}(X_{g} \cap U_{i_{1}} \cap \cdots \cap U_{i_{l}}) \cong \mathcal{F}(U_{i_{1}} \cap \cdots \cap U_{i_{l}})_{g\big{|}_{U_{i_{1}} \cap \cdots \cap U_{i_{l}}}} \cong \mathcal{F}(U_{i_{1}} \cap \cdots \cap U_{i_{l}})_{g} \cong \mathcal{F}(U_{i_{1}} \cap \cdots \cap U_{i_{l}})_{f}$.
    Thus for open covering $\{U_{i} \cap X_{g}\}_{i\in I}$ of $X_{g}$ and $\mathcal{F}\big{|}_{X_{g}}$,
    the Čech complex is localization of the Čech complex for open covering $\{U_{i}\}_{i}$ of $X$ and $\mathcal{F}$ with respect to $f$.
    By exactness of localization, get $H^{k}(f^{-1}(D(f)), \mathcal{F}\big{|}_{f^{-1}(D(f))}) = H^{k}(X, \mathcal{F})_{f}$.
\end{proof}
\begin{remark}
    Obviously, in the proof, we only need $f$ to proper.
    Thus this proposition still holds if projective is replaced by proper, as Remark 7.9.
\end{remark}
\subsection{Flat Base Change}
\begin{definition}[\textbf{\emph{Flat Module of Sheaves}}]
    Let $X$ be a scheme, $\mathcal{F}$ an $\mathcal{O}_{X}$ module.
    We say that $\mathcal{F}$ is flat if for all $x\in X$, $\mathcal{F}_{x}$ is flat over $\mathcal{O}_{X, x}$.
\end{definition}
\begin{remark}
    With this definition, we can restate the definition of flat morphism between schemes.
    We say $f: X \longrightarrow Y$ is flat if $\mathcal{O}_{X}$ is a flat $f^{-1}\mathcal{O}_{Y}$-module.
    More generally, an $\mathcal{O}_{X}$-module $\mathcal{F}$ is flat over $Y$ if $\mathcal{F}$ is a flat $f^{-1}\mathcal{O}_{Y}$-module.
\end{remark}
\begin{example}
    Assume $X$ is noetherian and $\mathcal{F}$ is coherent on $X$.
    Then $\mathcal{F}$ is flat if and only if it is locally free of finite rank,
    since flatness over a local ring is equivalent to free and $X$ is noetherian.
\end{example}
We are interested in cartesian diagrams of noetherian schemes like below 
\begin{equation*}
    \begin{tikzcd}
        X' \arrow[r, "v"] \arrow[d, "g"] &
        X \arrow[d, "f"] \\
        Y' \arrow[r, "u"] &
        Y
    \end{tikzcd}
\end{equation*}
\begin{proposition}
    Further assume $f: X \longrightarrow Y$ separated in the diagram above.
    Then for all $\mathcal{F}$ quasi-coherent sheaf on X, we have that \\
    (1)There are natural morphisms which are called base change morphism,
    \begin{equation*}
        u^{\ast}(R^{i}f_{\ast}\mathcal{F}) \longrightarrow R^{i}g_{\ast}(v^{\ast}\mathcal{F})
    \end{equation*}
    (2)if moreover $u: Y' \longrightarrow Y$ is flat,
    then the natural morphisms are isomorphic.
    \label{Proposition 7.11}
\end{proposition}
\begin{proof}
    Since everything is local on $Y$ and $Y'$, can assume that $Y = \spec A$ and $Y' = \spec A'$.
    Want natural morphisms 
    \begin{equation*}
        H^{i}(X, \mathcal{F}) \otimes_{A} A' \longrightarrow H^{i}(X', i^{\ast}\mathcal{F})
    \end{equation*}
    Consider the Čech complexes.
    Take affine open covering $\mathcal{U} = \{U_{i}\}_{i\in I}$ of X.
    By construction of fiber product, easy to see $v^{-1}(\mathcal{U}) = \{v^{-1}(U_{i})\}_{i\in I}$ is affine open covering of $X'$.
    Observe that $C^{\ast}(v^{-1}(\mathcal{U}), v^{\ast}\mathcal{F}) = C^{\ast}(\mathcal{U}, \mathcal{F}) \otimes_{A} A'$.
    Get $H^{i}(X, \mathcal{F}) \otimes_{A} A' \longrightarrow H^{i}(X', i^{\ast}\mathcal{F})$.
    By universal property of sheafification, get natural morphism $u^{\ast}(R^{i}f_{\ast}\mathcal{F}) \longrightarrow R^{i}g_{\ast}(v^{\ast}\mathcal{F})$.
    Further, if $A'$ is flat over $A$,
    then $\otimes_{A} A'$ is exact so that commutes with cohomology, inducing the isomorphism.
    Thus the local homomorphisms between stalks are isomorphic, so is the natural morphism between schemes.
\end{proof}
\begin{proposition}[\textbf{\emph{Projection Formula, Flat Case}}]
    Let $f: X \longrightarrow Y$ be a separated morphism between noetherian schemes.
    Then for all $\mathcal{F}$ quasi-coherent on X and $\mathcal{G}$ quasi-coherent on Y, we have that \\
    (1)there are natural morphisms 
    \begin{equation*}
        (R^{i}f_{\ast}\mathcal{F}) \otimes_{\mathcal{O}_{Y}} \mathcal{G} \longrightarrow R^{i}f_{\ast}(\mathcal{F} \otimes_{\mathcal{O}_{X}} f^{\ast}\mathcal{G})
    \end{equation*}
    (2)if $\mathcal{G}$ is flat over $Y$, then the natural morphisms are isomorphic.
    \label{Proposition 7.12}
\end{proposition}
\begin{remark}
    The proof is same as Proposition \ref{Proposition 7.11}
\end{remark}

\section{Divisors}
\label{section:Divisors}

\subsection{Weil Divisors}
For convenience, we assume that scheme $X$ is noetherian, integral, (separated) and regular in dimension 1.
Denote the condition by $(\ast)$.
\begin{definition}
    Let $X$ be a scheme satisfying condition $(\ast)$.
    A prime divisor on X is an intgral closed subscheme of codimension 1.
    A Weil divisor on X is an element of $Div(X)$, the free abelian group generated by prime divisors on X.
\end{definition}
\begin{definition}
    Let $X$ be a scheme satisfying condition $(\ast)$, $D\in Div(X)$.
    $D$ is called effective if coefficients of prime divisors are all non-negative in it.
    Define support of $D$ to be $Supp(D) (= |D|) := \cap_{n_{Y} \neq 0} Y$.
\end{definition}
Want to construct an equivalence relation in $Div(X)$.
Consider function field $k(X)$ of $X$.
By our assumptions, for all $Y$ prime divisor on $X$, there is a discrete valuation $v_{Y}: k(X)^{\times} \longrightarrow \mathbb{Z}$ on $\mathcal{O}_{X, \eta_{Y}}$,
where $\eta_{Y}$ is the generic point of $Y$.
\begin{definition}
    Let $X$ be a scheme satisfying condition $(\ast)$, $k(X)$ the function field of $X$, $f\in k(X)$.
    Define the divisor of $f$ to be $div(f) = \sum_{Y} v_{Y}(f)Y$.
    Devisors of the form $div(f)$ are called principal divisors, denoted by $Pr(X)$.
\end{definition}
\begin{lemma}
    Let $X$ be a scheme satisfying condition $(\ast)$, $k(X)$ the function field of $X$, $f\in k(X)$.
    Then $v_{Y}(f) = 0$ for all but finitely many $Y$.
    \label{Lemma 8.1}
\end{lemma}
\begin{proof}
    Let $U = \spec A \subseteq X$ be an affine open subset.
    Since X is noetherian, by Theorem \ref{Theorem 3.1}, $X \setminus U$ is union of finitely many irreducible components which are of codimension $\ge 1$ in $X$.
    Hence we can reduce to the case $X = \spec A$.
    Since $f\in k(X) = \fraction(A)$, assume $f = \frac{a}{b}$ where $a,b\in A$.
    Take $D(b) \subseteq \spec A$, then $f\in A_{b}$.
    Hence we can assume that $f\in A \setminus \{0\}$.
    Thus for all $Y$ prime divisor, $f\in \mathcal{O}_{X, \eta_{Y}}$ so that $v_{Y}(f) \ge 0$ and $v_{Y}(f) > 0$ if and only if $\eta_{Y} \subseteq V(f)$.
    While again by Theorem \ref{Theorem 3.1}, $V(f)$ can be covered by finitely many irreducible components which are of codimension $\ge 1$ in $X$, done!
\end{proof}
\begin{definition}
    Let $X$ be a scheme satisfying condition $(\ast)$.
    Two divisors $D_{1}, D_{2}$ on $X$ are linearly equivalent, denoted by $D_{1} \sim D_{2}$, if $D_{1} - D_{2}$ is principal.
    Define divisor class group of $X$ to be $Cl(X) := Div(X)/\sim$.
\end{definition}
\begin{proposition}
    Let $A$ be a noetherian integral domain, $X = \spec A$.
    Then $X$ is normal and $Cl(X) = 0$ if and only if $A$ is unique factorization domain.
    \label{Proposition 8.1}
\end{proposition}
\begin{proof}
    "$\Leftarrow$": Since $A$ is unique factorization domain, $A$ is integrally closed so that $X$ is normal.
    For all $Y$ prime divisor, the generic point $\eta_{Y}$ corresponds to a prime ideal $\mathfrak{p}_{Y}$ of height 1 in $\spec A$.
    Since $A_{\mathfrak{p}_{Y}}$ is a DVR, we can take its uniformizer $\omega_{Y} = \frac{a_{Y}}{b_{Y}}$, where $a_{Y},b_{Y}\in A$.
    Consider $a_{Y}\in \mathfrak{p}_{Y}$.
    Since $A$ is UFD, it is easy to show that $\mathfrak{p}_{Y} = (a_{Y})$.
    Since $\mathfrak{p}\in V(a_{Y})$ if and only if $\mathfrak{p}_{Y} \subseteq \mathfrak{p}$,
    as the proof of Lemma \ref{Lemma 8.1}, get $Y = div(a_{Y})$ is principal.
    Thus $Cl(X) = 0$.
    \par
    "$\Rightarrow$": By knowledge of commutative algebra, a noetherian integral domain is UFD if and only if all prime ideals of height 1 is principal.
    For $\mathfrak{p}\in \spec A$ of height 1 and $Y = V(\mathfrak{p})$, as $Cl(X) = 0$, 
    there exists $f\in k(X)^{\times}$ such that $div(f) = Y$.
    Since $v_{Y}(f) = 1$, get $f\in A_{\mathfrak{p}}$ and $\mathfrak{p}A_{\mathfrak{p}} = (f)$.
    If $\mathfrak{p}'$ is another prime ideal of height 1, then $v_{V(\mathfrak{p}')}(f) = 0$ so that $f\in A_{\mathfrak{p}'}^{\times}$.
    By Algebraic Hartogs, $A = \cap_{ht(\mathfrak{p}) = 1} A_{\mathfrak{p}}$.
    Get $f\in A$ so that $f\in A \cap \mathfrak{p}A_{\mathfrak{p}} = \mathfrak{p}$.
    Remains to show that $\mathfrak{p} = (f)$.
    For all $g\in \mathfrak{p} \setminus \{0\}$, we have that $\frac{g}{f}\in A_{\mathfrak{p}}$ and $V_{V(\mathfrak{p}')}(\frac{g}{f}) \ge 0$ so that $\frac{g}{f}\in A_{\mathfrak{p}'}$.
    Again with Algebraic Hartogs, get $\frac{g}{f}\in A$.
    Thus $\mathfrak{p} = (f)$.
\end{proof}
\begin{proposition}
    Let $k$ be a field.
    Then \\ 
    (1)the prime divisors of $\mathbb{A}_{k}^{n}$ are of the form $V(f)$ with $f$ irreducible polynomial. \\
    (2)the prime divisors of $\mathbb{P}_{k}^{n}$ are of the form $V_{+}(f)$ with $f$ irreducible homogeneous polynomial.
    \label{Proposition 8.2}
\end{proposition}
\begin{proof}
    For (1), since $\mathbb{A}_{k}^{n} = \spec(k[x_{1}, \cdots, x_{n}])$, any prime divisor corresponds a prime ideal of height 1.
    By knowledge of commutative algebra, a noetherian integral domain is UFD if and only if all prime ideals of height 1 is principal.
    Thus prime divisors of $\mathbb{A}_{k}^{n}$ are of the form $V(f)$ with $f$ irreducible polynomial.
    \par
    For (2), consider affine open covering $\{D_{+}(x_{i})\}$.
    For all prime divisor $Y$, take generic point $\eta$ of $Y$.
    Assume that $\eta\in D_{+}(x_{i})$.
    Then $Y \cap D_{+}(x_{i})$ is prime divisor in $D_{+}(x_{i})$.
    With same argument, $Y \cap D_{+}(x_{i})$ is of the form $V(f)$ with $f$ prime in $k[x_{0}, \cdots, x_{n}]_{(x_{i})}$.
    Assume that $f = \frac{g}{x_{i}^{k}}$, where $x_{i} \not\big{|} g$, $g$ is irreducible and $\deg(g) = k$.
    Since generic point $V(f)$ is $(f)$ and $(f)$ corresponds to $(g)$ in $\mathbb{P}_{k}^{n}$, get $\overline{\{(g)\}} = Y$.
    Thus $Y = V_{+}(g)$ with $g$ irreducible homogeneous polynomial.
\end{proof}
\begin{remark}
    With this proposition, we can define degree for prime divisor of $\mathbb{P}_{k}^{n}$.
    For prime divisor $Y = V_{+}(f)$, define $\deg(Y) = \deg(f)$.
    Get a group homomorphism $\deg: Div(\mathbb{P}_{k}^{n}) \longrightarrow \mathbb{Z}$.
\end{remark}
\begin{proposition}
    Let $k$ be a field and $X = \mathbb{P}_{k}^{n}$, $H = V_{+}(x_{0}) \subseteq X$ hyperplane.
    Then \\
    (1)every divisor $D$ of degreee $d$ is equivalent to $dH$ \\
    (2)for all $f\in k(X)^{\times}$, where $k(X)$ is the function field, we have that $\deg(div(f)) = 0$ \\
    (3)the degree homomorphism $\deg: Div(\mathbb{P}_{k}^{n}) \longrightarrow \mathbb{Z}$ descends to isomorphism $\deg: Cl(\mathbb{P}_{k}^{n}) \overset{\sim}{\longrightarrow} \mathbb{Z}$.
    \label{Proposition 8.3}
\end{proposition}
\begin{proof}
    For (3), $\deg$ is obviously surjective.
    It suffices to show that $\ker(\deg)$ is just principal divisors.
    Assume we have proved (1) and (2).
    By (1), $\ker(\deg) \subseteq Pr(X)$.
    By (2), $Pr(X) \subseteq \ker(\deg)$, done!
    \par
    For (2), for all $f\in K^{\times}$, $f$ can be reducedly written as $\frac{g}{h}$, where $g,h\in k[x_{0}, \cdots, x_{n}]$ homogeneous polynomials of same degree $d$.
    Factorize $g = g_{1}^{n_{1}}\cdots g_{r}^{n_{r}}$, where $g_{i}$ is irreducible homogeneous polynomial of degree $d_{i}$.
    By Proposition \ref{Proposition 8.2}, $g_{i}$ corresponds to a prime divisor $Y_{i} = V_{+}(g_{i})$ and $\deg(Y_{i}) = d_{i}$.
    We can define a divisor $div(g) = \sum_{i = 1}^{r} n_{i}Y_{i}$ whose degree is $d$.
    Also, there is a divisor $div(h)$.
    It is clear that $div(f) = div(g) - div(h)$.
    Thus $\deg(div(f)) = \deg(div(g)) - \deg(div(h)) = 0$.
    \par
    For (1), let $D$ be a divisor of degree $d$.
    Write $D = D_{1} - D_{2}$ where $D_{1}, D_{2}$ are two effective divisors.
    Assume $\deg(D_{1}) = d_{1}$ and $\deg(D_{2}) = d_{2}$.
    As proof of (2), we can write $D_{1} = div(g_{1})$ and $D_{2} = div(g_{2})$, where $g_{1}, g_{2}$ are two homogeneous polynomials.
    Then $D - dH = div(\frac{g_{1}}{x_{0}^{d}g_{2}})$ so that $D$ is equivalent to $dH$.
\end{proof}
\begin{remark}
    Can also show that $Cl(\mathbb{P}_{k}^{1} \times_{k} \mathbb{P}_{k}^{1}) \cong \mathbb{Z} \oplus \mathbb{Z}$.
    While $\mathbb{P}_{k}^{2} \cong \mathbb{Z}$, get $\mathbb{P}_{k}^{1} \times_{k} \mathbb{P}_{k}^{1}$ is not isomorphic to $\mathbb{P}_{k}^{2}$.
    However, it is not true that $Cl(X \times_{k} X) \cong Cl(X) \oplus Cl(X)$ in general, not even for curves.
    \par
    If $U \subseteq X$ is a nonempty open subset, then the restriction map $Cl(X) \longrightarrow Cl(U)$ is surjective,
    whose kernel consists of equivalence classes of divisors supported on $Z = X \setminus U$.
    If $Z$ is irreducible closed subset of codimension 1, then kernel is spanned by $Z$.
    For example, if $Y \subseteq \mathbb{P}_{k}^{2}$ is irreducible curve of degree $d$, 
    then $Cl(\mathbb{P}_{k}^{2} \setminus Y) \cong \mathbb{Z}/d\mathbb{Z}$.
\end{remark}
\subsection{Weil Divisors for Curve Case}
\begin{definition}[\textbf{\emph{Curves}}]
    Let $X$ be a scheme of finite type over field $k$.
    We say that $X$ is a curve over $k$ if it is geometrically integral of dimension 1.
\end{definition}
\begin{definition}
    Let $X$ be a curve which is smooth and proper over field $k$, $D = \sum_{i} n_{i}P_{i}\in Div(X)$.
    Define the degree of $D$ to be $\sum_{i} n_{i}[k(P_{i}) : k]$, where $k(P_{i})$ is the residue field of generic point of $P_{i}$.
\end{definition}
\begin{remark}
    Similar to projective abstract variety case, we will see that $\deg: Div(X) \longrightarrow \mathbb{Z}$ can also descend to $\deg: Cl(X) \longrightarrow \mathbb{Z}$.
    However, the homomorphism is not necessarily surjective if $k$ is not algebraically closed.
\end{remark}
\subsection{Cartier Divisors}
\begin{definition}
    Let $X$ be an arbitrary scheme.
    For all $U \subseteq X$ open subset, set $K(U)$ to be localization of $\mathcal{O}_{X}(U)$ with respect to the multiplicative system $S(U)$,
    where $S(U)$ is the subset of elements which are non zero-divisors in each $\mathcal{O}_{X, x}$ for all $x\in U$.
    In particular, if $U = \spec A$ is affine, then $K(U)$ is the localization of $A$ with respect to non zero-divisors.
    \par
    Define $\mathcal{K}$ to be the sheafification of $U \longmapsto K(U)$, called the function field of $X$.
    It is clear that if $X$ is integral, then $\mathcal{K}$ is the constant sheaf corresponding to $K(X)$.
    Can also define $\mathcal{K}^{\ast} \subseteq \mathcal{K}$ and $\mathcal{O}_{X}^{\ast} \subseteq \mathcal{O}_{x}$ to be the subsheaves of invertible elements.
\end{definition}
\begin{definition}
    Let $X$ be a scheme.
    A Cartier divisor $D$ on $X$ is a global section of the sheaf (of multiplicative groups) $\mathcal{K}^{\ast}/\mathcal{O}_{X}^{\ast}$.
    Denote $\Gamma(X, \mathcal{K}^{\ast}/\mathcal{O}_{X}^{\ast})$ by $CaDiv(X)$.
    \par
    $D\in CaDiv(X)$ is called principal if it is in the image of $\Gamma(X, \mathcal{K}^{\ast}) \longrightarrow \Gamma(X, \mathcal{K}^{\ast}/\mathcal{O}_{X}^{\ast})$.
    Two Cartier divisors are linearly equivalent if their difference is principal.
    Define $CaCl(X) = CaDiv(X)/im(\Gamma(X, \mathcal{K}^{\ast}))$, called the Cartier divisor class group of $X$.
\end{definition}
\begin{remark}
    More concretely, a Cartier divisor $D$ on X is represented by an (affine) open covering $\{U_{i}\}$ of $X$ and corresponding elements $f_{i}\in \mathcal{K}^{\ast}(U_{i})$ such that the restriction of $\frac{f_{i}}{f_{j}}$ to $U_{i} \cap U_{j}$ lies in $\mathcal{O}^{\ast}(U_{i} \cap U_{j})$ for all $i,j$.
    $D\in CaDiv(X)$ is called effective if it can be represented by $(U_{i}, f_{i})$ with $f_{i}\in \mathcal{K}^{\ast}(U_{i}) \cap \mathcal{O}_{X}(U_{i})$
\end{remark}
\begin{lemma}
    Let $X$ be a scheme, $D\in CaDiv(X)$.
    Define the support of $D$ to the set of points satisfying that the local equation $f_{i}$ of $D$ is not invertible in the stalk of $\mathcal{O}_{X}$, denoted by $Supp(D)$ or $|D|$.
    Then $Supp(D)$ is a closed subset of $X$.
    \label{Lemma 8.2} 
\end{lemma}
\begin{remark}
    Just a corollary of the fact that support of a section is closed.
\end{remark}
\begin{definition}[\textbf{\emph{Locally Factorial}}]
    Let $X$ be a scheme.
    We say that $X$ is locally factorial if all its stalks are UFD.
\end{definition}
\begin{example}
    Regular schemes are locally factorial.
\end{example}
\begin{remark}
    Since UFD is not a local property, we cannot cover a locally factorial scheme with affine open subsets $U_{i} = \spec(A_{i})$ with each $A_{i}$ UFD, such as curve $y^{2} = x^{3} + x$.
    In addition, since UFD is integrally closed, local factorial implies normal.
\end{remark}
\begin{theorem}
    Let $X$ be a scheme which is noetherian, integral, (separated) and locally factorial.
    Then there is isomorphism $Div(X) \overset{\sim}{\longleftarrow} CaDiv(X)$.
    Further, the principal (resp. effective) Weil divisors correspond to principal (resp. effective) Cartier divisors under the isomorphism.
    In particular, we get the isomorphism $Cl(X) \overset{\sim}{\longleftarrow} CaCl(X)$.
    \label{Theorem 8.1}
\end{theorem}
\begin{proof}
    Since $X$ is integral, $\mathcal{K} = K(X)_{X}$.
    Let $D$ be a Cartier divisor represented by $(U_{i}, f_{i})$.
    For all $Y$ prime Weil divisor, set $v_{Y}(D) = v_{Y}(f_{i})$ for $i$ such that $U_{i}$ meets $Y$.
    This is well defined since $\frac{f_{i}}{f_{j}}$ lies in $\mathcal{O}_{X}^{\ast}(U_{i} \cap U_{j})$.
    Since $X$ is noetherian, there are only finitely many prime divisor $Y$ such that $v_{Y}(D) \neq 0$. 
    Then $\sum_{Y} v_{Y}(D)Y$ is a Weil divisor.
    \par
    Conversely, let $D$ be a Weil divisor on $X$.
    Then for all $x\in X$, $D$ restricts to a Weil divisor $D_{x}$ on $\spec(\mathcal{O}_{X, x})$.
    For all $Y\ni x$ prime divisor, take an affine open neighbourhood $U = \spec A$ of $x$.
    Assume that $x$ corresponds to a prime ideal $\mathfrak{p}$.
    Then $Y \cap U \neq \varnothing$ so that $Y \cap U$ is a irreducible closed subset of codimension $1$ of $U$ and of the form $V(\mathfrak{q})$ with $\mathfrak{q} \subseteq \mathfrak{p}$ of height $1$.
    Then we restirct $Y$ to $V(\mathfrak{q}A_{\mathfrak{p}}) \subseteq \spec(A_{\mathfrak{p}})$.
    It is clear that the restriction is well defined.
    \par
    Note that $\mathcal{O}_{X, x}$ is UFD, by Proposition \ref{Proposition 8.1}, get $D_{x}$ is principal.
    Thus there exists $f_{x}\in K(Spec(\mathcal{O}_{X, x})) = \fraction(\mathcal{O}_{X, x}) = K(X)$ such that $D_{x} = div(f_{x})$.
    Since $f_{x}\in K(X)$, we can also consider Weil divisor $\widetilde{div(f_{x})}$ on $X$.
    Then $D$ and $\widetilde{div(f_{x})}$ have the same restriction to $\spec(\mathcal{O}_{X, x})$.
    Thus they differ by finitely many prime divisors not containing $x$.
    Consider the complementary set of union of such prime divisors, we get an open neighbourhood $U_{x}$ of $x$ such that $D$ and $\widetilde{div(f_{x})}$ have the same restriction to $U_{x}$.
    Note that $U_{x}$ form an open covering of $X$, want to check that $(U_{x}, f_{x})$ is a Cartier divisor on $X$.
    \par
    Take affine open covering $\{V_{i} = \spec(B_{i})\}$ of $U_{x} \cap U_{x'}$.
    Since $\widetilde{div(f_{x})}$ coincides with $\widetilde{div(f_{x'})}$ on $U_{x} \cap U_{x'}$, 
    for all $Y$ prime divisor meeting $U_{x} \cap U_{x'}$ with generic point $y$, it is clear that $\frac{f_{x}}{f_{x'}}$ is invertible in $\mathcal{O}_{X, y}$.
    Thus for each $\mathfrak{q}\in V_{i}$ of height $1$, $\frac{f_{x}}{f_{x'}}$ is invertible in $(B_{i})_{\mathfrak{q}}$.
    By Algebraic Hartogs, $B_{i} = \cap_{\mathfrak{q}} (B_{i})_{\mathfrak{q}}$.
    Get $\frac{f_{x}}{f_{x'}}$ is invertible in $B_{i}$ so that $\frac{f_{x}}{f_{x'}}\in \mathcal{O}_{X}^{\ast}(U_{i} \cap U_{j})$.
    It is easy to show that the two constructions are inverse to each other and correspondences of principal (resp. effective) divisors.
\end{proof}
\begin{remark}
    If $X$ is only noetherian, integral and normal, then there is still an injection $CaDiv(X) \hookrightarrow Div(X)$ with image equal to the set of Weil divisors which are locally principal.
    When we say a Weil divisor $D$ is locally principal, it means that there is an open covering $\{U_{i}\}$ such that $D\big{|}_{U_{i}}$ is principal divisor on $U_{i}$.
    Similarly, the injection also descends to $Cadiv(X) \hookrightarrow Div(X)$.
    However, without locally factorial, the map is not necessarily surjective.
\end{remark}
\subsection{Invertible Sheaves}
Recall that invertible sheaf on scheme $X$ is equal to locally free $\mathcal{O}_{X}$-module of rank $1$.
\begin{definition}
    Let $X$ be a scheme.
    Define the Picard group of $X$ to be the set of isomorphic classes of invertible sheaves on $X$ with multiplicative operation given by tensor product, denoted by $Pic(X)$.
\end{definition} 
\begin{example}
    For local ring $A$, it is easy to show that $Pic(Spec(A))$ is trivial.
\end{example}
\begin{definition}
    Let $D = (U_{i}, f_{i})$ be a Cartier divisor on scheme $X$.
    We can define an invertible sheaf $\mathcal{O}_{X}(D)$ on $X$ as the $\mathcal{O}_{X}$-submodule of $\mathcal{K}$ generated by $f_{i}^{-1}$ on $U_{i}$
\end{definition}
\begin{example}
    Let $k$ be a field and $X = \mathbb{P}_{k}^{d}$, $D = H$ hyperplane.
    Then $\mathcal{O}_{X}(D) \cong \mathcal{O}_{X}(1)$.
\end{example}
\begin{proposition}
    Let $X$ be a scheme.
    Then \\
    (1)The map $D \longmapsto \mathcal{O}_{x}(D)$ is a bijection between Cartier divisors and invertible subscheaves if $\mathcal{K}$. \\
    (2)for all Cartier divisors $D_{1}, D_{2}$, $\mathcal{O}_{X}(D_{1} - D_{2}) \cong \mathcal{O}_{X}(D_{1}) \otimes_{\mathcal{O}_{X}} \mathcal{O}_{X}(D_{2})^{-1}$ \\
    (3)for all Cartier divisors $D_{1}, D_{2}$, if $D_{1} - D_{2}$ is principal, then $\mathcal{O}_{X}(D_{1}) \cong \mathcal{O}_{X}(D_{2})$ as $\mathcal{O}_{X}$-modules.
    \label{Proposition 8.4}
\end{proposition}
\begin{proof}
    (1)Let $\mathcal{L}$ be an invertible subsheaf of $\mathcal{K}$.
    Assume that $\{U_{i}\}$ is the open covering of $X$ such that $\mathcal{L}\big{|}_{U_{i}}$ is free of rank $1$ over $\mathcal{O}_{U_{i}}$.
    Then for all $i$, we can take the preimage of $1$ under $\mathcal{L}(U_{i}) \overset{\sim}{\longrightarrow} \mathcal{O}_{X}(U_{i})$, denoted by $g_{i}$.
    Obviously, $g_{i}\in \mathcal{K}^{\ast}(U_{i})$.
    Set $f_{i} = g_{i}^{-1}$, want to show $(U_{i}, f_{i})$ is a Cartier divisor, 
    which is clear since two bases of $\mathcal{L}\big{|}_{U_{i} \cap U_{j}}$ differ by an element of $\mathcal{O}_{X}^{\ast}(U_{i} \cap U_{j})$. 
    Obviously, the construction is inverse of $D \longmapsto \mathcal{O}_{X}(D)$.
    \par
    (2)For two Cartier divisors $D_{1}, D_{2}$, when representing them, we can use the same open covering.
    Assume that $D_{1} = (U_{i}, f_{i})$ and $D_{2} = (U_{i}, g_{i})$.
    Then $D_{1} - D_{2} = (U_{i}, \frac{f_{i}}{g_{i}})$ so that $\mathcal{O}_{X}(D_{1} - D_{2})\big{|}_{U_{i}}$ is generated by $\frac{f_{i}}{g_{i}}$.
    Thus $\mathcal{O}_{X}(D_{1} - D_{2}) \subseteq \mathcal{K}$.
    Denote $\mathcal{O}_{X}(D_{1}) \otimes_{\mathcal{O}_{X}} \mathcal{O}_{X}(D_{2})^{-1}$ and $\mathcal{O}(D_{1} - D_{2})$ by $\mathcal{L}$ and $\mathcal{J}$.
    For all open subset $V \subseteq X$, consider the following commutative diagram
    \begin{equation*}
        \begin{tikzcd}
            0 \arrow[r] & 
            \mathcal{J}(V) \arrow[r] \arrow[d] &
            \prod_{i\in I} \mathcal{J}(V \cap U_{i}) \arrow[r] \arrow[d, "\sim"] &
            \prod_{i,j\in I} \mathcal{J}(V \cap U_{i} \cap U_{j}) \arrow[d, "\sim"] \\
            0 \arrow[r] &
            \mathcal{L}(V) \arrow[r] &
            \prod_{i\in i} \mathcal{L}(V \cap U_{i}) \arrow[r] &
            \prod_{i,j\in I} \mathcal{L}(V \cap U_{i} \cap U_{j})
        \end{tikzcd}
    \end{equation*}
    The last two isomorphisms between rows come from the isomorphisms between stalks.
    Then, by $5$ Lemma, get $\mathcal{J} \cong \mathcal{L}$.
    \par
    (3)Suffice to show that Cartier divisor $D$ is principal if and only if $\mathcal{O}_{X}(D) \cong \mathcal{O}_{X}$.
    If $D$ is principal and defined by $f\in \Gamma(X, \mathcal{K}^{\ast})$, then $\mathcal{O}_{X}(D)$ globally generated by $f^{-1}$.
    Thus $\mathcal{O}_{X}(D) \cong \mathcal{O}_{X}$.
    Conversely, if $\mathcal{O}_{X}(D) \cong \mathcal{O}_{X}$, then the image of $1\in \mathcal{O}_{X}(X)$ fives the $f^{-1}$.
\end{proof}
\begin{corollary}
    Let $X$ be a scheme.
    Then the map $D \longmapsto \mathcal{O}_{X}(D)$ descends to an injective group homomorphism $CaCl(X) \hookrightarrow Pic(X)$.
    \label{Corollary 8.1}
\end{corollary}
\begin{proposition}
    Let $X$ be an integral scheme.
    Then $CaCl(X) \overset{\sim}{\longrightarrow} Pic(X)$ is an isomorphism.
    \label{Proposition 8.5}
\end{proposition}
\begin{proof}
    Since $X$ is integral, $\mathcal{K} = K(X)_{X}$ is a constant sheaf.
    For any invertible sheaf $\mathcal{L}$ on $X$, set $\mathcal{L}' = \mathcal{L} \otimes_{\mathcal{O}_{X}} \mathcal{K}$.
    Assume $\{U_{i}\}$ is an open covering of $X$ trivializing $\mathcal{L}$.
    Thus for all $i$, $\mathcal{L}'\big{|}_{U_{i}} \cong \mathcal{K}\big{|}_{U_{i}}$ is a constant sheaf on $U_{i}$.
    As $X$ is irreducible, $\mathcal{L}'\big{|}_{U_{i}}$ glue up to a constant sheaf on $X$.
    Moreover, we have that $\mathcal{L}' \cong \mathcal{K}$.
    Note that for all $x\in X$, $\mathcal{K}_{x}$ is a localization of $\mathcal{O}_{X, x}$ so that $\mathcal{K}_{x}$ is flat over $\mathcal{O}_{X, x}$.
    Then $\mathcal{L} \hookrightarrow \mathcal{L} \otimes_{\mathcal{O}_{X}} \mathcal{K} \cong \mathcal{K}$ expresses $\mathcal{L}$ as a subsheaf on $\mathcal{K}$.
\end{proof}
\begin{corollary}
    Let $k$ be a field and $X = \mathbb{P}_{k}^{d}$.
    Then all invertible shaeves on $X$ are isomorphic to $\mathcal{O}_{X}(n)$ for some $n\in \mathbb{Z}$.
    \label{Corollary 8.2}
\end{corollary}
\begin{proof}
    By Proposition \ref{Proposition 8.5}, Theorem \ref{Theorem 8.1} and Proposition \ref{Proposition 8.3}, get
    \begin{equation*}
        Pic(X) \cong CaCl(X) \cong Cl(X) \cong \mathbb{Z}
    \end{equation*}
\end{proof}
In conclusion, if $X$ is a scheme which is noetherian, integral, (separated) and locally factorial,
then there is a commutative diagram
\begin{equation*}
    \begin{tikzcd}
        Cl(X) \arrow[dr, "f_{3}" swap] &&
        CaCl(X) \arrow[ll, "f_{1}" swap] \arrow[dl, "f_{2}"] \\
        & Pic(X) &
    \end{tikzcd}
\end{equation*}
$f_{1}$ is given by 
\begin{equation*}
    D = (U_{i}, g_{i}) \longmapsto \sum_{Y} v_{Y}(g_{i})Y
\end{equation*}
$f_{2}$ is given by 
\begin{equation*}
    D = (U_{i}, g_{i}) \longmapsto \mathcal{O}_{X}(D) \subseteq \mathcal{K}
\end{equation*}
$f_{3}$ is given by
\begin{equation*}
    D = \sum_{Y} n_{Y}Y \longmapsto (\mathcal{O}_{X}(D): U \longmapsto \{g\in \mathcal{K}^{\ast} \big{|} D\big{|}_{U} + div(f)\big{|}_{U} \text{ effective}\} \cup \{0\}) \subseteq \mathcal{K}
\end{equation*}
\subsection{Effective Divisors}
Let $X$ be a scheme.
Recall that $D = (U_{i}, f_{i})$ is an effective Cartier divisor on $X$ if and only if for all $i$, $f_{i}\in \mathcal{K}^{\ast}(U_{i}) \cap \mathcal{O}_{X}(U_{i})$.
Thus an effective Cartier divisor one-to-one corresponds to an ideal sheaf $\mathcal{I} \subseteq \mathcal{O}_{X}$ which is locally generated by a non zero-divisor $f_{i}$.
Moreover, $\mathcal{I} \cong \mathcal{O}_{X}(-D) = \mathcal{O}_{X}^{-1}$.
\begin{definition}[\textbf{\emph{Zero Divisor of Global Sections}}]
    Let $X$ be an integral scheme, $\mathcal{L}$ invertible sheaf on $X$, $s \neq 0\in \Gamma(X, \mathcal{L})$.
    For all open subset $U \subseteq X$ trivializing $\mathcal{L}$, there is an isomorphism $\varphi: \mathcal{L}\big{|}_{U} \overset{\sim}{\longrightarrow} \mathcal{O}_{U}$.
    Assume that $\{U_{i}\}$ is an open covering of $X$ trivializing $\mathcal{L}$ with isomorphisms $\varphi_{i}$.
    Define the zero divisor of $s$ to be $Z(s) := (U_{i}, \varphi_{i}(U_{i})(s))$.
    In particular, for $\mathcal{L} \cong \mathcal{O}_{X}$, get zero divisor of a regular function.
\end{definition}
\begin{remark}
    It is clear that $Z(s)$ is a well defined Cartier divisro.
\end{remark}
\begin{proposition}
    Let $X$ be a scheme which is geometrically integral, smooth and projective over field $k$, $D$ Cartier divisor on $X$, $\mathcal{L} = \mathcal{O}_{X}(D)$ invertible sheaf associated with $D$.
    Then \\
    (1)for all $s \neq 0\in \Gamma(X, \mathcal{L})$, the zero divisor $Z(s)$ is effective and linearly equivalent to $D$. \\
    (2)All effective divisors linearly equivalent to $D$ are of the form $Z(s)$ for some $s \neq 0\in \Gamma(X, \mathcal{L})$. \\
    (3)Two sections $s,s' \neq 0\in \Gamma(X, \mathcal{L})$ have the same zero divisor if and only if they differ by a nonzero constant in $k^{\times}$.
    \label{Proposition 8.6}
\end{proposition}
\begin{proof}
    (1)Since $X$ is integral, view $\mathcal{L}$ as a subsheaf of $\mathcal{K} = K(X)_{X}$.
    Then also since $X$ is integral, $s$ corresponds to an element $f\in \mathcal{K}^{\ast}(X) = K(X)^{\times}$.
    Assume that $D = (U_{i}, f_{i})$.
    Then $\mathcal{L}$ is locally generated by $f_{i}^{-1}$ and the isomorphism $\varphi_{i}$ is inducing by multiplication with $f$.
    Get $Z(s) = (U_{i}, f_{i}f) = D + div(f)$ is effective and linearly equivalent to $D$.
    \par
    (2)If $D'$ is effective and $D' = D + div(f)$ for some $f\in \mathcal{K}^{\ast}(X) = K(X)^{\times}$.
    Then $f_{i}f\in \mathcal{K}^{\ast}(U_{i}) \cap \mathcal{O}_{X}(U_{i})$ so that $f\big{|}_{U_{i}} = \frac{f_{i}f}{f_{i}}$ glue up to a global section of $\mathcal{L}$, denoted by $s$.
    Same argument, get $Z(s) = D'$.
    \par
    (3)Assume $s, s'$ correspond to $f,f'\in K(X)^{\times}$.
    Then $Z(s) = Z(s')$ if and only if $div(\frac{f}{f'}) = 0$ if and only if $\frac{f}{f'}\in \mathcal{O}_{X}^{\ast}(X)$.
    As $X$ is geometrically integral and projective over $k$, by Corollary \ref{Corollary 7.9}, get $\mathcal{O}_{X}^{\ast}(X) = k^{\times}$, done!
\end{proof}
\begin{definition}
    Let $X$ be a scheme which is geometrically integral, smooth and projective over field $k$, $D$ Cartier divisor on $X$.
    The set of divisors that are effective and linearly equivalent to $D$ is in bijection with $\Gamma(X, \mathcal{O}_{X}(D)) \setminus \{0\}/k^{\times}$.
    It is called the (complete) linear system associated with $D$, denoted by $|D|$.
\end{definition}
\subsection{Vector Bundles}
\begin{definition}[\textbf{\emph{Vector Bundles}}]
    Let $X$ be a scheme.
    A vector bundle of rank $r$ on $X$ is a scheme $E$ with a morphism $\pi: E \longrightarrow X$ satisfying that there exists an open covering $\{U_{i}\}$ of $X$ and commutative diagrams
    \begin{equation*}
        \begin{tikzcd}
            \pi^{-1}(U_{i}) \arrow[rr, "\varphi_{i}", "\sim" swap] \arrow[dr] &&
            \mathbb{A}_{U_{i}}^{r} \arrow[dl] \\
            & U_{i} &
        \end{tikzcd}
    \end{equation*}
    where isomorphism $\varphi_{i}$ is called the local trivialization, 
    such that the transition $\varphi_{i} \circ \varphi_{j}^{-1}$ is linear over $U_{i} \cap U_{j}$ i.e. for all affine open subset $V = \spec A \subseteq U_{i} \cap U_{j}$,
    $\varphi_{i} \circ \varphi_{j}^{-1}: \spec(A[y_{1}, \cdots, y_{r}]) \longrightarrow \spec(A[x_{1}, \cdots, x_{r}])$ is given by an $A$-linear isomorphism
    \begin{equation*}
        \begin{split}
            A[x_{1}, \cdots, x_{r}] & \longrightarrow A[y_{1}, \cdots, y_{r}] \\
            x_{i} & \longmapsto \sum_{j}a_{ij}y_{j}
        \end{split}
    \end{equation*}
    \par
    Equivalently giving transition functions $g_{ij}: U_{i} \cap U_{j} \longrightarrow (GL_{r}(\mathcal{O}_{X}(U_{i} \cap U_{j})))_{U_{i} \cap U_{j}}$ such that 
    $g_{ii} = id$, $g_{ij}^{-1} = g_{ji}$ and $g_{ik} = g_{jk} \circ g_{ij}$ on $U_{i} \cap U_{j} \cap U_{k}$.
\end{definition}
\begin{definition}
    Let $X$ be a scheme and $\pi: E \longrightarrow X$ be a vector bundle on $X$.
    A section of $E$ is a morphism $s: X \longrightarrow E$ such that $\pi \circ s = \identity_{X}$.
    If $E$ is determined by transitions functions $g_{ij}$, then $s$ is determined by morphisms $s_{i}: U_{i} \longrightarrow \mathbb{A}_{U_{i}}^{r}$ such that $s_{i} = g_{ij}s_{j}$.
\end{definition}
Now given $\pi: E \longrightarrow X$ vector bundle of rank $r$,
define its sheaf of sections to be 
\begin{equation}
    \mathcal{E}: U \longmapsto \{\text{sections of } \pi\big{|}_{U}: \pi^{-1}(U) \longrightarrow U\}
\end{equation}
Then $\mathcal{E}$ is locally free sheaf of rank $r$.
To see this, take open subset $U \subseteq X$ with commutative diagram 
\begin{equation*}
    \begin{tikzcd}
        \pi^{-1}(U_{i}) \arrow[rr, "\varphi_{i}", "\sim" swap] \arrow[dr] &&
        \mathbb{A}_{U_{i}}^{r} \arrow[dl] \\
        & U_{i} &
    \end{tikzcd}
\end{equation*}
Then sections of $\pi\big{|}_{U}$ are one-to-one corresponding to sections of $\mathbb{A}_{U}^{r} \longrightarrow U$.
Note that $\mathbb{A}_{U}^{r}$ is the fiber product of $U$ and $\mathbb{A}_{\mathbb{Z}}^{r}$ over $\mathbb{Z}$.
By universal property of fiber product, sections are one-to-one corresponding to a commutative diagram
\begin{equation*}
    \begin{tikzcd}
        U \arrow[rr] \arrow[dr] &&
        \mathbb{Z} \\
        & \mathbb{A}_{\mathbb{Z}}^{r} \arrow[ur] &
    \end{tikzcd}
\end{equation*}
which is equivalent to 
\begin{equation*}
    \begin{tikzcd}
        \mathcal{O}_{U}(U) &&
        \mathbb{Z} \arrow[dl] \arrow[ll] \\
        & \mathbb{Z}[x_{1}, \cdots, x_{r}] \arrow[ul] &
    \end{tikzcd}
\end{equation*}
Since the diagram is determined by the image of each $x_{i}$, it is clear that the rank is $r$.
\par
Conversely, a locally free $\mathcal{O}_{X}$-module $\mathcal{E}$ of rank $r$ comes from a vector bundle of rank $r$ unique up to isomorphism.
To see this, either use transition functions or directly define $E := \spec(Sym(\mathcal{E}^{\lor}))$, where $\mathcal{E}^{\lor} = \sheafhom_{\mathcal{O}_{X}}(\mathcal{E}, \mathcal{O}_{X})$ is the dual space.
For all affine open subset $U = \spec A \subseteq X$, take $\spec(Sym_{A}(\mathcal{E}^{\lor}(U)))$.
By gluing lemma, get $E$.
In conclusion, get the correspondence between isomorphic classes of locally free $\mathcal{O}_{X}$-module of rank $r$ and isomorphic classes of vector bundles of rank $r$.
However, this is not an equivalence of categories, as morphism of vector bundles are typically required to be of constant rank.
\begin{remark}
    For field $k$ and vector space $V$, $Sym_{k}(V^{\lor}) = \{\text{polynomials on V}\}$.
    Precise definition can be seen in Hartshorne, Chapter II, exercise 5.16.
\end{remark}
\begin{definition}[\textbf{\emph{Zero Schemes of Sections of Vector Bundles}}]
    Let $X$ be a scheme and $\pi: E \longrightarrow X$ be a vector bundle, $s: X \longrightarrow E$ section.
    Assume that $s$ is determined by $s_{i}: U_{i} \longrightarrow \mathbb{A}_{U_{i}}^{r}$.
    Since $s_{i}$ is equivalent to a ring homomorphism $A[x_{1}, \cdots, x_{r}] \longrightarrow \mathcal{O}_{X}(U_{i})$, $s_{i}$ is determined by $s_{i}^{1}, \cdots, s_{i}^{r}\in \mathcal{O}_{X}(U_{i})$.
    Define the zero scheme of $s$ to be a closed subscheme $Z(s) \subseteq X$ corresponding to the ideal sheaf generated by $s_{i}^{1}, \cdots, s_{i}^{r}$ on $U_{i}$.
\end{definition}
\subsection{Riemann-Roch Theorem for Curve Case}
Recall that a curve is a scheme of finite type over field $k$ which is geometrically integral of dimension $1$.
Thus irreducible closed subsets of codimension $1$ are just the closed points.
Let $X$ be a curve which is smooth and proper over $k$, $D = \sum_{i} n_{i}P_{i}$ a Weil divisor on $X$.
Then $\deg(D) = \sum_{i} n_{i}[k(P_{i}) : k]$.
Set $\ell(D) := \dim_{k}(\Gamma(X, \mathcal{O}_{X}(D))) = \dim_{k}(H^{0}(X, \mathcal{O}_{X}(D)))$.
Since $X$ is projective over $k$ and $\mathcal{O}_{X}(D)$ is coherent, get $\ell(D) < \infty$.
\begin{lemma}
    Let $X$ be a curve which is smooth and proper over $k$, $D$ a Weil divisor on $X$.
    If $\ell(D) \neq 0$, then $\deg(D) \ge 0$.
    If $\ell(D) \neq 0$ and $\deg(D) = 0$, then $D \sim 0$ i.e. $\mathcal{O}_{X}(D) \cong \mathcal{O}_{X}$.
    \label{Lemma 8.3}
\end{lemma}
\begin{proof}
    If $\ell(D) \neq 0$, then $\mathcal{O}_{X}(D) \neq 0$ and there exists an effective divisor $D' \sim D$ so that $\deg(D) = \deg(D') \ge 0$.
    Since $D'$ is effective, if moreover $\deg(D) = 0$, then $D' = 0$ so that $D \sim 0$.
\end{proof}
\begin{definition}[\textbf{\emph{Canonical Divisors}}]
    A canonical divisor $K_{X}$ is any Cartier divisor corresponding to canonical sheaf $\omega_{X}$.
    Set $g(X) (= p_{g}(X)) := \dim_{k}(\Gamma(X, \omega_{X}))$ to be the geometric genus of $X$.
\end{definition}
\begin{theorem}[\textbf{\emph{Riemann-Roch Theorem for Curve Case}}]
    Let $X$ be a curve which smooth and proper over field $k$, $D$ a Weil divisor on $X$.
    Then $\ell(D) - \ell(K_{X} - D) = \deg(D) + 1 - g(X)$.
    \label{Theorem 8.2}
\end{theorem}
\begin{remark}
    We will prove the theorem assuming $\ell(K_{X} - D) = \dim_{k}(H^{1}(X, \mathcal{O}_{X}(D)))$, which is a result from Serre duality.
    In particular, assuming the theorem, for $D = 0$, we have that $g(X) = \ell(K_{X}) + 1 - \ell(0)$.
    Since $\ell(0) = \dim_{k}(\Gamma(X, \mathcal{O}_{x})) = \dim_{k}(k) = 1$, get $g(x) = \ell(K_{X})$, called the arithmetirc genus of $X$.
    \par
    For $\mathcal{F}$ coherent sheaf on $X$, set $\chi(\mathcal{F}) := \sum_{i} (-1)^{i}\dim_{k}(H^{i}(X, \mathcal{F}))$, called the Euler characteristic of $\mathcal{F}$.
    By Serre duality, can rewrite Riemann-Roch Theorem as $\chi(\mathcal{O}_{X}(D)) = \deg(D) + 1 - g(X)$.
    Note that $1 - g(X) = \chi(\mathcal{O}_{X})$, get $\chi(\mathcal{O}_{X}(D)) - \chi(\mathcal{O}_{X}) = \deg(D)$.
\end{remark}
\begin{proof}
    As remark, $D = 0$ is ok.
    Let $P\in X$ be a closed point.
    Suffice to show that Riemann-Roch is ok for $D$ if and only if Riemann-Roch is ok for $D + P$.
    View $i: \{P\} \hookrightarrow  X$ as a closed subscheme with reduced scheme structure.
    Then $\mathcal{I}_{P} \cong \mathcal{O}_{X}(-P)$.
    Note that there is an exact sequence
    \begin{equation*}
        0 \longrightarrow \mathcal{O}_{X}(-P) \longrightarrow \mathcal{O}_{X} \longrightarrow i_{\ast}\mathcal{O}_{\{P\}} \longrightarrow 0
    \end{equation*}
    Tensot with $\mathcal{O}_{X}(D + P)$, get exact sequence
    \begin{equation*}
        0 \longrightarrow \mathcal{O}_{X}(D) \longrightarrow \mathcal{O}_{X}(D + P) \longrightarrow i_{\ast}\mathcal{O}_{\{P\}} \otimes_{\mathcal{O}_{X}} \mathcal{O}_{X}(D + P) \longrightarrow 0
    \end{equation*}
    Since no prime divisor can restrict to $\spec(\mathcal{O}_{X, P})$, $\mathcal{O}_{X}(D + P)$ is generated by $1$ on some open neighbourhood of $P$.
    Thus $i_{\ast}\mathcal{O}_{\{P\}} \otimes_{\mathcal{O}_{X}} \mathcal{O}_{X}(D + P) \cong i_{\ast}\mathcal{O}_{\{P\}}$.
    Consider the long exact sequence given by cohomology, get $\chi(\mathcal{O}_{X}(D + P)) = \chi(\mathcal{O}_{X}(D)) + \chi(i_{\ast}\mathcal{O}_{\{P\}})$,
    where $\chi(i_{\ast}\mathcal{O}_{\{P\}}) = \dim_{k}(H^{0}(X, i_{\ast}\mathcal{O}_{\{P\}})) = \dim_{k}(H^{0}(Spec(k(P)), \mathcal{O}_{Spec(k(P))})) = [k(P) : k]$.
    Thus $\chi(\mathcal{O}_{X}(D + P)) - \chi(\mathcal{O}_{X}(D)) = [k(P) : k]$.
    Since $\deg(D + P) = \deg(D) + [k(P) : k]$, done!
\end{proof}
\begin{example}
    Let $X$ be a curve which smooth and proper over field $k$, $D$ a Weil divisor on $X$. \\
    (1)If $\deg(D) > 0$, then dimension of complete linear system of $nD$ is $\dim_{k}(|nD|) = n\deg(D) - g(X)$ for $n$ large enough,
    since $\ell(K_{X} - nD) = 0$ for $n$ large enough. \\
    (2)Let $D = K_{X}$, then $\ell(K_{X}) - 1 = \deg(K_{X}) + 1 - g(X)$.
    Get $\deg(K_{X}) = 2g(X) - 2$. \\
    (3)If $g(X) = 1$, then $\deg(K_{X}) = 0$ and $\ell(K_{X}) = g(X) = 1$.
    Get $K_{X} \sim 0$. \\
    (4)Let $X$ be an elliptic curve i.e. $g(X) = 1$ and the $k$-points of $X$ isn't empty, $P_{0}\in X(k)$.
    Take $Pic^{0}(X) \subseteq Pic(X)$ to be the subgroup corresponding to divisor classes of degree zero.
    Then $X(k) \longrightarrow Pic^{0}(X)\quad P \longmapsto \mathcal{O}_{X}(P - P_{0})$ is bijective inducing a group structure over $X(k)$.
    Injection is obvious.
    For surjection, let $D$ be a divisor of degree $0$, then $\deg(D + P_{0}) = \deg(P_{0}) = 1$.
    By Riemann-Roch, $\ell(D + P_{0}) - \ell(K_{X} - D - P_{0}) = 1$.
    While $\deg(K_{X} - D - P_{0}) < 0$, get $\ell(K_{X} - D - P_{0}) = 0$ and $\ell(D + P_{0}) = 1$.
    Thus there exists effective divisor $P \sim D + P_{0}$ so that $P \longmapsto \mathcal{O}_{X}(D)$. \\
    (5)If $\deg(D) \ge 2g(X) + 1$, then $\mathcal{O}_{X}(D)$ is very ample.
    Thus $\mathcal{O}_{X}(D)$ is ample if and only if $\deg(D) > 0$.
\end{example}
\subsection{Higher Dimension Divisor Theory}
Let $X$ be a scheme which is geometrically integral, smooth and proper over $k$.
Define $Z^{i}(X)$ to be the free abelian group spanned by closed and integral subschemes of $X$ of codimension $i$.
Each element of $Z^{i}(X)$ is of the form $\alpha = \sum_{V} n_{V}V$, called an algebraic cycle of codimension $i$.
\begin{definition}[\textbf{\emph{Rational Equivalence}}]
    $\sim_{rat}$ is the equivalence on $Z^{i}(X)$ generated by $j_{\ast}(div(f))$,
    where $W \subseteq X$ is closed and integral subscheme of codimension $i - 1$, $\widetilde{W}$ is the normalization of $W$, 
    $j$ is the map $\widetilde{W} \longrightarrow W \hookrightarrow X$ and $f\in k(W) = k(\widetilde{W})$.
    Set $CH^{i}(X) := Z^{i}(X)/\sim_{rat}$ to be the Chow group of algebraic cycles of codimension $i$
\end{definition}
Want to define intersection between $Z^{i}(X)$ and $Z^{j}(X)$ for all $i,j$.
However, $V \cap W$ is not necessarily reduced as a scheme and the intersection is not always proper i.e. codimension of $V \cap W$ might be less than $i + j$.
There is an important theorem that can help us to solve these problems.
\begin{theorem}[\textbf{\emph{Chow's Moving Lemma}}]
    For all $\alpha\in Z^{i}(X)$ and $\beta\in Z^{j}(X)$, there exists $\alpha' \sim_{rat} \alpha$ such that every component of $\alpha'$ intersects every component of $\beta$ properly.
    \label{Theorem 8.3}
\end{theorem}
\begin{remark}
    Then we can define $\alpha \cdot \beta = [\alpha' \cap \beta]\in CH^{i + j}(X)$.
    However, there is another problem about this definition that if it is well defined.
    Even though the answer is correct, this is a hard problem, but there are other ways to avoid this problem.
    Fulton and Macpherson established a geometric, refined, moving lemma-free intersection theory.
    In particular, get well defined
    \begin{equation*}
        \cdot: CH^{i}(X) \times CH^{j}(X) \longrightarrow CH^{i + j}(X) \times (CH^{\ast}(X), \cdot)
    \end{equation*}
\end{remark}
There is also some generalization of Riemann-Roch in higher dimension case.
\begin{theorem}[\textbf{\emph{Hirzebruch-Riemann-Roch}}]
    Let $X$ be a scheme which is geometrically integral, smooth and proper over field $k$, $E$ vector bundle on $X$.
    Then $\chi(E) = \int_{X} ch(E) + d(X)$.
    \label{Theorem 8.4}
\end{theorem}
\begin{remark}
    $ch(E)$ is the Chern character of $E$ and $d(X)$ is the Todd class of $X$ which is independent of $E$.
\end{remark}

\section{Differentials}
\label{section:Differentials}

\subsection{Kähler Differentials in Algebraic Theory}
Let $X$ be a scheme, $x\in X$ closed point.
Recall that the tangent space of $X$ at $x$ is $T_{x}X := (\mathfrak{m}_{x}/\mathfrak{m}_{x}^{2})^{\lor}$.
However, in the world of algebraic geometry, it is easier to study cotangent space $\mathfrak{m}/\mathfrak{m}^{2}$.
Further, not only the tangent space, but also we want to get differentials in the world of algebraic geometry. 
Firstly, let's see three equivalent definitions of Kähler differentials.
\begin{definition}
    Let $A$ be a ring, $B$ an $A$-algebra.
    The module of relative differentials of $B$ over $A$, denoted by $\Omega_{B/A}$ (or $\Omega_{B/A}^{1}$),
    is defined by taking the quotient module of the free $B$-module spanned by symbols $\{db \big{|} b\in B\}$ under the following relations \\
    (1)$da$, $\forall a\in A$, \\
    (2)$d(b_{1} + b_{2}) - db_{1} - db_{2}$, $\forall b_{1},b_{2}\in B$, \\
    (3)$d(b_{1}b_{2}) - b_{1}db_{2} - b_{2}db_{1}$, $\forall b_{1}b_{2}\in B$. \\
    Get an $A$-linear map $d: B \longrightarrow \Omega_{B/A}\quad b \longmapsto db$.
\end{definition}
\begin{remark}
    This is an constructive definition, which directly shows us the structure of module of relative differentials.
    It is clear that those relation are copied from Differential Geometry.
\end{remark}
\begin{definition}
    Let $A$ be a ring, $B$ an $A$-algebra, $M\in Mod_{B}$.
    An $A$-derivation of $B$ into $M$ is a map $d: B \longrightarrow M$ such that \\
    (1)$da = 0$, $\forall a\in A$, \\
    (2)$d(b_{1} + b_{2}) = db_{1} + db_{2}$, $\forall b_{1},b_{2}\in B$. \\
    (3)$d(b_{1}b_{2}) = b_{1}db_{2} + b_{2}db_{1}$, $\forall b_{1},b_{2}\in B$. 
    \par
    Denote the set of $A$-derivations of $B$ into $M$ by $\derivation_{A}(B, M)$, which is automatically an $A$-module.
    Then $d: B \longrightarrow \Omega_{B/A}$ has the universal property that for all $M\in Mod_{B}$ and $A$-derivation $d': B \longrightarrow M$,
    there exists unique $B$-module homomorphism $\phi: \Omega_{B/A} \longrightarrow M$ such that the following diagram commutes 
    \begin{equation*}
        \begin{tikzcd}
            B \arrow[r, "d"] \arrow[d, "{d'}"] &
            \Omega_{B/A} \arrow[dl, "\phi"] \\
            M &
        \end{tikzcd}
    \end{equation*}
    The universal property gives an canonical isomorphism $Hom_{B}(\Omega_{B/A}, M) \overset{\sim}{\longrightarrow} \derivation_{A}(B, M)$.
\end{definition}
\begin{example}
    (1)Let $A$ be a ring, $B = A[x_{1}, \cdots, x_{n}]$.
    Then $\Omega_{B/A}$ is the free module spanned by $\{dx_{i}\}$. \\
    (2)If $B$ is either localization of quotient of $A$, then $\Omega_{B/A} = 0$.
    For localization case, assume $B = S^{-1}A$.
    Then for all $b\in B$, there exists $s\in S$ such that $bs\in A$ so that $s(db) = d(bs) = 0$.
    Note that $s\in B^{\times}$, get $db = 0$.
    For quotient case, obviously.
\end{example}
Consider ring homomorphism $A \longrightarrow B \longrightarrow C$.
There are natural homomorphisms $\alpha: \Omega_{B/A} \otimes_{B} C \longrightarrow \Omega_{C/A}\quad db \otimes c \longmapsto d(bc)$ and $\beta: \Omega_{C/A} \longrightarrow \Omega_{C/B}\quad dc \longmapsto dc$.
\begin{proposition}
    Let $A$ be a ring, $B$ an $A$-algebra. \\
    (1)(base change) 
    Let $A'$ be an $A$-algebra, $B' := B \otimes_{A} A'$.
    Then $\Omega_{B'/A'} \cong \Omega_{B/A} \otimes_{B} B'$ is a canonical isomorphism of $B$-modules. \\
    (2)(cotangent sequence) 
    Let $A \longrightarrow B \longrightarrow C$ be ring homomorphisms.
    Then there is an exact sequence
    \begin{equation*}
        \Omega_{B/A} \otimes_{B} C \overset{\alpha}{\longrightarrow} \Omega_{C/A} \overset{\beta}{\longrightarrow} \Omega_{C/B} \longrightarrow 0
    \end{equation*}
    \\
    (3)(localization)
    Let $S$ be a multiplicative system of $B$.
    Then $S^{-1}\Omega_{B/A} \cong \Omega_{S^{-1}B/A}$ is a canonical isomorphism of $S^{-1}B$-modules. \\
    (4)(conormal sequence)
    Let $I \subseteq B$ be an ideal, $C := B/I$.
    Then there is an exact sequence
    \begin{equation*}
        I/I^{2} \overset{\sigma}{\longrightarrow} \Omega_{B/A} \otimes_{B} C \overset{\alpha}{\longrightarrow} \Omega_{C/A} \overset{\beta}{\longrightarrow} \Omega_{C/B} = 0
    \end{equation*} 
    where $\sigma$ maps $\overline{b} \longmapsto db \otimes 1$.
    \label{Proposition 9.1}
\end{proposition}
\begin{proof}
    For (1), $d: B \longrightarrow \Omega_{B/A}$ induces a homomorphism $d' = d \otimes \identity_{A'}: B' \longrightarrow \Omega_{B/A} \otimes_{A} A' = \Omega_{B/A} \otimes_{B} B'$.
    It is clear that $d'$ is an $A'$-derivation.
    Suffice to show that $d'$ has the universal property.
    For all $M\in Mod_{B'}$ and $A'$-derivation $d'': B' \longrightarrow M$, define map $\phi: \Omega_{B/A} \otimes_{A} A' \longrightarrow M$ to be $db \otimes a' \longmapsto a'd(b \otimes 1)$.
    It is easy to check that $\phi$ is a $B'$-module homomorphism and $\phi \circ d' = d''$.
    \par
    For (2), use the property of Hom functor.
    It suffices to show that for all $M\in Mod_{C}$, the following sequence is exact
    \begin{equation*}
        0 \longrightarrow \hom_{C}(\Omega_{C/B}, M) \longrightarrow \hom_{C}(\Omega_{C/A}, M) \longrightarrow \hom_{C}(\Omega_{B/A} \otimes_{B} C, M)
    \end{equation*}
    As $\hom_{C}(\Omega_{C/B}, M) \cong \derivation_{B}(C, M)$, $\hom_{C}(\Omega_{C/A}, M) \cong \derivation_{A}(C, M)$ and $\hom_{C}(\Omega_{B/A} \otimes_{B} C, M) \cong \hom_{B}(\Omega_{B/A}, \hom_{C}(C, M)) \cong \hom_{B}(\Omega_{B/A}, M) \cong \derivation_{A}(B, M)$, get sequence
    \begin{equation*}
        \begin{aligned}
            0 \longrightarrow \derivation_{B}(C, M) & \longrightarrow & \derivation_{A}(C, M) & \longrightarrow && \derivation_{A}(B, M) \\
            (d: C \rightarrow M) & \longmapsto & (d: C \rightarrow M) & \longmapsto && (B \rightarrow C \overset{d}{\rightarrow} M)
        \end{aligned}
    \end{equation*}
    Obviously, the sequence is exact, done!
    \par
    For (3), taking $C = S^{-1}B$ in (2), get the surjectivety since $\Omega_{C/B} = 0$.
    For injectivety, compose with $\Omega_{S^{-1}B/A} \longrightarrow S^{-1}\Omega_{B/A}\quad d\frac{b}{s} \longmapsto \frac{db}{s}$ so that we get the $S^{-1}B$-module endomorphism of $S^{-1}\Omega_{B/A}$.
    By the universal property of $S^{-1}\Omega_{B/A}$, it should be identity map.
    Thus the canonical homomorphism is injective.
    \par
    For (4), recall that $I/I^{2} \cong I \otimes_{B} C$.
    Similar, suffice to show that for all $M\in Mod_{C}$, the following sequence is exact
    \begin{equation*}
        0 \longrightarrow \hom_{C}(\Omega_{C/A}, M) \longrightarrow \hom_{C}(\Omega_{B/A} \otimes_{B} C, M) \longrightarrow \hom_{C}(I/I^{2}, M)    
    \end{equation*}
    Since $\hom_{C}(\Omega_{C/A}, M) \cong \derivation_{A}(C, M)$, $\hom_{c}(\Omega_{B/A} \otimes_{B} C, M) \cong \derivation_{A}(B, M)$ as in (2) and $\hom_{C}(I/I^{2}, M) \cong \hom_{C}(I \otimes_{B} C, M) \cong \hom_{B}(I, \hom_{C}(C, M)) \cong \hom_{B}(I, M)$, get sequence
    \begin{equation*}
        \begin{aligned}
            0 \longrightarrow \derivation_{A}(C, M) & \longrightarrow & \derivation_{A}(B, M) & \longrightarrow && \hom_{B}(I, M) \\
            (d: C \rightarrow M) & \longmapsto & (C \rightarrow B \overset{d}{\rightarrow} M) &&& \\
            && (d': B \rightarrow M) & \longmapsto && (d'\big{|}_{I}: I \longrightarrow M)
        \end{aligned}
    \end{equation*}
    Obviously, the sequence is exact, done!
\end{proof}
\begin{corollary}
    Let $A$ be a ring, $B$ an algebra of finite type over $A$.
    Then $\Omega_{B/A}$ is a finite $B$-module and for all localization $C := S^{-1}B$ of $B$, $\Omega_{C/A}$ is a finite $C$-module.
    \label{Corollary 9.1}
\end{corollary}
\begin{reason}
    Follows from example 9.1 and Proposition \ref{Proposition 9.1} (3) and (4).
\end{reason}
\begin{definition}
    Let $A$ be a ring, $B$ an $A$-algebra.
    Consider the diagonal homomorphism $f: B \otimes_{A} B \longrightarrow B\quad b \otimes b' \longmapsto bb'$.
    Assume $I$ is the kernel of $f$.
    Then $I/I^{2}$ is a $B$-module by multiplication on the left.
    Define map $d: B \longrightarrow I/I^{2}\quad b \longmapsto I/I^{2}\quad b \longmapsto \overline{1 \otimes b - b \otimes 1}$.
\end{definition}
\begin{lemma}
    The map defined above is an $A$-derivation.
    \label{Lemma 9.1}
\end{lemma}
\begin{proof}
    For all $a\in A$, $da = \overline{1 \times a - a \otimes 1} = 0$.
    For all $b_{1},b_{2}\in B$, 
    \begin{equation*}
        \begin{split}
            d(b_{1} + b_{2}) & = \overline{1 \otimes (b_{1} + b_{2}) - (b_{1} + b_{2}) \otimes 1} \\
            & = \overline{1 \otimes b_{1} - b_{1} \otimes 1} + \overline{1 \otimes b_{2} - b_{2} \otimes 1} \\
            & = db_{1} + db_{2}
        \end{split}
    \end{equation*}
    and
    \begin{equation*}
        \begin{split}
            d(b_{1}b_{2}) - b_{1}db_{2} - b_{2}db_{1} & = \overline{1 \otimes b_{1}b_{2} - b_{1}b_{2} \otimes 1} - b_{1}\overline{1 \otimes b_{2} - b_{2} \otimes 1} - b_{2}\overline{1 \otimes b_{1} - b_{1} \otimes 1} \\
            & = \overline{1 \otimes b_{1}b_{2} - b_{1}b_{2} \otimes 1} - \overline{b_{1} \otimes b_{2} - b_{1}b_{2} \otimes 1} - \overline{b_{2} \otimes b_{1} - b_{1}b_{2} \otimes 1} \\
            & = \overline{(1 \otimes b_{1} - b_{1} \otimes 1)(1 \otimes b_{2} - b_{2} \otimes 1)} \\
            & = 0
        \end{split}
    \end{equation*}
\end{proof}
\begin{proposition}
    Assumption as above, then $(I/I^{2}, d) \cong (\Omega_{B/A}, d)$.
    \label{Proposition 9.2}
\end{proposition}
\begin{proof}
    By universal property, there exists unique $\phi: \Omega_{B/A} \longrightarrow I/I^{2}\quad db \longmapsto \overline{1 \otimes b - b \otimes 1}$.
    As $1 \otimes b - b \otimes 1$ are generators of $I$, $\phi$ is surjective.
    Consider $B \otimes_{A} B \longrightarrow \Omega_{B/A}\quad b \otimes b' \longmapsto bdb'$ restricting to $I$.
    It is easy to check that the image of $I^{2}$ is $O$.
    Get $\psi: I/I^{2} \longrightarrow \Omega_{B/A}$.
    Suffice to show that $\psi \circ \phi$ is identity.
    While $\psi \circ \phi(db) = \psi(\overline{1 \otimes b - b \otimes 1}) = db$, done!
\end{proof}
\subsection{Sheaf of Differentials}
\begin{definition}
    Let $f: X \longrightarrow Y$ be a morphism of schemes.
    Then there exists quasi-coherent sheaf $\Omega_{X/Y}$ on $X$ such that for all affine open subsets $V \subseteq Y$ and $U \subseteq f^{-1}(V)$, 
    we have that $\Omega_{X/Y}\big{|}_{U} \cong \widetilde{\Omega_{\mathcal{O}_{X}(U)/\mathcal{O}_{Y}(V)}}$ and for all $x\in X$, $\Omega_{X/Y, x} \cong \Omega_{\mathcal{O}_{X, x}/\mathcal{O}_{Y, f(x)}}$.
    $\Omega_{X/Y}$ is called the sheaf of relative differentials of $X$ over $Y$.
\end{definition}
Even though we can get $\Omega_{X/Y}$ by gluing lemma, the following proposition is a much faster way to see the uniqueness of sheaf of relative differentials.
\begin{proposition}
    Let $f: X \longrightarrow Y$ be a morphism of schemes.
    Consider the diagonal morphism $\Delta: X \longrightarrow X \times_{Y} X$.
    Note that by Remark 4.1, $\Delta(X)$ is closed in an open subset $U \subseteq X \times_{Y} X$.
    Set $\mathcal{I}$ to be the ideal sheaf corresponding to $\Delta(X)$ in $U$.
    Then $\Omega_{X/Y} \cong \Delta^{\ast}(\mathcal{I}/\mathcal{I}^{2})$.
    \label{Proposition 9.3}
\end{proposition}
\begin{remark}
    Even though we take the inverse image of $\mathcal{I}/\mathcal{I}^{2}$ under $\Delta$ in the equation, $\mathcal{I}/\mathcal{I}^{2}$ is automatically viewed as an $\mathcal{O}_{X}$-module.
\end{remark}
\begin{proof}
    For all affine open subset $V = \spec A \subseteq V$ and $U = \spec(B) \subseteq f^{-1}(V)$, then $\Delta(U) \subseteq W := U \times_{V} U$.
    Note that $\mathcal{O}_{X \times_{Y} X}(W) = B \otimes_{A} B$ and $\mathcal{I}\big{|}_{W} = \widetilde{I}$, where $I$ is the kernel of $B \otimes_{A} B \longrightarrow B\quad b \otimes b' \longmapsto bb'$.
    Thus $I/I^{2} \subseteq \mathcal{I}/\mathcal{I}^{2}(W)$.
    Define map $d: B \longrightarrow \Delta^{\ast}(\mathcal{I}/\mathcal{I}^{2})(U)$ to be $b \longmapsto (W, \overline{1 \otimes b - b \otimes 1})$.
    Same argument as proof of Lemma \ref{Lemma 9.1}, get $d$ is an $A$-derivation.
    By universal property, there exists unique homomorphism $\phi: \Omega_{B/A} \longrightarrow \Delta^{\ast}(\mathcal{I}/\mathcal{I}^{2})(U)$, which corresponds to a morphism $\varphi: \Delta^{\ast}(\mathcal{I}/\mathcal{I}^{2})\big{|}_{U} \longrightarrow \widetilde{\Omega_{B/A}}$.
    \par
    Suffice to show that $\varphi_{x}$ is isomorphic for each $x\in U$.
    Assume $x$ corresponds to prime ideal $\mathfrak{p}\in \spec(B)$ and $f(x)$ corresponds to prime ideal $\mathfrak{P}\in \spec(B \otimes_{A} B)$.
    Then $(\widetilde{\Omega_{B/A}})_{x} = \Omega_{B_{\mathfrak{p}}/A}$ by Proposition \ref{Proposition 9.1} (3) and $(\Delta^{\ast}(\mathcal{I}/\mathcal{I}^{2})\big{|}_{U})_{x} = I_{\mathfrak{P}}/I_{\mathfrak{P}}^{2} \otimes_{(B \otimes_{A} B)_{\mathfrak{P}}} B_{\mathfrak{p}}$.
    Note that 
    \begin{equation*}
        I_{\mathfrak{P}}/I_{\mathfrak{P}}^{2} \otimes_{(B \otimes_{A} B)_{\mathfrak{P}}} B_{\mathfrak{p}} \cong I/I^{2} \otimes_{B \otimes_{A} B} (B \otimes_{A} B)_{\mathfrak{P}} \otimes_{(B \otimes_{A} B)_{\mathfrak{P}}} B_{\mathfrak{p}} \cong I/I^{2} \otimes_{B \otimes_{A} B} B_{\mathfrak{p}}
    \end{equation*}
    it is clear that $\varphi(\frac{b}{s}) = \overline{1 \otimes b - b \otimes 1} \otimes \frac{1}{s}$.
    As $B \otimes_{A} B \longrightarrow B$ is surjective, $I/I^{2} \otimes_{B \otimes_{A} B} B_{\mathfrak{p}} \cong I/I^{2} \otimes_{B} B_{p} \cong (\Omega_{B/A})_{\mathfrak{p}}$ as $B$-modules.
    Thus $\varphi_{x}$ is isomorphic, done!
\end{proof}
\begin{proposition}
    Let $f: X \longrightarrow Y$ be a morphism of finite type with $X$ noetherian.
    Then $\Omega_{X/Y}$ is coherent on $X$.
    \label{Proposition 9.4}
\end{proposition}
\begin{reason}
    By Corollary \ref{Corollary 9.1}, for all local affine open subset $V = \spec A$ and $U = \spec(B)$, we have that $\Omega_{B/A}$ is a finite $B$-module, done!
\end{reason}
\begin{example}
    (1)Let $Y$ be a scheme, $X = \mathbb{A}_{Y}^{n}$.
    Then $\Omega_{X/Y} \cong \mathcal{O}_{X}^{n}$. \\
    (2)Let $A$ be a ring, $X = \mathbb{P}_{A}^{1} = \proj(A[x, y])$.
    Then $\Omega_{X/Y} \cong \mathcal{O}_{X}(-2)$.
\end{example}
\begin{proposition}
    Let $f: X \longrightarrow Y$ be a morphism of schemes. \\
    (1)(base change)
    Let $Y'$ be a $Y$-scheme, $X' = X \times_{Y} Y'$ with projection $p: X' \longrightarrow X$.
    Then $\Omega_{X'/Y'} \cong p^{\ast}\Omega_{X/Y}$. \\
    (2)(cotangent sequence)
    Let $X \overset{f}{\longrightarrow} Y \longrightarrow Z$ be morphisms of schemes.
    Then there is an exact sequence
    \begin{equation*}
        f^{\ast}\Omega_{Y/Z} \longrightarrow \Omega_{X/Z} \longrightarrow \Omega_{X/Y} \longrightarrow 0
    \end{equation*}
    \\
    (3)(localization)
    Let $U \subseteq X$ be an open subset.
    Then $\Omega_{X/Y}\big{|}_{U} \cong \Omega_{U/Y}$ and for all $x\in X$, $(\Omega_{X/Y})_{X} \cong \Omega_{\mathcal{O}_{X,x}/\mathcal{O}_{Y, f(x)}}$. \\
    (4)(conormal sequence)
    Let $i \hookrightarrow X$ be a closed immersion with ideal sheaf $\mathcal{I}$.
    Then there is an exact sequence
    \begin{equation*}
        \mathcal{I}/\mathcal{I}^{2} \overset{\delta}{\longrightarrow} i^{\ast}\Omega_{X/Y} \longrightarrow \Omega_{Z/Y} \longrightarrow 0
    \end{equation*}
    \label{Proposition 9.5}
\end{proposition}
\begin{proof}
    For (1), the question is local.
    Can reduce to the case $X = \spec(B)$, $Y = \spec A$ and $Y' = \spec(C)$.
    Then $\Omega_{X/Y} = \widetilde{\Omega_{B/A}}$ and $\Omega_{X'/Y'} = \widetilde{\Omega_{B \otimes_{A} C/C}}$.
    While by Proposition \ref{Proposition 9.1} (1), $\Omega_{B \otimes_{A} C/C} \cong \Omega_{B/A} \otimes_{B} (B \otimes_{A} C)$,
    get $p^{\ast}\widetilde{\Omega_{B/A}} \cong \widetilde{\Omega_{B/A} \otimes_{B} (B \otimes_{A} C)} \cong \widetilde{\Omega_{B \otimes_{A} C/C}}$, done!
    \par
    For (2), (3) and (4), the questions are also local, similarly by Proposition \ref{Proposition 9.1}, it is easy to prove for the reduced cases.
\end{proof}
\begin{remark}
    $\mathcal{I}/\mathcal{I}^{2}$ is naturally an $\mathcal{O}_{Z}$-module, called the conormal sheaf of $Z \subseteq X$.
\end{remark}
\begin{example}
    Let $k$ be a field and $Y = \spec k$, $X,Z$ geometrically integral and smooth over $k$.
    Will see that $\Omega_{X} := \Omega_{X/k}$, $\Omega_{Z}$ and $\mathcal{I}/\mathcal{I}^{2}$ are all locally free and there is an exact sequence
    \begin{equation*}
        0 \longrightarrow \mathcal{I}/\mathcal{I}^{2} \longrightarrow i^{\ast}\Omega_{X} \longrightarrow \Omega_{Z} \longrightarrow 0
    \end{equation*}
    Taking dual, get an exact sequence
    \begin{equation*}
        0 \longrightarrow \mathcal{T}_{Z} := \sheafhom_{\mathcal{O}_{Z}}(\Omega_{Z}, \mathcal{O}_{Z}) \longrightarrow i^{\ast}T_{X} \longrightarrow \mathcal{N}_{Z/X} := \sheafhom_{\mathcal{O}_{Z}}(\mathcal{I}/\mathcal{I}^{2}, \mathcal{O}_{Z}) \longrightarrow 0
    \end{equation*}
    which is obvious a algebraic version of exact sequence in Differential Geometry.
    However, when sheaf is not locally free, taking dual would lose some information.
    That's why considering cotangent space and conormal space is more convenient in algebraic geometry.
\end{example}
\begin{theorem}[\textbf{\emph{Euler Sequence}}]
    Let $A$ be a ring, $X = \mathbb{P}_{A}^{n} = \proj(A[x_{0}, \cdots, x_{n}])$.
    Then there is an exact sequence
    \begin{equation*}
        0 \longrightarrow \Omega_{X/A} \longrightarrow \mathcal{O}_{X}(-1)^{\oplus n + 1} \longrightarrow \mathcal{O}_{X} \longrightarrow 0
    \end{equation*}
\end{theorem}
\begin{proof}
    Denote $S = A[x_{0}, \cdots, x_{n}]$.
    Define graded homomorphism $\phi: S(-1)^{\oplus n + 1} \longrightarrow S$ mapping $(s_{0}, s_{1}, \cdots, s_{n})$ to $x_{0}s_{0} + x_{1}s_{1} + \cdots + x_{n}s_{n}$.
    Set $M = \ker(\phi)$, then there is an exact sequence
    \begin{equation*}
        0 \longrightarrow M \longrightarrow S(-1)^{\oplus n + 1} \longrightarrow S
    \end{equation*}
    Note that $\phi$ is surjective in all degree $\ge 1$, $\widetilde{\im(\phi)}\widetilde{\oplus_{d \ge 1} S_{d}} = \mathcal{O}_{X}$.
    Taking $\widetilde{\cdot}$, get 
    \begin{equation*}
        0 \longrightarrow \widetilde{M} \longrightarrow \mathcal{O}_{X}(-1)^{\oplus n + 1} \longrightarrow \mathcal{O}_{X} \longrightarrow 0
    \end{equation*}
    Remains to show that $\widetilde{M} \cong \Omega_{X/A}$.
    First observe that for all $i$, $M_{x_{i}}$ is a free $S_{x_{i}}$-module of rank $n$ with basis 
    \begin{equation*}
        \begin{aligned}
            (0, \cdots, & 1 & , \cdots, & -\frac{x_{k}}{x_{i}} & , \cdots, 0) \\
            & k && i &
        \end{aligned}
    \end{equation*}
    Thus $\widetilde{M}\big{|}_{D_{+}(x_{i})}$ is a free $\mathcal{O}_{D_{+}(x_{i})}$-module of rank $n$ with basis 
    \begin{equation*}
        \begin{aligned}
            (0, \cdots, & \frac{1}{x_{i}} & , \cdots, & -\frac{x_{k}}{x_{i}^{2}} & , \cdots, 0) \\
            & k && i &
        \end{aligned}
    \end{equation*}
    of degree $0$.
    Define isomorphism $\varphi_{i}: \Omega_{X/A}\big{|}_{D_{+}(x_{i})} \overset{\sim}{\longrightarrow} \widetilde{M}\big{|}_{D_{+}(x_{i})}$ mapping $d(\frac{x_{k}}{x_{i}})$ to 
    \begin{equation*}
        \begin{aligned}
            (0, \cdots, & \frac{1}{x_{i}} & , \cdots, & -\frac{x_{k}}{x_{i}^{2}} & , \cdots, 0) \\
            & k && i &
        \end{aligned}
    \end{equation*}
    Suffice to check that $\varphi_{i}$ can be glued up, which is clear.
\end{proof}
\subsection{Smoothness Revisit}
Let $k$ be a field of characteristic $p$, $l = k[t]/(t^{p} - a)$, $l/k$ inseparable.
Consider the conormal sequence 
\begin{equation*}
    I/I^{2} \longrightarrow \Omega_{k[t]/k} \otimes_{k[t]} l \longrightarrow \Omega_{l/k} \longrightarrow 0
\end{equation*}
where $I = (t^{p} - a)$.
Thus $\Omega_{l/k} = l \cdot dt/l \cdot d(t^{p} - a) = l$.
On the other hand, if $l/k$ is finite separable extension, then $\Omega_{l/k} = 0$.
There is a fact that if $l/k$ is finite field extension, then $\Omega_{l/k}$ is a finite dimensional $l$-vector space and $\dim_{l}(\Omega_{l/k}) \ge \trdeg_{k}(l)$.
And the equality is reached if and only if $l/k$ is separable.
Back to schemes, we have the following lemma.
\begin{lemma}
    Let $A$ be a $k$-algebra of finite type.
    Assume that $x\in \spec A$ is a closed point corresponding to a maximal ideal $\mathfrak{m} \subseteq A$ such that $A/\mathfrak{m} = k(x)$ is separable over $k$.
    Then the canonical map $\delta: \mathfrak{m}/\mathfrak{m}^{2} \longrightarrow \Omega_{A/k} \otimes_{A} k(x)$ is isomorphic.
    \label{Lemma 9.2}
\end{lemma}
\begin{reason}
    Consider the conormal sequence again and note by the previous fact, $\Omega_{k(x)/k} = 0$ so that $\delta$ is surjective.
    For injectivety, assume $A = k[x_{1}, \cdots, x_{n}]/I = B/I$ and $\mathfrak{M} \subseteq B$ is the inverse image of $\mathfrak{m}$.
    We have the following commutative diagram with exact rows
    \begin{equation*}
        \begin{tikzcd}
            I \arrow[r] \arrow[d] &
            \mathfrak{M}/\mathfrak{M}^{2} \arrow[r] \arrow[d, "{\delta'}"] &
            \mathfrak{m}/\mathfrak{m}^{2} \arrow[r] \arrow[d] &
            0 \\
            I \arrow[r] &
            \Omega_{B/k} \otimes_{B} k(x) \arrow[r] &
            \Omega_{A/k} \otimes_{A} k(x) \arrow[r] &
            0
        \end{tikzcd}
    \end{equation*}
    By 5 Lemma, it is reduced to prove $\delta'$ is surjective.
    Then repeat the argument, done!
\end{reason}
Let $X$ be a scheme of finite type over $k$.
Recall that $X$ is smooth at $x\in X$ if the points of $X_{\overline{k}}$ above $x$ are regular.
$X$ is smooth over $k$ if $X_{\overline{k}}$ is regular.
A morphism $f: X \longrightarrow S$ of finite type between locally noetherian schemes is smooth at $x\in X$ if it is flat at $x$ and $X_{s} \longrightarrow \spec(k(s))$ is smooth at $x$,
where $s = f(x)$.
$f$ is smooth if it is flat and for all $s\in S$, the fiber $X_{s}$ is smooth over $k(s)$.
\begin{lemma}
    Let $X$ be a scheme of finite type over $k$ which is pure of $\dim(X) = n$.
    Then for any field extension $l/k$, we have that $X_{l}$ is also pure of dimension $n$.
    \label{Lemma 9.3}
\end{lemma}
\begin{remark}
    Can been seen in Rising Sea.
\end{remark}
\begin{proposition}
    Let $X$ be a scheme of finite type over $k$ which is pure of $\dim(X) = n$.
    For all $x\in X$, he following conditions are equivalent \\
    (1)$X$ is smooth at $x$ \\
    (2)$\Omega_{X/k, x}$ is a free $\mathcal{O}_{X, x}$-module of rank $n$. \\
    (3)$X$ is smooth in an open neighbourhood of $x$.
    \label{Proposition 9.6}
\end{proposition}
\begin{proof}
    (3) $\Rightarrow$ (1) is obvious.
    \par
    For (2) $\Rightarrow$ (3), consider an affine open neighbourhood $U = \spec A$ of $x$.
    Then $\Omega_{X/k}\big{|}_{U}$ corresponds a finitely generated $A$-module.
    By argument about generators, it is clear that there is an open neighbourhood $V\ni x$ such that $\Omega_{X/k}\big{|}_{V}$ is free $\mathcal{O}_{V}$-module of rank $n$.
    By base change, get $\Omega_{V_{\overline{k}}/\overline{k}}$ also locally free of rank $n$.
    Note that for all closed point $y\in V_{\overline{k}}$, $k(y)/\overline{k}$ is separable.
    Similar to proof of Lemma \ref{Lemma 9.2}, we have $\mathfrak{m}_{y}/\mathfrak{m}_{y}^{2} \cong \Omega_{V_{\overline{k}}/\overline{k}, y} \otimes_{\mathcal{O}_{V_{\overline{k}}, y}} k(y)$.
    Thus $\dim_{k(y)}(\mathfrak{m}_{y}/\mathfrak{m}_{y}^{2}) = n$.
    By Lemma \ref{Lemma 9.3}, $\dim(\mathcal{O}_{V_{\overline{k}}, y}) = n$.
    Thus $V_{\overline{k}}$ is regular at closed points so that $V_{\overline{k}}$ is regular.
    Get $V$ is smooth.
    \par
    For (1) $\Rightarrow$ (2), let $x'\in X_{\overline{k}}$ be a point above $x$.
    Then with regularity at $x'$, there exists open neighbourhood $W\ni x'$ such that $W$ is regular.
    Can also assume that $W$ is of the form $U_{\overline{k}}$ for some open neighbourhood $U\ni x$.
\end{proof}
\begin{remark}
    For the proof of the $(1) \Rightarrow (2)$, there are several hints. \\
    (1)firstly, reduce to finite extension $l/k$.
    Show that there exists $W' \subseteq X_{l}$ such that $W'$ is regular. \\
    (2)As $X_{l} \longrightarrow X$ is flat and of finite type, it is open.
    Then project $W'$ to $X$ through it. \\
    (3)if one $z'\in X_{\overline{k}}$ above $z\in X$ is regular, then all points above $z$ are regular. \\
    (4)consider closed point $z\in U$ and $z'\in U_{\overline{k}}$ above $z$.
    Then by Lemma \ref{Lemma 9.2}, $\dim_{k(z)}(\Omega_{U/k, z} \otimes_{k} k(z)) = \dim_{k(z')}(\Omega_{U_{\overline{k}}, z'} \otimes_{\overline{k}} k(z')) = n$. 
\end{remark}
\begin{lemma}
    Let $X$ be a noetherian scheme, $\mathcal{F}$ coherent sheaf on $X$.
    Define $\varphi(x) = \dim_{k(x)}(\mathcal{F} \otimes_{\mathcal{O}_{X, x}} k(x))$.
    Then $\varphi$ is upper semicontinuous i.e. $\{x\in X \big{|} \varphi(x) \ge m\}$ is closed.
    If moreover $X$ is reduced and $\varphi$ is constant, then $\mathcal{F}$ is locally free.
    \label{Lemma 9.4}
\end{lemma}
\begin{proof}
    Both two statements follow from Nakayama.
    The proof of the first statement can be find in Hartshorne exercise II 5.8.
    Here, we only prove the second statement.
    For all $x\in X$, there exists affine open neighbourhood $U = \spec A$ of $x$ such that $\mathcal{F}\big{|}_{U} = \widetilde{M}$, where $M$ is a finitely generated $A$-module.
    Then there is a natural homomorphism $\phi: A^{n} \twoheadrightarrow M\quad (a_{1}, \cdots, a_{n}) \longmapsto \sum a_{i}m_{i}$.
    Suppose that $\phi$ is not injective.
    Then there exists $(a_{1}, \cdots, a_{n})\in \ker(\phi)$ with $a_{i} \neq 0$ for some $i$.
    As $A$ is reduced, there exists $\mathfrak{p}\in \spec A$ such that $a_{i}\notin \mathfrak{p}$.
    Thus $a_{i}$ is invertible in $A_{\mathfrak{p}}$ so that $M_{\mathfrak{p}}$ can be generated by less than $n$ elements, contradiction!
    Thus $M$ is free so that $\mathcal{F}$ is locally free.
\end{proof}
\begin{lemma}[\textbf{[\emph{Liu Qing}]}]
    Let $f: X \longrightarrow S$ be a morphism of finite type between locally noetherian schemes.
    Assume $s\in S$, $x\in X_{s}$ and set $n = \dim_{k(X)}(\Omega_{X_{s}/k(s), x} \otimes_{\mathcal{O}_{X_{s}, x}} k(x))$.
    Then there exist open neighbourhood $U\ni x$ and closed immersion $U \hookrightarrow Z$ with $Z$ smooth over $S$ at $x$ such that $Z_{s}$ is pure of $\dim(Z_{s}) = n$ and $\Omega_{Z/S, x}$ is free of rank $n$ over $\mathcal{O}_{Z, x}$.
    \label{Lemma 9.5}
\end{lemma}
\begin{proof}
    Can assume that $X$ and $S$ are affine.
    First write $X$ as a closed subscheme of an affine $S$-scheme $Y$ which is smooth over $S$ at $x$ and $Y_{s}$ is pure and $\Omega_{Y/S, x}$ is free, since $X$ is of finite type over $S$.
    In particular, we can take $Y = \mathbb{A}_{S}^{r}$. 
    Assume $\mathcal{I} \subseteq \mathcal{O}_{Y}$ is the ideal sheaf of $X$ in $Y$.
    Consider the conormal sequence for all $x\in X$.
    \begin{equation*}
        \mathcal{I}_{x}/\mathcal{I}_{x}^{2} \longrightarrow \Omega_{Y/S, x} \otimes_{\mathcal{O}_{Y, x}} \mathcal{O}_{X, x} \longrightarrow \Omega_{X/S, x} \longrightarrow 0
    \end{equation*}
    Tensor with $k(s)$ over $\mathcal{O}_{S, s}$, we can restrict to the fiber of $s$.
    \begin{equation*}
        \mathcal{I}_{x}/\mathcal{I}_{x}^{2} \otimes_{\mathcal{O}_{S, s}} k(s) \longrightarrow \Omega_{Y_{s}/k(s), x} \otimes_{\mathcal{O}_{Y_{s}, x}} \mathcal{O}_{X_{s}, x} \longrightarrow \Omega_{X_{s}/k(s), x} \longrightarrow 0
    \end{equation*}
    Agian, tensor with $k(x)$ over $\mathcal{O}_{X_{s}, x}$
    \begin{equation*}
        \mathcal{I}_{x}/\mathcal{I}_{x}^{2} \otimes_{\mathcal{O}_{S, s}} k(s) \otimes_{\mathcal{O}_{X_{s}, x}} k(x) \longrightarrow \Omega_{Y_{s}/k(s), x} \otimes_{\mathcal{O}_{Y_{s}, x}} k(x) \longrightarrow \Omega_{X_{s}/k(s), x} \otimes_{\mathcal{O}_{X_{s}, x}} k(x) \longrightarrow 0
    \end{equation*}
    If $\dim(Y) = n$, then we can take $Z = Y$.
    \par
    Suppose that $\dim(Y) = m > n$.
    As $Y$ is smooth over $S$ at $x$, by Proposition \ref{Proposition 9.6}, $\dim_{k(x)}(\Omega_{Y_{s}/k(s), x} \otimes_{\mathcal{O}_{Y_{s}, x}} k(x)) = m$.
    Thus there exists $f\in \mathcal{I}_{x}$ whose image $d\overline{f}$ can be completed to a basis $\{d\overline{f}, d\overline{f}_{2}, ..., d\overline{f}_{m}\}$.
    Furthermore, by restricting Y, can assume $f, f_{i}\in \mathcal{O}_{Y}$.
    Set $Z = V(f) \subseteq Y$, $X \subseteq Z \subseteq Y$.
    By Nakayama, $\{df_{1}, \cdots, df_{m}\}$ span $\Omega_{Y/S, x}$.
    As $\Omega_{Y/S}$ is free of rank $m$, $\{df_{1}, \cdots, df_{m}\}$ form a basis of $\Omega_{Y/S, x}$.
    Now $\Omega_{Z/S, x} = \Omega_{Y/S, x}/(df) \otimes_{\mathcal{O}_{Y, x}} \mathcal{O}_{Z, x}$ is free of rank $m - 1$.
    By base change, $\Omega_{Z_{s}/k(s), x}$ is free of rank $m - 1$.
    By Proposition \ref{Proposition 9.6}, $Z_{s}$ is smooth at $x$.
    As $f\in \mathcal{O}_{Y_{s}, x}$ is a non-zero-divisor in an integral domain, irreducible components of $Z_{s}$ are of dimension $m - 1$.
    Finally by "local criterion for flatness", the Hartshorne Chapter III Lemma 10.5 A, flatness of $\mathcal{O}_{Z, x}$ over $\mathcal{O}_{S, s}$ is ok.
    Now we can finish by induction.
\end{proof}
\begin{proposition}
    Let $f: X \longrightarrow S$ be a morphism of finite type between locally noetherian schemes.
    Assume $s\in S$, $x\in X_{s}$ and $X_{s}$ is pure of dimension $n$.
    If $X$ is smooth over $S$ at $x$, then $\Omega_{X/S}$ is free of rank $n$ in an open neighbourhood of $x$.
    \label{Proposition 9.7} 
\end{proposition}
\begin{proof}
    Let $U \hookrightarrow Z$ be as in Lemma \ref{Lemma 9.5}, then $\dim(X_{s}) = \dim(Z_{s}) = n$.
    Thus $X_{s} = Z_{s}$ in an open neighbourhood of $x$.
    As $X$ is flat over $S$ at $x$, $X = Z$ in an open neighbourhood of $x$.
    Get $\Omega_{X/S, x} = \Omega_{Z/S, x}$ is free of rank $n$.
    Then $\Omega_{X/S}$ is free of rank $n$ in an open neighbourhood of $x$.
\end{proof}
\begin{theorem}
    Let $f: X \longrightarrow S$ be a morphism of finite type between locally noetherian schemes with fibers pure of dimension $n$.
    Then $f$ is smooth if and only if $f$ is flat and $\Omega_{X/S}$ is locally free of rank $n$.
    \label{Theorem 9.1}
\end{theorem}
\begin{reason}
    The theorem is immediately given by Proposition \ref{Proposition 9.6} and Proposition \ref{Proposition 9.7}.
\end{reason}
\begin{remark}
    Flatness is needed.
    Otherwise if $X \hookrightarrow Y$ is a closed immersion, then $\Omega_{X/Y} = 0$.
\end{remark}
\begin{corollary}
    Let $f: X \longrightarrow Y$ be a dominant morphism between integral schemes of finite type over $k$ of characteristic $0$.
    Then there exists nonempty open subset $U \subseteq X$ such that $f\big{|}_{U}: U \longrightarrow Y$ is smooth.
    \label{Corollary 9.2}
\end{corollary}
\begin{proof}
    Since characteristic of $k$ is $0$, $K(X)$ is separable over $K(Y)$.
    Thus 
    \begin{equation*}
        \dim(\Omega_{X/Y, \eta_{X}}) = \trdeg_{K(Y)}(K(X)) = \dim(X) - \dim(Y) = \dim(X_{\eta_{Y}}) = n
    \end{equation*}
    Get $\Omega_{X/Y}$ is free of rank $n$ on a nonempty open subset of X.
    Note that $f$ is also flat with fiber pure of dimension $n$ on an nonempty open subset of X.
    By Theorem \ref{Theorem 9.1}, $f$ is smooth on a nonempty open subset of $X$.
\end{proof}
\begin{example}
    This example shows that of characteristic $0$ is needed in the previous corollary.
    Consider Frobinius map $\mathbb{A}_{\mathbb{F}_{p}}^{1} \longrightarrow \mathbb{A}_{\mathbb{F}_{p}}^{1}$ corresponding to ring homomorphism $\mathbb{F}_{p}[t] \longrightarrow \mathbb{F}_{p}[t]\quad t \longmapsto t^{p}$.
    Then it is nowhere smooth.
\end{example}
\subsection{Some Notions}
For simplicity, all schemes in this subsection are assumed geometrically intgral $k$-schemes of finite type and all morphisms are $k$-morphisms.
\begin{definition}
    Let $X$ be a smooth $k$-scheme pure of dimension $n$.
    Then by Proposition \ref{Proposition 9.7}, $\Omega_{X/k}$ is locally free of rank $n$.
    Set $\mathcal{T}_{X} := \sheafhom_{\mathcal{O}_{X}}(\Omega_{X/k}, \mathcal{O_{X}})$ to be the tangent sheaf, which is locally free of rank $n$.
    Also, we can define tangent bundle $T_{X} := \spec(Sym(\Omega_{X/k}))$, which is a vector bundle of rank $n$.
    \par
    Define sheaf of $r$-forms $\Omega_{X/k}^{r} := \wedge^{r} \Omega_{X/k}$ that is locally free of rank $\binom{n}{r}$.
    In particular, define canonical sheaf $\omega_{X} (= \omega_{X/k}) := \Omega_{X/k}^{n} = \det(\Omega_{X/k})$ to the sheaf of $n$-forms, which is invertible.
    Set $K_{X}$ to any Cartier divisor such that $\mathcal{O}_{X}(K_{X}) \cong \omega_{X}$.
    \par
    If $X$ is also projective (proper) over $k$, set $P_{g}(X) := \dim_{k}(\Gamma(X, \omega_{X}))$ to be the geometric genus.
\end{definition}
\begin{proposition}
    If $X,Y$ are smooth and projective over $k$ and $X,Y$ are $k$-birational, then $P_{g}(X) = P_{g}(Y)$ i.e. $P_{g}$ is birational invariant.
    \label{Proposition 9.8}
\end{proposition}
\begin{proof}
    Let $f: X \longrightarrow Y$ be the birational map.
    As $X$ is normal and $Y$ is proper, there exists open subset $U \subseteq X$ such that $f\big{|}_{X \setminus U}$ is injective and $X \setminus U$ is of codimension $\ge 2$.
    Get $f^{\ast}\Omega_{Y/k} \longrightarrow \Omega_{U/k}$ both locally free of rank $n = \dim(X) = \dim(Y)$, inducing map $f^{\ast}\omega_{Y} \longrightarrow \omega_{U}$.
    Thus there are two maps
    \begin{equation*}
        \varphi: \Gamma(Y, \omega_{Y}) \longrightarrow \Gamma(U, \omega_{U})
    \end{equation*} 
    and 
    \begin{equation*}
        \psi: \Gamma(X, \omega_{X}) \longrightarrow \Gamma(U, \omega_{U})
    \end{equation*}
    Claim that the first one is injective and the second one is bijective.
    \par
    Note that there exists open subset $V \subseteq U \subseteq X$ such that $f\big{|}_{V}$ is isomorphic onto image, get $\omega_{U}\big{|}_{V} \cong \omega_{Y}\big{|}_{f(V)}$.
    For all $s\in \ker(\varphi)$, have that $s\big{|}_{f(V)} = 0$ so that $s = 0$.
    Then $\varphi$ is injective.
    \par
    For the second claim, suffices to show that for all affine open subsets $V \subseteq X$ trivializing $\omega_{X}$, $\Gamma(V, \omega_{X}\big{|}_{V}) \longrightarrow \Gamma(U \cap V, \omega_{X}\big{|}_{U \cap V})$ is bijective.
    As $X$ is normal and $X \setminus U$ is of codimension $\ge 2$, this is a result form Algebraic Hartogs.
\end{proof}
\begin{proposition}
    Let $X,Y$ be smooth $k$-schemes, $f: X \longrightarrow Y$ $k$-morphism. \\
    (1)If $f$ is smooth, then the following sequence is exact
    \begin{equation*}
        0 \longrightarrow f^{\ast}\Omega_{Y/k} \longrightarrow \Omega_{X/k} \overset{\beta}{\longrightarrow} \Omega_{X/Y} \longrightarrow 0
    \end{equation*}
    (2)If $f$ is closed (or locally closed) immersion with ideal sheaf $\mathcal{I}$,
    then the following sequence is exact
    \begin{equation*}
        0 \longrightarrow \mathcal{I}/\mathcal{I}^{2} \longrightarrow f^{\ast}\Omega_{Y/k} \overset{\alpha}{\longrightarrow} \Omega_{X/k} \longrightarrow 0 
    \end{equation*}
    \label{Proposition 9.9}
\end{proposition}
\begin{proof}
    In both (1) and (2), $\ker(\beta)$ and $\ker(\alpha)$ are locally free, thus exactness on the left follows by computing ranks, provided we know $\mathcal{I}/\mathcal{I}^{2}$ locally free of the expected rank.
    As $\ker(\alpha)$ is locally free of rank $r = \codim_{Y}(X)$, for all $x\in X$ closed point, can choose $f_{1}, \cdots, f_{r}\in \mathcal{I}$ in an open neighbourhood of $x$ such that $df_{1}, \cdots, df_{r}$ form a basis of $\ker(\alpha)$.
    Set $\mathcal{I}' = (f_{1}, ..., f_{r})$ corresponds to another closed immersion $X' \overset{f'}{\hookrightarrow} Y$.
    Then the following sequence is exact.
    \begin{equation*}
        0 \longrightarrow \mathcal{I}'/{\mathcal{I}'}^{2} \longrightarrow {f'}^{\ast}\Omega_{Y/k} \overset{\alpha'}{\longrightarrow} \Omega_{X'/k} \longrightarrow 0
    \end{equation*}
    Thus $X'$ is smooth over $k$ of dimension $\dim(Y) - r$ in an open neighbourhood of $x$.
    But $X$ is smooth over $k$ of same dimension, get $X = X'$ and $\mathcal{I} = \mathcal{I}'$ in an open neighbourhood of $x$.
\end{proof}
\begin{remark}
    Taking dual and the associated vector bundle in (1), get exact sequence
    \begin{equation*}
        0 \longrightarrow T_{X/Y} \longrightarrow T_{X} \longrightarrow f^{\ast}T_{Y} \longrightarrow 0
    \end{equation*}
    \par
    In (2), define normal sheaf $\mathcal{N}_{X/Y} := \sheafhom_{\mathcal{O}_{X}}(\mathcal{I}/\mathcal{I}^{2}, \mathcal{O}_{X})$, corresponding to normal bundle $N_{X/Y} := \spec(Sym(\mathcal{I}/\mathcal{I}^{2}))$.
    Taking dual and the associated vector bundle in (2), get exact sequence
    \begin{equation*}
        0 \longrightarrow T_{X} \longrightarrow f^{\ast}T_{Y} \longrightarrow N_{X/Y} \longrightarrow 0
    \end{equation*}
\end{remark}
\begin{example}
    Let $D \subseteq X$ be an effective (Cartier) divisor, with both $X,D$ smooth over $k$.
    Then $\mathcal{I}/\mathcal{I}^{2} \cong \mathcal{O}_{X}(-D)\big{|}_{D}$, $\mathcal{N}_{D/X} \cong \mathcal{O}_{X}(D)\big{|}_{D}$ and the following sequence is exact
    \begin{equation*}
        0 \longrightarrow \mathcal{O}_{X}(-D)\big{|}_{D} \longrightarrow \Omega_{X/k}\big{|}_{D} \longrightarrow \Omega_{D/k} \longrightarrow 0
    \end{equation*}
\end{example}
\begin{proposition}
    Let $X$ be a scheme, $\mathcal{F}$ locally free sheaf on $X$ of rank $n < \infty$.
    Define $\det(\mathcal{F}) := \wedge^{n} \mathcal{F}$, which is invertible.
    Then \\
    (1)$\det(\mathcal{F}^{\lor}) \cong (\det(\mathcal{F}))^{\lor}$ is canonical isomorphism. \\
    (2)for all $f: Y \longrightarrow X$ morphism of schemes, $\det(f^{\ast}\mathcal{F}) \cong d^{\ast}\det(\mathcal{F})$. \\
    (3)If there is an exact sequence of locally free sheaves of rank $< \infty$
    \begin{equation*}
        0 \longrightarrow \mathcal{E} \longrightarrow \mathcal{F} \longrightarrow \mathcal{G} \longrightarrow 0
    \end{equation*}
    then $\det(\mathcal{E}) \otimes_{\mathcal{O}_{X}} \det(\mathcal{G}) \cong \det(\mathcal{F})$.
    \label{Proposition 9.10}
\end{proposition}
\begin{corollary}[\textbf{\emph{Adjunction Formula}}]
    Let $D \subseteq X$ be an effective (Cartier) divisor, with both $X,D$ smooth over $k$.
    Then $\omega_{D} \cong (\omega_{X} \otimes_{\mathcal{O}_{X}} \mathcal{O}_{X}(D))\big{|}_{D}$ and $K_{D} \sim (K_{X} + D)\big{|}_{D}$.
    \label{Corollary 9.3}
\end{corollary}
\begin{example}
    Recall that when $X = \mathbb{P}_{k}^{n}$, there is an exact sequence
    \begin{equation*}
        0 \longrightarrow \Omega_{X/k} \longrightarrow \mathcal{O}_{X}(-1)^{\oplus n + 1} \longrightarrow \mathcal{O}_{X} \longrightarrow 0
    \end{equation*}
    Thus $\omega_{X} \cong \mathcal{O}_{X}(-n - 1)$ and $\omega_{\mathbb{P}_{k}^{1}} \cong \mathcal{O}_{\mathbb{P}_{k}^{1}}(-2)$.
    If $D \subseteq \mathbb{P}_{k}^{n}$ is hypersurface smooth over $k$ of degree $d$,
    then by Adjunction Formula, $\omega_{D} \cong \mathcal{O}_{\mathbb{P}_{k}^{n}}(-n - 1 + d)\big{|}_{D}$.
    \par
    People classify hypersurfaces depending on degree $d$.
    \begin{equation*}
        \left\{
            \begin{aligned}
                & d < n + 1 & \text{Fano} \\
                & d = n + 1 & \text{Calabi-Yau} \\
                & d > n + 1 & \text{general type}
            \end{aligned}
        \right.
    \end{equation*} 
    When $d \ge n + 1$, we see that $P_{g}(d) > 0$ so that $D$ is not rational.
\end{example}

\section{Blowing-up}
\label{section:Blowing-up}

\subsection{Warm-up Construction}
Let $X$ be a noetherian scheme, $\mathcal{G}$ quasi-coherent sheaf on $X$ which is a graded algebra object in $Mod_{\mathcal{O}_{X}}$ i.e. $\mathcal{G} = \oplus_{d \ge 0} \mathcal{G}_{d}$.
Further assume that $\mathcal{G}_{0} = \mathcal{O}_{X}$, $\mathcal{G}_{1}$ is coherent $\mathcal{O}_{X}$-module and $\mathcal{G}$ is locally generated by $\mathcal{G}_{1}$ as $\mathcal{G}_{0} = \mathcal{O}_{X}$-algebra.
Thus for all $d$, $\mathcal{G}_{d}$ is coherent.
\par
(1)Can define $C = \spec\mathcal{G}$ called the cone of $\mathcal{G}$ by gluing the various Spec over an affine open covering of $X$.
Then there is a natural affine morphism $\pi: C \longrightarrow X$.
For example, if $\mathcal{E}$ is a locally free sheaf over $X$, then $Sym(\mathcal{E}^{\lor})$ is graded $\mathcal{O}_{X}$-algebra so that vector bundle $E = \spec(Sym(\mathcal{E}^{\lor}))$ is the cone of $Sym(\mathcal{E}^{\lor})$.
\par
(2)Also can define $\mathbb{P}(C) = \proj(\mathcal{G})$, called the projective cone of $\mathcal{G}$ by gluing the various $\proj$ over an affine open covering of $X$.
Then there is a natural proper morphism $p: \mathbb{P}(C) \longrightarrow X$ and a natural invertible sheaf $\mathcal{O}_{\mathbb{P}(C)}(1)$ on $\mathbb{P}(C)$.
For example, if $\mathcal{E}$ is a locally free sheaf over $X$, then $Sym(\mathcal{E}^{\lor})$ is graded $\mathcal{O}_{X}$-algebra so that projectiver bundle $\mathbb{E} = \proj(Sym(\mathcal{E}^{\lor}))$ is the projective cone of $Sym(\mathcal{E}^{\lor})$.
Note that here $\mathbb{P}(E)$ is the $\mathbb{P}(\mathcal{E}^{\lor})$ in Hartshorne.
\par
(3)If $\mathcal{G} \longrightarrow \mathcal{G}'$ is a surjective morphism of graded $\mathcal{O}_{X}$-algebras,
consider $C = \spec\mathcal{G}$ and $C' = \spec(\mathcal{G}')$, then there is closed immersions $C' \hookrightarrow C$ and $\mathbb{P}(C') \hookrightarrow \mathbb{P}(C)$ such that $\mathcal{O}_{\mathbb{P}(C)}(1)\big{|}_{\mathbb{P}(C')} \cong \mathcal{O}_{\mathbb{P}(C')}(1)$.
In particular, as $\mathcal{G} \longrightarrow \mathcal{G}_{0} = \mathcal{O}_{X}$ is surjective, get zero section $0: X \hookrightarrow C$.
\par
(4)(base change)Let $C = \spec\mathcal{G}$ on X be the cone of $\mathcal{G}$ and $f: X' \longrightarrow X$ be a morphism with $X'$ noetherian.
Then $f^{\ast}C := C \times_{X} X' \cong \spec(f^{\ast}\mathcal{G})$ and $\mathbb{C} \times_{X} X' \cong \proj(f^{\ast}\mathcal{G})$, where $f^{\ast}\mathcal{G}$ is a graded $\mathcal{O}_{X'}$-algebra.
\subsection{Blowing-up}
\begin{definition}
    Let $X$ be a noetherian scheme and $Y \hookrightarrow X$ be a closed immersion.
    Assume $\mathcal{I} \subseteq \mathcal{O}_{X}$ is the corresponding ideal sheaf of the closed immersion. \\
    (1)The normal cone of $Y$ in $X$ is the cone $C_{Y/X} := \spec(\oplus_{d \ge 0} \mathcal{I}^{d}/\mathcal{I}^{d + 1})$,
    where $\mathcal{I}^{0} = \mathcal{O}_{X}$ and $\mathcal{I}^{0}/\mathcal{I}^{1}$ is isomorphic as the push forward of $\mathcal{O}_{Y}$ so that $\oplus_{d \ge 0} \mathcal{I}^{d}/\mathcal{I}^{d + 1}$. \\
    (2)The blow-up of $X$ along $Y$ is the projective $\widetilde{X} = Bl_{Y}(C) := \proj(\oplus_{d \ge 0} \mathcal{I}^{d})$ and $\mathcal{O}_{\widetilde{X}}(1)$ is an invertible sheaf on $\widetilde{X}$.
\end{definition} 
\begin{proposition}
    Let $X$ be a noetherian scheme and $Y \hookrightarrow X$ be a closed immersion.
    Assume $\mathcal{I} \subseteq \mathcal{O}_{X}$ is the corresponding ideal sheaf of the closed immersion.
    Then we have that \\
    (1)$E = p^{-1}(Y) \cong \mathbb{P}(C_{Y/X})$ called the exceptional divisor,
    where $p: \widetilde{X} \longrightarrow X$ is the natural morphism \\
    (2)ideal sheaf of $E$ is isomorphic to $\mathcal{O}_{\widehat{X}}(1)$. \\
    (3)$U = X \setminus Y$.
    Then $p: p^{1}(U) \longrightarrow U$ is isomorphic.
    \label{Proposition 10.1}
\end{proposition}
\begin{proof}
    For (1), as $\oplus_{d \ge 0} \mathcal{I}^{d} \otimes_{\mathcal{O}_{X}} \mathcal{O}_{X}/\mathcal{I} \cong \oplus_{d \ge 0} \mathcal{I}^{d}/\mathcal{I}^{d + 1}$, take $\proj$, done! 
    \par
    For (2), ideal sheaf of $E$ is isomorphic to $\mathcal{I}\mathcal{O}_{\widetilde{X}}$.
    As $\mathcal{I}\oplus_{d \ge 0} \mathcal{I}^{d} = \oplus_{d \ge 1} \mathcal{I}^{d}$, ideal sheaf is isomorphic to $\mathcal{O}_{\widetilde{X}}(1)$. 
    \par
    For (3), we have that $p^{-1}(U) = \proj(\mathcal{O}_{U} \oplus \mathcal{O}_{U} \oplus \cdots) \cong \proj(\mathcal{O}_{U}[t]) \cong U$ and $\mathcal{I}\big{|}_{U} \cong \mathcal{O}_{X}\big{|}_{U}$.
\end{proof}
\begin{example}
    (1)Let $X = \spec A$, $Y$ closed subscheme of $X$ corresponding to ideal $I \subseteq A$.
    Assume that $f_{1}, \cdots, f_{n}$ are generators of $I$ and $S = \oplus_{d \ge 0} I^{d}$.
    There is a surjective morphism of graded $A$-algebra $\phi: A[y_{1}, \cdots, y_{n}] \longrightarrow S\quad y_{i} \longmapsto t_{i}\in S_{1} = I$.
    Here $t_{i}$ corresponds to $f_{i}$.
    Set $J := \ker(\phi)$, then $f_{i}y_{j} - f_{j}y_{i}\in J_{1}$.
    Then $\phi$ induces a closed immersion $\widetilde{X} = \proj(S) \hookrightarrow \mathbb{P}_{A}^{n - 1}$.
    \\
    (2)in particular, if $I = (f)$, where $f$ is not invertible and not zero-divisor, then $\phi: A[t] \longrightarrow S$ is isomorphic so that $\widetilde{X} \longrightarrow \mathbb{P}_{A}^{0} = \spec A = X$ is isomorphic.
    More generally, for $X$ noetherian scheme and $Y \hookrightarrow X$ closed immersion, $p: \widetilde{X} \longrightarrow X$ is isomorphic if and only if $Y$ is an effective Cartier divisor.
    On the other hand, when $X = \spec A$ is noetherian affine scheme, $\widetilde{X} = \varnothing$ if and only if $I$ is nilpotent.
    \\
    (3)Note that $S$ is an integral domain (resp. reduced) if and only if $A$ is an integral domain (resp. reduced).
    Thus for $X$ noetherian scheme, if $X$ is integral and $\mathcal{I} \neq 0$, then $\widetilde{X}$ is intgral and $p: \widetilde{X} \longrightarrow X$ is birational.
    \\
    (4)Let $X = \mathbb{A}_{k}^{n} = \spec(k[x_{1}, \cdots, x_{n}])$ and $Y = \{(0, \cdots, 0)\}$.
    Then corresponding iedal $I = (x_{1}, \cdots, x_{n})$ and $J = \ker(\phi) = (x_{i}y_{j} - x_{i}y_{i})_{1 \le i \le n, 1 \le j \le n}$.
    \\
    (5)Let $X = \spec A$.
    Set $z_{i} := \frac{y_{i}}{y_{1}}\in \mathcal{O}_{\widetilde{X}(D_{+}(y_{i}))}$.
    Then $J_{(y_{1})} = \{P\in A[z_{2}, \cdots, z_{n}] \big{|} \exists d \ge 0 \text{ such that } f_{1}^{d}P\in (f_{1}z_{2} - f_{2}, \cdots, f_{1}z_{n} - f_{n})\}$.
    \\
    (6)We have seen that $J' := (f_{i}z_{j} - f_{j})_{1 \le i \le n, q \le j \le n} \subseteq J$.
    Suppose that $\{f_{i}\}$ form a minimal set of generators and $Z := V_{+}(J^{'})$ is integral.
    Then $\widehat{X} = V_{+}(J) \hookrightarrow Z = V_{+}(J')$ is isomorphic.
    It suffices to check on each $D_{+}(y_{i}) \cap Z$.
    Note that $f_{i}\notin (f_{j})_{j \neq i}$ so that $f_{i} \neq 0$ on $D_{+}(y_{i}) \cap Z$.
    As $Z$ is integral, $f_{i}$ is not zero-divisor on $D_{+}(y_{i}) \cap Z$.
    Then apply (5).
    \\
    (7)Let $A = k[x, y]/(y^{2} - x^{3} - x^{2})$ and $X = \spec A$, $I = (x, y) \subseteq A$, $Y = \{(0, 0)\}$.
    Then $C_{Y/X} = \spec(k[x, y]/(y^{2} - X^{2}))$.
    That's why it looks like two lines when we stand at point $(0, 0)$.
    \\
    (8)Let $A = k[x, y]/(y^{2} - x^{3})$ and $X= \spec A$, $I = (x, y) \subseteq A$, $Y = \{(0, 0)\}$.
    Then $C_{Y/X} = \spec(k[x, y]/(y^{2}))$.
    That's why it looks like two coincided lines when we stand at point $(0, 0)$.
    \\
    (9)Let $A = k[x, y, z]/(xy - z^{2})$ and $X = \spec A$, $I = (y, z) \subseteq A$, $Y = \spec(k[x])$.
    Then $C_{Y/X} = \spec(k[x][y, z]/(xy))$.
    Thus when $x \neq 0$, $C_{Y/X}$, the blow up is still one point, while when $x = 0$, the blow up is a line.
    \label{example:many_examples_about_blow_up}
\end{example}
\begin{proof}[\textbf{\emph{Proof of (5)}}]
    "$\subseteq$": Let $P\in J_{(y_{1})} \subseteq A[z_{2}, \cdots, z_{n}]$.
    Can write for some $d \ge 0$ that $f_{1}^{d}P = \sum_{i = 2}^{n} Q_{i}(f_{1}z_{i} - f_{i}) + a$, where $a\in A$.
    As $P\in J_{(y_{1})}$ and $f_{1}z_{i} - f_{i}\in J_{(y_{1})}$, get $a\in J_{(y_{1})}$ i.e. $\phi(a) = 0\in S_{(t_{1})}$.
    Thus there exists $r \ge 0$ such that $at_{1}^{r} = 0$ in $S$ so that $af_{1}^{r} = 0$ in $A$.
    Replace $d$ by $d + r$, get $a = 0$ and $f_{1}^{d}P\in (f_{1}z_{2} - f_{2}, \cdots, f_{1}z_{n} - f_{n})$.
    \par
    "$\supseteq$": Let $P\in A[z_{2}, \cdots, z_{n}]$ such that $f_{1}^{d}P\in (f_{1}z_{2} - f_{2}, \cdots, f_{1}z_{n} - f_{n})$.
    Write $P = \frac{Q(y_{1}, ..., y_{n})}{y_{1}^{r}}$ with $Q(y_{1}, \cdots, y_{n})$ homogeneous of degree $r$.
    Thus $f_{1}^{d}y_{1}^{e}Q(y_{1}, \cdots, y_{n})\subseteq J$ for some $e$.
    Get $t_{1}^{d + e}Q(t_{1}, \cdots, t_{n}) = 0$ in $S$ so that $y_{1}^{d + e}Q(y_{1}, \cdots, Y_{n}) = 0$ in $J$.
    Thus $P = \frac{y_{1}^{d + e}Q(y_{1}, \cdots, y_{n})}{y_{1}^{d + e + r}}\in J_{(y_{1})}$.
\end{proof}
\begin{proposition}[\textbf{\emph{Base Change}}]
    Let $X$ be a noetherian scheme, $Y \hookrightarrow X$ closed immersion, $f: X' \hookrightarrow X$ morphism with $X'$ noetherian.
    Then there is a closed immersion $Bl_{Y'}(X') \hookrightarrow Bl_{Y}(X) \times_{X} X'$.
    Let $\widetilde{f}: Bl_{Y'}(X') \longrightarrow Bl_{Y}(X)$ be the composition of the closed immersion with projection,
    then $\widetilde{f}^{-1}(E) = E'$.
    \label{Proposition 10.2}
\end{proposition}
\begin{proof}
    Reduce to affine case that $X = \spec A$, $Y = \spec(A/I)$ and $X' = \spec(A')$.
    Then we have exact sequence
    \begin{equation*}
        I \otimes_{A} A' \longrightarrow A' \longrightarrow A/IA' \longrightarrow 0
    \end{equation*}
    Note that $I \otimes_{A} A' \longrightarrow IA'$ is surjective, get $(\oplus_{d \ge 0} I^{d}) \otimes_{A} A' \longrightarrow \oplus_{d \ge 0} (IA')^{d}$ surjective, inducing the wanted closed immersion.
    What's more, by commutative diagram, it is clear that $E' = \widetilde{f}^{-1}(E)$.
\end{proof}
\begin{remark}
    There are two special cases \\
    (1)If $d: X' \longrightarrow X$ is flat, then $Bl_{Y'}(X') \cong Bl_{Y}(X) \times_{X} X'$. \\
    (2)If $Y \hookrightarrow X$ and $Z \hookrightarrow X$ are both closed immersions,
    then with the following commutative diagram 
    \begin{equation*}
        \begin{tikzcd}
            Y \cap Z \arrow[r, hookrightarrow] \arrow[d, hookrightarrow] &
            Z \arrow[d, hookrightarrow] \\
            Y \arrow[r, hookrightarrow] &
            X 
        \end{tikzcd}
    \end{equation*}
    get that
    \begin{equation*}
        \begin{tikzcd}
            Bl_{Y \cap Z}(Z) \arrow[rr, hookrightarrow] \arrow[dr, hookrightarrow] &&
            Bl_{Y}(X) \\
            & Bl_{Y}(X) \times_{X} Z \arrow[ur, hookrightarrow] &
        \end{tikzcd}
    \end{equation*}
    where $Bl_{Y \cap Z}(Z)$ is called the strict transform and $Bl_{Y}(Z) \times_{X} Z$ is called the total transform.
\end{remark}
\begin{theorem}
    Let $X$ be a noetherian scheme, $Y \hookrightarrow X$ closed immersion, $p: \tilde{X} = Bl_{Y}(X) \longrightarrow X$ natural morphism.
    For all $f: Z \longrightarrow X$ morphism with $Z$ noetherian such that the pull back $W$ of $Z$ along $Y \hookrightarrow X$ is an effective Cartier divisor,
    there exists unique morphism $g: Z \longrightarrow \widetilde{X}$ such that the following diagram commutes
    \begin{equation*}
        \begin{tikzcd}
            Z \arrow[r, "g"] \arrow[dr, "f"] &
            \widetilde{X} \arrow[d, "p"] \\
            & X
        \end{tikzcd}
    \end{equation*}
    \label{Theorem 10.1}
\end{theorem}
\begin{proof}
    As discuss in Example \ref{example:many_examples_about_blow_up} (2), since $W$ is an effective divisor, we have that $Bl_{W}(Z) \cong Z$.
    Thus the existence of $g$ follows form Proposition \ref{Proposition 10.2}.
    For uniqueness, reduce to the affine case that $X = \spec A$ and $Y = \spec(A/I)$.
    Then the image of $I \longrightarrow \mathcal{O}_{Z}(Z)$ generates the ideal sheaf $\mathcal{J}$ corresponding to $W$.
    Let $f_{1}, \cdots, f_{n}$ be generators of $I$, mapping to global sections $s_{1}, \cdots, s_{n}$ of $\mathcal{J}$ which also generate $\mathcal{J}$.
    In fact, such $\mathcal{J}$ is invertible.
    Thus by Proposition \ref{Proposition 6.10}, there exists unique morphism $h: Z \longrightarrow \mathbb{P}_{A}^{n - 1}$ such that $\mathcal{J} \cong h^{\ast}\mathcal{O}_{\mathbb{P}_{A}^{n - 1}}(1)$ and $s_{i} = h^{\ast}(x_{i})$.
    Easy to show that $h$ factors through $\widetilde{X}$ so that $g$ is uniquely determined by $h$.
\end{proof}
\begin{proposition}
    Let $X$ be a scheme of finite type ober field $k$, $Y \hookrightarrow X$ closed immersion.
    If $X$ is pure of dimension $n$, then $C_{Y/X}$ is also pure of dimension $n$.
    \label{Proposition 10.3}
\end{proposition}
\begin{proof}
    Consider $Y \times_{k} \{0\} \hookrightarrow X \times_{k} \{0\} \hookrightarrow X \times_{k} \mathbb{A}_{k}^{1}$,
    where $\{0\}$ is the one-point closed subset of $\mathbb{A}_{k}^{1}$.
    Then $C_{Y \times_{k} \{0\}/X \times_{k} \mathbb{A}_{k}^{1}} \cong C_{Y/X} \oplus \mathbbm{1}$, here the definition of $C_{Y/X} \oplus \mathbbm{1}$ can be seen in the following remark.
    Consider $Bl_{Y \times_{k} \{0\}}(X \times_{k} \mathbb{A}_{k}^{1}) \overset{birational}{\longrightarrow} X \times_{k} \mathbb{A}_{k}^{1}$.
    Thus $\dim(Bl_{Y \times_{k} \{0\}}(X \times_{k} \mathbb{A}_{k}^{1})) = \dim(X \times_{k} \mathbb{A}_{k}^{1}) = \dim(X) + 1$.
    As $E = \mathbb{P}(C_{Y/X} \oplus \mathbbm{1})$ is an effective Cartier divisor which is pure of codimension $1$ and $C_{Y/X}$ is an open subset of $E$,
    get $\dim(C_{Y/X}) = \dim(E) = \dim(X) + 1 - 1 = \dim(X)$.
\end{proof}
\begin{remark}
    For $\mathcal{G}$ graded $\mathcal{O}_{Y}$-algebra, we define graded $\mathcal{O}_{Y}$-algebra $\mathcal{G}[t]$, whose graded structure is given by $\mathcal{G}[t]_{d} := \mathcal{G}_{d} \oplus \mathcal{G}_{d - 1}t \oplus \cdots \oplus \mathcal{G}_{0}t^{d}$.
    If $C = \spec\mathcal{G}$, then define $C \oplus \mathbbm{1} := \spec(\mathcal{G}[t])$.
\end{remark}
Let $X$ be a scheme of finite type over field $k$, $Y \hookrightarrow X$ closed immersion.
Our goal is to construct closed immersion $Y \times_{k} \mathbb{A}_{k}^{1} \hookrightarrow M^{o}$ such that the following diagram commutes
\begin{equation*}
    \begin{tikzcd}
        Y \times_{k} \mathbb{A}_{k}^{1} \arrow[rr, hookrightarrow] \arrow[dr] &&
        M^{o} \arrow[dl, "f"] \\
        & \mathbb{A}_{k}^{1} &
    \end{tikzcd}
\end{equation*}
where $f$ is flat,
satisfying that over $\mathbb{A}_{k}^{1} \setminus \{0\}$, $M^{o}\big{|}_{\mathbb{A}_{k}^{1} \setminus \{0\}} \cong X \times_{k} (\mathbb{A}_{k}^{1} \setminus \{0\})$ and over $\{0\}$, $M_{0}^{o} \cong C_{Y/X}$.
\begin{proof}[\textbf{\emph{Construction}}]
    Step 1: Consider $Y \times_{k} \{0\} \hookrightarrow X \times_{k} \mathbb{A}_{k}^{1}$ with ideal sheaf $\mathcal{J}$.
    Define $M := Bl_{Y \times_{k} \{0\}}(X \times_{k} \mathbb{A}_{k}^{1})$.
    Want to show that $M \longrightarrow X \times_{k} \mathbb{A}_{k}^{1} \longrightarrow \mathbb{A}_{k}^{1}$ is flat.
    Note that flatness is equivalent to torsion-freeness over PID.
    As $\mathcal{J}$ is torsion-free, get it is flat over $\mathcal{O}_{\mathbb{A}_{k}^{1}}$ so that $\oplus_{d \ge 0} \mathcal{J}^{d}$ is flat over $\mathcal{O}_{\mathbb{A}_{k}^{1}}$ and so is $M$.
    \par
    On the other hand, $Y \times_{k} \{0\}$ is an effective Cartier divisor of $Y \times_{k} \mathbb{A}_{k}^{1}$, as Example \ref{example:many_examples_about_blow_up} (2), get $Bl_{Y \times_{k} \{0\}}(Y \times_{k} \mathbb{A}_{k}^{1}) \cong Y \times_{k} \mathbb{A}_{k}^{1} \hookrightarrow M$.
    Thus $M\big{|}_{\mathbb{A}_{k}^{1} \setminus \{0\}} \cong X \times_{k} (\mathbb{A}_{k}^{1} \setminus \{0\})$.
    But what's about the $M_{0}$?
    Note $E = \mathbb{P}(C_{Y/X} \oplus \mathbbm{1}) \subseteq M_{0}$, $\widetilde{X} = Bl_{Y \times_{k} \{0\}}(X \times_{k} \{0\}) = Bl_{Y}(X) \subseteq M_{0}$ and $\mathbb{P}(C_{Y/X})$ is contatined in both of them.
    Claim that $M_{0} = E \cup \widetilde{X}$, $M_{0} \setminus \widetilde{X} = C_{Y/X}$ and $E \cap \widetilde{X} = \mathbb{P}(C_{Y/X})$.
    \par
    Step 2: With the above claim, take $M^{o} = M \setminus \widetilde{X}$, then it is clear that $M_{0}^{o} = C_{Y/X}$ so that it is what we want.
    To prove the claim, it suffices to do some local calculation.
    Let $X = \spec A$, $Y = \spec(A/I)$, $\mathbb{A}_{k}^{1} = \spec(k[t])$ and $Bl_{Y \times_{k} \{0\}}(X \times_{k} \mathbb{A}_{k}^{1}) = \proj(S)$, where $S_{d} = J^{d} = (I, t)^{d} = I^{d} \oplus I^{d - 1}t \oplus \cdots \oplus At^{d} \oplus At^{d + 1} \oplus \cdots$.
    Then $\proj(S)$ can be covered by $\spec(S_{(a)})$, where $a\in S_{1} = (I, t)$.
    In $\spec(S_{(a)})$, get $E = V(\frac{a}{1})$ and $\widetilde{X} = V(\frac{t}{a})$.
    As $\frac{t}{1} = \frac{a}{1}\frac{t}{a}$, get $M_{0}\big{|}_{Spec(S_{(a)})} = E \cup \widetilde{X}\big{|}_{Spec(S_{(a)})}$.
    \par
    On the other hand, since $\spec(S_{(t)}) = M \setminus \widetilde{X}$ and $S_{(t)} = \cdots \oplus I^{2}t^{-2} \oplus It^{-1} \oplus A \oplus At \oplus \cdots $,
    we have that $S_{(t)}/tS_{(t)} = \oplus_{d \ge 0} I^{d}/T^{d + 1}$ so that $M_{0} \setminus \widetilde{X} = C_{Y/X}$.
    And by base change, get $E \cap \widetilde{X} = \mathbb{P}(C_{Y/X})$ which is the exceptional divisor of $\widetilde{X}$.
\end{proof} 

\section{Serre Duality}
\label{section:Serre Duality}

\begin{definition}[\textbf{\emph{$\delta$-functor}}]
    Let $\mathcal{C}, \mathcal{C}'$ be abelian categories.
    A (covairant) $\delta$-functor $T: \mathcal{C} \longrightarrow \mathcal{C}'$ is a collection of additive functors $(T^{i}: \mathcal{C} \rightarrow \mathcal{C}')_{i \ge 0}$ together with morphisms $(\delta^{i}: T^{i}(C) \rightarrow T^{i + 1}(A))_{i \ge 0}$ for all exact sequence $0 \longrightarrow A \longrightarrow B \longrightarrow C \longrightarrow 0$ in $\mathcal{C}$ satisfying that \\
    (1)There is a long exact sequence 
    \begin{equation*}
        \cdots \longrightarrow T^{i}(A) \longrightarrow T^{i}(B) \longrightarrow T^{i}(C) \overset{\delta^{i}}{\longrightarrow} T^{i + 1}(A) \longrightarrow \cdots
    \end{equation*}
    \\
    (2)(functoriality)
    For all commutative diagram with exact rows
    \begin{equation*}
        \begin{tikzcd}
            0 \arrow[r] &
            A \arrow[r] \arrow[d] &
            B \arrow[r] \arrow[d] &
            C \arrow[r] \arrow[d] &
            0 \\
            0 \arrow[r] &
            A' \arrow[r] &
            B' \arrow[r] &
            C' \arrow[r] &
            0
        \end{tikzcd}
    \end{equation*}
    we have commutative diagram
    \begin{equation*}
        \begin{tikzcd}
            T^{C} \arrow[r, "\delta^{i}"] \arrow[d] &
            T^{i + 1}(A) \arrow[d] \\
            T^{i}(C') \arrow[r, "\delta^{i}"] &
            T^{i + 1}(A')
        \end{tikzcd}
    \end{equation*}
    \par
    The $\delta$-functor $T = ((T^{i})_{i \ge 0}, (\delta^{i})_{i \ge 0})$ is called universal if for all $\delta$-functor $T': \mathcal{C} \longrightarrow \mathcal{C}'$ and natural transformation $f^{0}: T^{0} \longrightarrow {T'}^{0}$,
    there exists unique natural transformation $T \longrightarrow T'$ extending $f^{0}$ i.e. $\{f^{i}: T^{i} \rightarrow {T'}^{i}\}_{i \ge 0}$ commuting with $\delta$.
\end{definition}
\begin{remark}
    Given $\mathcal{F}: \mathcal{C} \longrightarrow \mathcal{C}'$ (covariant) left exact functor,
    then there exists at most one universal $\delta$-functor $T$ with $T^{0} = \mathcal{F}$ up to unique morphism.
\end{remark}
\begin{theorem}
    Let $\mathcal{C}, \mathcal{C}'$ be abelian categories, $T: \mathcal{C} \longrightarrow \mathcal{C}'$ a $\delta$-functor.
    If for all $A\in \mathcal{C}$, there exists monomorphism $A \hookrightarrow J$ such that $T^{i}(J) = 0$ for all $i > 0$,
    then $T$ is universal.
    \label{Theorem 11.1}
\end{theorem}
\begin{example}
    If $\mathcal{C}$ has enough injective objects, then for all $\mathcal{F}$ left exact, $(R^{i}\mathcal{F})_{i \ge 0}$ is the wanted universal $\delta$-functor with $R^{0}\mathcal{F} \cong \mathcal{F}$.
    Conversely, if $T: \mathcal{C} \longrightarrow \mathcal{C}'$ is a universal $\delta$-functor, 
    then $T^{0}$ is left exact and $T^{i} \cong R^{i}T^{0}$ for all $i \ge 0$.
\end{example}
Recall that for ringed space $(X, \mathcal{O}_{X})$ and $\mathcal{O}_{X}$-module $\mathcal{F}$, we have that $\hom_{\mathcal{O}_{X}}(\mathcal{F}, \cdot)$ and $\sheafhom_{\mathcal{O}_{X}}(\mathcal{F}, \cdot)$ are covariant left exact.
\begin{definition}[\textbf{\emph{Extension Functor}}]
    Let $(X, \mathcal{O}_{X})$ be a ringed space, $\mathcal{F}$ an $\mathcal{O}_{X}$-module.
    Define $\ext_{\mathcal{O}_{X}}^{i}(\mathcal{F}, \cdot) := R^{i}\hom_{\mathcal{O}_{X}}(\mathcal{F}, \cdot)$ and $\sheafext_{\mathcal{O_{X}}}^{i}(\mathcal{F}, \cdot) := R^{i}\sheafhom_{\mathcal{O}_{X}}(\mathcal{F}, \cdot)$.
\end{definition}
\begin{remark}
    If $\mathcal{F}, \mathcal{H}$ are two $\mathcal{O}_{X}$-modules, then $\ext_{\mathcal{O}_{X}}^{1}(\mathcal{H}, \mathcal{F})$ is one-to-one corresponding to $\{\text{extensions }0 \rightarrow \mathcal{F} \rightarrow \mathcal{G} \rightarrow \mathcal{H} \rightarrow 0\}/\sim$.
\end{remark}
\begin{lemma}
    Let $(X, \mathcal{O}_{X})$ be a ringed space.
    If $\mathcal{I}$ is an injective $\mathcal{O}_{X}$-module, then $\mathcal{I}\big{|}_{U}$ is injective $\mathcal{O}_{U}$-module for all open subset $U \subseteq X$.
    \label{Lemma 11.1}
\end{lemma}
\begin{proof}
    For injection $\mathcal{F} \hookrightarrow \mathcal{G}$ in $Mod(\mathcal{O}_{X})$ and morphism $\mathcal{F} \longrightarrow \mathcal{I}\big{|}_{U}$, consider $j_{!}\mathcal{F} \hookrightarrow j_{!}\mathcal{G}$,
    where $j: U \hookrightarrow X$ is the inclusion and $j_{!}\mathcal{F}, j_{!}\mathcal{G}$ extend $\mathcal{F}, \mathcal{G}$ by $0$ outside $U$ respectively.
    Then there is a morphism $j_{!}\mathcal{G} \longrightarrow \mathcal{I}$ such that the following diagram commutes
    \begin{equation*}
        \begin{tikzcd}
            j_{!}\mathcal{F} \arrow[r, hookrightarrow] \arrow[d] &
            j_{!}\mathcal{G} \arrow[d] \\
            j_{!}(\mathcal{I}\big{|}_{U}) \arrow[r, hookrightarrow] &
            \mathcal{I}
        \end{tikzcd}
    \end{equation*} 
    take restriction of $j_{!}\mathcal{G} \longrightarrow \mathcal{I}$ to $U$, done!
\end{proof}
\begin{proposition}
    Let $(X, \mathcal{O}_{X})$ be a ringed space, $\mathcal{F}, \mathcal{G}$ $\mathcal{O}_{X}$-modules.
    Then for all $U \subseteq X$ open subset, we have that $\sheafext_{\mathcal{O}_{X}}^{i}(\mathcal{F}, \mathcal{G})\big{|}_{U} \cong \sheafext_{\mathcal{O}_{U}}^{i}(\mathcal{F}\big{|}_{U}, \mathcal{G}\big{|}_{U})$.
    \label{Proposition 11.1}
\end{proposition}
\begin{proof}
    Both sides $\sheafext_{\mathcal{O}_{X}}^{i}(\mathcal{F}, \cdot)\big{|}_{U}$ and $\sheafext_{\mathcal{O}_{U}}^{i}(\mathcal{F}\big{|}_{U}, \cdot\big{|}_{U})$ are $\delta$-functor.
    They agree for $i = 0$ and both vanish for $i > 0$ and $\mathcal{G}$ injective.
    Apply Theorem \ref{Theorem 11.1}, they both are universal, thus they are identical up to isomorphism.
\end{proof}
\begin{remark}
    In fact, we have that $\sheafext_{\mathcal{O}_{X}}^{0}(\mathcal{O}_{X}, \mathcal{G}) \cong \mathcal{G}$ and $\sheafext_{\mathcal{O}_{X}}^{i}(\mathcal{O}_{X}, \mathcal{G}) = 0$ for $i > 0$ since $\sheafhom_{\mathcal{O}_{X}}(\mathcal{O}_{X}, \cdot) \cong id$.
    In addition, $\ext_{\mathcal{O}_{X}}^{i}(\mathcal{O}_{})$
    In addition, $\ext_{\mathcal{O}_{X}}^{i}(\mathcal{O}_{X}, \mathcal{G}) = H^{i}(X, \mathcal{G})$ for all $i \ge 0$ since $\hom_{\mathcal{O}_{X}}(\mathcal{O}_{X}, \cdot) \cong \Gamma(X, \cdot)$.
\end{remark}
\begin{proposition}
    Let $(X, \mathcal{O}_{X})$ be a ringed space, $0 \longrightarrow \mathcal{F}' \longrightarrow \mathcal{F}'' \longrightarrow 0$ exact sequence in $Mod(\mathcal{O}_{X})$.
    Then for all $\mathcal{O}_{X}$-module $\mathcal{G}$, there is a long exact sequence
    \begin{equation*}
        \cdots \longrightarrow \ext_{\mathcal{O}_{X}}^{i}(\mathcal{F}'', \mathcal{G}) \longrightarrow \ext_{\mathcal{O}_{X}}^{i}(\mathcal{F}, \mathcal{G}) \longrightarrow \ext_{\mathcal{O}_{X}}^{i}(\mathcal{F}', \mathcal{G}) \longrightarrow \ext_{\mathcal{O}_{X}}^{i + 1}(\mathcal{F}'', \mathcal{G}) \longrightarrow \cdots
    \end{equation*}
    similar result holds for $\sheafext$ too.
    \label{Proposition 11.2}
\end{proposition}
\begin{proof}
    Take injective resolution $0 \longrightarrow \mathcal{G} \longrightarrow \mathcal{I}^{\ast}$.
    Since $\hom_{\mathcal{O}_{X}}(\cdot, \mathcal{I}^{i})$ is exact, by Lemma \ref{Lemma 11.1}, get $\sheafhom_{\mathcal{O}_{X}}(\cdot, \mathcal{I}^{i})$ also exact.
    Thus there is an exact sequence of complexes
    \begin{equation*}
        0 \longrightarrow \hom_{\mathcal{O}_{X}}(\mathcal{F}'', \mathcal{I}^{\ast}) \longrightarrow \hom_{\mathcal{O}_{X}}(\mathcal{F}, \mathcal{I}^{\ast}) \longrightarrow \hom_{\mathcal{O}_{X}}(\mathcal{F}', \mathcal{I}^{\ast}) \longrightarrow 0
    \end{equation*}
    take associated long exact sequence, done!
    Same argument for $\sheafhom$ gives the long exact sequence for $\sheafext$.
\end{proof}
\begin{proposition}
    Let $(X, \mathcal{O}_{X})$ be a ringed space, $\mathcal{F}, \mathcal{G}$ $\mathcal{O}_{X}$-modules.
    It there exists exact sequence
    \begin{equation*}
        \cdots \longrightarrow \mathcal{L}_{1} \longrightarrow \mathcal{L}_{0} \longrightarrow \mathcal{F} \longrightarrow 0
    \end{equation*}
    with $\mathcal{L}_{i}$ locally free of finite rank, 
    which is called a locally free resolution $\mathcal{L}_{\ast} \longrightarrow \mathcal{F} \longrightarrow 0$,
    then $\sheafext_{\mathcal{O}_{X}}^{i}(\mathcal{F}, \mathcal{G}) \cong h^{i}(\sheafhom_{\mathcal{O}_{X}}(\mathcal{L}_{\ast}, \mathcal{G}))$.
    \label{Proposition 11.3}
\end{proposition}
\begin{proof}
    Both sides $\sheafext_{\mathcal{O}_{X}}^{i}(\mathcal{F}, \cdot)$ and $h^{i}(\sheafhom_{\mathcal{O}_{X}}(\mathcal{L}_{\ast}, \cdot))$ are $\delta$-functors.
    They agree for $i = 0$ and both vanish for $i > 0$ and $\mathcal{G}$ injective.
    Apply Theorem \ref{Theorem 11.1}, done!
\end{proof}
\begin{corollary}
    Let $(X, \mathcal{O}_{X})$ be a ringed space, $\mathcal{L}$ loally free of finite rank, $\mathcal{G}$ an $\mathcal{O}_{X}$-module.
    Then $\sheafext_{\mathcal{O}_{X}}^{i}(\mathcal{L}, \mathcal{G}) = 0$ for $i > 0$.
    \label{Corollary 11.1}
\end{corollary}
\begin{remark}
    (1)For $X$ quasi-projective over $\spec A$ with $A$ noetherian, any coherent $\mathcal{F}$ is quotient of locally free of finite rank.
    Thus to compute $\sheafext_{\mathcal{O}_{X}}^{i}(\mathcal{F}, \mathcal{G})$, where $\mathcal{F}, \mathcal{G}$ are coherent, can stay within $Coh(X)$. \\
    (2)Proposition \ref{Proposition 11.3} does not mean $\sheafext_{\mathcal{O}_{X}}^{i}(\cdot, \mathcal{G})$ is a derived functor in a naive sense.
    In fact, $Mod(\mathcal{O}_{X})$ and $Qcoh(X)$ rarely have enough projective objects. \\
    (3)About $\mathcal{F} \otimes_{\mathcal{O}_{X}} \cdot$, for $\mathcal{O}_{X}$-module $\mathcal{G}$, we can take a flat resolution $\mathcal{K}_{\ast} \longrightarrow \mathcal{G} \longrightarrow 0$ and define $\sheaftor_{i}^{\mathcal{O}_{X}}(\mathcal{F}, \mathcal{G}) := h_{i}(\mathcal{F} \otimes_{\mathcal{O}_{X}} \mathcal{K}_{\ast})$.
    Need to check this definition is well defined.
\end{remark}
\begin{lemma}
    Let $(X, \mathcal{O}_{X})$ be a ringed space, $\mathcal{L}$ locally free sheaf of finite rank, $\mathcal{L}^{\lor} = \sheafhom_{\mathcal{O}_{X}}(\mathcal{L}, \mathcal{O}_{X})$.
    Then for all $\mathcal{I}$ injective, $\mathcal{L} \otimes_{\mathcal{O}_{X}} \mathcal{I}$ is injective.
    \label{Lemma 11.2}
\end{lemma}
\begin{proof}
    Note that $\hom_{\mathcal{O}_{X}}(\cdot, \mathcal{L} \otimes_{\mathcal{O}_{X}} \mathcal{I}) \cong \hom_{\mathcal{O}_{X}}(\cdot \otimes_{\mathcal{O}_{X}} \mathcal{L}^{\lor}, \mathcal{I})$,
    since $\cdot \otimes_{\mathcal{O}_{X}} \mathcal{L}^{\lor}$ is exact, get $\mathcal{L} \otimes_{\mathcal{O}_{X}} \mathcal{I}$ is injective.
\end{proof}
\begin{proposition}
    Let $(X, \mathcal{O}_{X})$ be a ringed space, $\mathcal{L}$ locally free sheaf of finite rank, $\mathcal{L}^{\lor} = \sheafhom_{\mathcal{O}_{X}}(\mathcal{L}, \mathcal{O}_{X})$.
    Then for all $\mathcal{O}_{X}$-modules $\mathcal{F}$ and $\mathcal{G}$,
    we have that $\ext_{\mathcal{O}_{X}}^{i}(\mathcal{F} \otimes_{\mathcal{O}_{X}} \mathcal{L}, \mathcal{G}) \cong \ext_{\mathcal{O}_{X}}^{i}(\mathcal{F}, \mathcal{L}^{\lor} \otimes_{\mathcal{O}_{X}} \mathcal{G})$ and $\sheafext_{\mathcal{O}_{X}}^{i}(\mathcal{F} \otimes_{\mathcal{O}_{X}} \mathcal{L}, \mathcal{G}) \cong \sheafext_{\mathcal{O}_{X}}^{i}(\mathcal{F}, \mathcal{G}) \otimes_{\mathcal{O}_{X}} \mathcal{L}^{\lor}$.
    \label{Proposition 11.4}
\end{proposition}
\begin{proof}
    Both sides as functors in $\mathcal{G}$ are $\delta$-functors.
    They agree for $i = 0$ and both vanish for $i > 0$ and $\mathcal{G}$ injective.
    Apply Theorem \ref{Theorem 11.1}, done!
\end{proof}
\begin{proposition}
    Let $X$ be a noetherian scheme, $x\in X$, $\mathcal{F}$ coherent sheaf on $X$, $\mathcal{G}$ an $\mathcal{O}_{X}$-module.
    Then $\sheafext_{\mathcal{O}_{X}}^{i}(\mathcal{F}, \mathcal{G})_{x} \cong \ext_{\mathcal{O}_{X, x}}^{i}(\mathcal{F}_{x}, \mathcal{G}_{x})$.
    \label{Proposition 11.5} 
\end{proposition}
\begin{proof}
    Can assume $X$ affine.
    Take a free resolution $\mathcal{L}_{\ast} \longrightarrow \mathcal{F} \longrightarrow 0$ exact.
    Take stalks at $x$, get $(\mathcal{L}_{x})_{\ast} \longrightarrow \mathcal{F}_{x} \longrightarrow 0$ exact.
    Use the two resolutions to compute both sides, as $\mathcal{L}$ is free and $\sheafhom_{\mathcal{O}_{X}}(\mathcal{L}, \mathcal{G})_{x} \cong \hom_{\mathcal{O}_{X, x}}(\mathcal{L}_{x}, \mathcal{G}_{x})$, done! 
\end{proof}
\begin{remark}
    When $i = 0$, it shows that when $\mathcal{F}$ is coherent, hom functor commutes with directed limit.
\end{remark}
\begin{proposition}
    Let $X$ be a noetherian scheme, $\mathcal{F}, \mathcal{G}$ quasi-coherent sheaves on $X$.
    Then we have that \\
    (1)if $\mathcal{F}$ is coherent, then $\sheafext_{\mathcal{O}_{X}}^{i}(\mathcal{F}, \mathcal{G})$ is quasi-coherent for all $i \ge 0$ \\
    (2)if both $\mathcal{F}, \mathcal{G}$ are coherent, then $\sheafext_{\mathcal{O}_{X}}^{i}(\mathcal{F}, \mathcal{G})$ is coherent for all $i \ge 0$.
    \label{Proposition 11.6}
\end{proposition}
\begin{reason}
    Can assume $X$ is affine.
    Take a free resolution $\mathcal{L}_{\ast} \longrightarrow \mathcal{F} \longrightarrow 0$ and compute.
\end{reason}
\begin{proposition}
    Let $X$ be a projective scheme over $\spec A$, $A$ noetherian, $\mathcal{F}, \mathcal{G}$ coherent sheaves on $X$.
    Then there exists $n_{0} > 0$ relative to $\mathcal{F}$, $\mathcal{G}$ and $i$ such that $\ext_{\mathcal{O}_{X}}^{i}(\mathcal{F}, \mathcal{G}(n)) = \Gamma(X, \sheafext_{\mathcal{O}_{X}}^{i}(\mathcal{F}, \mathcal{G}(n)))$ for all $n \ge n_{0}$.
    \label{Proposition 11.7}
\end{proposition}
\begin{proof}
    If $i = 0$, by definition, we get it for all $\mathcal{F}$, $\mathcal{G}$ and $n$.
    If $i > 0$ and $\mathcal{F} = \mathcal{O}_{X}$, then the left side is isomorphic to $H^{i}(X, \mathcal{G}(n))$ while the right side is $0$.
    By Serre Theorem, $H^{i}(X, \mathcal{G}(n))$ vanish for large enough $n$ so that result is ok for $\mathcal{F} = \mathcal{O}_{X}$.
    If $i > 0$ and $\mathcal{F}$ is locally free (of finite rank), can reduce to $\mathcal{F} = \mathcal{O}_{X}$.
    \par
    For arbitrary $\mathcal{F}$, write it as a quotient of locally free sheaf $\mathcal{E}$ of finite rank.
    Set $\mathcal{R} = \ker(\mathcal{E} \rightarrow \mathcal{F})$, then we have the following exact sequence
    \begin{equation*}
        0 \longrightarrow \mathcal{R} \longrightarrow \mathcal{E} \longleftarrow \mathcal{F} \longrightarrow 0
    \end{equation*}
    By Proposition \ref{Proposition 11.2}, get
    \begin{equation*}
        \hom_{\mathcal{O}_{X}}(\mathcal{E}, \mathcal{G}(n)) \longrightarrow \hom_{\mathcal{O}_{X}}(\mathcal{R}, \mathcal{G}(n)) \longrightarrow \ext_{\mathcal{O}_{X}}^{1}(\mathcal{F}, \mathcal{G}(n)) \longrightarrow 0 \longrightarrow \cdots
    \end{equation*}
    Similar sequence for $\sheafext$ functor also holds.
    By Serre Theorem, as $\mathcal{E}$ is locally free of finite rank, for large enough $n$, $\ext_{\mathcal{O}_{X}}^{i}(\mathcal{R}, \mathcal{G}(n)) \cong \ext_{\mathcal{O}_{X}}^{i + 1}(\mathcal{F}, \mathcal{G}(n))$ and $\sheafext_{\mathcal{O}_{X}}^{i}(\mathcal{R}, \mathcal{G}(n)) \cong \sheafext_{\mathcal{O}_{X}}^{i + 1}(\mathcal{F}, \mathcal{G}(n))$ for all $i > 0$.
    In addition, by Proposition \ref{Proposition 11.3}, there is a commutative diagram with exact rows 
    \begin{equation*}
        \begin{tikzcd}[sep = small]
            \hom_{\mathcal{O}_{X}}(\mathcal{E}, \mathcal{G}(n)) \arrow[r] \arrow[d, "\sim"] &
            \hom_{\mathcal{O}_{X}}(\mathcal{R}, \mathcal{G}(n)) \arrow[r] \arrow[d, "\sim"] &
            \ext_{\mathcal{O}_{X}}^{1}(\mathcal{F}, \mathcal{G}(n)) \arrow[r] \arrow[d] &
            0 \arrow[r] &
            0 \\
            \Gamma(X, \sheafhom_{\mathcal{O}_{X}}(\mathcal{E}, \mathcal{G}(n))) \arrow[r] &
            \Gamma(X, \sheafhom_{\mathcal{O}_{X}}(\mathcal{R}, \mathcal{G}(n))) \arrow[r] &
            \Gamma(X, \sheafext_{\mathcal{O}_{X}}^{1}(\mathcal{F}, \mathcal{G}(n))) \arrow[r] &
            0 \arrow[r] &
            0
        \end{tikzcd}
    \end{equation*}
    By 5 Lemma, get $\ext_{\mathcal{O}_{X}}^{1}(\mathcal{F}, \mathcal{G}(n)) \longrightarrow \Gamma(\sheafext_{\mathcal{O}_{X}}^{1}(\mathcal{F}, \mathcal{G}(n)))$ is isomorphic.
    Note $\mathcal{R}$ is also coherent, by induction, done!
\end{proof}
\subsection{Serre Duality for Projective Space}
\label{subsection:Serre Duality for Projective Space}
\begin{theorem}
    Let $k$ be a field and $X = \mathbb{P}_{k}^{n}$, $\omega_{X/k} = \wedge^{n} \Omega_{X/k}$ canonical sheaf.
    Then we have that \\
    (1)$H^{n}(X, \omega_{X/k}) \cong k$. \\
    (2)For all coherent sheaf $\mathcal{F}$ on $X$, the natural pairing 
    \begin{equation*}
        \hom(\mathcal{F}, \omega_{X/k}) \times H^{n}(X, \mathcal{F}) \longrightarrow H^{n}(X, \omega_{X/k})
    \end{equation*}
    is a perfect pairing of finite-dimensional vector spaces over $k$. \\
    (3)For all coherent sheaf $\mathcal{F}$ on $X$ and $i \ge 0$, there exists isomorphism $\ext^{i}(\mathcal{F}, \omega_{X/k}) \cong H^{n - i}(X, \mathcal{F}) \cong \ext^{n - i}(\mathcal{O}_{X}, \mathcal{F})^{\lor}$ functorial in $\mathcal{F}$.
    \label{Theorem 11.2}
\end{theorem}
\begin{proof}
    For (1), by Euler sequence, we have that $\omega_{X/k} \cong \mathcal{O}_{X}(-n - 1)$ so that $H^{n}(X, \omega_{X/k}) \cong H^{n}(X, \mathcal{O}_{X}(-n - 1)) \cong k$.
    \par
    For (2), first consider the case when $\mathcal{F} = \mathcal{O}_{X}(q)$.
    Then we get $\hom(\mathcal{O}_{X}(q), \omega_{X/k}) \cong H^{0}(X, \mathcal{O}_{X}(-n - 1 - q))$.
    By Theorem \ref{Theorem 7.6}, the following pairing is a perfect pairing
    \begin{equation*}
        H^{0}(X, \mathcal{O}_{X}(-n - 1 - q)) \times H^{n}(X, \mathcal{O}_{X}(q)) \longrightarrow H^{n}(X, \mathcal{O}_{X}(-n - 1)).
    \end{equation*}
    If $\mathcal{F}$ is a direct sum of $\mathcal{O}_{X}(q)$, as cohomology commutes with direst sum, it is also ok.
    For arbitrary $\mathcal{F}$, by Corollary \ref{Corollary 6.1}, there is an exact sequence
    \begin{equation*}
        \mathcal{E}_{1} \longrightarrow \mathcal{E}_{0} \longrightarrow \mathcal{F} \longrightarrow 0
    \end{equation*}
    where $\mathcal{E}_{1}$ and $\mathcal{E}_{0}$ are direct sum of $\mathcal{O}_{X}(q)$.
    Then by Grothendieck Vanishing Theorem, there is a commutative diagram with exact rows,
    \begin{equation*}
        \begin{tikzcd}
            0 \arrow[r] &
            \hom(\mathcal{F}, \omega_{X/k}) \arrow[r] \arrow[d] &
            \hom(\mathcal{E}_{0}, \omega_{X/k}) \arrow[r] \arrow[d, "\sim"] &
            \hom(\mathcal{E}_{1}, \omega_{X/k}) \arrow[d, "\sim"] \\
            0 \arrow[r] &
            H^{n}(X, \mathcal{F})^{\lor} \arrow[r] &
            H^{n}(X, \mathcal{E}_{0})^{\lor} \arrow[r] &
            H^{n}(X, \mathcal{E}_{1})^{\lor}
        \end{tikzcd}
    \end{equation*}
    By 5 Lemma, $\hom(\mathcal{F}, \omega_{X/k}) \cong H^{n}(X, \mathcal{F})^{\lor}$.
    \par
    For (3), both sides $\ext^{i}(\cdot, \omega_{X/k})$ and $H^{n - i}(X, \cdot)^{\lor}$ are contravariant $\delta$-functors from $Coh(X)$ to $Vect(k)$.
    They agree for $i = 0$ by (2).
    Remains to show that for all coherent sheaf $\mathcal{F}$ on $X$, there exists exact sequence $\mathcal{E} = \oplus \mathcal{O}_{X}(q) \longrightarrow \mathcal{F} \longrightarrow 0$ for some $q$.
    Can choose $q$ much less that $0$ such that $\ext^{i}(\mathcal{E}, \omega_{X/k}) = 0$ and $H^{n - i}(X, \mathcal{E})^{\lor} = 0$ for all $i > 0$.
    Then apply Theorem \ref{Theorem 11.1}, done.
\end{proof}
\subsection{Serre Duality in General}
\label{subsection:Serre Duality in General}
Want to know what is the Serre duality for general $X$ projective scheme over $k$ of dimension $n$.
\begin{remark}
    Version (1): there exists coherent sheaf $\omega_{X}^{o}$ on $X$ together with linear map $t: H^{n}(X, \omega_{X}^{o}) \longrightarrow k$ called the trace homomorphism such that for all coherent sheaf $\mathcal{F}$ on $X$,
    the natural pairing 
    \begin{equation*}
        \hom(\mathcal{F}, \omega_{X}^{o}) \times H^{n}(X, \mathcal{F}) \longrightarrow H^{n}(X, \omega_{X}^{o}) \overset{t}{\longrightarrow} k
    \end{equation*}
    is a perfect pairing.
    \par
    Version (1$'$): there exists coherent sheaf $\omega_{X}^{o}$ on $X$ such that for all coherent sheaf $\mathcal{F}$ on $X$,
    there exists isomorphism $\hom(\mathcal{F}, \omega_{X}^{o}) \overset{\sim}{\longrightarrow} H^{n}(X, \mathcal{F})^{\lor}$ functorial in $\mathcal{F}$.
    In fact, for given $\mathcal{F} \longrightarrow \omega_{X}^{o}$, with the following commutative diagram , it is clear that version (1) is equivalent to version (1$'$).
    \begin{equation*}
        \begin{tikzcd}
            \hom(\mathcal{F}, \omega_{X}^{o}) \arrow[r, "\sim"] &
            H^{n}(X, \mathcal{F})^{\lor} \\
            \hom(\omega_{X}^{o}, \omega_{X}^{o}) \arrow[u] \arrow[r, "\sim"] &
            H^{n}(X, \omega_{X}^{o})^{\lor} \arrow[u]
        \end{tikzcd}
    \end{equation*}
    \par
    Version (2): $\omega_{X}^{o}$ in version (1) also satisfies that for all coherent sheaf $\mathcal{F}$ on $X$ and $i \ge 0$,
    there exists isomorphism $\ext^{i}(\mathcal{F}, \omega_{X}^{o}) \overset{\sim}{\longrightarrow} H^{n - 1}(X, \mathcal{F})^{\lor} \cong \ext^{n - i}(\mathcal{O}_{X}, \mathcal{F})^{\lor}$ functorial in $\mathcal{F}$.
    \par
    Version (2-): $\omega_{X}^{o}$ in version (1) also satisfies that for all locally free sheaf $\mathcal{E}$ of finite rank on $X$ and $i \ge 0$,
    there exists isomorphism $H^{i}(X, \mathcal{E}^{\lor} \otimes_{\mathcal{O}_{X}} \omega_{X}^{o}) \cong H^{n - i}(X, \mathcal{E})^{\lor}$ functorial in $\mathcal{E}$.
    This is what used in the proof of Riemann-ROch Theorem for curves.
    \par
    Version (2+): $\omega_{X}^{o}$ in version (1) also satisfies that for all coherent sheaf $\mathcal{F}$, locally free sheaf $\mathcal{E}$ of finite rank on $X$ and $i \ge 0$,
    there exists isomorphism 
    \begin{equation*}
        \ext^{i}(\mathcal{F} \otimes_{\mathcal{O}_{X}} \omega_{X}^{o}) \cong \ext^{i}(\mathcal{F}, \mathcal{E} \otimes_{\mathcal{O}_{X}} \omega_{X}^{o}) \cong \ext^{i}(\mathcal{E}, \mathcal{F})^{\lor} \cong H^{n - i}(X, \mathcal{E}^{\lor} \otimes_{\mathcal{O}_{X}} \mathcal{F})  
    \end{equation*}
    functorial in both $\mathcal{F}$ and $\mathcal{E}$.
    \par
    Version (2$'$): $\omega_{X}^{o}$ in version (1) also satisfies that for all coherent sheaf $\mathcal{F}$ on $X$,
    there exists a natural pairing 
    \begin{equation*}
        \ext^{i}(\mathcal{F}, \omega_{X}^{o}) \times H^{n - i}(X, \mathcal{F}) \cong \ext^{n - i}(\mathcal{O}_{X}, \mathcal{F}) \longrightarrow H^{n}(X. \omega_{X}^{o}) \cong \ext^{n}(\mathcal{O}_{X}, \omega_{X}^{o}) \overset{t}{\longrightarrow} k
    \end{equation*}
    called the Yoneda pairing, which is a perfect pairing for $X$ under assumptions.
    \par
    Version (3): duality generalized to the derived category.
    \par
    Finally, if $X$ is smooth and projective over $k$ of pure dimension, want $\omega_{X}^{o} \cong \omega_{X}$ is the canonical sheaf of $X$.
\end{remark}
\begin{example}
    Version (2) needs more assumption on $X$, here is a counterexample.
    Let $X$ be two projective planes meeting at one point $x$ as a subscheme of $\mathbb{P}_{k}^{4}$.
    Suppose that there exists $\omega_{X}^{o}$ such that $H^{1}(X, \omega_{X}^{o}(q)) \cong H^{1}(X, \mathcal{O}_{X}(-q))^{\lor}$.
    By Serre Vanishing Theorem, $H^{1}(X, \mathcal{O}_{X}(-q))$ for large enough $q$.
    \par
    But if let $H \subseteq \mathbb{P}_{k}^{4}$ be a hyperplane not containing $x$, then $H \cap X$ is union of two disjoint lines.
    Set $Z := qH \cap X$ where $q$ is large enough.
    Then there is an exact sequence
    \begin{equation*}
        0 \longrightarrow \mathcal{O}_{X}(-Z) = \mathcal{O}_{X}(-q) \longrightarrow \mathcal{O}_{X} \longrightarrow \mathcal{O}_{Z} \longrightarrow 0
    \end{equation*}
    Then consider long exact sequence of cohomology groups, get 
    \begin{equation*}
        H^{0}(X, \mathcal{O}_{X}) \longrightarrow H^{0}(Z, \mathcal{O}_{Z}) \longrightarrow H^{1}(X, \mathcal{O}_{X}(-q)) = 0
    \end{equation*}
    While $H^{0}(X, \mathcal{O}_{X}) \cong k$ is one-dimensional and $H^{0}(Z, \mathcal{O}_{Z})$ is two-dimensional, 
    it is impossible that there is a surjection from $H^{0}(X, \mathcal{O}_{X})$ to $H^{0}(Z, \mathcal{O}_{Z})$, contradiction!
\end{example}
Let $X$ be a projective scheme over field $k$, $\dim(X) = n$.
\begin{definition}[\textbf{\emph{Dualizing Sheaf}}]
    A dualizing sheaf on $X$ is a coherent sheaf $\omega_{X}^{o}$ on $X$ together with a trace homomorphism $t: H^{n}(X, \omega_{X}^{o}) \longrightarrow k$ such that for all coherent sheaf $\mathcal{F}$ on $X$, the natural pairing 
    \begin{equation*}
        \hom(\mathcal{F}, \omega_{X}^{o}) \times H^{n}(X, \mathcal{F}) \longrightarrow H^{n}(X, \omega_{X}^{o}) \overset{t}{\longrightarrow} k
    \end{equation*}
    is a perfect pairing.
\end{definition}
\begin{proposition}
    If a dualizing sheaf exists, then it is unique i.e. if $\omega_{X}'$ together with trace homomorphism $t'$ is another one,
    then there exists unique isomorphism $\phi: \omega_{X}^{o} \overset{\sim}{\longrightarrow} \omega_{X}'$ such that $t = t' \circ H^{n}(X, \phi)$.
    \label{Proposition 11.8}
\end{proposition}
\begin{reason}
    Since $\omega_{X}'$ is dualizing sheaf, we have that $\hom(\omega_{X}^{o}, \omega_{X}') \cong H^{n}(X, \omega_{X}')^{\lor}$, take $\phi$ to be the preimage of $t'$.
    By functoriality, $t' \circ H^{n}(X, \phi) = t$.
\end{reason}
\begin{remark}
    In fact, we can say that $\omega_{X}^{o}$ together with $t$ represents the functor $\mathcal{F}\in Coh(X) \longmapsto H^{n}(X, \mathcal{F})^{\lor}\in Vect(k)$.
\end{remark}
\begin{lemma}
    Let $X \subseteq \mathbb{P}_{k}^{N}$ be a closed subscheme of codimension $r$.
    Then $\sheafext^{i}(\mathcal{O}_{X}, \omega_{\mathbb{P}_{k}^{N}}) = 0$ for all $i < r$
    \label{Lemma 11.3}.
\end{lemma}
\begin{proof}
    Set $\mathcal{F}^{i} := \sheafext_{\mathcal{O}_{\mathbb{P}_{k}^{N}}}^{i}(\mathcal{O}_{X}, \omega_{\mathbb{P}_{k}^{N}})$, which is a coherent sheaf on $\mathbb{P}_{k}^{N}$.
    After twisting by $\mathcal{O}_{\mathbb{P}_{k}^{N}}(q)$ for large enough $q$, by Theorem \ref{Theorem 6.3}, we get $\mathcal{F}^{i}(q)$ is generated by global sections.
    Suffices to prove $\Gamma(\mathbb{P}_{k}^{N}, \mathcal{F}^{i}(q)) = 0$ for all $i < r$ and large enough $q$.
    By Proposition \ref{Proposition 11.7}, $\Gamma(\mathbb{P}_{k}^{N}, \mathcal{F}^{i}(q)) \cong \ext_{\mathcal{O}_{\mathbb{P}_{k}^{N}}}^{i}(\mathcal{O}_{X}, \omega_{\mathbb{P}_{k}^{N}}(q))$ for large enough $q$.
    \par
    By Serre duality for $\mathbb{P}_{k}^{N}$ and Grothendieck Vanishing Theorem, get 
    \begin{equation*}
        \ext^{i}_{\mathcal{O}_{\mathbb{P}_{k}^{N}}}(\mathcal{O}_{X}, \omega_{\mathbb{P}_{k}^{N}}(q)) \cong H^{N - i}(\mathbb{P}_{k}^{N}, \mathcal{O}_{X}(-q))^{\lor} \cong H^{N - i}(X, \mathcal{O}_{X}(-q))^{\lor} = 0
    \end{equation*}
    for $i < r$ since $N - i > \dim(X)$.
\end{proof}
\begin{lemma}
    Let $\mathcal{I}^{0} \rightarrow \mathcal{I}^{1} \rightarrow \cdots$ be a complex of injective objects in an abelian category $\mathcal{C}$ such that $h^{i}(\mathcal{I}^{\bullet}) = 0$ for $0 \le i < r$.
    Then can write $\mathcal{I}^{\bullet} \cong \mathcal{I}_{1}^{\bullet} \oplus \mathcal{I}_{2}^{\bullet}$ such that $\mathcal{I}_{1}^{\bullet}$ is in degrees $0 \le i \le r$ and exact and $\mathcal{I}_{2}^{\bullet}$ is in degree $i \ge r$.
    \label{Lemma 11.4}
\end{lemma}
\begin{lemma}
    Let $X \subseteq \mathbb{P}_{k}^{N}$ be a closed subscheme of codimension $r$.
    Set wanted dualizing sheaf $\omega_{X}^{o} := \sheafext_{\mathcal{O}_{\mathbb{P}_{k}^{N}}}^{r}(\mathcal{O}_{X}, \omega_{\mathbb{P}_{k}^{N}})$ viewed as $\mathcal{O}_{X}$-module
    Then for all $\mathcal{O}_{X}$-module $\mathcal{F}$, there exists isomorphism $\hom_{\mathcal{O}_{X}}(\mathcal{F}, \omega_{X}^{o}) \cong \ext_{\mathcal{O}_{\mathbb{P}_{k}^{N}}}^{r}(\mathcal{F}, \omega_{\mathbb{P}_{k}^{N}})$ functorial in $\mathcal{F}$.
    \label{Lemma 11.5}
\end{lemma}
\begin{proof}
    Let $0 \rightarrow \omega_{\mathbb{P}_{k}^{N}} \rightarrow \mathcal{I}^{\bullet}$ be an injective resolution of $\omega_{\mathbb{P}_{k}^{N}}$.
    Then $\ext_{\mathcal{O}_{\mathbb{P}_{k}^{N}}}^{i}(\mathcal{F}, \omega_{\mathbb{P}_{k}^{N}}) = h^{i}(\hom_{\mathcal{O}_{\mathbb{P}_{k}^{N}}}(\mathcal{F}, \mathcal{I}^{\bullet}))$.
    As $\mathcal{F}$ is an $\mathcal{O}_{X}$-module, $\mathcal{F} \rightarrow \mathcal{I}^{i}$ factors through $\mathcal{J}^{i} := \sheafhom_{\mathcal{O}_{\mathbb{P}_{k}^{N}}}(\mathcal{O}_{X}, \mathcal{I}^{i})$ so that $\hom_{\mathcal{O}_{\mathbb{P}_{k}^{N}}}(\mathcal{F}, \mathcal{I}^{i}) \cong \hom_{\mathcal{O}_{X}}(\mathcal{F}, \mathcal{J}^{i})$.
    Thus $\ext_{\mathcal{O}_{\mathbb{P}_{k}^{N}}}^{i}(\mathcal{F}, \omega_{\mathbb{P}_{k}^{N}}) \cong h^{i}(\hom_{\mathcal{O}_{X}}(\mathcal{F}, \mathcal{J}^{\bullet}))$.
    Moreover, each $\mathcal{J}^{i}$ is an injective $\mathcal{O}_{X}$-module since $\hom_{\mathcal{O}_{X}}(\cdot, \mathcal{J}^{i}) \cong \hom_{\mathcal{O}_{\mathbb{P}_{k}^{N}}}(\cdot, \mathcal{I}^{i})$ as functors from $Mod(\mathcal{O}_{X})$ to $Ab$.
    By Lemma \ref{Lemma 11.3}, $h^{i}(\mathcal{J}^{\ast}) = 
    \left\{
        \begin{aligned}
            & 0 & \forall i < r \\
            & \omega_{X}^{o} & i = r
        \end{aligned}
    \right.
    $.
    By Lemma \ref{Lemma 11.4}, can write $\mathcal{J}^{\ast} \cong \mathcal{J}_{1}^{\bullet} \oplus \mathcal{J}_{2}^{\bullet}$ such that $\mathcal{J}_{1}^{\bullet}$ is in degrees $0 \le i \le 1$ and exact and $\mathcal{J}_{2}^{\bullet}$ is in degrees $i \ge r$.
    Then $\omega_{X}^{o} = h^{r}(\mathcal{J}^{\bullet}) \cong \ker(d_{2}^{r}: \mathcal{J}_{2}^{r} \rightarrow \mathcal{J}_{2}^{r + 1})$ and for all $\mathcal{O}_{X}$-module $\mathcal{F}$,
    we have that 
    \begin{equation*}
        \hom_{\mathcal{O}_{X}}(\mathcal{F}, \omega_{X}^{o}) \cong \hom_{\mathcal{O}_{X}}(\mathcal{F}, \ker(d_{2}^{r})) \cong h^{r}(\hom_{\mathcal{O}_{X}}(\mathcal{F}, \mathcal{J}^{\bullet})) \cong \ext_{\mathcal{O}_{\mathbb{P}_{k}^{N}}}^{r}(\mathcal{F}, \omega_{\mathbb{P}_{k}^{N}})
    \end{equation*}
    done!
\end{proof}
\begin{remark}
    In fact, same argument as Lemma \ref{Lemma 11.5} shows that $\ext_{\mathcal{O}_{\mathbb{P}_{k}^{N}}}^{i}(\mathcal{F}, \omega_{\mathbb{P}_{k}^{N}}) = 0$ and $\sheafext_{\mathcal{O}_{\mathbb{P}_{k}^{N}}}^{i}(\mathcal{F}, \omega_{\mathbb{P}_{k}^{N}}) = 0$ for all $i < r$.
    Also, we have that 
    \begin{equation*}
        \sheafhom_{\mathcal{O}_{X}}(\mathcal{F}, \omega_{X}^{o}) \cong \sheafhom_{\mathcal{O}_{X}}(\mathcal{F}, \ker(d_{2}^{r})) \cong h^{r}(\sheafhom_{\mathcal{O}_{X}}(\mathcal{F}, \mathcal{J}^{\bullet})) \cong \sheafext_{\mathcal{O}_{\mathbb{P}_{k}^{N}}}^{r}(\mathcal{F}, \omega_{\mathbb{P}_{k}^{N}})
    \end{equation*}
    which gives that $\sheafhom_{\mathcal{O}_{X}}(\mathcal{F}, \omega_{X}^{o}) \cong \sheafext_{\mathcal{O}_{\mathbb{P}_{k}^{N}}}^{r}(\mathcal{F}, \omega_{\mathbb{P}_{k}^{N}})$ functorial in $\mathcal{F}$.
\end{remark}
\begin{proposition}
    Let $X$ be a projective scheme over field $k$ of dimension $n$.
    Then $X$ has a dualizing sheaf.
    \label{Proposition 11.9}
\end{proposition}
\begin{proof}
    Since $X \subseteq \mathbb{P}_{k}^{N}$ of codimension $r$ for some $N$, take $\omega_{X}^{o}$ in the Lemma \ref{Lemma 11.4}.
    Then for all $\mathcal{O}_{X}$-module $\mathcal{F}$, there exists isomorphism $\hom_{\mathcal{O}_{X}}(\mathcal{F}, \omega_{X}^{o}) \cong \ext_{\mathcal{O}_{\mathbb{P}_{k}^{N}}}^{r}(\mathcal{F}, \omega_{\mathbb{P}_{k}^{N}})$.
    As $H^{n}(X, \mathcal{F})^{\lor} \cong H^{N - r}(\mathbb{P}_{k}^{N}, \mathcal{F})^{\lor}$, suffice to show that $\ext_{\mathcal{O}_{\mathbb{P}_{k}^{N}}}^{r}(\mathcal{F}, \omega_{\mathbb{P}_{k}^{N}}) \cong H^{N - r}(\mathbb{P}_{k}^{N}, \mathcal{F})^{\lor}$, which is just Serre duality.
    Thus there exists isomorphism $\hom(\mathcal{F}, \omega_{X}^{o}) \overset{\sim}{\longrightarrow} H^{n}(X, \mathcal{F})^{\lor}$ functorial in $\mathcal{F}\in Coh(X)$.
    This is version (1$'$) of generalization of Serre duality.
\end{proof}
\subsection{Some Algebra}
\begin{definition}[\textbf{\emph{Regular Sequence}}]
    Let $A$ be a ring, $M \neq 0\in Mod_{A}$.
    A sequence $\{a_{1}, \cdots, a_{r}\} \subseteq A$ is called a regular sequence for $M$ if $a_{1}$ is not a zero-divisor in $M$, $a_{2}$ is not a zero divisor in $M/a_{1}M$ and so on with $M/(a_{1}, \cdots, a_{r})M \neq 0$.
\end{definition}
\begin{definition}[\textbf{\emph{Depth of Module}}]
    Let $A$ be a local ring, $\mathfrak{m} \subseteq A$ maximal ideal, $M\in Mod_{A}$.
    Define the depth of $M$ is the maximal length of regular sequence for $M$, denoted by $\depth(M)$.
\end{definition}
\begin{definition}[\textbf{\emph{Cohen-Macaulay Ring}}]
    Let $A$ be a noetherian local ring.
    $A$ is called Cohen-Macaulay if $\depth(A) = \dim(A)$ as $A$-module.
\end{definition}
\begin{proposition}
    Let $A$ be a noetherian local ring, $\mathfrak{m} \subseteq A$ maximal ideal.
    Then we have that \\
    (1)if $A$ is regular, then $A$ is Cohen-Macaulay \\
    (2)if $A$ is Cohen-Macaulay, then $A_{\mathfrak{p}}$ is Cohen-Macaulay for all prime ideal $\mathfrak{p}\in \spec A$. \\
    (3)if $A$ is Cohen-Macaulay, then $\{a_{1}, \cdots, a_{r}\}$ form a regular sequence for $A$ if and only if $\dim(A/(a_{1}, \cdots, a_{r})) = \dim(A) - r$.
    In particular, regular sequence for $A$ is independent on order and $A/(a_{1}, \cdots, a_{r})$ is also Cohen-Macaulay. \\
    (4)if $\{a_{1}, \cdots, a_{r}\}$ form a regular sequence for $A$, then $A$ is Cohen-Macaulay if and only if $A/(a_{1}, \cdots, a_{r})$ is Cohen-Macaulay.
    \label{Proposition 11.10}
\end{proposition}
\begin{proposition}
    Let $A$ be a ring, $I = (a_{1}, \cdots, a_{r}) \subseteq A$ ideal.
    Then we have that \\
    (1)if $\{a_{1}, \cdots, a_{r}\}$ form a regular sequence for $A$, 
    then the natural surjective homomorphism of graded rings 
    \begin{equation*}
        \begin{split}
            A/I[x_{1}, \cdots, x_{r}] & \longrightarrow \oplus_{d \ge 0} I^{d}/I^{d + 1} \\
            x_{i} & \longmapsto a_{i}
        \end{split}
    \end{equation*}
    is an isomorphism. 
    In particular, $I/I^{2}$ is free $A/I$-module of rank r. \\
    (2)if $A$ is noetherian local ring such that $A/I$ is regular,
    then $\{a_{1}, \cdots, a_{r}\}$ form a regular sequence for $A$ if and only if $A$ is regular.
    \label{Proposition 11.11}
\end{proposition}
\begin{definition}[\textbf{\emph{Projective Dimension}}]
    Let $A$ be a ring, $M\in Mod_{A}$.
    Define the projective dimension of $M$ to be the minimal length of projective resolution of $M$, denoted by $\projdim(M)$.
\end{definition}
\begin{proposition}
    Let $A$ be a ring, $M\in Mod_{A}$.
    Then we have \\
    (1)$M$ is projective $\iff$ $\ext^{i}(M, N) = 0$ for all $i > 0$ and $N\in Mod_{A}$ $\iff$ $\ext^{1}(M, N) = 0$ for all $N\in Mod_{A}$. \\
    (2)$\projdim(M) \le n$ if and only if $\ext^{i}(M, N) = 0$ for all $i > n$ and $N\in Mod_{A}$. \\
    (3)if $A$ is noetherian local ring and $M$ is finitely generated with $\projdim(M) < \infty$,
    then $\projdim(M) + \depth(M) = \depth(A)$. \\
    (3)if $A$ is noetherian local ring and $M$ is finitely generated with $\projdim(M) < \infty$,
    then $\projdim(M) \ge n$ if and only if $\ext^{i}(M, A) = 0$ for all $i > n$.
    \label{Proposition 11.12} 
\end{proposition}
\subsection{Serre Duality in General (cont.)}
\begin{definition}[\textbf{\emph{Cohen-Macaulay Schemes}}]
    Let $X$ be a locally noetherian scheme.
    $X$ is called Cohen-Macaulay if all its stalks are Cohen-Macaulay.
\end{definition}
\begin{theorem}[\textbf{\emph{Serre Duality for Projective Schemes}}]
    Let $X$ be a projective scheme over $k$, $\dim(X) = n$, $\omega_{X}^{o}$ dualizing sheaf, $\mathcal{O}_{X}(1)$ very ample invertible sheaf on $X$.
    Then we have that \\
    (1)for all coherent sheaf $\mathcal{F}$ on $X$ and $i > 0$, there exists natural homomorphism
    \begin{equation*}
        \theta^{i}: \ext^{i}(\mathcal{F}, \omega_{X}^{o}) \longrightarrow H^{n - i}(X, \mathcal{F})^{\lor} 
    \end{equation*}
    functorial in $\mathcal{F}$ such that $\theta^{0}$ is given by the definition of $\omega_{X}^{o}$. \\
    (2)the following conditions ar equivalent: \\
    (i)X is pure and Cohen-Macaulay \\
    (ii)for all locally sheaf $\mathcal{F}$ of finite rank on $X$, $i < n$ and $q$ large enough, $H^{i}(X, \mathcal{F}(-q)) = 0$. \\
    (iii)$\theta^{i}$ are isomorphic for all $i \ge 0$.
    \label{Theorem 11.3}
\end{theorem}
\begin{proof}
    For (1), since $X$ is projective, by Corollary \ref{Corollary 6.1}, there is an exact sequence
    \begin{equation*}
        \oplus \mathcal{O}_{X}(-q) = \mathcal{E} \longrightarrow \mathcal{F} \longrightarrow o
    \end{equation*}
    for $q$ large enough.
    Then by Serre Theorem, $\ext^{i}(\mathcal{E}, \omega_{X}^{o}) \cong H^{i}(X, \omega_{X}^{o}(q)) = 0$ for all $i > 0$ and $q$ large enough.
    By Theorem \ref{Theorem 11.1}, $\ext^{i}(\cdot, \omega_{x}^{o})$ is universal contravariant $\delta$-functor.
    By universal property, there exists unique morphism of $\delta$-functors $\theta^{i}$ extending $\theta^{0}$.
    \par
    For (2), first prove (i) induce (ii).
    Assume that $X \subseteq \mathbb{P}_{k}^{N}$ for some $N$ of codimension $r$.
    Then for all locally free sheaf $\mathcal{F}$ of finite rank on $X$ and $x\in X$ closed point,
    we have that $\depth(\mathcal{F}_{x}) = \depth(\mathcal{O}_{X, x}) = n = \dim(\mathcal{O}_{X, x})$ since $X$ is pure of dimension $n$ and Cohen-Macaulay.
    Moreover, as $\mathcal{O}_{\mathbb{P}_{k}^{N}, x}$ is a regular local ring of dimension $N$, get $\\depth(\mathcal{F}_{x}) = n$ also as $\mathcal{O}_{\mathbb{P}_{k}^{N}, x}$-module.
    By Proposition \ref{Proposition 11.12}, $\projdim(\mathcal{F}_{x}) = N - n = r$ for all closed point $x\in X$.
    Then also by Proposition \ref{Proposition 11.12}, $(\sheafext_{\mathcal{O}_{\mathbb{P}_{k}^{N}}}^{i}(\mathcal{F}, \mathcal{G}))_{x} \cong \ext_{\mathcal{O}_{\mathbb{P}_{k}^{N}}}^{i}(\mathcal{F}_{x}, \mathcal{G}_{x}) = 0$ so that $\sheafext_{\mathcal{O}_{\mathbb{P}_{k}^{N}}}^{i}(\mathcal{F}, \mathcal{G}) = 0$ for all $i > N - n$.
    On the other hand, for $q$ large enough, $H^{i}(X, \mathcal{F}(-q))^{\lor} \cong \ext_{\mathcal{O}_{\mathbb{P}_{k}^{N}}}^{N - i}(\mathcal{F}, \omega_{\mathbb{P}_{k}^{N}}(q))$.
    Thus $H^{i}(X, \mathcal{F}(-q))^{\lor} = 0$ for $i < n$ and $q$ large enough.
    \par
    For $(ii) \Rightarrow (i)$, argue backwards with $\mathcal{F} = \mathcal{O}_{X}$ and get $\sheafext_{\mathcal{O}_{\mathbb{P}_{k}^{N}}}^{i}(\mathcal{O}_{X}, \omega_{\mathbb{P}_{k}^{N}}) = 0$ for all $i > N - n \ge r$.
    Then for all closed point $x\in X$, $\ext_{\mathcal{O}_{\mathbb{P}_{k}^{N}}}^{i}(\mathcal{O}_{X, x}, \mathcal{O}_{\mathbb{P}_{k}^{N}, x}) = 0$ for all $i > N - n$.
    By Proposition \ref{Proposition 11.12}, $\projdim(\mathcal{O}_{X, x}) \le N - n$ so that $\depth(\mathcal{O}_{X, x}) \ge n = \dim(X)$.
    While $\depth(\mathcal{O}_{X, x}) \le n$, get $\mathcal{O}_{X, x}$ Cohen-Macaulay of dimension $n$ for all closed point $x\in X$.
    By Proposition \ref{Proposition 11.10}, $\mathcal{O}_{X, y}$ is Cohen-Macaulay for all $y\in X$ since $\mathcal{O}_{X, y}$ is localization of $\mathcal{O}_{X, x}$ for some closed poitn $x$.
    \par
    For $(ii) \Rightarrow (iii)$, suffice to show $H^{n - i}(X, \cdot)^{\lor}$ is universal $\delta$-functor.
    For all $\mathcal{F}$ coherent sheaf on $X$, we have exact sequence
    \begin{equation*}
        \oplus \mathcal{O}_{X}(-q) = \mathcal{E} \longrightarrow \mathcal{F} \longrightarrow 0
    \end{equation*}
    for $q$ large enough.
    By (ii), get $H^{n - i}(X, \mathcal{E})^{\lor} = \oplus H^{n - i}(X, \mathcal{O}_{X}(-q))^{\lor} = 0$ for all $i > 0$ and $q$ large enough.
    Then apply Theorem \ref{Theorem 11.1}.
    \par
    For $(iii) \Rightarrow (ii)$, for all locally free sheaf $\mathcal{F}$ of finite rank on $X$, by (iii) and Serre Theorem, get
    \begin{equation*}
        H^{i}(X, \mathcal{F}(-q))^{\lor} \overset{\theta^{n - i}}{\longrightarrow} \ext^{n - i}(\mathcal{F}(-q), \omega_{X}^{o}) \cong H^{n - i}(X, \mathcal{F}^{\lor} \otimes \omega_{X}^{o}(q)) = 0
    \end{equation*}
    for all $i > 0$ and large enough.
\end{proof}
\begin{remark}
    Let $X \subseteq \mathbb{P}_{k}^{N}$ be closed subscheme of codimension $r$ as before.
    Then $X$ is pure and Cohen-Macaulay if and only if $\sheafext_{\mathcal{O}_{\mathbb{P}_{k}^{N}}}^{i}(\mathcal{O}_{X}, \omega_{\mathbb{P}_{k}^{N}}) = 
    \left\{
        \begin{aligned}
            & 0 & i \neq r \\
            & \omega_{X}^{o} & i = r
        \end{aligned}
    \right.$
\end{remark}
\begin{definition}
    A closed immersion $X \hookrightarrow Y$ of locally noetherian schemes with ideal sheaf $\mathcal{I}$ is called a regular immersion (of codimension $r$) if for all $x\in X$,
    $\mathcal{I}_{x} \subseteq \mathcal{O}_{Y, x}$ is generated by a regular sequence (of $r$ elements).
\end{definition}
\begin{remark}
    In the old language, regular immersion into smooth $k$-scheme is called "local complete intersection".
\end{remark}
\begin{proposition}
    All schemes discussed are locally noetherian. \\
    (1)Let $X \hookrightarrow Y$ be a regular immersion of codimension $r$ with ideal sheaf $\mathcal{I}$.
    Then by Proposition \ref{Proposition 11.11}, $\mathcal{I}/\mathcal{I}^{2}$ is locally free of rank $r$ and $C_{X/Y} \cong N_{X/Y}$. \\
    (2)(composition) Let $f: X \hookrightarrow Y$ and $g: Y \hookrightarrow Z$ be regular immersions.
    Then by Proposition \ref{Proposition 11.10}, $g \circ f: X \hookrightarrow Z$ is still a regular immersion,
    in which case, the following sequence is exact
    \begin{equation*}
        0 \longrightarrow N_{X/Y} \longrightarrow N_{X/Z} \longrightarrow f^{\ast}N_{Y/Z} \longrightarrow 0
    \end{equation*} 
    \\
    (3)(flat base change) Let $f: X \hookrightarrow Y$ be a regular immersion and $g: Y' \longrightarrow Y$ be a flat morphism.
    Then $f': X' = X \times_{Y} Y' \hookrightarrow Y'$ is a regular immersion with $\mathcal{N}_{X'/Y'} = {g'}^{\ast}\mathcal{N}_{X/Y}$. \\
    (4)Let $f: X \hookrightarrow Y$ be a regular immersion of $S$-schemes.
    Assume that $X$ is smooth over $S$.
    Then by Proposition \ref{Proposition 11.11}, $f$ is regular immersion if and only if $Y$ is smooth over $S$ in a neighbourhood of $X$,
    in which case, the following sequence is exact
    \begin{equation*}
        0 \longrightarrow T_{X/S} \longrightarrow f^{\ast}T_{Y/S} \longrightarrow N_{X/Y} \longrightarrow 0
    \end{equation*} 
    \label{Proposition 11.13}
\end{proposition}
\begin{example}
    (1)Regular immersion of codimension $1$ is equivalent to effective Cartier divisor. \\
    (2)Let $f: X \longrightarrow S$ be a (separated) flat morphism of finite type between locally noetherian schemes with fibers pure of dimension $n$.
    Then $f$ is smooth if and only if $\Delta_{X/S}: X \longrightarrow X \times_{S} X$ is regular immersion of codimension $n$.
    "$\Rightarrow$" is by Proposition \ref{Proposition 11.11}.
    For "$\Leftarrow$", assume that $\mathcal{I}$ is the ideal sheaf corresponds to $\Delta_{X/S}$.
    Note that $\mathcal{I}/\mathcal{I}^{2} \cong \Omega_{X/S}$.
    Then Proposition \ref{Proposition 9.6} tells us that $X$ is smooth over $S$. \\
    (3)Let $X \hookrightarrow Y$ be regular immersion of codimension $r$.
    Then by Proposition \ref{Proposition 11.10}, $X$ is pure and Cohen-Macaulay if $Y$ is pure and Cohen-Macaulay.
\end{example}
\begin{theorem}
    Let $X \hookrightarrow \mathbb{P}_{k}^{N}$ be a regular immersion of codimension $r$ with ideal sheaf $\mathcal{I}$.
    Since $\mathbb{P}_{k}^{N}$ is pure and Cohen-Macaulay, then $X$ is pure and Cohen-Macaulay.
    Then $\omega_{X}^{o} \cong \omega_{\mathbb{P}_{k}^{N}}\big{|}_{X} \otimes_{\mathcal{O}_{X}} \wedge^{r} (\mathcal{I}/\mathcal{I}^{2})^{\lor}$.
    In particular, $\omega_{X}^{o}$ is an invertible sheaf.
    \label{Theorem 11.4}
\end{theorem}
\begin{corollary}
    Let $X \subseteq \mathbb{P}_{k}^{N}$ be a smooth projective scheme over $k$ pure of dimension $n$.
    Then $\omega_{X}^{o} \cong \omega_{X}$.
    \label{Corollary 11.2}
\end{corollary}
\begin{proof}
    Let $X \subseteq \mathbb{P}_{k}^{N}$.
    Can assume that $X$ is pure of codimension $r$ with ideal sheaf $\mathcal{I}$.
    As both $X$ and $\mathbb{P}_{k}^{N}$ are smooth over $k$, by Proposition \ref{Proposition 11.11}, $X \hookrightarrow \mathbb{P}_{k}^{N}$ is a regular immersion of codimension $r$.
    Get exact sequence 
    \begin{equation*}
        0 \longrightarrow \mathcal{I}/\mathcal{I}^{2} \longrightarrow \Omega_{\mathbb{P}_{k}^{N}}\big{|}_{X} \longrightarrow \Omega_{X} \longrightarrow 0
    \end{equation*}
    Take determinant, get $\omega_{\mathbb{P}_{k}^{N}}\big{|}_{X} = \omega_{X} \otimes \wedge^{r} (\mathcal{I}/\mathcal{I}^{2})$.
    Then by previous theorem, $\omega_{X} \cong \omega_{\mathbb{P}_{k}^{N}}\big{|}_{X} \otimes_{\mathcal{O}_{X}} \wedge^{r} (\mathcal{I}/\mathcal{I}^{2})^{\lor} \cong \omega_{X}^{o}$.
\end{proof}
\begin{corollary}
    Take determinant, get $\omega_{\mathbb{P}_{k}^{N}}\big{|}_{X} = \omega_{X} \otimes \wedge^{r} (\mathcal{I}/\mathcal{I}^{2})$.
    Let $X \subseteq \mathbb{P}_{k}^{N}$ be a smooth projective scheme over $k$ pure of dimension $n$.
    Then $H^{q}(X, \Omega_{X}^{p}) \cong H^{n - q}(X, \Omega_{X}^{n - p})^{\lor}$.
    \label{Corollary 11.3}
\end{corollary}
\begin{reason}
    This comes from $\Omega_{X}^{n - p} \cong (\Omega_{X}^{p})^{\lor} \otimes_{\mathcal{O}_{X}} \omega_{x}$ and $\Omega_{X}^{p} \cong (\Omega_{X}^{n - p})^{\lor} \otimes_{\mathcal{O}_{X}} \omega_{X}$.
\end{reason}
\begin{remark}
    If further $X$ is geometrically integral, then $H^{n}(X, \omega_{X}) \cong H^{0}(X ,\mathcal{O}_{X})^{\lor} \cong k$.
\end{remark}
\begin{definition}[\textbf{\emph{Koszul Complex}}]
    Let $A$ be a ring, $a_{1}, \cdots, a_{r}\in A$.
    Define the Koszul complex $K_{\bullet} = K_{\bullet}(a_{1}, \cdots, a_{r})$ of $A$-modules like following \\
    (1)$K_{1}$ is a free $A$-module of rank $r$ with basis $e_{1}, \cdots, e_{r}$. \\
    (2)$K_{i} := \wedge^{i} K_{1}$ for all $i \ge 0$. \\
    (3)boundary map $d_{i}: K_{i} \longrightarrow K_{i - 1}$ is given by $e_{j_{1}} \wedge \cdots \wedge e_{j_{i}} \longmapsto \sum_{k = 1}^{i} (-1)^{k - 1}a_{j_{k}}e_{j_{1}} \wedge \cdots \wedge \widehat{e_{j_{k}}} \wedge \cdots \wedge e_{j_{i}}$.
    \par
    For $M\in Mod_{A}$, set $K_{\bullet}(M) = K_{\bullet} \otimes_{A} M$.
    In particular, $h_{0}(K_{\bullet}(M)) \cong M/(a_{1}, \cdots, a_{r})M$.
\end{definition}
\begin{proposition}[\textbf{\emph{Regular sequence is Koszul regular}}]
    Let $A$ be a ring, $M\in Mod_{A}$, $a_{1}, \cdots, a_{r}\in A$.
    If $a_{1}, \cdots, a_{r}$ form a regular sequence for $M$,
    then $h_{i}(K_{\bullet}(M)) = 0$ for all $i > 0$ so that $K_{\bullet}(M) \longrightarrow M/(a_{1}, \cdots, a_{r})M \longrightarrow 0$ is exact.
    \label{Proposition 11.14}
\end{proposition}
\begin{proof}[\textbf{\emph{Proof of Theorem \ref{Theorem 11.4}}}]
    Recall that $\omega_{X}^{o} = \sheafext_{\mathcal{O}_{\mathbb{P}_{k}^{N}}}^{r}(\mathcal{O}_{X}, \omega_{\mathbb{P}_{k}^{N}})$.
    For all $x\in X$, $\mathcal{I}_{x}$ is generated by a regular sequence $a_{1}, \cdots, a_{r}\in \mathcal{O}_{\mathbb{P}_{k}^{N}, x}$.
    Then by Proposition \ref{Proposition 11.14}, the Koszul complex $K_{\bullet}(\mathcal{O}_{\mathbb{P}_{k}^{N}, x})$ is a finite free resolution of $\mathcal{O}_{\mathbb{P}_{k}^{N}, x}/(a_{1}, \cdots, a_{r}) \cong \mathcal{O}_{X, x}$.
    \par
    Since everything is noetherian, get a finite free resolution $K_{\bullet}(\mathcal{O}_{\mathbb{P}_{k}^{N}})$ of $\mathcal{O}_{X}$ over a neighbourhood $U$ of $x$.
    Over $U$, we have that $\sheafext_{\mathcal{O}_{\mathbb{P}_{k}^{N}}}^{r}(\mathcal{O}_{X}, \omega_{\mathbb{P}_{k}^{N}}) \cong h^{r}(\sheafhom_{\mathcal{O}_{\mathbb{P}_{k}^{N}}}(K_{\bullet}(\mathcal{O}_{\mathbb{P}_{k}^{N}}, \omega_{\mathbb{P}_{k}^{N}}))) \cong \omega_{\mathbb{P}_{k}^{N}}/(a_{1}, \cdots, a_{r})\omega_{\mathbb{P}_{k}^{N}}$ i.e. $\sheafext_{\mathcal{O}_{\mathbb{P}_{k}^{N}}}^{r}(\mathcal{O}_{X}, \omega_{\mathbb{P}_{k}^{N}}) \cong \omega_{\mathbb{P}_{k}^{N}}\big{|}_{X}$.
    But this isomorphism depends on the choice of $a_{1}, \cdots, a_{r}$.
    If $b_{i} = \sum_{j} c_{ij}a_{j}$ is another choice, then get a functor of $\det(c_{ij})$ for $K_{r} = \wedge^{r} K_{1}$ and the isomorphism differs by $\det(c_{ij})$.
    \par
    To fix this, consider $\mathcal{I}/\mathcal{I}^{2}$ which is locally free of rank $r$.
    In fact, it is free over $U$ with different basis $a_{1}, \cdots,  a_{r}$ and $b_{1}, \cdots, b_{r}$.
    Then $\wedge^{r} (\mathcal{I/\mathcal{I}^{2}})$ is free of rank $1$ over $U$ with two different basis $a_{1} \wedge \cdots \wedge a_{r}$ and $b_{1} \wedge \cdots \wedge b_{r}$, which differ by $\det(c_{ij})$.
    In conclusion, over $U$, we have that $\sheafext_{\mathcal{O}_{\mathbb{P}_{k}^{N}}}^{r}(\mathcal{O}_{X}, \omega_{\mathbb{P}_{k}^{N}}) \cong \omega_{\mathbb{P}_{k}^{N}}\big{|}_{X} \otimes_{\mathcal{O}_{X}} \wedge^{r} (\mathcal{I}/\mathcal{I}^{2})^{\lor}$.
    Glue these isomorphisms up to get $\omega_{X}^{o} \cong \omega_{\mathbb{P}_{k}^{N}}\big{|}_{X} \otimes_{\mathcal{O}_{X}} \wedge^{r}(\mathcal{I}/\mathcal{I}^{2})^{\lor}$.
\end{proof}

\subsection{Second Approach to Serre Duality}
\label{subsection:Serre_Duality_Second_Approach_to_Serre_Duality}

Let $X$ be a projective scheme over $k$ of dimension $n$.
Instead of $X \subseteq \mathbb{P}_{k}^{N}$, consider finite morphism $X \longrightarrow \mathbb{P}_{k}^{n}$.
Then we need some starting points \\
(1)(Noetherian Normalization) there exists finite morphism $f: X \longrightarrow \mathbb{P}_{k}^{n}$. \\
(2)(Miracle Flatness) if further $X$ is pure and Cohen-Macaulay, then $f$ is flat.
\begin{definition}
    Let $f: X \longrightarrow Y$ be a finite morphism between locally noetherian schemes.
    Then \\
    (1)For all quasi-coherent sheaf $\mathcal{G}$ on $Y$, $\sheafhom_{\mathcal{O}_{Y}}(f_{\ast}\mathcal{O}_{X}, \mathcal{G})$ is canonically a quasi-coherent $\mathcal{O}_{X}$-module,
    which yields a functor $f^{!}: Qcoh(Y) \longrightarrow Qcoh(X)$ called upper shrick. \\
    (2)$(f_{\ast}, f^{!})$ form a left-right adjoint pair i.e.
    \begin{equation*}
        \hom_{\mathcal{O}_{X}}(\mathcal{F}, f^{!}\mathcal{G}) \cong \hom_{\mathcal{O}_{Y}}(f_{\ast}\mathcal{F}, \mathcal{G})
    \end{equation*}
    functorial in $\mathcal{F}$ and $\mathcal{G}$. \\
    (3)there exists natural isomorphism 
    \begin{equation*}
        f_{\ast}\sheafhom_{\mathcal{O}_{X}}(\mathcal{F}, f^{!}\mathcal{G}) \cong \sheafhom_{\mathcal{O}_{Y}}(f_{\ast}\mathcal{F}, \mathcal{G})
    \end{equation*}
    functorial in $\mathcal{F}$ and $\mathcal{G}$.
\end{definition}
\begin{remark}
    Same statement holds for coherent sheaves if $X, Y$ are noetherian.
\end{remark}
\begin{proof}[\textbf{\emph{Proof of Serre Duality}}]
    Let $X$ be a projective scheme over $k$ pure of dimension $n$ and Cohen-Macaulay.
    Consider finite and flat morphism $f: X \longrightarrow \mathbb{P}_{k}^{n}$.
    Set $\omega_{X}^{o} := f^{!}\omega_{\mathbb{P}_{k}^{n}} = \sheafhom_{\mathcal{O}_{\mathbb{P}_{k}^{n}}}(f_{\ast}\mathcal{O}_{X}, \omega_{\mathbb{P}_{k}^{n}})$.
    First assume $\mathcal{F}$ locally free of finite rank on $X$.
    Then we have that
    \begin{equation*}
        \begin{split}
            \ext^{i}(\mathcal{F}, \omega_{X}^{o}) & \cong H^{i}(X, \mathcal{F}^{\lor} \otimes_{\mathcal{O}_{X}} \omega_{X}^{o}) \\
            & \cong H^{i}(X, \mathcal{F}^{\lor} \otimes_{\mathcal{O}_{X}} f^{!}\omega_{\mathbb{P}_{k}^{n}}) \\
            & \cong H^{i}(\mathbb{P}_{k}^{n}, f_{\ast}(\mathcal{F}^{\lor} \otimes_{\mathcal{O}_{X}} f^{!}\omega_{\mathbb{P}_{k}^{n}})) \text{ as $f$ is affine} \\
            & \cong H^{i}(\mathbb{P}_{k}^{n}, f_{\ast}\sheafhom_{\mathcal{O}_{X}}(\mathcal{F}, f^{!}\omega_{\mathbb{P}_{k}^{n}})) \\
            & \cong H^{i}(\mathbb{P}_{k}^{n}, \sheafhom_{\mathcal{O}_{\mathbb{P}_{k}^{n}}}(f_{\ast}\mathcal{F}, \omega_{\mathbb{P}_{k}^{n}})) \\
            & \cong H^{i}(\mathbb{P}_{k}^{n}, (f_{\ast}\mathcal{F})^{\lor} \otimes_{\mathcal{O}_{\mathbb{P}_{k}^{n}}} \omega_{\mathbb{P}_{k}^{n}}) \\
            & \cong H^{n - i}(\mathbb{P}_{k}^{n}, f_{\ast}\mathcal{F})^{\lor} \text{ by Serre duality for projective space} \\
            & \cong H^{n - i}(X, \mathcal{F})^{\lor} \text{ as $f$ is affine} 
        \end{split}
    \end{equation*}
    Then complete the proof by universal $\delta$-functors and Theorem \ref{Theorem 11.1}.
\end{proof}
\begin{remark}
    To discribe whar really happens, we need to use the language of derived category.
    For all $f: X \longrightarrow Y$ morphism between separated schemes of finite type over $k$,
    there exists $f^{!}: DCoh(Y) \longrightarrow DCoh(X)$ satisfying that \\
    (1)(composition) Let $f: X \longrightarrow Y$ and $g: Y \longrightarrow Z$ be two morphisms,
    there exists natural isomorphism $(g \circ f)^{!} \overset{\sim}{\longrightarrow} f^{!} \circ g^{!}$. \\
    (2)(adjointion) For $f: X \longrightarrow Y$ proper, $(Rf_{\ast}, f^{!})$ form a left-right adjoint pair i.e.
    \begin{equation*}
        \hom_{DCoh(X)}(\mathcal{F}^{\bullet}, f^{!}\mathcal{G}^{\bullet}) \cong \hom_{DCoh(Y)}(Rf_{\ast}\mathcal{F}^{\bullet}, \mathcal{G}^{\bullet})
    \end{equation*}
    functorial in $\mathcal{F}^{\bullet}$ and $\mathcal{G}^{\bullet}$.
\end{remark}
\begin{example}
    (1)If $f: X \longrightarrow Y$ is a finite morphism, 
    then $f^{!}\mathcal{G}^{\bullet} \cong R\sheafhom_{\mathcal{O}_{Y}}(f_{\ast}\mathcal{O}_{X}, \mathcal{G}^{\bullet})$ in $DCoh(X)$ \\
    (2)If $f: X \longrightarrow Y$ is smooth of relative dimension $d$, 
    then $f^{!}\mathcal{G}^{\bullet} \cong Lf^{\ast}\mathcal{G}^{\bullet} \otimes \otimes \omega_{X/Y}[d]$, 
    where $\omega_{X/Y}$ is the relative canonical ideal.
\end{example}
Now for all $X$ separated scheme of finite type over field $k$ with morphism $p: X \longrightarrow \spec k$.
Set $\omega_{x}^{\bullet} := p^{!}\mathcal{O}_{Spec(k)}$ called dualizing complex.
\begin{example}
    (1)$\omega_{\mathbb{P}_{k}^{n}}^{\bullet} \cong \omega_{\mathbb{P}_{k}^{n}}[n]$ \\
    (2)$X$ is pure of dimension $n$ and Cohen-Macaulay if and only if $\omega_{x}^{\bullet} \cong \omega_{X}^{o}[n]$.
\end{example}
Then adjointion $\hom_{DCoh(X)}(\mathcal{F}^{\bullet}, \omega_{X}^{\bullet}) \cong \hom_{k}(R\Gamma(X, \mathcal{F}^{\bullet}), k)$ induces Serre duality if $X$ is proper over $k$.
\begin{theorem}[\textbf{\emph{Grothendieck Duality}}]
    Let $f: X \longrightarrow Y$ be a proper morphism.
    Then $Rf_{\ast}R\sheafhom_{\mathcal{O}_{X}}(\mathcal{F}^{\bullet}, f^{!}\mathcal{G}^{\bullet}) \cong R\sheafhom_{\mathcal{O}_{Y}}(Rf_{\ast}\mathcal{F}^{\bullet}, \mathcal{G}^{\bullet})$ functorial in $\mathcal{F}^{\bullet}$ and $\mathcal{G}^{\bullet}$.
    \label{Theorem 11.5}
\end{theorem}

\section{Cohomology and Base Change}
\label{section:Cohomology and Base Change}

Recall give cartesian diagram of noetherian schemes with $f: X \longrightarrow Y$ separated and $\mathcal{F}$ coherent sheaf on $X$.
\begin{equation*}
    \begin{tikzcd}
        X' \arrow[r, "v"] \arrow[r, "g"] &
        Y \arrow[d, "f"] \\
        Y' \arrow[r, "u"] &
        Y
    \end{tikzcd}
\end{equation*}
We get base change morphism 
\begin{equation*}
    u^{\ast}(R^{i}f_{\ast}\mathcal{F}) \longrightarrow R^{i}g_{\ast}(v^{\ast}\mathcal{F}), \forall i \ge 0
\end{equation*}
If $f: X \longrightarrow Y$, then $u^{\ast}f_{\ast}\mathcal{F} \longrightarrow g_{\ast}v^{\ast}\mathcal{F}$ is isomorphic.
Also, if $u: Y' \longrightarrow Y$ is flat, then all the base change morphisms are isomorphic.
Consider cartesian diagram of fibre
\begin{equation*}
    \begin{tikzcd}
        X_{y} \arrow[r, hookrightarrow] \arrow[d] &
        X \arrow[d, "f"] \\
        \spec(k(y)) \arrow[r] &
        Y
    \end{tikzcd}
\end{equation*}
Then base change morphisms change to be 
\begin{equation*}
    R^{i}f_{\ast}\mathcal{F} \otimes_{\mathcal{O}_{Y}} k(y) \longrightarrow H^{i}(X_{y}, \mathcal{F}\big{|}_{X_{y}})
\end{equation*}
Want to know when the base change morphisms are isomorphic and how does $H^{i}(X_{y}, \mathcal{F}\big{|}_{X_{y}})$ change as $y\in Y$ varies.
\begin{example}
    Let $E$ be a curve of genus $1$ over $k$, $P\in E(k)$, $(E, P)$ elliptic curve.
    Let $X = E \times_{k} E$, $Y = E$ and $f$ be the diagonal map, $\Delta \subseteq X$ diagonal set.
    Set $\mathcal{L} := \mathcal{O}_{X}(\Delta -\{P\} \times_{k} E)$.
    For $g\in Y - E$, $\mathcal{L}\big{|}_{X_{y}}$ is trivial if and only if $y = P$.
    Then $H^{0}(X_{y}, \mathcal{L}\big{|}_{X_{y}}) =
    \left\{
        \begin{aligned}
            & k & y = P \\
            & 0 & otherwise
        \end{aligned}
    \right.$.
    \par
    On the other hand, $R^{0}f_{\ast}\mathcal{L} \cong f_{\ast}\mathcal{L}$ is torsion-free coherent sheaf on $Y$.
    Get $f_{\ast}\mathcal{L}$ is locally free of finite rank.
    To determine the rank, flat base change to the generic point $\eta$ with $\spec(k(\eta)) \longrightarrow Y$ flat.
    Get $f_{\ast}\mathcal{L} \otimes_{\mathcal{O}_{Y}} k(\eta) \cong H^{0}(X_{\eta}, \mathcal{L}\big{|}_{X_{\eta}}) = 0$ so that $f_{\ast}\mathcal{L} = 0$. 
\end{example}
\subsection{Hilbert Polynomial}
\begin{definition}[\textbf{\emph{Hilbert Polynomial}}]
    Let $X$ be a projective scheme over field $k$, $\mathcal{O}_{X}(1)$ very ample invertible sheaf on $X$, $\mathcal{F}$ coherent sheaf on $X$.
    Then the function $P_{\mathcal{F}}(m) := \chi(X, \mathcal{F}(m)) = \sum_{i = 0}^{\infty} (-1)^{i}\dim_{k}(H^{i}(X \mathcal{F}(m)))$,
    where $\chi(X, \mathcal{F})$ is the Euler characteristic of $\mathcal{F}$,
    if a polynomial in $m$ of degree equal to $\dim(Supp(\mathcal{F}))$, callled the Hilbert polynomial of $\mathcal{F}$.
\end{definition}
\begin{remark}
    By Serre Vanishing Theorem, $\dim_{k}(H^{0}(X, \mathcal{F}(m)))$ is a polynomial in $m$ for $m$ large enough.
\end{remark}
\begin{proof}
    By flat base change, can assume $k$ algebraically closed so that $k$ is infinite.
    Also, by pushing forward, can assume $X = \mathbb{P}_{k}^{N}$.
    Then for all coherent sheaf $\mathcal{F}$ on $X$, there exists hyperplane $H \subseteq X$ together with exact sequence
    \begin{equation*}
        0 \longrightarrow \mathcal{O}_{X}(-1) \longrightarrow \mathcal{O}_{X} \longrightarrow \mathcal{O}_{H} \longrightarrow 0
    \end{equation*}
    inducing exact sequence
    \begin{equation*}
        0 \longrightarrow \mathcal{F}(-1) \longrightarrow \mathcal{F} \longrightarrow \mathcal{F} \otimes_{\mathcal{O}_{X}} \mathcal{O}_{H} \longrightarrow 0
    \end{equation*}
    This is because finitely generated module over noetherian ring has finitely many associated prime ideals.
    While $k$ is infinite, can find $H$ avoiding all these associated prime ideals.
    Denote $\mathcal{I}_{H} := \mathcal{O}_{X}(-1)$ and $\mathcal{G} := \mathcal{F} \otimes_{\mathcal{O}_{X}} \mathcal{O}_{H}$.
    Then $\chi(X, \mathcal{F}(m)) = \chi(X, \mathcal{F}(m - 1)) + \chi(X, \mathcal{G}(m))$ so that $P_{\mathcal{F}}(m) - P_{\mathcal{F}}(m - 1) = P_{\mathcal{G}}(m)$.
    Also $\dim(Supp(\mathcal{G})) = \dim(Supp(\mathcal{F})) - 1$.
    By following lemma, done!
\end{proof}
\begin{lemma}
    (1)Any numerical polynomial can be written as $P(m) = \sum_{i = 0}^{n} a_{i}\binom{m}{i}$ where $a_{i}\in \mathbb{Z}$ and $n = \deg(P(m))$. \\
    (2)If $Q(m)$ is a numerical polynomial of degree $n - 1$ and $P(m)$ is a function on the integers satisfying that $P(m + 1) - P(m) = Q(m)$,
    then $P(m)$ is a numerical polynomial for degree $n$.
    \label{Lemma 12.1}
\end{lemma}
\begin{remark}
    For $\mathcal{F} = \mathcal{O}_{X}$, write $P_{X}(m) := P_{\mathcal{O}_{X}}(m)$ and $\deg(P_{X}(m)) = \dim(X)$.
    Define the degree of $X$ to be $\deg(X) := n!(\text{leading coefficient of } P_{X}(m))$, where $n = \dim(X)$.
\end{remark}
\begin{theorem}
    Let $f: X \longrightarrow Y$ be a projective morphism between noetherian schemes.
    If $\mathcal{F}$ is a coherent sheaf on $X$ which is flat over $Y$, then the Hilbert polynomial $P_{y}(m) := P_{\mathcal{F}\big{|}_{X_{y}}}(m)$ is locally constant as $y\in Y$ varies.
    If moreover $Y$ is reduced, then the converse holds.
    \label{Theorem 12.1}
\end{theorem}
\begin{proof}
    Since question is local on $Y$, can assume $Y = \spec A$.
    Also, by pushing forward, can assume $X = \mathbb{P}_{A}^{N}$.
    Can even assume $A$ is a noetherian local ring since flatness is local.
    Consider the following statements. \\
    (1)$\mathcal{F}$ is flat on $Y = \spec A$ \\
    (2)$H^{0}(X, \mathcal{F}(m))$ is a free $A$-module of finite rank for $m$ large enough \\
    (3)$P_{y}(m)$ is independent of $y\in Y = \spec A$.
    \par
    Claim that $(1) \iff (2) \Rightarrow (3)$ and $(3) \Rightarrow (2)$ if $A$ is reduced.
    For $(1) \Rightarrow (2)$, compute $H^{i}(X, \mathcal{F}(m))$ using Čech cohomology.
    Take $\mathcal{U} = \{U_{i}\}$ standard open covering of $X$, then $H^{i}(X, \mathcal{F}(m)) \cong H^{i}(C^{\bullet}(\mathcal{U}, \mathcal{F}(m)))$.
    As $\mathcal{F}$ is flat, each $C^{i}(\mathcal{U}, \mathcal{F}(m))$ is flat $A$-module.
    Moreover $H^{i}(X, \mathcal{F}(m)) = 0$ for $i > 0$ and $m$ large enough.
    Get exact sequence
    \begin{equation*}
        0 \longrightarrow H^{0}(X, \mathcal{F}(m)) \longrightarrow C^{0}(\mathcal{U}, \mathcal{F}(m)) \longrightarrow C^{1}(\mathcal{U}, \mathcal{F}(m)) \longrightarrow \cdots
    \end{equation*}
    for $m$ large enough.
    Note that for exact sequence in category of $A$-modules 
    \begin{equation*}
        0 \longrightarrow L \longrightarrow M \longrightarrow N \longrightarrow 0
    \end{equation*}
    if $M$ and $N$ are flat, then $L$ is flat.
    Repeatedly using the fact, get $H^{0}(X, \mathcal{F}(m))$ is finitely generated flat $A$-module for $m$ large enough.
    As $A$ is local, $H^{0}(X, \mathcal{F}(m))$ is free of finite rank for $m$ large enough.
    \par
    For $(2) \Rightarrow (1)$, set $M = \oplus_{m > m_{0}} H^{0}(X, \mathcal{F}(m))$ for large enough $m_{0}$.
    Then $M$ is flat $A$-module so that $\widetilde{M}$ is flat over $Y$.
    For $(2) \Rightarrow (3)$, suffice to show $P_{y}(m) = \rank_{A}(H^{0}(X, \mathcal{F}(m)))$ for $m$ large enough.
    We show that $H^{0}(X, \mathcal{F}(m)) \otimes_{A} k(y) \cong H^{0}(X_{y}, \mathcal{F}(m)\big{|}_{X_{y}})$ for $y\in Y$ and $m$ large enough, which doesn't need flatness.
    First consider closed point $y\in Y = \spec A$, then $k(y) = A/\mathfrak{m}$ where $\mathfrak{m}$ is the maximal ideal.
    There is an exact sequence
    \begin{equation*}
        A^{q} \longrightarrow A \longrightarrow k(y) \longrightarrow 0
    \end{equation*}
    View this sequence in category of modules of sheaves over $Y$.
    After pulling back and tensored by $\mathcal{F}$, get exact sequence
    \begin{equation*}
        \mathcal{F}^{q} \longrightarrow \mathcal{F} \longrightarrow \mathcal{F} \otimes_{\mathcal{O}_{X}} f^{\ast}\widetilde{k(y)} \longrightarrow 0
    \end{equation*}
    Then by Serre Vanishing Theorem, there is a commutative diagram with exact rows
    \begin{equation*}
        \begin{tikzcd}
            H^{0}(X, \mathcal{F}(m)) \otimes_{A} A^{q} \arrow[r] \arrow[d, "\sim"] &
            H^{0}(X, \mathcal{F}(m)) \arrow[r] \arrow[d, "\sim"] &
            H^{0}(X, \mathcal{F}(m)) \otimes_{A} k(y) \arrow[r] \arrow[d] &
            0 \\
            H^{0}(X, \mathcal{F}(m)^{q}) \arrow[r] &
            H^{0}(X, \mathcal{F}(m)) \arrow[r] &
            H^{0}(X, \mathcal{F} \otimes_{\mathcal{O}_{X}} f^{\ast}\widetilde{k(y)}) \arrow[r] &
            0 
        \end{tikzcd}
    \end{equation*}
    for $m$ large enough.
    By universal property of cokernel, we get that $H^{0}(X, \mathcal{F}(m)) \otimes_{A} k(y) \cong H^{0}(X, \mathcal{F} \otimes_{\mathcal{O}_{X}} f^{\ast}\widetilde{k(y)})$.
    In fact, by some argument of graded module, we have that $\mathcal{F} \otimes_{\mathcal{O}_{X}} f^{\ast}\widetilde{k(y)} \cong \mathcal{F}\big{|}_{X_{y}}$.
    Thus $H^{0}(X, \mathcal{F}(m)) \otimes_{A} k(y) \cong H^{0}(X_{y}, \mathcal{F}(m)\big{|}_{X_{y}})$ for $m$ large enough.
    \par
    For general point $y\in Y$, set $Y' = \spec(\mathcal{O}_{Y, y})$ which is flat over $Y$ and $g: X' \longrightarrow X$ is the pull back of $Y' \longrightarrow Y$ along $f$.
    Thus by flat base change, we can reduce $Y$ to $Y'$ so that now $y$ is closed point.
    Then by same argument, get $H^{0}(X_{y}, \mathcal{F}(m)\big{|}_{X_{y}}) \cong H^{0}(X', g^{\ast}\mathcal{F}(m)) \otimes_{A'} k(y) \cong H^{0}(X, \mathcal{F}(m))$ for $m$ large enough. 
    \par
    For $(3) \Rightarrow (2)$ with $A$ reduced, recall that a finitely generated module $M$ over reduced local ring $A$ is free if and only if $\dim_{k(Y)}(M \otimes_{A} k(y))$ is independent of $y\in \spec A$.
    Thus $H^{0}(X, \mathcal{F}(M))$ is free over $A$ for $m$ large enough if $P_{y}(m)$ is independent of $y\in \spec A$, as base change morphisms are isomorphic for $m$ large enough.
\end{proof}
\begin{corollary}
    Let $f: X \longrightarrow Y$ be a projective morphism between noetherian schemes, $\mathcal{F}$ coherent sheaf on $X$.
    Then $\mathcal{F}$ is flat over $Y$ if and only if $f_{\ast}(\mathcal{F}(m))$ is locally free of finite rank for $m$ large enough.
    \label{Corollary 12.1}
\end{corollary}
\begin{remark}
    This is in fact $(1) \iff (2)$.
\end{remark}
The same proof also gives 
\begin{proposition}[\textbf{\emph{Base Change without Flatness}}]
    Give a cartesian diagram of noetherian schemes.
    \begin{equation*}
        \begin{tikzcd}
            X' \arrow[r, "v"] \arrow[d, "g"] &
            Y \arrow[d, "f"] \\
            Y' \arrow[r, "u"] &
            Y
        \end{tikzcd}
    \end{equation*}
    Let $f: X \longrightarrow Y$ be a projective morphism, $\mathcal{F}$ coherent sheaf on $X$.
    Then the base change morphism
    \begin{equation*}
        u^{\ast}f_{\ast}(\mathcal{F}(m)) \longrightarrow f_{\ast}v^{\ast}(\mathcal{F}(m))
    \end{equation*}
    is isomorphic for $m$ large enough.
    \label{Proposition 12.1}
\end{proposition}
\begin{proof}
    Reduce to the case that $Y = \spec A$, $Y' = \spec(A')$ and $X = \mathbb{P}_{A}^{N}$.
    As $\mathcal{F}$ is coherent, there is an exact sequence
    \begin{equation*}
        \mathcal{E}_{1} \longrightarrow \mathcal{E}_{0} \longrightarrow \mathcal{F} \longrightarrow 0
    \end{equation*}
    where $\mathcal{E}_{i}$ are bothe of the form $\oplus \mathcal{O}_{X}(q)$.
    By pulling back, get exact sequence
    \begin{eqnarray}
        v^{\ast}\mathcal{E}_{1} \longrightarrow v^{\ast}\mathcal{E}_{0} \longrightarrow v^{\ast}\mathcal{F} \longrightarrow 0
    \end{eqnarray}
    Get commutative diagram with exact rows
    \begin{equation*}
        \begin{tikzcd}
            H^{0}(X, \mathcal{E}_{1}(m)) \otimes_{A} A' \arrow[r] \arrow[d, "\sim"] &
            H^{0}(X, \mathcal{E}_{0}(m)) \otimes_{A} A' \arrow[r] \arrow[d, "\sim"] &
            H^{0}(X, \mathcal{F}(m)) \otimes_{A} A' \arrow[r] \arrow[d] &
            0 \\
            H^{0}(X', v^{\ast}(\mathcal{E}_{1}(m))) \arrow[r] &
            H^{0}(X', v^{\ast}(\mathcal{E}_{0}(m))) \arrow[r] &
            H^{0}(X', v^{\ast}(\mathcal{F}(m))) \arrow[r] &
            0
        \end{tikzcd}
    \end{equation*}
    for $m$ large enough.
    Then by universal property of cokernel, done!
\end{proof}
\subsection{Three Important Theorems}
Let $f: X \longrightarrow Y$ be a projective morphism between noetherian schemes, $\mathcal{F}$ coherent sheaf on $X$ flat on $Y$.
Recall that we have base change morphisms for all $y\in Y$ as following,
\begin{equation*}
    \phi_{y}^{i}: (R^{i}f_{\ast}\mathcal{F})_{y} \otimes_{\mathcal{O}_{Y, y}} k(y) \longrightarrow H^{i}(X_{y}, \mathcal{F}\big{|}_{X_{y}})
\end{equation*}
where by Nakayama, dimension of the left side is upper semicontinuous.
\begin{theorem}[\textbf{\emph{Semicontinuousity}}]
    Let $f: X \longrightarrow Y$ be a projective morphism between noetherian schemes, $\mathcal{F}$ coherent sheaf on $X$ flat on $Y$.
    Then the function $y \longmapsto \dim_{k(y)}(H^{i}(X_{y}, \mathcal{F}\big{|}_{X_{y}}))$ is upper semicontinuous for all $i \ge 0$.
    \label{Theorem 12.2}
\end{theorem}
\begin{theorem}[\textbf{\emph{Grauert}}]
    Let $f: X \longrightarrow Y$ be a projective morphism between noetherian schemes, $\mathcal{F}$ coherent sheaf on $X$ flat on $Y$.
    Further assume $Y$ is reduced.
    Suppose $\dim_{k(y)}(H^{i}(X_{y}, \mathcal{F}\big{|}_{X_{y}}))$ is locally constant for some $i$. 
    Then $R^{i}f_{\ast}\mathcal{F}$ is locally free and $\phi_{y}^{i}$ is isomorphic for all $y\in Y$.
    \label{Theorem 12.3}
\end{theorem}
\begin{remark}
    The converse is obviously true with $Y$ reduced.
    When $Y$ is not reduced, there is a counterexample that $X = Y = \spec(k[t]/(t^{2}))$ and $\mathcal{F} = \overline{(t)}$.
\end{remark}
By these two theorems, if $Y$ is integral, then $\dim_{k(y)}(H^{i}(X_{y}, \mathcal{F}\big{|}_{X_{y}}))$ is constant in an open dense subset of $Y$ relative to $i$ where $R^{i}f_{\ast}\mathcal{F}$ is locally free.
\begin{theorem}[\textbf{\emph{Cohomology and Base Change}}]
    Let $f: X \longrightarrow Y$ be a projective morphism between noetherian schemes, $\mathcal{F}$ coherent sheaf on $X$ flat on $Y$.
    Suppose that for some $y\in Y$ and $i$, base change morphism $\phi_{y}^{i}$ is surjective.
    Then \\
    (1)there exists open neighbourhood $U\ni y$ such that $\phi_{y'}^{i}$ is isomorphic for all $y'\in U$. \\
    (2)$\phi_{y}^{i - 1}$ is also surjective if and only if $R^{i}f_{\ast}\mathcal{F}$ is locally free over an open neighbourhood $U\ni y$.
    \label{Theorem 12.4}
\end{theorem}
\begin{remark}
    There is a stronger statement of (1) \\
    (1$'$)there exists open neighbourhood $U\ni Y$ such that for all cartesian diagram of noetherian schemes 
    \begin{equation*}
        \begin{tikzcd}
            X' \arrow[r, "v"] \arrow[d, "g"] &
            X_{U} \arrow[d, "f"] \\
            Y' \arrow[r, "u"] &
            U
        \end{tikzcd}
    \end{equation*}
    there is an isomorphism $U^{\ast}(R^{i}f_{\ast}\mathcal{F}) \overset{\sim}{\longrightarrow} R^{i}g_{\ast}(v^{\ast}\mathcal{F})$.
\end{remark}
\begin{example}
    Let $f: X \longrightarrow Y$ be a projective morphism between noetherian schemes, $\mathcal{F}$ coherent sheaf on $X$ flat on $Y$. \\
    (1)If $H^{i}(X_{y}, \mathcal{F}\big{|}_{X_{y}}) = 0$ for some $i$ and all $y\in Y$, then by Theorem \ref{Theorem 12.4} and Nakayama Lemma, 
    $R^{i}f_{\ast}\mathcal{F} = 0$ and $R^{i - 1}f_{\ast}\mathcal{F}$ commutes with arbitrary base change.
    In particular, if $H^{1}(X_{y}, \mathcal{F}\big{|}_{X_{y}}) = 0$ for all $y\in Y$, then $R^{1}f_{\ast}\mathcal{F} = 0$ and $f_{\ast}\mathcal{F}$ is locally free of finite rank.
    \par
    (2)If $R^{i}f_{\ast}\mathcal{F} = 0$ for all $i \ge i_{0}$,
    then by Theorem \ref{Theorem 12.4} and Grothendieck Vanishing Theorem, 
    $H^{i}(X_{y}, \mathcal{f}\big{|}_{X_{y}}) = 0$ for all $y\in Y$ and $i \ge i_{0}$.
\end{example}
We present the proof in Vakil, which relies on the following observation by Mumford.
\begin{theorem}[\textbf{\emph{Mumford}}]
    Let $A$ be a noetherian ring, $C^{\bullet}$ complex of $A$-modules bounded on the right i.e. there exists $N$ large enough such that $C^{i} = 0$ for all $i > N$.
    Suppose that all cohomology groups $h^{i}(C^{\bullet})$ are finite $A$-modules.
    Then there exists complex $K^{\bullet}$ of free $A$-modules of finite rank such that $K^{i} = 0$ for all $i > N$ and $\alpha: K^{\bullet} \longleftarrow C^{\bullet}$ morphism of complexes such that $h^{i}(\alpha): h^{i}(K^{\bullet}) \overset{\sim}{\longrightarrow} h^{i}(C^{\bullet})$ is isomorphic for all $i$.
    \par
    Further, if $C^{i}$ is flat $A$-module for all $i$, then for all ring homomorphism $A \longrightarrow B$,
    $h^{i}(\alpha \otimes_{A} B): h^{i}(K^{\bullet} \otimes_{A} B) \overset{\sim}{\longrightarrow} h^{i}(C^{\bullet} \otimes_{A} B)$ is isomorphic for all $i$.
    \label{Theorem 12.5}
\end{theorem}
Back to proof of theorems, can assume that $Y = \spec A$, $X = \mathbb{P}_{A}^{N}$ and $f: \mathbb{P}_{A}^{N} \longrightarrow \spec A$ is the projection.
We apply Theorem \ref{Theorem 12.5} to Čech complex $C^{\bullet} := C^{\bullet}(\mathcal{U}, \mathcal{F})$, where $\mathcal{U} = \{U_{i}\}$ is standard open covering.
Get complex $K^{\bullet}$ of free $A$-module of finite rank
\begin{equation*}
    \cdots \longrightarrow K^{i - 1} \overset{\delta^{i - 1}}{\longrightarrow} K^{i} \overset{\delta^{i}}{\longrightarrow} K^{i + 1} \longrightarrow \cdots \longrightarrow K^{N} \longrightarrow 0
\end{equation*}
\begin{proof}[\textbf{\emph{Proof of Semicontinuousity}}]
    For $y\in Y = \spec A$, 
    \begin{equation*}
        \begin{split}
            \dim_{k(y)}H^{i}(X_{y}, \mathcal{F}\big{|}_{X_{y}}) = & \dim_{k(y)}(\ker(\delta^{i} \otimes_{A} k(y))) - \dim_{k(y)}(\im(\delta^{i - 1} \otimes_{A} k(y))) \\
            = & \dim_{k(y)}(K^{i} \otimes_{A} k(y)) - \dim_{k(y)}(\im(\delta^{i} \otimes_{A} k(y))) \\
            & - \dim_{k(y)}(\im(\delta^{i - 1} \otimes_{A} k(y)))
        \end{split}
    \end{equation*}
    Note that $\dim_{k(y)}(K^{i} \otimes_{A} k(y))$ is constant and $\dim_{k(y)}(\im(\delta^{i} \otimes_{A} k(y)))$ is lower semicontinuous, done!
\end{proof}
\begin{definition}
    Let $X$ be a noetherian scheme, $\mathcal{E}, \mathcal{F}$ locally free sheaves on $X$ of rank $r + a$ and $r + b$ respectively.
    A morphism $\phi: \mathcal{E} \longrightarrow \mathcal{F}$ is called of constant rank $r$ (or say morphism of vector bundles) if there exists open covering $\{U_{i}\}$ on $X$ such that for all $i$, there is a commutative diagram
    \begin{equation*}
        \begin{tikzcd}[sep = large]
            \mathcal{E}\big{|}_{U_{i}} \arrow[rr, "\phi\big{|}_{U_{i}}"] \arrow[d, "\sim"] &&
            \mathcal{F}\big{|}_{U_{i}} \arrow[d, "\sim"] \\
            \mathcal{O}_{U_{i}}^{\oplus r + a} \arrow[r, "projection"] &
            \mathcal{O}_{U_{i}}^{\oplus r} \arrow[r, "inclusion"] &
            \mathcal{O}_{U_{i}}^{\oplus r + b}
        \end{tikzcd}
    \end{equation*}
\end{definition}
\begin{remark}
    If $\phi: \mathcal{E} \longrightarrow \mathcal{F}$ is of constant rank $r$, 
    then $\ker(\phi)$, $\im(\phi)$, $\coker(\phi)$ and $\coim(\phi)$ are all locally free of rank $r$.
    Moreover, for all $x\in X$, $\phi_{x} \otimes_{\mathcal{O}_{X, x}} \identity_{k(x)}: \mathcal{E}_{x} \otimes_{\mathcal{O}_{X, x}} k(x) \longrightarrow \mathcal{F}_{x} \otimes_{\mathcal{O}_{X, x}} k(x)$ is of constant rank $r$ as morphism of locally free sheaves on $\spec(k(x))$.
    Also, the converse holds if $X$ is reduced.
    In addition, being of constant rank commutes with base change.
\end{remark}
\begin{lemma}
    $\phi: \mathcal{E} \longrightarrow \mathcal{F}$ is of constant rank if and only if $\coker(\phi)$ is locally free of constant rank.
    In particular, if $\phi$ is surjective, then it is of constant rank.
    \label{Lemma 12.2}
\end{lemma}
\begin{remark}
    This is a exercise using Nakayama Lemma.
\end{remark}
\begin{lemma}
    $\phi: \mathcal{E} \longrightarrow \mathcal{F}$ is of constant rank over an open neighbourhood of $x\in X$ if and only if $(\ker(\phi))\big{|}_{x} = \ker(\phi)_{x} \otimes_{\mathcal{O}_{X, x}} k(x) \longrightarrow \ker(\phi\big{|}_{x}) = \ker(\phi_{x} \otimes_{\mathcal{O}_{X, x}} \identity_{k(x)})$ is surjective.
    \label{Lemma 12.3} 
\end{lemma}
\begin{proof}
    "$\Rightarrow$": Obviously. 
    \par
    "$\Leftarrow$": choose bases $\mathcal{E}\big{|}_{x} \cong k(x)^{\oplus r + a}$ and $\mathcal{F}\big{|}_{x} \cong k(x)^{\oplus r + b}$ under which $\phi\big{|}_{x}: \mathcal{E}\big{|}_{x} \longrightarrow \mathcal{F}\big{|}_{x}$ given by projection and inclusion.
    Lift first $r$ basis elements of $\mathcal{E}\big{|}_{x}$ to local sections $e_{1}, \cdots, e_{r}$ of $\mathcal{e}$.
    Lift last $b$ basis elements of $\mathcal{F}\big{|}_{x}$ to local sections $f_{1}, \cdots, f_{b}$ of $\mathcal{F}$.
    Since $(\ker(\phi))\big{|}_{x} \longrightarrow \ker(\phi\big{|}_{x})$ is surjective, we can lift last $a$ basis elements of $\mathcal{E}\big{|}_{x}$ to local sections $k_{1}, \cdots, k_{a}$ of $\ker(\phi) \subseteq \mathcal{E}$.
    Then $e_{1}, \cdots, e_{r}, k_{1}, \cdots, k_{a}$ give $\mathcal{E} \cong \mathcal{O}_{X}^{\oplus r + s}$ and $\phi(e_{1}), \cdots, \phi(e_{r}), f_{1}, \cdots, f_{b}$ give $\mathcal{F} \cong \mathcal{O}_{X}^{\oplus r + b}$ over an open neighbourhood of $x$ under which $\phi$ takes the desired form.
\end{proof}
\begin{proof}[\textbf{\emph{Proof of Grauert}}]
    As $\dim_{k(y)}(H^{i}(X_{y}, \mathcal{F}\big{|}_{X_{y}}))$ is constant, both $\delta^{i - 1}\big{|}_{y}$ and $\delta^{i}\big{|}_{y}$ are of constant rank for all $y\in Y = \spec A$.
    As $Y$ is reduced, both $\delta^{i - 1}$ and $\delta^{i}$ are of constant rank.
    Consider exact sequence
    \begin{equation*}
        0 \longrightarrow h^{i}(K^{\bullet}) \longrightarrow \coker(\delta^{i - 1}) \longrightarrow \im(\delta^{i}) \longrightarrow 0
    \end{equation*}
    Note that $\coker(\delta^{i})$ and $\im(\delta^{i})$ are both locally free of finite rank and $\coker(\delta^{i - 1}) \longrightarrow \im(\delta^{i})$ is surjective, 
    by Lemma \ref{Lemma 12.2}, $h^{i}(K^{\bullet})$ is locally free of finite rank and commutes with arbitrary base change. 
\end{proof}
\begin{lemma}
    Given a morphism of complexes 
    \begin{equation*}
        \begin{tikzcd}
            K^{i - 1} \arrow[r, "\delta_{K}^{i - 1}"] \arrow[d, twoheadrightarrow] &
            K^{i} \arrow[r, "\delta_{K}^{i}"] \arrow[d] &
            K^{i + 1} \arrow[d] \\
            L^{i - 1} \arrow[r, "\delta_{L}^{i - 1}"] &
            L^{i} \arrow[r, "\delta_{L}^{i}"] &
            L^{i + 1}
        \end{tikzcd}
    \end{equation*}
    We have that $h^{i}(K^{\bullet}) \longrightarrow h^{i}(L^{\bullet})$ is surjective if and only if $\ker(\delta_{K}^{i}) \longrightarrow \ker(\delta_{L}^{i})$ is surjective.
    \label{Lemma 12.4}
\end{lemma}
\begin{reason}
    This is an easy exercise by Snake Lemma.
\end{reason}
Consider the following commutative diagram
\begin{equation*}
    \begin{tikzcd}
        K^{i - 1} \arrow[r, "\delta^{i - 1}"] \arrow[d] &
        K^{i} \arrow[r, "\delta^{i}"] \arrow[d] &
        K^{i + 1} \arrow[d] \\
        K_{y}^{i - 1} \arrow[r] \arrow[d, twoheadrightarrow] &
        K_{y}^{i} \arrow[r] \arrow[d] &
        K_{y}^{i + 1} \arrow[d] \\
        K^{i - 1}\big{|}_{y} \arrow[r, "\delta^{i - 1}\big{|}_{y}"] &
        K^{i}\big{|}_{y} \arrow[r, "\delta^{i}\big{|}_{y}"] &
        K^{i + 1}\big{|}_{y}
    \end{tikzcd}
\end{equation*}
\begin{proof}[\textbf{\emph{Proof of Cohomology and Base Change}}]
    For $y\in Y = \spec A$, $h^{i}(K^{\bullet}) \longrightarrow h^{i}(K^{\bullet}\big{|}_{y})$ is surjective,
    by Lemma \ref{Lemma 12.4}, is equivalent to that $(\ker(\delta^{i}))\big{|}_{y} \longrightarrow \ker(\delta^{i}\big{|}_{y})$ is surjective,
    and by Lemma \ref{Lemma 12.3}, is equivalent to that $\delta^{i}$ is of constant rank near $y\in Y$.
    Thus $\ker(\delta^{i})$ is locally free of finite rank and commutes with base change near $y$.
    Consider exact sequence
    \begin{equation*}
        K^{i - 1} \longrightarrow \ker(\delta^{i}) \longrightarrow h^{i}(K^{\bullet}) \longrightarrow 0
    \end{equation*}
    Note that $K^{i - 1}$ and $\ker(\delta^{i})$ both are locally free and commute with base change near $y$,
    get $h^{i}(K^{\bullet})$ commutes with base change near $y$.
    This proves (1) and (1$'$).
    \par
    For (2), $h^{i - 1}(K^{\bullet})\big{|}_{y} \longrightarrow h^{i - 1}(K^{\bullet}\big{|}_{y})$ is also surjective if and only if $\delta^{i - 1}$ is of constant rank near $y$.
    Obviously, if $\delta^{i - 1}$ is of constant rank near $y$, then $h^{i}(K^{\bullet})$ is locally free of finite rank near $y$.
    For the converse, if $h^{i}(K^{\bullet})$ is locally free of finite rank near $y$, then by Lemma \ref{Lemma 12.2}, $K^{i - 1} \longrightarrow \ker(\delta^{i})$ is of constant rank so that the composition $\delta^{i - 1}$ is also of constant rank.
\end{proof}
\begin{remark}[\textbf{\emph{Derived Base Change}}]
    We have discussed lots about base change morphisms, but where does the base change isomorphisms always hold?
    The answer is derived/homotopy cartesian diagram.
    \begin{equation*}
        \begin{tikzcd}
            X' \arrow[r, "v"] \arrow[d, "g"] &
            X \arrow[d, "f"] \\
            Y' \arrow[r, "u"] &
            Y
        \end{tikzcd}
    \end{equation*}
    Locally $X = \spec(B)$, $Y = \spec A$, $Y' = \spec(A')$ and $X' = \spec(B \derivedotimes_{A} A')$, where $\derivedotimes$ is the derived tensor product.
    Assume that all these derived schemes are qcqs.
    Then for all quasi-coherent sheaf $\mathcal{F}$ on $X$, $Lu^{\ast}(R^{i}f_{\ast}\mathcal{F}) \overset{\sim}{\longrightarrow} R^{i}g_{\ast}(Lv^{\ast}\mathcal{F})$ is isomorphic in $DQcoh(Y')$.
\end{remark}

\section{Case Study: Hilbert Schemes}
\label{section:Case Study: Hilbert Schemes}

Want to construct a scheme parametrizing all closed subschemes of $\mathbb{P}_{k}^{n}$.
Naively, we have that 
\begin{equation*}
    \{\text{closed subschemes of $\mathbb{P}_{k}^{n}$}\} \hookrightarrow \{\text{vector subspaces of $k[x_{1}, \cdots, x_{n}]$}\}
\end{equation*}
Problems are that RHS is infinite-dimensional Grassmannian manifold, this is not a one-to-one correspondence and the set $\{\text{finite-dimensional subspaces of $k[x_{1}, \cdots, x_{n}]$}\}/\sim$ is not clear to be a scheme.
To solve these problems, we need to define Hilbert functor and representability of functors.
\subsection{Hilbert Functor}
\label{subsection:Hilbert Functor}
By now, we assume all schemes discussed are noetherian.
\begin{definition}[\textbf{\emph{Representable Functor}}]
    Let $S$ be a scheme, $Sch_{S}$ category of $S$-schemes, $Set$ category of sets.
    Suppose $F: Sch_{S} \longrightarrow Set$ is a contravariant functor.
    We say an $S$-scheme $X$ and an element $U\in F(X)$ represent $F$ if for all $S$-scheme $T$, 
    $\hom_{X}(T, X) \longrightarrow F(T)\quad g \longmapsto g^{\ast}U := F(g)(U)$ is isomorphic, 
    $U$ is called the universal element/family over $X$.
    In particular, $(X, U)$ is unique up to isomorphism.
\end{definition}
\begin{definition}[\textbf{\emph{Hilbert Functor}}]
    Let $f: X \longrightarrow S$ be a projective morphism.
    The Hilbert functor $\hilbertfunctor_{X/S}: Sch_{S} \longrightarrow Set$ is given by
    \begin{equation*}
        T \longmapsto \{\text{closed subschemes $V \subseteq X \times_{S} T$ flat over $T$}\}
    \end{equation*}
\end{definition}
\begin{remark}
    More generally, $X \longrightarrow S$ is not necessarily projective.
    For $X \longrightarrow S$ strongly projective i.e. it factors through $\mathbb{P}(E)$ for some vector bundle on $S$ as following
    \begin{equation*}
        \begin{tikzcd}[sep = huge]
            X \arrow[r, hookrightarrow, "closed", "immersion" swap] \arrow[dr] &
            \mathbb{P}(E) \arrow[d] \\
            & S &
        \end{tikzcd}
    \end{equation*}
    we can also define similar Hilbert functor.
    \par
    In addition, it is clear that closed subscheme $V\in \hilbertfunctor_{X/S}(T)$ is one-to-one corresponding to a quotient coherent sheaf $\mathcal{O}_{X \times_{S} T} \twoheadrightarrow \mathcal{F}$ for some $\mathcal{F}$ flat over $T$,
    which gives us a way to rewrite the Hilbert functor.
    And the new version can be viewed as a special case of Quot functor.
\end{remark}
Recall flatness induces local constancy of Hilbert polynomial.
Fix $P = P(m)$ a numerical polynomial.
There is a finer Hilbert functor $\hilbertfunctor_{X/S}^{P}: Sch_{S} \longrightarrow Set$ given by 
\begin{equation*}
    T \longmapsto \{V\in \hilbertfunctor_{X/S}(T) \big{|} P_{V_{t}}(m) = P, \forall t\in T\}.
\end{equation*}
Similarly, as Remark 13.1, we can also rewrite $\hilbertfunctor_{X/S}^{P}$.
In particular, we have that $\hilbertfunctor_{X/S} = \sqcup_{P} \hilbertfunctor_{X/S}^{P}$.
\begin{example}
    Let $S = \spec k$, $X = \mathbb{P}_{k}^{n}$ and $T = \spec k$.
    Then $\hilbertfunctor_{\mathbb{P}_{k}^{n}/Spec(k)}(Spec(k)) = \{\text{closed subschemes of }\mathbb{P}_{k}^{n}\}$.
\end{example}
The main theorem we wanted can be stated as following
\begin{theorem}
    Let $f: X \longrightarrow S$ be a strongly projective morphism, $P$ a fixed numerical polynomial.
    The functor $\hilbertfunctor_{X/S}^{P}$ is represented by an $S$-scheme $\hilbertscheme_{X/S}^{P}$ and a closed subscheme $U_{X/S}^{P} \subseteq X \times_{S} \hilbertscheme_{X/S}^{P}$.
    Moreover, $\hilbertscheme_{X/S}^{P} \longrightarrow S$ is strongly projective factoring through $\mathbb{P}(E)$ for some vector bundle $E$ on $S$.
    \label{Theorem 13.1}
\end{theorem}
\begin{example}
    (1)Let $P = 1$ be constant.
    Then $\hilbertscheme_{X/S}^{1} = X$. \\
    (2)Let $C$ be a curve smooth and projective over $k$ and $P = n$ be constant.
    Then $\hilbertscheme_{C}^{n} = C^{(n)} := (C \times_{k} \cdots \times_{k} C)/S_{n}$. \\
    (3)Let $S$ be a surface smooth and projective over $k$ and $P = n$ be constant.
    Then $\hilbertscheme_{S}^{n} = S^{[n]} \longrightarrow S^{(n)}$ is not in general isomorphic for $n > 1$.
    In particular, when $n = 2$, we have that $\hilbertscheme_{S}^{2} = Blow_{\Delta}(S \times_{k} S)/S_{2}$.
\end{example}
\subsection{Special Case: Grassmannians}
\label{subsection:Special Case: Grassmannians}
\begin{definition}
    Let $S$ be a scheme, $E$ vect bundle on $S$, $r$ natural number.
    The Grassmannian functor $\grassmanmianfunctor{r}{E}: Sch_{S} \longrightarrow Set$ is given by 
    \begin{equation*}
        \begin{split}
            T \longmapsto & \{\text{subbundles of rank $r$ of $E \times_{S} T$}\} \\
            = & \{\text{projective subbundles of rank $r - 1$ of $\mathbb{P}(E) \times_{S} T$}\} \\
            \subseteq & \hilbertfunctor_{\mathbb{P}(E)/S}(T) 
        \end{split}
    \end{equation*}
\end{definition}
\begin{lemma}
    Let $V$ be a finite-dimensional vector space, $W \subseteq V$ subspace of dimension $r$.
    Then under the wedge map $V \otimes \wedge^{r} V \overset{\wedge}{\longrightarrow} \wedge^{r + 1} V$,
    $W$ and $\wedge^{r} W$ are annihilator of each other.
    \label{Lemma 13.1}
\end{lemma}
\begin{theorem}
    Let $S$ be a scheme, $E$ vect bundle on $S$, $r$ natural number.
    Then $\grassmanmianfunctor{r}{E}$ is represented by an $S$-scheme $\grassmannianscheme{r}{E}$ and a subbundle $U \subseteq E \times_{S} \grassmannianscheme{r}{E}$.
    \label{Theorem 13.2}
\end{theorem}
\begin{example}
    Let $r = 1$.
    Then $\grassmannianscheme{1}{E} = \mathbb{P}(E)$.
\end{example}
\begin{proof}[\textbf{\emph{Sketch Proof of Theorem \ref{Theorem 13.2}}}]
    Via "Plucier embedding".
    Set $Y = \wedge^{r} E$, then there is a map $p: Y \longrightarrow S$.
    Consider $\sigma: Y \longrightarrow p^{\ast}(\wedge^{r} E)\quad y \longmapsto "y"\in (p^{\ast}(\wedge^{r} E))_{y} = (\wedge^{r} E)_{s}$, where $s\in S$ and $y\in Y_{s} = (\wedge^{r} E)_{s}$.
    And $E \otimes \wedge^{r} E \overset{\wedge}{\longrightarrow} \wedge^{r + 1} E$ on $S$ induces $\phi := \wedge\sigma: p^{\ast}E \longrightarrow p^{\ast}(\wedge^{r + 1} E)$ on $Y$.
    For all $y\in Y_{s}$, $\phi_{y} := (\wedge\sigma)_{y}: E_{s} = (p^{\ast}E)_{y} \longrightarrow (p^{\ast}(\wedge^{r + 1} E))_{y} = \wedge^{r + 1} E_{s}$.
    \par
    Now, $\phi$ is locally given by a matrix $(m_{ij})$, where $m_{ij}$ are local functions on $Y$.
    By Lemma \ref{Lemma 13.1}, get $\rank(\phi) \ge n - r$ everywhere away from $0$, where $n = \rank(E)$.
    Thus $\rank(\phi) = n - r$ at $y\in Y \setminus \{0\}$ if and only if $\ker(\phi_{y})$ is $r$-dimensional and $y\in \wedge^{r} (\ker(\phi_{y})) \setminus \{0\}$.
    Let $Y_{r} \subseteq Y$ be the closed subscheme locally defined by $(n - r + 1) \times (n - r + 1)$ minors of $(m_{ij})$.
    Then $K := \ker(\phi\big{|}_{Y_{r} \setminus \{0\}}) \subseteq p^{\ast}E\big{|}_{Y_{r} \setminus \{0\}}$ is locally free of rank $r$ on $Y_{r} \setminus \{0\}$ i.e. a vector bundle.
    \par
    Finally, set $\mathbb{P}(Y) := \mathbb{(\wedge^{r} E)} \overset{q}{\longrightarrow} S$.
    $Y_{r}$ descends to closed subscheme $\grassmannianscheme{r}{E} \subseteq \mathbb{P}(Y)$ and $K$ descends to subbundle $U \subseteq q^{\ast}E\big{|}_{\grassmannianscheme{r}{E}} \cong E \times_{S} \grassmannianscheme{r}{E}$.
    Claim that $(\grassmannianscheme{r}{E}, U)$ represent $\grassmanmianfunctor{r}{E}$.
    Suppose that $g: T \longrightarrow \grassmannianscheme{r}{E}$ and get subbundle $g^{\ast}U \subseteq E \times_{S} T$ of rank $r$.
    And $\wedge^{r} (g^{\ast}U) \subseteq \wedge^{r} E \times_{S} T$ is the unique line bundle which annihilates $g^{\ast}U$.
    \par
    Conversely, for $F$ subbundle of rank $r$ of $E \times_{S} T$.
    Then line bundle $\wedge^{r} F \subseteq \wedge^{r} E \times_{S} T$ corresponds to $g: T \longrightarrow \mathbb{P}(Y)$ since projective cone represents the functor $\grassmanmianfunctor{1}{Y}$.
    Consider $E_{T} \otimes \wedge^{r} F \longrightarrow \wedge^{r + 1} E \times_{S} T$, whose kernel is locally free of rank $r$ by Lemma \ref{Lemma 13.1}.
    Thus $g: T \longrightarrow \mathbb{P}(Y)$ factors through $\grassmannianscheme{1}{E}$ and $F = g^{\ast}U$.
\end{proof}
\begin{corollary}
    $\grassmannianscheme{r}{E}$ is strongly projective factoring through $\mathbb{P}(Y)$.
    \label{Corollary 13.1}
\end{corollary}
\subsection{Castelnuovo-Mumford Regularity}
\begin{theorem}
    For all numerical polynomial $P$, there exists integer $m_{P}$ relative to $P$ such that for all ideal sheaf $\mathcal{I} \subseteq \mathcal{O}_{\mathbb{P}_{k}^{n}}$ with Hilbert polynomial $P$ and for all $q \ge m_{P}$,
    we have that \\
    (1)$H^{i}(\mathbb{P}_{k}^{n}, \mathcal{I}(q)) = 0$ for all $i > 0$. \\
    (2)$\mathcal{I}(q)$ is generated by global sections. \\
    (3)$H^{0}(\mathbb{P}_{k}^{n}, \mathcal{I}(q)) \otimes H^{0}(\mathbb{P}_{k}^{n}, \mathcal{O}_{\mathbb{P}_{k}^{n}}(1)) \longrightarrow H^{0}(\mathbb{P}_{k}^{n}, \mathcal{I}(q + 1))$ is surjective. 
    \label{Theorem 13.3}
\end{theorem}
\begin{example}
    Consider $\mathcal{F}_{a} := \mathcal{O}_{\mathbb{P}_{k}^{1}}(-a) \oplus \mathcal{O}_{\mathbb{P}_{k}^{1}}(a)$ for $a \ge 0$.
    Then $P_{\mathcal{F}_{a}}(m) = 2m + 2$ which is independent on $a$.
    But for all $a$, there is no $q$ such that $H^{1}(\mathbb{P}_{k}^{1}, \mathcal{F}_{a}(q)) = 0$ and there is no $q$ such that $\mathcal{F}_{a}(q)$ is globally generated.
    Thus, the condition that $\mathcal{I}$ is an ideal sheaf is necessary.
\end{example}
Recall the Hilbert functor $\hilbertfunctor_{X/S}^{P}$.
For simplicity, assume that $S = \spec k$, $X = \mathbb{P}_{k}^{n}$ and $T = \spec k$.
Let $V \subseteq \mathbb{P}_{k}^{n}$ be a closed subscheme with $P_{V} = P$, $\mathcal{I}_{V}$ ideal sheaf, $P_{\mathcal{I}_{V}} = Q := P_{\mathbb{P}_{k}^{n}} - P$.
Choose $m_{Q}$ in Theorem \ref{Theorem 13.3}.
Then $\mathcal{I}_{V}(m_{Q})$ is globally generated.
Consider exact sequence
\begin{equation*}
    0 \longrightarrow \mathcal{I}_{V}(m_{Q}) \longrightarrow \mathcal{\mathbb{P}_{k}^{n}}(m_{Q}) \longrightarrow \mathcal{O}_{V}(m_{Q}) \longrightarrow 0
\end{equation*}
Get $H^{0}(\mathbb{P}_{k}^{n}, \mathcal{I}_{V}(m_{Q})) \subseteq H^{0}(\mathbb{P}_{k}^{n}, \mathcal{O}_{\mathbb{P}_{k}^{n}}(m_{Q}))$ gives a subspace.
While $\mathcal{I}_{V}(m_{Q})$ is globally generated, the correspondence is one-to-one.
\par
Moreover, by Theorem \ref{Theorem 13.3}, we also get the dimension $\dim_{k}(H^{0}(\mathbb{P}_{k}^{n}, \mathcal{I}_{V}(m_{Q}))) = Q(m_{Q})$.
Set-theorectically, this gives an injection from the set of closed subschemes of $\mathbb{P}_{k}^{n}$ with Hilbert polynomial $P$ to $\grassmannianscheme{Q(m_{Q})}{H^{0}(\mathbb{P}_{k}^{n}, \mathcal{O}_{\mathbb{P}_{k}^{n}}(m_{Q}))}$.
We furthur need to put a scheme structure on the image.
\begin{definition}[\textbf{\emph{Castelnuovo-Munford Regularity}}]
    Let $\mathcal{F}$ be a coherent sheaf on $\mathbb{P}_{k}^{n}$.
    We say that $\mathcal{F}$ is $m$-regular if $H^{i}(\mathbb{P}_{k}^{n}, \mathcal{F}(m - i)) = 0$ for all $i > 0$.
\end{definition}
\begin{example}
    $\mathcal{O}_{\mathbb{P}_{k}^{n}}(a)$ is $m$-regular for $m \ge -a$.
\end{example}
\begin{proposition}
    If $\mathcal{F}$ is a $m$-regular coherent sheaf on $\mathbb{P}_{k}^{n}$.
    Then \\
    (1)$H^{i}(\mathbb{P}_{k}^{n}, \mathcal{F}(q)) = 0$ for all $i > 0$ and $q \ge m - i$ or say that $\mathcal{F}$ is $q$-regular for all $q \ge m$. \\
    (2)$\mathcal{F}(q)$ is generated by global sections for all $q \ge m$. \\
    (3)$H^{0}(\mathbb{P}_{k}^{n}, \mathcal{F}(q)) \otimes H^{0}(\mathbb{P}_{k}^{n}, \mathcal{O}_{\mathbb{P}_{k}^{n}}(1)) \longrightarrow H^{0}(\mathbb{P}_{k}^{n}, \mathcal{F}(q + 1))$ is surjective for all $q \ge m$.
    \label{Proposition 13.1}
\end{proposition}
\begin{proof}
    Induction on $n$.
    For $n = 0$, obviously ok.
    Take a general hyperplane $H \subseteq \mathbb{P}_{k}^{n}$ such that the following sequence is exact
    \begin{equation*}
        0 \longrightarrow \mathcal{F}(-1) \longrightarrow \mathcal{F} \longrightarrow \mathcal{F}\big{|}_{H} \longrightarrow 0
    \end{equation*}
    By twisting, get exact sequence
    \begin{equation*}
        0 \longrightarrow \mathcal{F}(q - 1) \longrightarrow \mathcal{F}(q) \longrightarrow \mathcal{F}\big{|}_{H}(q) \longrightarrow 0
    \end{equation*}
    Take cohomology group, get exact sequence
    \begin{equation*}
        H^{i}(\mathbb{P}_{k}^{n}, \mathcal{F}(m - i)) \longrightarrow H^{i}(H, \mathcal{F}\big{|}_{H}(m - i)) \longrightarrow H^{i + 1}(\mathbb{P}_{k}^{n}, \mathcal{F}(m - i - 1))
    \end{equation*}
    As $\mathcal{F}$ is $m$-regular, the left and right sides are both $0$ for all $i > 0$.
    Thus $\mathcal{F}\big{|}_{H}$ is $m$-regular on $H$.
    Then by induction hypothesis, $\mathcal{F}\big{|}_{H}$ is $q$-regular for all $q \ge m$.
    \par
    For (1), consider another exact sequence
    \begin{equation*}
        H^{i}(\mathbb{P}_{k}^{n}, \mathcal{F}(m - i)) \longrightarrow H^{i}(\mathbb{P}_{k}^{n}, \mathcal{F}(m - i + 1)) \longrightarrow H^{i}(H, \mathcal{F}\big{|}_{H}(m - i + 1))
    \end{equation*}
    Note that by Castelnuovo-Mumford regularity, the left and right sides are both $0$, get $\mathcal{F}$ is $(m + 1)$-regularity.
    Complete the proof by induction.
    \par
    For (3), consider the following commutative diagram with lower row exact
    \begin{equation*}
        \begin{tikzcd}[sep = small]
            & H^{0}(\mathbb{P}_{k}^{n}, \mathcal{F}(m)) \otimes H^{0}(\mathbb{P}_{k}^{n}, \mathcal{O}_{\mathbb{P}_{k}^{n}}(1)) \arrow[r, twoheadrightarrow, "v"] \arrow[d, "g"] &
            H^{0}(H, \mathcal{F}\big{|}_{H}(m)) \otimes H^{0}(H, \mathcal{O}_{H}(1)) \arrow[d, twoheadrightarrow, "f"] \\
            H^{0}(\mathbb{P}_{k}^{n}, \mathcal{F}(m)) \arrow[r, "w"] & 
            H^{0}(\mathbb{P}_{k}^{n}, \mathcal{F}(m + 1)) \arrow[r, "u"] &
            H^{0}(H, \mathcal{F}\big{|}_{H}(m + 1))
        \end{tikzcd}
    \end{equation*}
    As $\mathcal{F}$ is $m$-regular, $H^{1}(\mathbb{P}_{k}^{n}, \mathcal{F}(m - 1)) = 0$ so that $H^{0}(\mathbb{P}_{k}^{n}, \mathcal{F}(m)) \longrightarrow H^{0}(H, \mathcal{F}\big{|}_{H}(m))$ is surjective.
    Thus $v$ is also surjective by right exactness of tensor product.
    Since $\mathcal{F}\big{|}_{H}$ is also $m$-regular, by induction hypothesis, we get $f$ is surjective.
    Now $v \circ g = f \circ v$ is surjective.
    Note that $\im(w) \subseteq im(g)$, by exactness of lower row, get $g$ surjective.
    Complete the proof by induction. 
    While (2) is a direct consequence of (3) since by Theorem \ref{Theorem 6.3}, $\mathcal{F}(q)$ is globally generated for large enough $q$.
\end{proof}
\begin{proof}[\textbf{\emph{Proof of Theorem}}]
    By Proposition \ref{Proposition 13.1}, suffices to show that there exists integer $m_{P}$ relative to $P$ such that any ideal sheaf $\mathcal{I} \subseteq \mathcal{O}_{\mathbb{P}_{k}^{n}}$ with Hilbert polynomial $P$ is $m_{P}$-regular.
    Induction on $n$.
    For $n = 0$, obviously ok.
    Take a general hyperplane $H \subseteq \mathbb{P}_{k}^{n}$ such that the following sequence is exact
    \begin{equation*}
        0 \longrightarrow \mathcal{I}(m) \longrightarrow \mathcal{I}(m + 1) \longrightarrow \mathcal{I}\big{|}_{H}(m + 1) \longrightarrow 0
    \end{equation*}
    Suppose that $P(m) = \sum_{i = 0}^{n} a_{i}\binom{m}{i}$ where $a_{i}\in \mathbb{Z}$.
    Then
    \begin{equation*}
        \begin{split}
            \chi(H, \mathcal{I}\big{|}_{H}(m + 1)) & = \chi(\chi(H, \mathcal{I}\big{|}_{H}(m + 1)) , \mathcal{I}(m + 1)) - \chi(\chi(H, \mathcal{I}\big{|}_{H}(m + 1)) , \mathcal{I}(m)) \\
            & = \sum_{i = 0}^{n} a_{i}(\binom{m + 1}{i} - \binom{m}{i}) \\
            & = \sum_{i = 0}^{n - 1} a_{i + 1}\binom{m}{i}
        \end{split}
    \end{equation*}
    By induction hypothesis, there exist integer $m_{1}$ depending on $P_{\mathcal{I}\big{|}_{H}}$ such that $\mathcal{I}\big{|}_{H}$ is $m_{1}$-regular.
    Consider the following exact sequence of cohomology groups
    \begin{equation*}
        H^{i - 1}(H, \mathcal{I}\big{|}_{H}(m + 1)) \longrightarrow H^{i}(\mathbb{P}_{k}^{n}, \mathcal{I}(m)) \longrightarrow H^{i}(\mathbb{P}_{k}^{n}, \mathcal{I}(m + 1)) \longrightarrow H^{i}(H, \mathcal{I}\big{|}_{H}(m + 1))
    \end{equation*}
    Thus if $i \ge 2$ and $m \ge m_{1} - i$, then $H^{i}(\mathbb{P}_{k}^{n}, \mathcal{I}(m)) \cong H^{i}(\mathbb{P}_{k}^{n}, \mathcal{I}(m + 1))$.
    By Serre Vanishing Theorem, get $H^{i}(\mathbb{P}_{k}^{n}, \mathcal{I}(m)) = 0$ for all $i \ge 2$ and $m \ge m_{1} - i$.
    \par
    For $i = 1$, we can only conclude that $H^{1}(\mathbb{P}_{k}^{n}, \mathcal{F}(m)) \longrightarrow H^{1}(\mathbb{P}_{k}^{n}, \mathcal{F}(m + 1))$ is surjective for $m \ge m_{1} - 1$.
    Claim that if the map is isomorphic for some $m \ge m_{1} - 1$, then it is isomorphic for all $q \ge m$.
    Consider the following commutative diagram
    \begin{equation*}
        \begin{tikzcd}
            H^{0}(\mathbb{P}_{k}^{n}, \mathcal{F}(m + 1)) \otimes H^{0}(\mathbb{P}_{k}^{n}, \mathcal{O}_{\mathbb{P}_{k}^{n}}(1)) \arrow[r, twoheadrightarrow, "v"] \arrow[d] &
            H^{0}(H, \mathcal{F}\big{|}_{H}(m + 1)) \otimes H^{0}(H, \mathcal{O}_{H}(1)) \arrow[d, twoheadrightarrow, "f"] \\
            H^{0}(\mathbb{P}_{k}^{n}, \mathcal{F}(m + 2)) \arrow[r, "u"] &
            H^{0}(H, \mathcal{F}\big{|}_{H}(m + 2))
        \end{tikzcd}
    \end{equation*}
    As $v$ and $f$ are both surjective, get $u$ also surjective.
    Thus the map is isomorphic for $m + 1$.
    By induction, we prove the claim.
    \par
    With the claim, by Serre Vanishing Theorem, get $H^{1}(\mathbb{P}_{k}^{n}, \mathcal{I}(m)) = 0$ for all $m \ge m_{1} + \dim_{k}(H^{1}(\mathbb{P}_{k}^{n}, \mathcal{I}(m_{1} - 1)))$.
    Set $m_{2} := m_{1} + \dim_{k}(H^{1}(\mathbb{P}_{k}^{n}, \mathcal{I}(m_{1} - 1)))$.
    Then $\mathcal{I}$ is $m_{2}$-regular.
    What we need to do is to bound $\dim_{k}(H^{1}(\mathbb{P}_{k}^{n}, \mathcal{I}(m_{1} - 1)))$.
    Note that 
    \begin{equation*}
        \begin{split}
            \dim_{k}(H^{1}(\mathbb{P}_{k}^{n}, \mathcal{I}(m_{1} - 1))) & = \dim_{k}(H^{0}(\mathbb{P}_{k}^{n}, \mathcal{I}(m_{1} - 1))) - \chi(\mathbb{P}_{k}^{n}, \mathcal{I}(m_{1} - 1)) \\
            & \le \dim_{k}(H^{0}(\mathbb{P}_{k}^{n}, \mathcal{O}_{\mathbb{P}_{k}^{n}}(m_{1} - 1))) - \chi(\mathbb{P}_{k}^{n}, \mathcal{I}(m_{1} - 1))
        \end{split}
    \end{equation*}
    Thus $\mathcal{I}$ is $m_{P}$-regular for some $m_{P}$ depending on $P$.
\end{proof}
\subsection{Flattening Stratification}
Recall that a loally closed immersion $X \longrightarrow Y$ is a composition
\begin{equation*}
    X \overset{closed}{\longrightarrow} U \overset{open}{\longrightarrow} Y
\end{equation*}
\begin{definition}[\textbf{\emph{Stratification}}]
    A stratification of a scheme $S$ is a locally finite collection $\{S_{n}\}$ of locally closed subschemes of $S$ such that $S = \sqcup_{n} S_{n}$.
\end{definition}
\begin{remark}
    Here, we don't need the countable version of stratification.
    In the following, when we talk about stratification, we always mean there is a finite disjoint union.
\end{remark}
\begin{theorem}[\textbf{\emph{Flattening Stratification}}]
    Let $f: X \longrightarrow S$ be a strongly projective morphism, $\mathcal{F}$ coherent sheaf on $X$.
    Then there exists stratification $S = \sqcup_{P} S_{P}$ satisfying that a morphism $g: T \longrightarrow S$ factors as $T \longrightarrow S_{P} hookrightarrow S$ if and only if the pull back $g^{\ast}\mathcal{F}$ on $X \times_{S} T$ is flat over $T$ with fiberwise Hilbert polynomial $P$.
    In particular, such a stratification is unique.
    \label{Theorem 13.4}
\end{theorem}
\begin{example}
    Let $X = S$, $f = \identity_{S}: S \longrightarrow S$ and $\mathcal{F}$ be a coherent sheaf on $S$.
    Then there exists stratification $S = \sqcup_{r} S_{r}$ such that $\mathcal{F}\big{|}_{S_{r}}$ is locally free of rank r.
    Moreover, a morphism $g: T \longrightarrow S$ factors $T \longrightarrow S_{r} \hookrightarrow S$ if and only if $g^{\ast}\mathcal{F}$ on $T$ is locally free of rank $r$.
\end{example}
For the general case, recall that $\mathcal{F}$ is flat over $S$ if and only if $f_{\ast}(\mathcal{F}(m))$ is locally free of finite rank for $m$ large enough.
To start our proof of the theorem, first consider a result from Commutative Algebra.
\begin{proposition}[\textbf{\emph{Generic Flatness}}]
    Let $f: X \longrightarrow Y$ be a morphism of finite type between noetherian schemes with $Y$ integral, $\mathcal{F}$ coherent sheaf on $X$.
    Then there exists an open dense subset $U \subseteq Y$ such that $\mathcal{F}\big{|}_{U}$ is flat over $U$.
    \label{Proposition 13.2}
\end{proposition}
This proposition easily reduces to the following lemma.
\begin{lemma}
    Let $A$ be a noetherian integral domain, $B$ finitely generated $A$-algebra, $M$ finite $B$-module.
    Then there exists $f\in A$ such that $M_{f} = M \otimes_{A} A_{f}$ is a free $A_{f}$-module.
\end{lemma}
\begin{remark}
    The process of reducing to this lemma has used Corollary \ref{Corollary 12.1},
    so we can replace flatness by freeness.
\end{remark}
\begin{proof}
    Note that if $0 \longrightarrow L \longrightarrow M \longrightarrow N \longrightarrow 0$ is an exact sequence of $B$-modules, 
    $L_{f}$ is free over $A_{f}$ and $N_{g}$ is free over $A_{g}$, then $M_{fg}$ is free over $A_{fg}$.
    As $M$ is finite over $B$, we can take composition series
    \begin{equation*}
        0 = M_{0} \subset M_{1} \subset \cdots \subset M_{r} = M
    \end{equation*} 
    where $M_{i}/M_{i - 1} \cong A/\mathfrak{p}_{i}$ for some prime ideal $\mathfrak{p}_{i}$.
    Then we reduce to $M = B$ is finitely generated $A$-algebra and integral domain.
    \par
    Set $K = \fraction(A)$ and $L = \fraction(B)$.
    Induct on $n = \trdeg(L/K)$.
    For $n = 0$, $B \otimes_{A} K$ is a finite dimensional $K$-vector space.
    Thus there exists $f\in A$ which is a product of denominators such that $B_{f}$ is a finite $A_{f}$-module.
    Conclude by taking composition series of $B_{f}$.
    \par
    For $n > 0$, apply Noetherian Normalization Theorem to $B \otimes_{A} K$.
    Then there exist $f_{1}, \cdots, f_{n}\in B$ such that $B \otimes_{A} K$ is integral over $K[f_{1}, \cdots, f_{n}]$.
    Thus there exists $f\in A$ which is a product of denominators such that $B_{f}$ is integral over $A_{f}[f_{1}, \cdots, f_{n}]$.
    Since finiteness is equivalent to finitely generated and integral, we get $B_{f}$ is fintie over $A_{f}[f_{1}, \cdots, f_{n}]$.
    Can find $b_{1}, \cdots, b_{m}\in B_{f}$ generating a maximal free $A_{f}[f_{1}, \cdots, f_{n}]$-submodule.
    This gives an exact sequence
    \begin{equation*}
        0 \longrightarrow A_{f}[f_{1}, \cdots, f_{n}]^{m} \longrightarrow B_{f} \longrightarrow C \longrightarrow 0
    \end{equation*}
    We are left to deal with $C$.
    Conclude by taking composition series of $C$ as $A_{f}[f_{1}, \cdots, f_{n}]$-module and reducing to integral domain with $\trdeg(L/K) < n$.
\end{proof}
\begin{proof}[\textbf{\emph{Proof of Theorem}}]
    Step 1: Want a stratification $S = V_{1} \sqcup \cdots \sqcup V_{t}$ such that $\mathcal{F}\big{|}_{f^{-1}(V_{i})}$ is flat over $V_{i}$.
    Can assume that $S$ is reduced and $S_{1}, \cdots, S_{n}$ are irreducible components.
    Take $Y = S_{1} \setminus (\cup_{i = 2}^{n} S_{i})$.
    Apply generic flatness, get open dense subset $U \subseteq Y \subseteq S$ such that $\mathcal{F}\big{|}_{f^{-1}(U)}$ is flat over $U$.
    Repeat this process with $V = S \setminus U$.
    By noetherian induction, we get $S = V_{1} \sqcap \cdots \sqcup V_{t}$, where $V_{i}$ are integral locally closed subschemes such that $\mathcal{F}\big{|}_{f^{-1}(V_{i})}$ is flat over $V_{i}$.
    \par
    For simplicity, set $f_{i} := f\big{|}_{f^{-1}(V_{i})}$ and $\mathcal{F}_{i} := \mathcal{F}\big{|}_{f^{-1}(V_{i})}$.
    By construction, $\mathcal{F}_{i}$ is flat over $V_{i}$ with fiberwise Hilbert polynomial $P_{i}(m)$.
    Thus there exist finitely many numerical polynomial $P_{1}(m), \cdots, P_{t}(m)$ such that for all $s\in S$, $P_{\mathcal{F}\big{|}_{s}}(m) = P_{i}(m)$ for some $i$.
    By Serre Vanishing, for all $o = 1, \cdots, t$, there exists $m_{i}$ such that $R^{j}(f_{i})_{\ast}\mathcal{F}_{i}(m) = 0$ for all $j > 0$ and $m \ge m_{i}$.
    By cohomology and base change, get $H^{j}(X_{i}, \mathcal{F}\big{|}_{s}(m)) = 0$ for all $j > 0$, $m \ge m_{i}$ and $s\in V_{i}$,
    $(f_{i})_{\ast}\mathcal{F}_{i}(m)$ is locally free of rank $P_{i}(m)$ for all $m \ge m_{i}$ and $((f_{i})_{\ast}\mathcal{F}_{i})_{s} \otimes_{\mathcal{O}_{V_{i}, s}} k(s) \overset{\sim}{\longrightarrow} H^{0}(X_{i}, \mathcal{F}\big{|}_{s}(m))$ of dimension $P_{i}(m)$ is isomorphic for all $m \ge m_{i}$.
    \par
    Taking $N >> \max\{m_{i}\}$, we have gotten some results \\
    (1)There exist finitely many numerical polynomials $P_{1}, \cdots, P_{t}$ such that for all $s\in S$, $P_{\mathcal{F}\big{|}_{s}}(m) = P_{i}(m)$ for some $i$. \\
    (2)$H^{j}(X_{s}, \mathcal{F}\big{|}_{s}(m)) = 0$ for all $j > 0$ and $m \ge N$. \\
    (3)$(f_{\ast}\mathcal{F}(m))_{s} \otimes_{\mathcal{O}_{S, s}} k(s) \overset{\sim}{\longrightarrow} H^{0}(X_{s}, \mathcal{F}\big{|}_{s}(m))$ is isomorphic for all $m \ge N$, which has dimension $P_{i}(m)$ for some $i$.
    \par
    Step 2: Fix $n$ such that $\deg(P_{i}) \le n$ for all $P_{i}$ in (1).
    Note that we can determine $P_{i}$ by $n + 1$ values $P_{i}(N), \cdots, P_{i}(N + n)$.
    Consider $\mathcal{E}_{j} := f_{\ast}\mathcal{F}(N + j)$ on $S$, where $j = 0, 1, \cdots, n$.
    Recall that there exists locally closed subscheme $W_{jr} \hookrightarrow S$ which is universal for the property that $\mathcal{E}_{j}\big{|}_{W_{ir}}$ is locally free of rank $r$.
    Consider for $P(m)$ numerical polynomial, define $W_{P}^{0} := \cap_{j = 0}^{n} W_{jP(N + j)}$.
    By definition, a morphism $g: T \longrightarrow S$ factors through $W_{P}^{0}$ if and only if $g^{\ast}f_{\ast}\mathcal{F}(N + j)$ is locally free of rank $P(N + j)$ for all $j$.
    In particular, $s\in W_{P}^{0}$ if and only if $P_{\mathcal{F}\big{|}_{s}}(m) = P(m)$.
    Then by (2) and (1), $S = \sqcup_{P} W_{P}^{0}$ but $W_{P}^{0}$ might not have the correct scheme structure.
    \par
    Step 3: To correct scheme structure, we need to consider more values of $P$.
    Set $W_{P}^{k} := \cap_{j = 0}^{n + k} W_{jP(N + j)}$.
    Get a sequence
    \begin{equation*}
        W_{P}^{0} \supseteq W_{P}^{1} \supseteq \cdots
    \end{equation*}
    Note that $W_{P}^{k + 1} \subseteq W_{P}^{k}$ is locally closed subscheme and they have same support, get $W_{P}^{k + 1} \hookrightarrow W_{P}^{k}$ is closed immersion.
    Thus the sequence induces a sequence of ideal sheaves
    \begin{equation*}
        \mathcal{I}_{P}^{1} \subseteq \mathcal{I}_{P}^{2} \subseteq \cdots
    \end{equation*}
    By noetherian condition, the sequence stabilizes to an ideal $\mathcal{I}_{P}$ cutting out a closed subscheme $S_{P} \subseteq W_{P}^{0}$.
    \par
    Claim that $S = \sqcup_{P} S_{P}$ given the flattening stratification.
    As $S_{P} = \cap_{j = 0}^{\infty} W_{jP(N + j)}$, by definition, a morphism $g: T \longrightarrow S$ factors through $S_{P}$ if and only if $g^{\ast}f_{\ast}\mathcal{F}(N + j)$ is locally free of rank $P(N + j)$ for all $j \ge 0$.
    Consider the following commutative diagram
    \begin{equation*}
        \begin{tikzcd}
            X \times_{S} T \arrow[r] \arrow[d, "f_{T}"] &
            X \arrow[d] \\
            T \arrow[r] &
            S
        \end{tikzcd}
    \end{equation*}
    By base change without flatness, $g^{\ast}f_{\ast}\mathcal{F}(N + j) \cong (f_{T})_{\ast}g^{\ast}\mathcal{F}(N + j)$.
    Thus by Corollary \ref{Corollary 12.1}, a morphism $g: T \longrightarrow S$ factors through $S_{P}$ if and only if $g^{\ast}\mathcal{F}$ on $X \times_{S} T$ is flat over T with fiberwise Hilbert polynomial $P$.
    And $S = \sqcup_{P} S_{P}$ gives the flattening stratification.
\end{proof}
\subsection{Construction of Hilbert Scheme}
Let $S= \spec k$, $X = \mathbb{P}_{k}^{n}$ and $T = \spec k$.
Assume $V \subseteq \mathbb{P}_{k}^{n}$ is a closed subscheme with $P_{V} = P$ and $\mathcal{I}_{V} \subseteq \mathcal{O}_{\mathbb{P}_{k}^{n}}$ is the corresponding ideal sheaf with $P_{\mathcal{I}_{V}} = Q := P_{\mathbb{P}_{k}^{n}} - P$.
Choose $m_{Q}$ as in Castelnuovo-Munford Theorem.
Consider the following exact sequence
\begin{equation*}
    0 \longrightarrow \mathcal{I}_{V}(m_{Q}) \longrightarrow \mathcal{O}_{\mathbb{P}_{k}^{n}}(m_{Q}) \longrightarrow \mathcal{O}_{V}(m_{Q}) \longrightarrow 0
\end{equation*}
Get $H^{0}(\mathbb{P}_{k}^{n}), \mathcal{I}_{V}(m_{Q}) \subseteq H^{0}(\mathbb{P}_{k}^{n}, \mathcal{O}_{\mathbb{P}_{k}^{n}}(m_{Q}))$ is a subspace of dimension $Q(m_{Q})$,
which determines $\mathcal{I}_{V}(m_{Q})$ since $\mathcal{I}_{V}(m_{Q})$ globally generated.
As $\mathcal{I}_{V} = \mathcal{I}_{V}(m_{Q}) \otimes_{\mathcal{O}_{\mathbb{P}_{k}^{n}}} \mathcal{O}_{\mathbb{P}_{k}^{n}}(-m_{Q})$, the subspace also determines $\mathcal{I}_{V}$.
Thus we have an injection
\begin{equation*}
    \hilbertfunctor_{\mathbb{P}_{k}^{n}}^{P}(k) \hookrightarrow \grassmanmianfunctor{Q(m_{Q})}{H^{0}(\mathbb{P}_{k}^{n}, \mathcal{O}_{\mathbb{P}_{k}^{n}}(m_{Q}))}(k)
\end{equation*}
Now, with this idea, we take 5 steps to complish our proof.
\begin{proof}[\textbf{\emph{Construction of Hilbert Scheme}}]
    Step 1: For $X \longrightarrow S$ strongly projective, assume it factors through $\mathbb{P} = \mathbb{P}(E)$,
    where $E$ is a vector bundle on $S$ of rank $n + 1$.
    As fiber product preserve closed immersion, it is clear that there is an injection from set of closed subschemes of $X \times_{S} T$ flat over $T$ to the set of closed subschemes of $\mathbb{P} \times_{S} T$ flat over $T$ for any $T\in Sch_{S}$.
    Thus we get $\hilbertfunctor_{X/S} \hookrightarrow \hilbertfunctor_{\mathbb{P}/S}$ as a subfunctor.
    \par
    Step 2: Let $f: \mathbb{P} \longrightarrow S$ be a projection.
    For all morphism $g: T \longrightarrow S$, there is a commutative diagram
    \begin{equation*}
        \begin{tikzcd}
            \mathbb{P} \times_{S} T \arrow[r] \arrow[d, "f_{T}"] &
            \mathbb{P} \arrow[d, "f"] \\
            T \arrow[r, "g"] &
            S
        \end{tikzcd}
    \end{equation*}
    Let $V \subseteq \mathbb{P} \times_{S} T$ be a closed subscheme flat over $T$ with fiberwise Hilbert polynomial $P$.
    Denote $Q := P_{\mathbb{P}_{k}^{n}} - P$ and take $m_{Q}$ as in Castelnuovo-Munford Theorem.
    Consider the following exact sequences
    \begin{equation*}
        0 \longrightarrow \mathcal{I}_{V} \longrightarrow \mathcal{O}_{\mathbb{P} \times_{S} T} \longrightarrow \mathcal{O}_{V} \longrightarrow 0
    \end{equation*}
    and 
    \begin{equation*}
        0 \longrightarrow (f_{T})_{\ast}\mathcal{I}_{V}(m_{Q}) \longrightarrow (f_{T})_{\ast}\mathcal{O}_{\mathbb{P} \times_{S} T}(m_{Q}) \longrightarrow (f_{T})_{\ast}\mathcal{O}_{V}(m_{Q}) \longrightarrow R^{1}(f_{T})_{\ast}\mathcal{I}_{V}(m_{Q}) \longrightarrow \cdots
    \end{equation*}
    By Castelnuovo-Mumford, we know that the long sequence is in fact a short sequence.
    Moreover, by cohomology and base change, $(f_{T})_{\ast}\mathcal{I}_{V}(m_{Q})$ and $(f_{T})_{\ast}\mathcal{O}_{V}(m_{Q})$ are locally free of finite rank, respectively $Q(m_{Q})$ and $P(m_{Q})$.
    Thus $(f_{T})_{\ast}\mathcal{I}_{V}(m_{Q}) \hookrightarrow (f_{T})_{\ast}\mathcal{O}_{\mathbb{P} \times_{S} T}(m_{Q}) \cong g^{\ast}f_{\ast}\mathcal{O}_{\mathbb{P}}(m_{Q})$ is a subbundle of rank $Q(m_{Q})$.
    With this fact, get morphism $\hilbertfunctor_{\mathbb{P}/S}^{P} \longrightarrow \grassmanmianfunctor{Q(m_{Q})}{F}$, 
    where $F$ is the vector bundle associated with $f_{\ast}\mathcal{O}_{\mathbb{P}}(m_{Q})$.
    \par
    Step 3: Consider the following commutative diagram
    \begin{equation*}
        \begin{tikzcd}
            \mathbb{P} \times_{S} \grassmannianscheme{Q(m_{Q})}{F} \arrow[r, "p_{1}"] \arrow[d, "p_{1}"] &
            \mathbb{P} \arrow[d, "f"] \\
            \grassmannianscheme{Q(m_{Q})}{F} \arrow[r, "h"] &
            S
        \end{tikzcd}
    \end{equation*}
    Let $U_{Gr} \hookrightarrow h^{\ast}F$ be the universal subbundle.
    Then consider
    \begin{equation*}
        \begin{split}
            p_{2}^{\ast}U_{Gr} & \longrightarrow p_{2}^{\ast}h^{\ast} = p_{2}^{\ast}h^{\ast}(f_{\ast}\mathcal{O}_{\mathbb{P}}(m_{Q})) = p_{1}^{\ast}f^{\ast}f_{\ast}\mathcal{O}_{\mathbb{P}}(m_{Q}) \\
            & \longrightarrow p_{1}^{\ast}\mathcal{O}_{\mathbb{P}(m_{Q})}
        \end{split}
    \end{equation*}
    To some extent, we can view the image as "universal $\mathcal{I}_{V}(m_{Q})$".
    Set coherent sheaf $\mathcal{G} := \coker$ as "universal $\mathcal{O}_{V}(m_{Q})$".
    Apply Flattening Stratification Theorem to $\mathcal{G}(-m_{Q})$ on $\mathbb{P} \times-{S} \grassmannianscheme{Q(m_{Q})}{F} \overset{p_{2}}{\longrightarrow} \grassmannianscheme{Q(m_{Q})}{F}$.
    Get there exists "universal" locally closed subscheme $\hilbertscheme_{\mathbb{P}/S}^{P} \overset{i}{\hookrightarrow} \grassmannianscheme{Q(m_{Q})}{F}$ such that $i^{\ast}\mathcal{G}(-m_{Q})$ on $\mathbb{P} \times_{S} \hilbertscheme_{\mathbb{P}/S}^{P}$ is flat over $\hilbertscheme_{\mathbb{P}/S}^{P}$ with Hilbert polynomial $P$.
    And we have that $\mathcal{O}_{\mathbb{P} \times_{S} \grassmannianscheme{Q(m_{Q})}{F}} \twoheadrightarrow i^{\ast}\mathcal{G}(-m_{Q})$, whose kernel is an ideal sheaf $\mathcal{I}$.
    Take the corresponding closed subscheme $U_{\mathbb{P}/S}^{P}$.
    \par
    Claim that $(\hilbertscheme_{\mathbb{P}/S}^{P}, U_{\mathbb{P}/S}^{P})$ represents $\hilbertfunctor_{\mathbb{P}/S}^{P}$.
    By construction, $U_{\mathbb{P}/S}^{P}\in \hilbertfunctor_{\mathbb{P}/S}^{P}(\hilbertscheme_{\mathbb{P}/S}^{P})$.
    Conversely, let $V \subseteq \mathbb{P} \times_{S} T$ be a closed subscheme flat over $T$ with fiberwise Hilbert polynomial $P$.
    By step 2, get subbundle $(f_{T})_{\ast}\mathcal{I}_{V}(m_{Q}) \hookrightarrow g^{\ast}F$ of rank $Q(m_{Q})$.
    For all morphism $g: T \longrightarrow S$, consider its lifting $\widetilde{g}$ such that the following diagram commutes
    \begin{equation*}
        \begin{tikzcd}
            T \arrow[r, "\widetilde{g}"] \arrow[dr, "g"] &
            \grassmannianscheme{Q(m_{Q})}{F} \arrow[d, "h"] \\
            & S
        \end{tikzcd}
    \end{equation*}
    and $\widetilde{g}^{\ast}[U_{Gr} \hookrightarrow h^{\ast}F] \cong [(f_{T})_{\ast}\mathcal{I}_{V}(m_{Q}) \hookrightarrow g^{\ast}F]$.
    Thus $\widetilde{g}^{\ast}\mathcal{G} \cong \mathcal{O}_{V}(m_{Q})$ so that $\widetilde{g}^{\ast}\mathcal{G}(-m_{Q}) \cong \mathcal{O}_{V}$ flat over $T$ with fiberwise Hilbert polynomial $P$.
    By Flattening Stratification, $\widetilde{g}$ factors as $T \longrightarrow \hilbertscheme_{\mathbb{P}/S}^{P} \overset{i}{\hookrightarrow} \grassmannianscheme{Q(m_{Q})}{F}$.
    \par
    Step 4: Want to represent $\hilbertfunctor_{X/S}^{P}$ by a subscheme of $\hilbertscheme_{\mathbb{P}/S}^{P}$.
    Set $U' := U_{\mathbb{P}/S}^{P} \cap (X \times_{S} \hilbertscheme_{\mathbb{P}/S}^{P}) \subseteq X \times_{S} \hilbertscheme_{\mathbb{P}/S}^{P}$.
    By Flattening Stratification Theorem, there exists closed subscheme $\hilbertscheme_{X/S}^{P} \overset{j}{\hookrightarrow} \hilbertscheme_{\mathbb{P}/S}^{P}$ satisfying a morphism $T \longrightarrow \hilbertscheme_{\mathbb{P}/S}^{P}$ factors through $\hilbertscheme_{X/S}^{P}$ if and only if the pullback $U' \times_{\hilbertscheme_{\mathbb{P}/S}^{P}} T$ is flat over $T$ with fiberwise Hilbert polynomial $P$.
    Set $U_{X/S}^{P} := j^{\ast}U' \subseteq X \times_{S} \hilbertscheme_{X/S}^{P}$.
    \par
    Claim that $(\hilbertscheme_{X/S}^{P}, U_{X/S}^{P})$ represents $\hilbertfunctor_{X/S}^{P}$.
    Similar to the previous claim, by construction, $U_{X/S}^{P}\in \hilbertfunctor_{X/S}^{P}(\hilbertscheme_{X/S}^{P})$.
    Conversely, let $V \subseteq X \times_{S} T \subseteq \mathbb{P} \times_{S} T$ be a closed subscheme flat over $T$ with fiberwise Hilbert polynomial $P$.
    For all morphism $g: T \longrightarrow S$, there exists a lifting $\widetilde{g}$ such that the following diagram commutes
    \begin{equation*}
        \begin{tikzcd}
            T \arrow[r, "\widetilde{g}"] \arrow[dr, "g"] &
            \hilbertscheme_{\mathbb{P}/S}^{P} \arrow[d] \\
            & S
        \end{tikzcd}
    \end{equation*}
    and $V = \widetilde{g}^{\ast}U_{\mathbb{P}/S}^{P}$.
    Thus $V = \widetilde{g}^{\ast}(U_{\mathbb{P}/S}^{P} \cap (X \times_{S} \hilbertscheme_{\mathbb{P}/S}^{P})) = \widetilde{g}^{\ast}U' \subseteq X \times_{S} T$ is flat over $T$ with fiberwise Hilbert polynomial $P$.
    By Flattening Stratification Theorem, $\widetilde{g}$ factors through $\hilbertscheme_{X/S}^{P}$.
    \par
    Step 5:
    Now, we are only left to show the properity of Hilbert Scheme.
    By Construction, $\hilbertscheme_{X/S}^{P} \hookrightarrow \grassmannianscheme{Q(m_{Q})}{F}$ is loally closed immersion.
    Also, $\grassmannianscheme{Q(m_{Q})}{F}$ is strongly projective over $S$.
    Remains to show that $\hilbertscheme_{X/S}^{P}$ is proper over $S$.
    Use valuative criterion
    \begin{equation*}
        \begin{tikzcd}
            \spec K \arrow[r, "\widetilde{g \circ a}"] \arrow[d, "a"] &
            \hilbertscheme_{X/S}^{P} \arrow[d] \\
            \spec(R) \arrow[ur, dashrightarrow, "\widetilde{g}"] \arrow[r, "g"] &
            S 
        \end{tikzcd}
    \end{equation*}
    where $\widetilde{g \circ a}$ is one-to-one corresponding to a closed subscheme $V \subseteq X \times_{S} \spec K$ with Hilbert polynomial $P$.
    Want $\widetilde{g}$ which is one-to-one corresponding to a closed subscheme $\overline{V} \subseteq X \times_{S} \spec(R)$ flat over $R$ with fiberwise Hilbert polynomial $P$ whose restriction to $X \times_{S} \spec K$ is $V$.
    And this is given by a proposition in Hartshorne Chapter III 9.8.
\end{proof}
\end{document}